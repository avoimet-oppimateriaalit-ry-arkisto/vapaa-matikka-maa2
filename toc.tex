\Opensolutionfile{ans}[content/LIITE_vastaukset]

\providecommand{\lukufilter}[2]{#2} % ylikirjoitetaan kaanna_luku.sh -skriptistä.
\newcommand{\osa}[1]{\chapter{#1}} % osa
\newcommand{\nosa}[1]{\chapter*{#1} \addcontentsline{toc}{chapter}{#1}} %numeroimaton osa
\newcommand{\luku}[2]{\section{#2} \lukufilter{#1}{\input{content/TEORIA_#1} \input{content/TEHT_#1}}} % luku
\newcommand{\nluku}[2]{\section*{#2} \addcontentsline{toc}{section}{#2} \lukufilter{#1}{\input{content/#1}}} % numeroimaton luku
\newcommand{\vast}{\section*{Vastaukset} \addcontentsline{toc}{section}{Vastaukset} \begin{vastaussivu} \input{content/LIITE_vastaukset} \end{vastaussivu}}

\newpage
\nluku{LIITE_esipuhe}{Esipuhe}

\osa{Polynomi}
    \luku{polynomi}{Polynomi}
    \luku{polynomien_kertolasku}{Polynomeilla laskeminen}
	\luku{muistikaavat}{Muistikaavat}
	\section{Tulon nollasääntö ja tulon merkkisääntö}

\subsection*{Tulon merkkisääntö}

Pitkän matematiikan 1. kurssilla on esitetty seuraava sääntö kahden reaaliluvun tulolle:

\laatikko[Tulon merkkisääntö kahdelle tulon tekijälle]{
    \begin{itemize}
        \item Jos tulon tekijät ovat samanmerkkisiä, tulo on positiivinen.
        \begin{itemize}
	        \item Kahden positiivisen luvun tulo on positiivinen.
	        \item Kahden negatiivisen luvun tulo on positiivinen.
        \end{itemize}
        \item Jos tulon tekijät ovat erimerkkisiä, tulo on negatiivinen.
        \begin{itemize}
	        \item Positiivisen ja negatiivisen luvun tulo on negatiivinen.
        \end{itemize}
    \end{itemize}
}

Tulon merkkisäännöstä seuraa, että reaaliluvun neliö ei voi olla negatiivinen, koska kahden samanmerkkisen luvun tulo aina on positiivinen. Lyhyemmin ilmaistuna $x^2 \geq 0$, eli reaaliluvun neliö on aina \termi{epänegatiivinen}{epänegatiivinen}.

\begin{esimerkki}
Osoita, että funktio $P(x)=x^2+7$ saa vain positiivisia arvoja.
    \begin{esimratk}
	Koska $x^2 \geq 0$, lausekkeen $x^2+7$ arvo on vähintään $7$. Fuktio saa siis
	vain positiivisia arvoja.
    \end{esimratk}
\end{esimerkki}

Tulon merkkisääntö yleistyy mille tahansa määrälle tulontekijöitä. Mikäli tulossa on pariton $(1, 3, 5, \ldots)$ määrä negatiivisia tekijöitä, tulo on negatiivinen. Muulloin tulo on positiivinen, eli esimerkiksi parillisesta potenssista ei voi tulla negatiivista vastausta.

\begin{esimerkki}
Mikä on funktion $f:\mathbb{R} \rightarrow \mathbb{R}, f(t)=-t^4+3$ suurin arvo?
    \begin{esimratk}
Koska muuttuja $t$ on reaaliluku, ei sen parillinen potenssi voi olla merkkisäännön mukaan negatiivinen, vaan $t^4$:n arvo on vähintään $0$. Tämän perusteella $-t^4$:n arvo on epäpositiivinen eli korkeintaan nolla. Jos $-t^4$:n suurin arvo on nolla, niin lisäämällä tähän luvun $3$, saadaan funktion $f(t)=-t^4+3$ suurimmaksi mahdolliseksi arvoksi $3$.
    \end{esimratk}
\end{esimerkki}

%Hyödynnämme merkkisääntöä myöhemmin, kun teemme epäyhtälöistä merkkikaavioita.

\subsection*{Tulon nollasääntö}

Reaaliluku voi olla vain joko positiivinen, nolla tai negatiivinen. Tulon merkkisäännöstä seuraa, että positiivisten ja negatiivisten lukujen tulo on aina positiivinen tai negatiivinen, ei koskaan nolla. Jos siis tulo on $0$, tulon tekijöistä ainakin yhden täytyy olla $0$. Toisaalta jos jokin tulon tekijöistä on $0$, myös tulo on automaattisesti $0$.

Nämä tiedot yhdistämällä saadaan \termi{tulon nollasääntö}{tulon nollasääntö}:

\laatikko[Tulon nollasääntö]{

Ainakin yksi tulon tekijöistä on $0$. $\Longleftrightarrow$ Tulo on $0$.
}

Tulon nollasäännössä on olennaista, että päättely toimii \emph{molempiin suuntiin}.

\begin{esimerkki}
Sievennä lauseke $(x^5-7)\cdot y \cdot 0\cdot(3a-5b)^2$.
    \begin{esimratk}
Koska tulossa on tekijänä $0$, vastaus on $0$.
    \end{esimratk}
\end{esimerkki}

\begin{esimerkki} Ratkaistaan yhtälö $(x+5) \cdot x =0 $.
    \begin{align*}
        (x+5)\cdot x &=0 \quad \ppalkki \text{ tulon nollasääntö} \\
        x +5= 0 \text{ tai } x &=0 \\
        x= -5 \text{ tai } x &=0.
    \end{align*}
    Ratkaisuja on siis kaksi, $x= -5$ tai $x= 0$.
\end{esimerkki}

\begin{esimerkki} Ratkaistaan yhtälö $2(x+5)=0$.

Nyt tulon nollasäännön perusteella tiedetään, että $2=0$ tai $x+5=0$. Koska selvästi $2\neq 0$, jää ainoaksi ratkaisuksi $x+5=0$ eli $x=-5$.
\end{esimerkki}


\begin{esimerkki} Mitä voidaan yhtälön $xyz=0$ perusteella päätellä tuntemattomista $x$, $y$ ja $z$?

	\begin{esimratk}
Tulon nollasäännön perusteella $x=0$, $y=0$ tai $z=0$. Nollia voi siis olla 1--3 kappaletta.
	\end{esimratk}
\end{esimerkki}

\begin{tehtavasivu}

\paragraph*{Opi perusteet}

\begin{tehtava}
	Laske.
	\begin{alakohdat}
		\alakohta{$2 \cdot 3$}
		\alakohta{$2 \cdot (-3)$}
		\alakohta{$-2 \cdot 3$}
		\alakohta{$-2 \cdot (-3)$}
		\alakohta{$17 \cdot 666 \cdot 0 \cdot (-31)$}
	\end{alakohdat}
	
	\begin{vastaus}
		\begin{alakohdat}
			\alakohta{$6$}
			\alakohta{$-6$}
			\alakohta{$-6$}
			\alakohta{$6$}
			\alakohta{$0$}
		\end{alakohdat}
	\end{vastaus}
\end{tehtava}

\begin{tehtava}
	Olkoon $a>0$, $b<0$, ja $c=0$. Mitä voit päätellä tulon merkistä?
	\begin{alakohdat}
		\alakohta{$a \cdot a$}
		\alakohta{$b \cdot a$}
		\alakohta{$b \cdot b$}
		\alakohta{$b \cdot c$}
	\end{alakohdat}
	
	\begin{vastaus}
		\begin{alakohdat}
			\alakohta{tulo $>0$}
			\alakohta{tulo $<0$}
			\alakohta{tulo $>0$}
			\alakohta{tulo on $0$}
		\end{alakohdat}
	\end{vastaus}
\end{tehtava}


\begin{tehtava}
    Ratkaise seuraavat yhtälöt käyttämällä tulon nollasääntöä.
    \begin{alakohdat}
        \alakohta{$x(3+x)=0$}
        \alakohta{$(x-4)(x+3)=0$}
		\alakohta{$0=x^2(x-5)$}
    \end{alakohdat}
    \begin{vastaus}
        \begin{alakohdat}
            \alakohta{$x=0$ tai $x=-3$}
            \alakohta{$x=4$ tai $x=-3$}
            \alakohta{$x=0$ tai $x=5$}
        \end{alakohdat}
    \end{vastaus}
\end{tehtava}

\paragraph*{Hallitse kokonaisuus}

\begin{tehtava}
	Olkoon $a > 0$. Mitkä vaihtoehdoista $b>0$, $b<0$ ja $b=0$ ovat mahdollisia, jos
	tiedetään, että
	\begin{alakohdat}
		\alakohta{$a \cdot b > 0$}
		\alakohta{$a \cdot b \leq 0$}
		\alakohta{$b \cdot b > 0$}
		\alakohta{$b \cdot b < 0$?}
	\end{alakohdat}
	\begin{vastaus}
		\begin{alakohdat}
			\alakohta{$b>0$}
			\alakohta{$b < 0$ ja $b = 0$}
			\alakohta{$b>0$ ja $b<0$}
			\alakohta{Mikään vaihtoehto ei kelpaa.}
		\end{alakohdat}
	\end{vastaus}
\end{tehtava}

\begin{tehtava}
    Ratkaise seuraavat yhtälöt käyttämällä tulon nollasääntöä.
    \begin{alakohdat}
        \alakohta{$y(y+4)=0$}
        \alakohta{$(x-2)(x-1)(x+5)=0$}
        \alakohta{$(x+1)(x^2+2)=0$}
    \end{alakohdat}
    \begin{vastaus}
        \begin{alakohdat}
            \alakohta{$y=0$ tai $y=-4$}
            \alakohta{$x=2$, $x=1$ tai $x=-5$}
            \alakohta{$x=-1$}
        \end{alakohdat}
    \end{vastaus}
\end{tehtava}

\begin{tehtava} 
Osoita, että funktio $f(x)=x^4+3x^2+1$ saa vain positiivisia arvoja.
    \begin{vastaus}
     $x^4\geq 0$ ja $x^2 \geq 0$, joten $f(x) \geq 1$.
    \end{vastaus}
\end{tehtava}

\paragraph*{Lisää tehtäviä}

\begin{tehtava}
	Olkoon $a \geq 0$, $b \leq 0$, ja $c=0$. Mitä voit päätellä tulon merkistä?
	\begin{alakohdat}
		\alakohta{$a \cdot a$}
		\alakohta{$b \cdot a$}
		\alakohta{$b \cdot c$}
	\end{alakohdat}
	
	\begin{vastaus}
		\begin{alakohdat}
			\alakohta{tulo $\geq 0$}
			\alakohta{tulo $\leq 0$}
			\alakohta{tulo on $0$}
		\end{alakohdat}
	\end{vastaus}
\end{tehtava}

\begin{tehtava}
    Ratkaise seuraavat yhtälöt käyttämällä tulon nollasääntöä.
    \begin{alakohdat}
        \alakohta{$(g-2)\cdot (t+1)=0$}
        \alakohta{$x(x-5)=0$}
        \alakohta{$(2w+2)^2=0$}
    \end{alakohdat}
    \begin{vastaus}
        \begin{alakohdat}
            \alakohta{$g=2$ tai $t=-1$,\qquad Tehtävässä ei ole selvää, minkä muuttujan suhteen yhtälö pitäisi ratkaista, joten se on ratkaistu molempien muuttujien suhteen.}
            \alakohta{$x=0$ tai $x=5$}
            \alakohta{$w=-1$}
        \end{alakohdat}
    \end{vastaus}
\end{tehtava}

\begin{tehtava}
    Sievennä seuraava lauseke: $(a-x)\cdot(b-x)\cdot(c-x)\cdot...\cdot(\mathring{a}-x)\cdot(\ddot{a}-x)\cdot(\ddot{o}-x)$.
    \begin{vastaus}
        Tulossa esiintyy tekijänä $(x-x)=0$. Niinpä tulon nollasäännön mukaan
        \begin{align*}
            &(a-x)\cdot(b-x)\cdot(c-x)\cdot...\cdot(x-x)\cdot...\cdot(\ddot{a}-x)\cdot(\ddot{o}-x) \\
            =&(a-x)\cdot(b-x)\cdot(c-x)\cdot...\cdot 0\cdot...\cdot(\ddot{a}-x)\cdot(\ddot{o}-x) \\
            =&0
        \end{align*}
    \end{vastaus}
\end{tehtava}

%Laatinut V-P Kilpi 2013-11-10
\begin{tehtava}
Onko olemassa reaalilukua, jonka neliö on yhtä suuri kuin sen summa itsensä kanssa?
\begin{vastaus}
On, luku $2$.
\end{vastaus}
\end{tehtava}

\begin{tehtava}
$\star$ Osoita, että $x^2+\frac{1}{x^2}\geq 2$, kun $x \neq 0$.
    \begin{vastaus}
     Aloita tiedosta $\left(x-\frac{1}{x}\right)^2 \geq 0$ ja sievennä.
    \end{vastaus}
\end{tehtava}

%^ ja v -tehtävät epäyhtälölukuun?

\begin{tehtava} 
$\star$ Osoita, että kun $a \geq 0$ ja $b \geq 0$, pätee \\ $\frac{a+b}{2} \geq \sqrt{ab}$. Milloin yhtäsuuruus on voimassa?
    \begin{vastaus}
     Opastus: Aloita tiedosta $\left(\sqrt{a}-\sqrt{b}\right)^2 \geq 0$ ja sievennä. Yhtäsuuruus pätee, kun $a = b$.
    \end{vastaus}
\end{tehtava}

%tai vai vai vai joko tai, jos vai jos ja vain jos -tehtäviä :)
%potensseja tehtäviin!

\end{tehtavasivu}

	\section{Tekijöihinjako}

%Pitkän matematiikan 1. kurssilla on käsitelty lukujen jakamista tekijöihin.

Luvun $12$ \termi{tekijä}{tekijät} luonnolisten lukujen joukossa ovat luvut $1$, $2$, $3$, $4$, $6$ ja $12$. Ne ovat lukuja, joista saadaan luku $12$ kertomalla ne jollain luonnollisella luvulla. Toisin sanottuna luku $12$ voidaan jakaa millä tahansa näistä luvuista ilman jakojäännöstä.

\begin{esimerkki}
Luku $12$ voidaan kirjoittaa tekijöidensä tulona monella eri tavalla. Esimerkiksi
\begin{align*}
&12 = 2 \cdot 6, \\
&12= 4 \cdot 3 \text{ tai} \\
&12= 2 \cdot 2 \cdot 3.
\end{align*}
\end{esimerkki}

\subsection{Polynomin jakaminen tekijöihin}

\qrlinkki{http://opetus.tv/maa/maa2/polynomin-jakaminen-tekijoihin/}{Opetus.tv: \emph{polynomin jakaminen tekijöihin} (9:50 ja 5:44)}

Polynomeja voidaan jakaa tekijöihin kuten lukujakin. Polynomien tapauksessa tekijöihinjako tarkoittaa polynomin esittämistä saman- tai pienempiasteisten polynomien tulona, \termi{tulomuoto}{tulomuodossa}. Aste on aina pienempi, ellei kyse ole pelkästä vakiokertoimen ottamisesta yhteiseksi tekijäksi.

\subsubsection*{Yhteinen tekijä}

Kun polynomin jokaisessa termissä on sama tekijä, se voidaan ottaa yhteiseksi tekijäksi.

\begin{esimerkki}
Jaa tekijöihin polynomi $10x^3-20x^2$.
\begin{align*}
& 10x^3-20x^2 &\emph{otetaan $10$ yhteiseksi tekijäksi}\\
=& 10(x^3-2x^2) &\emph{otetaan $x^2$ yhteiseksi tekijäksi} \\
=& 10x^2(x-2)  
\end{align*}
\end{esimerkki}

\begin{esimerkki}
Jaa tekijöihin \quad a) $x^3+x$ \quad b) $3x^2+6x.$
\begin{alakohdat}
    \alakohta{$x^3+x = x(x^2+1)$}
    \alakohta{$3x^2+6x = 3x(x+2)$.}
\end{alakohdat}
\end{esimerkki}

\subsubsection*{Muistikaavat}

\begin{esimerkki}
Jaa tekijöihin \quad a) $x^2-4$ \quad b) $x^2+8x+16.$
	\begin{esimratk}
Jaetaan polynomit tekijöihin hyödyntämällä muistikaavoja
\begin{alakohdat}
    \alakohta{$x^2-4 = x^2-2^2 = (x+2)(x-2)$}
    \alakohta{$x^2+8x+16 = x^2+ 2\cdot 4 \cdot x + 4^2 = (x+4)^2$}
\end{alakohdat}
	\end{esimratk}
\end{esimerkki}

\begin{esimerkki}
Jaa tekijöihin polynomi $5x^3-20x^2+20x$.
\begin{align*}
& 5x^3-20x^2+20x \ \ \ \ &\emph{otetaan $5$ yhteiseksi tekijäksi}\\
=& 5(x^3-4x^2+4x)&\emph{otetaan $x$ yhteiseksi tekijäksi}  \\
=& 5x(x^2-4x+4) &\emph{sovelletaan muistikaavaa} \\
=& 5x(x^2-2\cdot 2x+2^2)  \\
=& 5x(x-2)^2
\end{align*}
\end{esimerkki}

%On suositeltavaa tarkistaa itse, että yllä esitetyt tekijöihinjaot todella toimivat. Polynomien tekijöihinjaon toimivuus
%on helppoa tarkistaa -- täytyy vain laskea väitettyjen tekijöiden tulo ja katsoa, onko se alkuperäinen polynomi. Vaikka
%tarkistus onkin helppoa, tässä vaiheessa ei luultavasti vielä ole selvää, miten tekijöihinjaon voisi saada selville
%-- paitsi toisinaan arvaamalla, mutta tähän kysymykseen vastataan myöhemmin tällä kurssilla.

%Polynomien tekijöihinjako ei ole yksiselitteinen, mutta monesti hyödyllisintä on jakaa polynomi tekijöihin samoin kuin
%esimerkkitapauksissa eli niin, että ensimmäisenä on vakiotermi ja kaikissa muissa tekijäpolynomeissa korkeimman asteen
%termin kerroin on 1.
%
%Esimerkki selkeyttänee asiaa. Polynomi $6x^2+30x+36$ voidaan jakaa tekijöihin vaikkapa seuraavilla tavoilla:
%
%\begin{esimerkki}
%\qquad \\
%\begin{itemize}
%    \item $6(x+2)(x+3)$
%    \item $3(2x+4)(x+3)$
%    \item $3(x+2)(2x+6)$
%    \item $2(3x+6)(x+2)$
%    \item $(6x+12)(x+3)$
%    \item $(\frac12 x+1)(12x+36)$
%\end{itemize}
%\end{esimerkki}
%
%Kaikki nämä tavat ovat ''oikein,'' mutta lähes aina ensimmäinen muoto $6(x+2)(x+3)$ on kätevin.
%
%Toisinaan polynomeille voi löytää tekijöitä soveltamalla joitakin seuraavista keinoista:
%
%\begin{itemize}
%\item Otetaan korkeimman asteen termin kerroin yhteiseksi tekijäksi: \\
%$5x^4+3x^2+x-9 = 5(x^4+\frac{3}{5} x^2+\frac{1}{5} x-\frac{9}{5})$
%\item Otetaan $x$ tai sen potenssi yhteiseksi tekijäksi, jos mahdollista: \\
%$x^5+x^3+3x = x(x^4+x^2+3)$
%$x^7+x^6+5x^4+2x^2 = x^2(x^5+x^4+5x^2+2)$
%\item Sovelletaan muistikaavaa käänteisesti \\
%$x^2-5=x^2-\sqrt{5}^2=(x+\sqrt{5})(x-\sqrt{5})$ \\
%$x^2+8x+16=x^2+2\cdot 4x+4^2=(x+4)^2$ \\
%$x^2+x+\frac14=x^2+2\cdot \frac12 x+(\frac12)^2=(x+\frac12)^2$
%\end{itemize}

%Kaikkien polynomien tekijöihinjako ei kuitenkaan näillä menetelmillä onnistu. Myöhemmin tässä kirjassa opitaan, miten toisen asteen polynomin voidaan jakaa tekijöihin nollakohtiensa avulla.

\subsubsection*{Ryhmittely}

Seuraavassa esimerkissä tekijöihin jako on toteutettu termien ryhmittelyn avulla. Se on joissain tapauksissa näppärä tapa jakaa polynomi tekijöihin, mutta oikean ryhmittelyn keksimiseen ei ole helppoa sääntöä.

\begin{esimerkki}
Jaa tekijöihin $x^3+3x^2+x+3$.
\begin{align*}
x^3+3x^2+x+3 &=x^2(x+3)+1(x+3) \\ &=(x^2+1)(x+3)
\end{align*}
\end{esimerkki}

\begin{esimerkki}
Jaa tekijöihin $x^{11}+2x^{10}+3x+6$.
\begin{align*}
& x^{11}+2x^{10}+3x+6\\
& =x^{10}(x+2)+3(x+2)\\
&=(x^{10}+3)(x+2)
\end{align*}
\end{esimerkki}

\subsection{Yhtälön ratkaisu tekijöihin jakamalla}

Tulon nollasääntö on yksi tärkeimmistä syistä siihen, miksi polynomien tekijöihinjako on hyödyllistä.

Jos vaikkapa haluamme ratkaista yhtälön $2x^3-14x^2+32x-24=0$ ja satumme tietämään, että $2x^3-14x^2+32x-24=2(x-3)(x-2)^2$, voimme helposti päätellä, että polynomi saa arvon $0$ jos ja vain jos $x-3=0$ tai $x-2=0$. Yhtälön ainoat ratkaisut ovat siis $x=3$ ja $x=2$.

\begin{esimerkki}
Ratkaise yhtälö $x^3-2x^2=0$ tulon nollasäännön avulla.
\begin{esimratk}
\begin{align*}
x^3-2x^2 &= 0 && \ppalkki \text{ otetaan } x^2 \text{ yhteiseksi tekijäksi} \\
x^2\cdot(x-2) &= 0 && \ppalkki \text{ tulon nollasääntö} \\
x^2=0 \textrm{\quad tai}& \quad x-2=0 \\
x=0 \textrm{\quad tai}& \quad x=2 \\
\end{align*}
\end{esimratk}
\begin{esimvast}
$x=0$ tai $x=2$.
\end{esimvast}
\end{esimerkki}

%\begin{esimerkki}
%Ratkaise yhtälö $6x^3-36x^2+54x=0$ tulon nollasäännön avulla.
%\begin{esimratk}
%\begin{align*}
%6x^3-36x^2+54x &= 0 \\
%6(x^3-6x^2+9x) &= 0 && \ppalkki \text{ otetaan } 6 \text{ yhteiseksi tekijäksi} \\
%6x(x^2-6x+9) &= 0 && \ppalkki \text{ otetaan } x \text{ yhteiseksi tekijäksi} \\
%6x(x^2-2\cdot 3\cdot x+3^2)  &= 0 \\
%6x(x-3)^2 &= 0 & \\
%6x(x-3)(x-3) &= 0 && \ppalkki \text{ tulon nollasääntö}\\
%6x=0 \textrm{\quad tai}& \quad x-3=0 \\
%x=0 \textrm{\quad tai}& \quad x=3 \\
%\end{align*}
%\end{esimratk}
%\begin{esimvast}
%$x=0$ tai $x=3$.
%\end{esimvast}
%\end{esimerkki}

Myöhemmin tällä kurssilla esitellään polynomien jakolause, joka antaa syvällisemmän yhteyden polynomien nollakohtien ja tekijöiden välille.

\subsubsection*{Murtolausekkeiden sieventäminen}

Kun murtolausekkeen osoittaja ja nimittäjä jaetaan tekijöihinsä, kummassakin esiintyvät tekijät voidaan supistaa pois.

\begin{esimerkki}
    Sievennä \quad 
    a) $\dfrac{4x+2y}{6}$ \quad
    b)$\dfrac{x^2+x}{2x+2}$ \quad
    c) $\dfrac{x^2-16}{x+4}$
    \begin{esimratk}
        \begin{alakohdat}
            \alakohta{$\dfrac{4x+2y}{6}=\dfrac{2(2x+y)}{2\cdot 3}=\dfrac{2x+y}{3}$}
            \alakohta{$\dfrac{x^2+x}{2x+2}=\dfrac{x(x+1)}{2(x+1)}=\dfrac{x}{2}$}
            \alakohta{Käytetään muistikaavaa: $\dfrac{x^2-16}{x+4}=\dfrac{(x+4)(x-4)}{x+4} = x-4$.}
        \end{alakohdat}
    \end{esimratk}
    \begin{esimvast}
        a) $\dfrac{2x+y}{3}$ \quad
        b) $\dfrac{x}{2}$ \quad
        c) $x-4$.
    \end{esimvast}
\end{esimerkki}

\begin{tehtavasivu}

\subsubsection*{Opi perusteet}

\begin{tehtava}
    Esitä tulona ottamalla yhteinen tekijä.
    \begin{alakohdat}
        \alakohta{$2x+6$}
        \alakohta{$x^2 -4x$}
        \alakohta{$3x^2 - 6x$}
    \end{alakohdat}
    \begin{vastaus}
        \begin{alakohdat}
        \alakohta{$2(x+3)$}
        \alakohta{$x(x-4)$}
        \alakohta{$3x(x-2)$}
        \end{alakohdat}
    \end{vastaus}
\end{tehtava}

\begin{tehtava}
    Jaa tekijöihin.
    \begin{alakohdat}
        \alakohta{$10a+5ab$}
        \alakohta{$x^4 -x^3$}
        \alakohta{$xy+x^2y$}
    \end{alakohdat}
    \begin{vastaus}
        \begin{alakohdat}
        \alakohta{$5a(2+b)$}
        \alakohta{$x^3(x-1)$}
        \alakohta{$xy(1+x)$}
        \end{alakohdat}
    \end{vastaus}
\end{tehtava}

\begin{tehtava}
    Sievennä.
    \begin{alakohdat}
        \alakohta{$\dfrac{3x-9}{3}$}
        \alakohta{$\dfrac{x^2-4x}{5x}$}
        \alakohta{$\dfrac{ab+a}{b^2+b}$}
    \end{alakohdat}
    \begin{vastaus}
        \begin{alakohdat}
        \alakohta{$x-3$}
        \alakohta{$\frac{x-4}{3}$}
        \alakohta{$\frac{a}{b}$}
        \end{alakohdat}
    \end{vastaus}
\end{tehtava}

\subsubsection*{Hallitse kokonaisuus}

%\begin{tehtava}
%    Jaa tekijöihin.
%    \begin{alakohdat}
%    	\alakohta{$x^3 - x$}
%        \alakohta{$x^2 - x + \frac{1}{4}$}
%        \alakohta{$9-x^4$}
%    \end{alakohdat}
%    \begin{vastaus}
%        \begin{alakohdat}
%            \alakohta{$x(x-1)^2$}
%            \alakohta{$(x-\frac{1}{2})^2$}
%            \alakohta{$(3+x^2)(3-x^2)$}
%        \end{alakohdat}
%    \end{vastaus}
%\end{tehtava}


\begin{tehtava}
    Jaa tekijöihin.
    \begin{alakohdat}
        \alakohta{$-15x^5 +10y$}
        \alakohta{$x^3y^2 +x^2y^3$}
        \alakohta{$-4a^3 -2a^2 +2ab$}
    \end{alakohdat}
    \begin{vastaus}
        \begin{alakohdat}
        \alakohta{joko $5(-3x^5 +2y)$ tai $-5(3x^5 -2y)$}
        \alakohta{$x^2y^2(x+y)$}
        \alakohta{joko $2a(-2a^2 -a +b)$ tai $-2a(2a^2 +a -b)$}
        \end{alakohdat}
    \end{vastaus}
\end{tehtava}

\begin{tehtava}
    Jaa tekijöihin muistikaavojen avulla.
    \begin{alakohdat}
        \alakohta{$x^2+6x+9$}
        \alakohta{$y^2 - 2y+1$}
        \alakohta{$x^2 -25$}
    \end{alakohdat}
    \begin{vastaus}
        \begin{alakohdat}
        \alakohta{$(x+3)^2$}
        \alakohta{$(y-1)^2$}
        \alakohta{$(x-5)(x+5)$}
        \end{alakohdat}
    \end{vastaus}
\end{tehtava}

\begin{tehtava}
    Jaa tekijöihin ryhmittelemällä sopivasti.
    \begin{alakohdat}
        \alakohta{$x^3 +x^2 +x +1$}
        \alakohta{$a^3 +a^2b +2a +2b$}
        \alakohta{$8m^6-2m^4+4m^2-1$}
    \end{alakohdat}
    \begin{vastaus}
        \begin{alakohdat}
        \alakohta{$(x^2+1)(x+1)$}
        \alakohta{$(a^2+2)(a+b)$}
        \alakohta{$(2m^4 +1)(4m^2 -1)=(2m^4 +1)(2m+1)(2m-1)$}
        \end{alakohdat}
    \end{vastaus}
\end{tehtava}

\begin{tehtava}
    Sievennä.
    \begin{alakohdat}
        \alakohta{$\dfrac{x^3-2x^2}{2-x}$}
        \alakohta{$\dfrac{x^2+6x+9}{x^2+3x}$}
        \alakohta{$\dfrac{4-x^2}{x^2-2x}$}
    \end{alakohdat}
    \begin{vastaus}
        \begin{alakohdat}
        \alakohta{$-x^2$}
        \alakohta{$\frac{x+3}{x}$}
        \alakohta{$-\frac{x+2}{x}$}
        \end{alakohdat}
    \end{vastaus}
\end{tehtava}

\begin{tehtava}
    Jaa tekijöihin.
    \begin{alakohdat}
    	\alakohta{$x^2 -4$}
    	\alakohta{$x^2 -3$}
    	\alakohta{$5x^2 -3$}
		\alakohta{$16-x^4$}
    \end{alakohdat}
    \begin{vastaus}
        \begin{alakohdat}
            \alakohta{$(x+2)(x-2)$}
            \alakohta{$(x+\sqrt{3})(x-\sqrt{3})$}
            \alakohta{$(\sqrt{5}x+\sqrt{3})(\sqrt{5}x-\sqrt{3})$}
            \alakohta{$(4+x^2)(4-x^2)=(4+x^2)(2-x)(2+x)$}
        \end{alakohdat}
    \end{vastaus}
\end{tehtava}

\begin{tehtava}
	Jaa tekijöihin.
	\begin{alakohdat}
		\alakohta{$x^3-x$}
		\alakohta{$5ab+ b+10a+2$}
		\alakohta{$16x^2y^2+8xy+1$}
	\end{alakohdat}
	\begin{vastaus}
		\begin{alakohdat}
			\alakohta{$x(x+1)(x-1)$}
			\alakohta{$(5a+1)(b+2)$}
			\alakohta{$(4xy+1)^2$}
		\end{alakohdat}
	\end{vastaus}
\end{tehtava}

\begin{tehtava}
	Ratkaise yhtälö jakamalla tekijöihin.
	\begin{alakohdat}
		\alakohta{$x^2-16 = 0$}
		\alakohta{$x^2+7x = 0$}
		\alakohta{$x^2-6x+9 = 0$}
	\end{alakohdat}
	\begin{vastaus}
		\begin{alakohdat}
			\alakohta{$x=4$ tai $x=-4$}
			\alakohta{$x=0$ tai $x=-7$}
			\alakohta{$x=3$}
		\end{alakohdat}
	\end{vastaus}
\end{tehtava}

\begin{tehtava} 
Jaa tekijöihin \\ $(3x^2-7y^2+5)^2-(x^2-9y^2-5)^2$.
    \begin{vastaus}
		$8(x-2y)(x+2y)(x^2+y^2+7)$. \\
    Opastus: Älä kerro aluksi sulkuja auki vaan käytä heti muistikaavaa.
    \end{vastaus}
\end{tehtava}

\subsubsection*{Lisää tehtäviä}

\begin{tehtava}
    Jaa tekijöihin muistikaavojen avulla.
    \begin{alakohdat}
        \alakohta{$x^2-4x+4$}
        \alakohta{$9y^2 + 6y+1$}
        \alakohta{$49-4x^2$}
    \end{alakohdat}
    \begin{vastaus}
        \begin{alakohdat}
        \alakohta{$(x-2)^2$}
        \alakohta{$(3y+1)^2$}
        \alakohta{$(7-2x)(7+2x)$}
        \end{alakohdat}
    \end{vastaus}
\end{tehtava}

\begin{tehtava}
	Ratkaise yhtälö jakamalla tekijöihin.
	\begin{alakohdat}
		\alakohta{$x^3-x^2 = 0$}
		\alakohta{$x^3+3x^2-4x-12 = 0$}
		\alakohta{$x^2-4x+4 = 4$}
	\end{alakohdat}
	\begin{vastaus}
		\begin{alakohdat}
			\alakohta{$x=0$ tai $x=1$}
			\alakohta{$x=-3$, $x=2$ tai $x=-2$}
			\alakohta{$x=0$ tai $x=4$}
		\end{alakohdat}
	\end{vastaus}
\end{tehtava}

\begin{tehtava}
	Ratkaise yhtälöt.
	\begin{alakohdat}
		\alakohta{$-x^4+4x^2=0$}
		\alakohta{$x^5-16x^3=0$}
	\end{alakohdat}
	\begin{vastaus}
		\begin{alakohdat}
			\alakohta{$x=-2$, $x=0$ tai $x=2$ (Tekijöihin jakamalla yhtälö sievenee muotoon $x^2(2+x)(2-x)=0$.)}
			\alakohta{$x=-4$, $x=0$ tai $x=4$ (Tekijöihin jakamalla yhtälö sievenee muotoon $x^3(x+4)(x-4)=0$.)}
		\end{alakohdat}
	\end{vastaus}
\end{tehtava}

\end{tehtavasivu}

	\section{Polynomifunktion kuvaaja}
Polynomifunktiota voi
havainnollistaa koordinaatistoon piirretyn kuvaajan avulla:

%\begin{kuvaajapohja}{1.5}{-2}{2}{-3}{3}
%\kuvaaja{-x-1}{$P(x) = -x-1$}{red}
%\kuvaaja{x**2+x}{$Q(x) = x^2+x$}{blue}
%\kuvaaja{x**3-3*x-1}{$R(x) = x^3-3x-1$}{green}
%\end{kuvaajapohja}

% \begin{kuvaajapohja}{1.5}{-2}{2}{-3}{3}
% %\kuvaaja{-x-1}{$P(x) = -x-1$}{red}
% \kuvaaja{x**2+x}{$Q(x) = x^2+x$}{black}
% \kuvaaja{x**3-3*x-1}{$R(x) = x^3-3x-1$}{black}
% \end{kuvaajapohja}
% %

\begin{kuva}
kuvaaja.pohja(-2, 2, -3, 3, korkeus = 9, nimiX = '$x$')
kuvaaja.piirra("x**2+x", nimi = "$Q(x) = x^2+x$", kohta = 1, suunta = -45)
kuvaaja.piirra("x**3-3*x-1", nimi = "$R(x) = x^3-3x-1$", kohta = 1.8)
\end{kuva}

\subsection*{Kuvaajan piirtäminen}

\qrlinkki{http://opetus.tv/maa/maa2/suoran-piirtaminen/}{Opetus.tv: \emph{suoran piirtäminen} (5:47)}

Funktioiden kuvaajia voi piirtää tietokoneella, graafisella laskimella tai käsin. Kussakin tapauksessa periaate on sama: valitaan joitakin muuttujan arvoja, lasketaan funktion arvot ja merkitään pisteet $(x,y)$-koordinaatistoon. Tietokoneet ja laskimet laskevat funktion arvoja niin tiheään, että näyttää syntyvän yhtenäinen kuvaaja. Käsin piirrettäessä tyydytään muutamaan pisteeseen ja hahmotellaan kuvaaja niiden avulla.

% Esimerkin kuvat.
\begin{luoKuva}{esimkuva1}
kuvaaja.pohja(-4, 4, -3, 6, leveys = 3.5, nimiX = "$x$")
piste((-3, 5.5))
piste((-2, 2))
piste((-1, -0.5))
piste((0, -2))
piste((1, -2.5))
piste((2, -2))
piste((3, -0.5))
\end{luoKuva}
\begin{luoKuva}{esimkuva2}
kuvaaja.pohja(-4, 4, -3, 6, leveys = 3.5, nimiX = "$x$")
piste((-3, 5.5))
piste((-2, 2))
piste((-1, -0.5))
piste((0, -2))
piste((1, -2.5))
piste((2, -2))
piste((3, -0.5))
kuvaaja.piirra("0.5*x**2-x-2")
\end{luoKuva}

\begin{esimerkki}
Hahmotellaan polynomifunktion $f(x) = \dfrac{1}{2}x^2 - x - 2$ kuvaaja.
Lasketaan ensin joitakin funktion $f(x) = \dfrac{1}{2}x^2 - x - 2$ arvoja ja piirretään niitä vastaavat pisteet
koordinaatistoon. Lopuksi hahmotellaan kuvaaja, joka kulkee pisteiden kautta.

\begin{tabular}{c c c}
	\begin{tabular}{|c|r @{,} l|}
	\hline $x$ & \multicolumn{2}{c|}{$f(x)$} \\
	\hline
	-3 & 5&5 \\
	-2 & 2&0 \\
	-1 & -0&5 \\
	0 & -2&0 \\
	1 & -2&5 \\
	2 & -2&0 \\
	3 & -0&5 \\
	\hline
	\end{tabular}
	&
	\vcent{\naytaKuva{esimkuva1}}
	&
	\vcent{\naytaKuva{esimkuva2}}
\end{tabular}
% 
% \begin{tabular}{c c c}
% 	\begin{tabular}{|c|r @{,} l|}
% 	\hline $x$ & \multicolumn{2}{c|}{$f(x)$} \\
% 	\hline
% 	-3 & 5&5 \\
% 	-2 & 2&0 \\
% 	-1 & -0&5 \\
% 	0 & -2&0 \\
% 	1 & -2&5 \\
% 	2 & -2&0 \\
% 	3 & -0&5 \\
% 	\hline
% 	\end{tabular}
% 	&
% 	\vcent{\begin{kuvaajapohja}{0.6}{-4}{4}{-3}{6}
% 	\kuvaajapiste{-3}{5.5}
% 	\kuvaajapiste{-2}{2}
% 	\kuvaajapiste{-1}{-0.5}
% 	\kuvaajapiste{0}{-2}
% 	\kuvaajapiste{1}{-2.5}
% 	\kuvaajapiste{2}{-2}
% 	\kuvaajapiste{3}{-0.5}
% 	\end{kuvaajapohja}}
% 	&
% 	\vcent{\begin{kuvaajapohja}{0.6}{-4}{4}{-3}{6}
% 	\kuvaajapiste{-3}{5.5}
% 	\kuvaajapiste{-2}{2}
% 	\kuvaajapiste{-1}{-0.5}
% 	\kuvaajapiste{0}{-2}
% 	\kuvaajapiste{1}{-2.5}
% 	\kuvaajapiste{2}{-2}
% 	\kuvaajapiste{3}{-0.5}
% 	\kuvaaja{0.5*x**2-x-2}{}{black}
% 	\end{kuvaajapohja}}
% \end{tabular}

\end{esimerkki}

\subsection*{Kuvaajan tulkintaa}

%Ensimmäisen asteen polynomin kuvaaja on luonnollisesti aina suora. 
%miten niin luonnollisesti?

Kuvaajan avulla voidaan tehdä johtopäätöksiä funktion ominaisuuksista.
Esimerkiksi funktion arvoja voidaan lukea kuvaajasta.

\begin{esimerkki}
Seuraavassa on esitetty polynomifunktion $P(x)=-3x^2+2x+4$ kuvaaja.

\begin{kuvaajapohja}[\kuvaajaAsetusEiRuudukkoa]{0.7}{-3}{3}{-5}{5}
\kuvaajapiste{2}{-4}
\kuvaajakohtaarvo{2}{-4}{}{}
\kuvaaja{-3*x**2+2*x+4}{$P(x)$}{black}
\end{kuvaajapohja}

Kuvaajasta voi lukea funktion arvoja tai ainakin niiden likiarvoja. Kuvaajan perusteella näyttää siltä, että $P(2)=-4$. Näin todellakin on, sillä \\ $P(1)=-3\cdot 1^2+2\cdot 1+4=1$.
\end{esimerkki}


\begin{esimerkki}
Kuvaajasta ei välttämättä näe tarkkoja arvoja. Seuraavassa on esitetty erään polynomifunktion $P(x)$ kuvaaja. Kuvaajan perusteella näyttäisi siltä, että $P(1)=1$, mutta tarkkaa arvoa kuvaajasta ei voi päätellä.

\begin{kuva}
kuvaaja.pohja(-1.9, 1.7, -1.4, 2, leveys = 4, nimiX = "$x$", ruudukko = False)
kuvaaja.piirra("19./20*x**2+19./20*x-1", nimi = "$P(x)$", kohta = 1)
\end{kuva}
% \begin{kuvaajapohja}[\kuvaajaAsetusEiRuudukkoa]{1}{-2}{2}{-2}{2}
% \kuvaaja{19./20*x**2+19./20*x-1}{$P(x)$}{black}
% \end{kuvaajapohja}
 
Itse asiassa edellinen kuvaaja kuuluu funktiolle $P(x)=\dfrac{19}{20} x^2+\dfrac{19}{20} x-1$. Nyt tiedetään, että funktion $P(x)$ arvo kohdassa $x=1$ on
$$\dfrac{19}{20}\cdot 1^2+\dfrac{19}{20} \cdot 1-1=\dfrac{9}{10}$$
eikä $1$, kuten kuvaajan perusteella voisi luulla. Kuvaajasta ei siis voi lukea tarkkoja tietoja funktiosta.
\end{esimerkki}

\newpage

\subsubsection*{Nollakohta}

Funktion \termi{nollakohta}{nollakohta} on sellainen muuttujan arvo, jolla funktio saa arvon nolla. Esimerkiksi funktiolla $Q(x)=x^2-1$
on nollakohdat $-1$ ja $1$, sillä $Q(-1)=0$ ja $Q(1)=0$.

Funktion kuvaaja antaa tietoa nollakohdista. Niiden kohdalla kuvaaja leikkaa $x$-akselin.

\begin{esimerkki}
Funktion $P(x) = \dfrac{1}{3}x^2-4x+\dfrac{5}{2}$ kuvaajasta nähdään, että funktiolla on ainakin kaksi nollakohtaa. Toinen niistä on lähellä lukua $1$ ja toinen lukua $11$.

\begin{kuva}
kuvaaja.pohja(-5, 15, -10, 5, korkeus = 6, nimiX = "$x$")
piste((0.66146, 0))
piste((11.3385, 0))
kuvaaja.piirra("1./3*x**2-4*x+5./2", nimi = "$P(x) = \dfrac{1}{3}x^2-4x+\dfrac{5}{2}$", kohta = 11, suunta = -45)
\end{kuva}
% \begin{kuvaajapohja}{0.3}{-5}{15}{-10}{5}
% \kuvaajapiste{0.66146}{0}
% \kuvaajapiste{11.3385}{0}
% \kuvaaja{1./3*x**2-4*x+5./2}{$P(x) = \dfrac{1}{3}x^2-4x+\dfrac{5}{2}$}{black}
% \end{kuvaajapohja}
\end{esimerkki}

%Nollakohta tarkoittaa sitä annetun polynomin muuttujan arvoa, jolla koko
%polynomi saa arvon nolla. Kuvaajasta sen voi helposti lukea niinä kohtina,
%joissa kuvaaja leikkaa muuttujan koordinaattiakselin. Funktion $P(x)$ ja
%$xy$-koordinaatiston tapauksessa funktion nollakohdat ovat täsmälleen ne
%$x$-koordinaatit, joilla funktion kuvaaja leikkaa $x$-akselin.

%\subsection{Taylorin sarja}
%Eräs mielenkiintoinen ja hyvin tunnettu potenssisarja on Taylorin sarja.
%Se on päättymätön potenssisarja, jolla voidaan approksimoida muiden funktioiden
%arvoja.
%
%Yleisesti Taylorin sarjalla saadaan (rajatta derivoituvan) funktion $f$ arvo
%pisteessä $x_0$:
%
%\begin{align*}
%	f(x_0) = \sum\limits_{n=0}^\infty a_n(x-x_0)^n
%\end{align*}
%
%missä
%
%\begin{align*}
%a_n = \frac{f^n(x_0)}{n!}
%\end{align*}
%
%Koska sarja on äärettömän pitkä, sarjan arvoja edelleen arvioidaan Taylorin
%polynomilla, joka on muotoa
%
%\begin{align*}
%	P_k(x) = \sum\limits_{n=0}^k a_k(x-x_0)^k
%\end{align*}
%
%Polynomin avulla voidaan laskea esimerkiksi likiarvo funktiolle
%$(1-x)^{-1} = \frac{1}{1-x}$ pisteen a ympäristössä, kun $a \neg 1$:
%
%\begin{align*}
%	\frac{1}{1-x} \approx \frac{1}{1-a} + \frac{x-a}{(1-a)^2} +
%\frac{(x-a)^2}{(1-a)^3} + \frac{(x-a)^3}{(1-a)^4} ...
%\end{align*}
%
%\missingfigure{Funktion $(x-1)^-1$ kuvaaja}
%\missingfigure{Funktion $\frac{1}{1-a} + \frac{x-a}{(1-a)^2} +
%\frac{(x-a)^2}{(1-a)^3} +$ kuvaaja}

\begin{tehtavasivu}

\paragraph*{Opi perusteet}


\begin{tehtava}
\begin{kuva}
	kuvaaja.pohja(-2,4.5,-2,8,5,nimiX="$x$",nimiY="$P(x)$")
	kuvaaja.piirra("x**2-4*x+3")
\end{kuva}
Kuvassa on polynomifunktion $P$ kuvaajaa. Määritä on kuvaajan perusteella
\begin{alakohdat}
\alakohta{$P(-1)$}
\alakohta{$P(1)$}
\alakohta{$P(2)$}
\alakohta{polynomin $P$ nollakohdat}
\alakohta{millä $x$:n arvoilla $P(x)=3$.}
\end{alakohdat}
\begin{vastaus}
\begin{alakohdat}
\alakohta{$8$}
\alakohta{$0$}
\alakohta{$-1$}
\alakohta{Nollakohdat ovat $x=1$ ja $x=3$.}
\alakohta{Kun $x= 0$ tai $x=4$.}
\end{alakohdat}
\end{vastaus}
\end{tehtava}

\begin{tehtava}
    Hahmottele polynomien kuvaajat koordinaatistoon käsin. Laske testipisteitä tarpeen mukaan. Tarkista tietokoneella tai graafisella laskimella.
    \begin{alakohdat}
        \alakohta{$P(x) = x-2$}
        \alakohta{$Q(x) = 4-x^2$}
        \alakohta{$R(x) = \frac{1}{4}x^3$}
    \end{alakohdat}   
    \begin{vastaus}
    	\begin{alakohdat}
    	\alakohta{ \begin{kuvaajapohja}{0.4}{-4}{4}{-4}{4}
				\kuvaaja{2*x-2}{}{red}
			  \end{kuvaajapohja}}
    	\alakohta{ \begin{kuvaajapohja}{0.4}{-4}{4}{-3}{5}
				\kuvaaja{4-x**2}{}{red}
			  \end{kuvaajapohja}}
		\alakohta{ \begin{kuvaajapohja}{0.4}{-4}{4}{-4}{4}
				\kuvaaja{0.25*x**3}{}{red}
			  \end{kuvaajapohja}}
		\end{alakohdat}
    \end{vastaus}
\end{tehtava}

\paragraph*{Hallitse kokonaisuus}

\begin{tehtava}
Piirrä polynomifunktion kuvaaja ja päättele sen avulla polynomin nollakohdat.
Käytä graafista laskinta, tietokonetta tai piirrä käsin.
\begin{alakohdat}
\alakohta{$x^3+x^2+x+1$}
\alakohta{$x^2-6x+2$}
\end{alakohdat}
\begin{vastaus}
\begin{alakohdat}
\alakohta{$x=-1$}
\alakohta{$x \approx 0,35$ ja $x \approx 5,65$}
\end{alakohdat}
\end{vastaus}
\end{tehtava}

\begin{tehtava} $\star$
	Monia funktioita voidaan esittää likimääräisesti polynomeina (ns.
Taylorin polynomi). Esimerkiksi (kun $-1<x<1$)

	\begin{tabular}{lcll}
	$\frac{1}{1+x^2}$ &$\approx$ & $1-x^2+x^4-x^6+x^8-x^{10}$ \\
	$\sqrt{1+x}$ & $\approx $ & $ 1+\frac{x}{2}
	-\frac{x^2}{8}+\frac{x^3}{16}-\frac{5x^4}{128}$
	\end{tabular}

	Piirrä alkuperäinen funktio ja polynomi samaan kuvaajaan tietokoneella
tai graafisella laskimella. Kokeile, kuinka polynomin viimeisten termien pois
jättäminen vaikuttaa tarkkuuteen. Mitä havaitset? (Termejä voi laskea lisääkin,
mutta siihen ei puututa tässä.)

	\begin{vastaus}
		Mitä enemmän termejä, sitä parempi vastaavuus.
	\end{vastaus}
\end{tehtava}

\paragraph*{Lisää tehtäviä}

\begin{tehtava}
    Piirrä polynomien kuvaajat käsin.
    \begin{alakohdat}
        \alakohta{$x+4$}
        \alakohta{$2x-3$}
        \alakohta{$5$}
        \alakohta{$x^2+x-2$}
%        \alakohta{$x^3-x+3$}
    \end{alakohdat}   
    \begin{vastaus}
    	\begin{alakohdat}
    	\alakohta{ \begin{kuvaajapohja}{0.4}{-4}{4}{-1}{7}
				\kuvaaja{x+4}{}{red}
			  \end{kuvaajapohja}}
    	\alakohta{ \begin{kuvaajapohja}{0.4}{-4}{4}{-5}{3}
				\kuvaaja{2*x-3}{}{red}
			  \end{kuvaajapohja}}
    	\alakohta{ \begin{kuvaajapohja}{0.4}{-4}{4}{-1}{7}
				\kuvaaja{5}{}{red}
			  \end{kuvaajapohja}}
		\alakohta{ \begin{kuvaajapohja}{0.4}{-4}{4}{-3}{5}
				\kuvaaja{x**2+x-2}{}{red}
			  \end{kuvaajapohja}}
%		\alakohta{ \begin{kuvaajapohja}{0.4}{-4}{4}{-3}{5}
%				\kuvaaja{x**3-x+2}{}{red}
%			  \end{kuvaajapohja}}
		\end{alakohdat}
    \end{vastaus}
\end{tehtava}


\begin{tehtava}
    Piirrä polynomien kuvaajat tietokoneella tai laskimella.
    \begin{alakohdat}
        \alakohta{$x^2-6x+3$}
        \alakohta{$\frac{1}{5}x^3-x^2+2$}
        \alakohta{$-0,5x^4+0,25x^3+2,5x^2-1$}

    \end{alakohdat}
    \begin{vastaus}
    	\begin{alakohdat}
		\alakohta{ \begin{kuvaajapohja}{0.4}{-1}{7}{-7}{5}
				\kuvaaja{x**2-6*x+3}{}{red}
			  \end{kuvaajapohja}}
    	\alakohta{ \begin{kuvaajapohja}{0.4}{-2.5}{5.5}{-4}{4}
				\kuvaaja{0.2*x**3-x**2+2}{}{red}
			  \end{kuvaajapohja}}
		\alakohta{ \begin{kuvaajapohja}{0.4}{-3}{3}{-4}{4}
				\kuvaaja{-0.5*x**4+0.25*x**3+2.5*x**2-1}{}{red}
			  \end{kuvaajapohja}}
		\end{alakohdat}
    \end{vastaus}
\end{tehtava}

%\begin{tehtava}
%    Piirrä polynomien kuvaajat.
%    \begin{alakohdat}
%        \alakohta{$x+4$}
%        \alakohta{$2x-9$}
%        \alakohta{$5x+2$}
%        \alakohta{$6x+1$}
%    \end{alakohdat}
%    \begin{vastaus}
%    	\begin{alakohdat}
%        \alakohta{ \begin{kuvaajapohja}{0.4}{-4}{4}{-1}{7}
%				\kuvaaja{x+4}{}{red}
%			  \end{kuvaajapohja}}
%    	\alakohta{ \begin{kuvaajapohja}{0.4}{-2}{6}{-6}{2}
%				\kuvaaja{2*x-9}{}{red}
%			  \end{kuvaajapohja}}
%		\alakohta{ \begin{kuvaajapohja}{0.4}{-4}{4}{-2}{6}
%				\kuvaaja{5*x+2}{}{red}
%			  \end{kuvaajapohja}}
%		\alakohta{ \begin{kuvaajapohja}{0.4}{-4}{4}{-2}{6}
%				\kuvaaja{6*x+1}{}{red}
%			  \end{kuvaajapohja}}
%		\end{alakohdat}
%    \end{vastaus}
%\end{tehtava}
%
%\begin{tehtava}
%    Piirrä polynomien kuvaajat.
%    \begin{alakohdat}
%        \alakohta{$x^2-1$}
%        \alakohta{$2x^2$}
%        \alakohta{$4x^2+4$}
%        \alakohta{$x^2-6x+3$}
%    \end{alakohdat}
%    \begin{vastaus}
%    	\begin{alakohdat}
%        \alakohta{ \begin{kuvaajapohja}{0.4}{-4}{4}{-2}{6}
%				\kuvaaja{x+4}{}{red}
%			  \end{kuvaajapohja}}
%    	\alakohta{ \begin{kuvaajapohja}{0.4}{-4}{4}{-1}{7}
%				\kuvaaja{2*x-9}{}{red}
%			  \end{kuvaajapohja}}
%		\alakohta{ \begin{kuvaajapohja}{0.4}{-4}{4}{-1}{11}
%				\kuvaaja{5*x+2}{}{red}
%			  \end{kuvaajapohja}}
%		\alakohta{ \begin{kuvaajapohja}{0.4}{-1}{7}{-7}{5}
%				\kuvaaja{6*x+1}{}{red}
%			  \end{kuvaajapohja}}
%		\end{alakohdat}
%    \end{vastaus}
%\end{tehtava}

\end{tehtavasivu}


% tämä alla oleva omaksi filukseen! ja näihin ratkaisut! ... mutta miten?

\section{Testaa tietosi!}

\subsection*{Osaatko selittää?}

\begin{enumerate}

\item Miksi polynomin vakiotermin (tai vakiopolynomin) asteluku on nolla?
\item Miksi puhutaan vain termien summasta, vaikka polynomeissa voi esiintyä miinusmerkkejä?
\item Miten usean muuttujan polynomin asteluku lasketaan?
\item Mitä tarkoitetaan sillä, että luku on epänegatiivinen?
\item Mitä tarkoitetaan tulon nollasäännöllä?
\item Kun polynomi jaetaan tekijöihin, mitä voit varmasti sanoa tekijäpolynomien asteista verrattuna alkuperäiseen polynomiin?
\item Miten selvität polynomin $(x^3-2x)^{15}$ asteen purkamatta sulkuja?

%lisää MIKSI-tehtäviä!

\end{enumerate}

\subsection*{Kertauskysymyksiä}

Valitse yksi oikea vaihtoehto

\begin{enumerate}

\item Mitkä ovat polynomin $x^4-x^2+\sqrt{2}x-1337$ termien kertoimet? \\
a) \\
b) \\ %miten tää kannattaa rakentaa...?

\item Monettako astetta on 2 muuttujan polynomi $x^5y-x^4y^3-715517y$?

\item Polynomifunktion $P:\rr \rightarrow \rr$, $P(x)=-x^3-x^2-x-1$ arvo kohdassa $x=-1$ on...
\item Mikä polynomi saadaan, kun avataan sulut lausekkeesta $(2x-1)^2$?
\item Tulosta $ab$ tiedetään, että... mikä seuraavista väitteistä pitää välttämätät paikkansa?
\item Polynomin tekijöistä...
\item Polynoimfunktion kuvaajan tulkintaa... nollakohtien lukumäärä
\item Polynomifunktion kuvaajan tulkintaa...

\end{enumerate}



\osa{Ensimmäinen aste}
    \luku{epayhtalo}{Epäyhtälöiden teoriaa}
    \luku{ensimmaisen_asteen_yhtalo}{Kertausta: ensimmäisen asteen yhtälö}
    \luku{ensimmaisen_asteen_epayhtalo}{Ensimmäisen asteen epäyhtälö}

\osa{Toinen aste}

	\section{Toisen asteen polynomifunktio}

\qrlinkki{http://opetus.tv/maa/maa2/toisen-asteen-polynomifunktio/}{Opetus.tv: \emph{toisen asteen polynomifunktio} (7:59)}

Toisen asteen polynomifunktio on muotoa
\begin{align*}
P(x)=ax^2+bx+c,
\end{align*}
missä vakiot $b$ ja $c$ voivat olla mitä tahansa reaalilukuja $(b, \ c \in \rr)$ ja $a$ voi olla mikä tahansa reaaliluku, paitsi luku nolla $(a \in \rr, \ a \neq 0)$.

Toisen asteen polynomifunktion kuvaajaa nimitetään \termi{paraabeli}{paraabeliksi}.

Funktion $P(x)=ax^2+bx+c$ kuvaaja on joka ylös- tai alaspäin aukeava paraabeli.
Paraabelin suuntaan vaikuttaa ainoastaan toisen asteen termin kerroin $a$: kun
$a>0$, paraabeli aukeaa ylöpäin, ja kun $a<0$, paraabeli aukeaa alaspäin.

\begin{center}
\begin{tabular}{cc}

\begin{tabular}{c}
	\begin{lukusuora}{-2}{2}{4}
	\lukusuoraisobbox
	\lukusuorakuvaaja{x**2-1}
	\end{lukusuora}
	\\ ylöspäin aukeava paraabeli, \\ $a > 0$
\end{tabular}
&
\begin{tabular}{c}
	\begin{lukusuora}{-2}{2}{4}
	\lukusuoraisobbox
	\lukusuorakuvaaja{-x**2+1}
	\end{lukusuora}
	\\ alaspäin aukeava paraabeli, \\ $a < 0$
\end{tabular}

\end{tabular}
\end{center}

Vakiotermi $c$ vaikuttaa kuvaajan korkeuteen.

%FIXME: kuvaajassa x^2+3 ja x^2 menevät päällekäin 
\begin{luoKuva}{paraabelit}
	kuvaaja.pohja(-5, 5, -1, 8, leveys=7)
	
	kuvaaja.piirra("x**2+3", nimi="$x^2+3$")
	kuvaaja.piirra("x**2", nimi="$x^2$")
\end{luoKuva}

\begin{luoKuva}{paraabelit2}
	kuvaaja.pohja(-5, 5, -4, 6, leveys=7)
	
	kuvaaja.piirra("x**2-3*x", nimi="$x^2-3x$")
	kuvaaja.piirra("x**2", nimi="$x^2$")
\end{luoKuva}


%\begin{luoKuva}{vakio}
%	kuvaaja.pohja(-5, 5, -5, 5, leveys=7)
%	
%	kuvaaja.piirra("*x**2+2", nimi="$x^2+2$")
%	kuvaaja.piirra("x**2", nimi="$x^2$")
%\end{luoKuva}

\begin{center}
	\naytaKuva{paraabelit}
\end{center}

Ensimmäisen asteen termin kerroin $b$ vaikuttaa huipun sijaintiin sekä pysty- että vaakasuunnassa.

\begin{center}
	\naytaKuva{paraabelit2}
\end{center}

Tarkempi perustelu edellä esitetyille seikoille löytyy lisämateriaalista,
sivulta \pageref{paraabeli_tod}.

%\begin{center}
%	\naytaKuva{vakio}
%\end{center}


%
%\begin{luoKuva}{ekaaste}
%	kuvaaja.pohja(-5, 5, -5, 5, leveys=7)
%	
%	kuvaaja.piirra("*x**2-3*x", nimi="$x^2-3x$")
%	kuvaaja.piirra("x**2", nimi="$x^2$")
%\end{luoKuva}
%
%\begin{center}
%	\naytaKuva{ekaaste}
%\end{center}



\begin{tehtavasivu}

\paragraph*{Opi perusteet}

\begin{tehtava}
  Aukeavatko seuraavat paraabelit ylös- vai alaspäin?
  \begin{alakohdat}
    \alakohta{$4x^2 + 100x - 3$}
    \alakohta{$-x^2 + 1337$}
    \alakohta{$5x^2 - 7x + 5$}
    \alakohta{$-6(-3x^2 + 5)$}
    \alakohta{$-13x(9 - 17x)$}
    \alakohta{$100(1-x^2)$}
  \end{alakohdat}

  \begin{vastaus}
    \begin{alakohdat}
      \alakohta{Ylös}
      \alakohta{Alas}
      \alakohta{Ylös}
      \alakohta{Ylös}
      \alakohta{Ylös}
      \alakohta{Alas}
    \end{alakohdat}
  \end{vastaus}
\end{tehtava}

\begin{tehtava}
Funktiot $P(x)$ ja $Q(x)$ ovat toisen asteen polynomeja.\\
\begin{kuvaajapohja}{1}{-2}{3}{-1}{3}
\kuvaaja{x*(2-x)}{$P(x)$}{black}
\kuvaaja{x**2+1}{$Q(x)$}{black}
\end{kuvaajapohja} \\
Päättele kuvaajan perusteella
\begin{alakohdat}
\alakohta{mihin suuntaan paraabelit aukeavat}
\alakohta{funktion $P$ nollakohdat}
\alakohta{yhtälön $Q(x)=2$ ratkaisu}
\alakohta{polynomin $Q(x)$ vakiotermi}
\end{alakohdat}

\begin{vastaus}
\begin{alakohdat}
\alakohta{$P$ alaspäin, $Q$ ylöspäin.}
\alakohta{nollakohdat: $x=0$ ja $x=2$}
\alakohta{$x=-1$ tai $x=1$}
\alakohta{$1$, sillä kun $x=0$, $Q(x)=1$.}
\end{alakohdat}
\end{vastaus}
\end{tehtava}

\paragraph*{Hallitse kokonaisuus}

\begin{tehtava}
Kuvassa on funktion $P(x)=x^2$ kuvaaja.\\
\begin{kuvaajapohja}{1.5}{-1.5}{1.5}{-1}{3}
\kuvaaja{x**2}{$P(x)=x^2$}{black}
\end{kuvaajapohja} \\
Hahmottele kuvaajan avulla funktioiden
\begin{alakohdat}
\alakohta{$x^2-1$}
\alakohta{$2-x^2$}
\alakohta{$\frac{1}{2}x^2$}
\alakohta{$(x-2)^2$}
\end{alakohdat}
kuvaajat.
\begin{vastaus}
\begin{alakohdat}
\alakohta{
\begin{kuvaajapohja}{1}{-1.5}{1.5}{-2}{2}
\kuvaaja{x**2-1}{$P(x)=x^2-1$}{black}
\end{kuvaajapohja}}
\alakohta{
\begin{kuvaajapohja}{1}{-1.5}{1.5}{-1}{3}
\kuvaaja{2-x**2}{$P(x)=2-x^2$}{black}
\end{kuvaajapohja}}
\alakohta{
\begin{kuvaajapohja}{1}{-1.5}{1.5}{-1}{3}
\kuvaaja{0.5*x**2}{$P(x)=\frac12x^2$}{black}
\end{kuvaajapohja}}
\alakohta{
\begin{kuvaajapohja}{1}{-0.5}{3.5}{-1}{3}
\kuvaaja{(x-2)**2}{$P(x)=(x-2)^2$}{black}
\end{kuvaajapohja}}
\end{alakohdat}
\end{vastaus}
\end{tehtava}

\end{tehtavasivu}
	\section{Toisen asteen yhtälö}

\qrlinkki{http://opetus.tv/maa/maa2/toisen-asteen-yhtalo/}{Opetus.tv: \emph{toisen asteen yhtälö} (9:02, 11:09 ja 9:30)}

\begin{esimerkki}
Selvitetään, milloin funktio $f(x)=x^2+2x+1$ leikkaa x-akselin.

Funktion kuvaaja leikkaa x-akselin, kun $f(x)=0$. Tätä muuttujan $x$ arvoa kutsutaan funktion $f$ \textbf{nollakohdaksi}. Piirretään funktion $f$ kuvaaja ja etsitään ne kohdat, joissa funktio leikkaa x-akselin. %insert kuvaaja.pic
%funktio käsite on tuttu, joten itse esittelisin tämän asian näin -Lauri

\begin{kuvaajapohja}{1.5}{-2.5}{0.5}{-0.5}{3}
\kuvaaja{x**2+2*x+1}{$f(x)=x^2+2x+1$}{red}
\end{kuvaajapohja}
\end{esimerkki}

Kuvaajasta havaitaan, että funktion $f$ nollakohta on $x \approx -1$. Tätä funktion nollikohtien ratkaisumenetelmää kutsutaan graafiseksi ratkaisemiseksi. Graafinen ratkaisu on aina likimääräinen eli arvio oikeasta ratkaisusta.

Määritettäessä toisen asteen polynomifunktion nollakohtia päädytään \textbf{toisen asteen yhtälöön}, joka on aina saatettavissa yleiseen muotoon
\begin{align*}
ax^2+bx+c=0.
\end{align*}
\laatikko{Toisen asteen yhtälöllä \[ax^2+bx+c=0\] on reaalilukuratkaisuja joko 0, 1 tai 2 kappaletta.}

Seuraavassa kuvat eri tapauksista:

\begin{tabular}{c c}

\begin{tabular}{c}
	$2$ ratkaisua, $a$ positiivinen\\
	\begin{lukusuora}{-1}{1}{4}
	\lukusuoraisobbox
	\lukusuoraparaabeli{-0.5}{0.5}{-1}
	\end{lukusuora}
\end{tabular}

&

\begin{tabular}{c}
	$2$ ratkaisua, $a$ negatiivinen\\
	\begin{lukusuora}{-1}{1}{4}
	\lukusuoraisobbox
	\lukusuoraparaabeli{-0.5}{0.5}{1}
	\end{lukusuora}
\end{tabular}

\\ \qquad & \qquad \\

\begin{tabular}{c}
	$1$ ratkaisu, $a$ positiivinen\\
	\begin{lukusuora}{-2}{2}{4}
	\lukusuoraisobbox
	\lukusuorakuvaaja{x**2}
	\end{lukusuora}
\end{tabular}

&

\begin{tabular}{c}
	$1$ ratkaisu, $a$ negatiivinen\\
	\begin{lukusuora}{-2}{2}{4}
	\lukusuoraisobbox
	\lukusuorakuvaaja{-x**2}
	\end{lukusuora}
\end{tabular}

\\ \qquad & \qquad \\

\begin{tabular}{c}
	ei ratkaisuja, $a$ positiivinen\\
	\begin{lukusuora}{-2}{2}{4}
	\lukusuoraisobbox
	\lukusuorakuvaaja{x**2+0.3}
	\end{lukusuora}
\end{tabular}

&

\begin{tabular}{c}
	ei ratkaisuja, $a$ negatiivinen\\
	\begin{lukusuora}{-2}{2}{4}
	\lukusuoraisobbox
	\lukusuorakuvaaja{-x**2-0.3}
	\end{lukusuora}
\end{tabular}

\\ \qquad & \qquad \\

\end{tabular}
% fixme {termi-komento ei toimi vaillinaisen ekvationin kohdalla}

Graafisen ratkaisemisen sijasta toisen asteen yhtälö ratkaistaan yleensä laskemalla, mihin tutustutaan seuraavaksi.
%Seuraavaksi tutustumme toisen asteen yhtälön algebralliseen ratkaisemiseen.

\subsection{Vaillinaiset yhtälöt}
Jos toisen asteen yhtälöstä $ax^2+bx+c=0$ puuttuu joko termi $bx$ tai $c$, kyseessä on niin sanottu \emph{vaillinainen toisen asteen yhtälö}. Se on muotoa
\[ax^2+c=0 \quad \text{ tai } \quad ax^2+bx=0.\]
Vaillinaisten yhtälöiden ratkaiseminen on paljon yleistä tapausta yksinkertaisempaa.

\subsubsection*{Toisen asteen yhtälö $ax^2+c=0$}
Muotoa $ax^2+c = 0$ oleva toisen asteen yhtälö saadaan helposti ratkaistua neliöjuuren avulla.

\begin{esimerkki}
Ratkaistaan yhtälö $5x^2-45=0$.
\begin{align*}
5x^2-45 &= 0 &&\ppalkki + 45 \\
5x^2 &= 45 &&\ppalkki : 5 \\
x^2 &= 9 &&\text{ratkaistaan käyttäen neliöjuurta ($9 > 0$)} \\
x &= \pm \sqrt{9} = \pm 3.
\end{align*}
\end{esimerkki}

\begin{esimerkki}
Ratkaistaan yhtälö $13x^2-42=-3$:
\begin{align*}
13x^2-42 &= -3 &&\ppalkki + 42 \\
13x^2 &= 39 &&\ppalkki : 13 \\
x^2 &= 3 &&\text{ratkaistaan käyttäen neliöjuurta ($3 > 0$)} \\
x &= \pm \sqrt{3}.
\end{align*}
\end{esimerkki}

\begin{esimerkki}
Ratkaistaan yhtälö $x^2+4=3$:
\begin{align*}
x^2+4 &= 3 &&\ppalkki - 4 \\
x^2 &= -1
\end{align*}
Koska $x^2 \geq 0$ kaikilla $x$, yhtälöllä ei ole ratkaisua.
\end{esimerkki}

\subsubsection*{Toisen asteen yhtälö $ax^2+bx=0$}
Jos yhtälöstä $ax^2+bx+c=0$ puuttuu vakiotermi $c$, yhtälö saa muodon 
$$ax^2+bx=0.$$ Tällainen yhtälö ratkeaa jakamalla tekijöihin ja käyttämällä tulon nollasääntöä:
\begin{align*}
ax^2+bx&=0 \ \ \ \ \ &&\ppalkki\text{otetaan yhteinen tekijä} \\
x(ax+b)&=0 \ \ \ \ \ &&\ppalkki\text{tulon nollasääntö} \\
x&=0 \text{ tai } ax+b=0 \\
x&=0 \text{ tai } ax=-b \\
x&=0 \text{ tai } x=-\frac{b}{a}.
\end{align*}
\begin{esimerkki}
Ratkaise yhtälö $x^2-11x=0$.
\begin{align*}
x^2-11x&=0 \ \ \ \ \  &&\ppalkki\text{otetaan yhteinen tekijä } x\\
x(x-11)&=0 \ \ \ \ \ &&\ppalkki\text{tulon nollasääntö} \\
x&=0 \text{ tai } x-11=0 \\
x&=0 \text{ tai } x=11
\end{align*}
\end{esimerkki}

\begin{esimerkki}
Ratkaise yhtälö $55x^2+8x=0$.
\begin{align*}
55x^2+8x&=0 \ \ \ \ \ &&\ppalkki\text{otetaan yhteinen tekijä} \\
x(55x+8)&=0 \ \ \ \ \ &&\ppalkki\text{tulon nollasääntö} \\
x&=0 \text{ tai } 55x+8=0 \\
x&=0 \text{ tai } 55x=-8 \\
x&=0 \text{ tai } x=-\frac{8}{55}
\end{align*}
\end{esimerkki}

\subsection{Neliöksi täydentäminen}
Toisen asteen yhtälöä $ax^2+bx+c=0$, jossa $a,b,c \neq 0$, kutsutaan täydelliseksi toisen asteen yhtälöksi. Tällaiset yhtälöt voidaan ratkaista muistikaavojen avulla täydentämällä ne neliöiksi. Tarkastellaan vaikkapa yhtälöä
\[x^2+2x-3=0.\]
Yhtälön vasen puoli on melkein sama kuin binomin $x+1$ neliö, sillä $(x+1)^2=x^2+2x+1$. Vain vakiotermissä on eroa. Korjataan asia ja ratkaistaan yhtälö:

\begin{align*}
x^2+2x-3 & = 0  &&\ppalkki +4 \\
x^2+2x+1 & = 4  &&\ppalkki \text{ muistikaava: } (x+1)^2=x^2+2x+1. \\
(x+1)^2 & = 4  && \ppalkki \text{ neliöjuuri } (4 > 0)\\
x+1 & = \pm 2 \\
x & = \pm 2 - 1 \\
x & = 2-1 \text{ tai } x= -2-1 \\ 
x & = 1 \text{ tai } x= -3. 
\end{align*}

Miksi edellä osattiin ajatella juuri oikeaa muistikaavaa $(x+1)^2=x^2+2x+1$? Syynä on se, että yhtälön vasemmalla puolella olevan polynomin alkuosaa $x^2+2x$ ei saada minkään muun binomin neliöstä. Neliöksi täydentäminen vaatii siis muistikaavojen hyvää hallintaa.

Kaikki toisen asteen yhtälöt voidaan ratkaista neliöksi täydentämällä. Yleensä sen sijasta tosin käytetään \emph{toisen asteen yhtälön ratkaisukaavaa}, joka esitellään seuraavassa luvussa. Ratkaisukaava perustellaan neliöksi täydentämällä, minkä vuoksi neliöksi täydentäminen opetellaan ensin.

\begin{esimerkki} 
Ratkaise yhtälö $x^2+4x-16 = 0$. 

Kirjoitetaan yhtälö muotoon $x^2+2\cdot 2\cdot x-16 = 0$ ja verrataan kahta ensimmäistä termiä muistikaavaan, jotta nähdään, minkä binomin neliöksi lauseke voidaan muokata.
\[ \begin{array}{lcl}
x^2+2\cdot x\cdot 2  &=& (\quad + \quad)^2\\
a^2 +2\cdot a\cdot b +b^2 &=& (a+b)^2
\end{array} \]
Lausekkeita vertaamalla nähdään vastaavuus $a = x$, $b = 2$. Neliöstä puuttuva termi $b^2$ on siis $2^2=4$. Täydennetään nyt neliöksi:
\begin{align*}
x^2+4x-16 &= 0 && \ppalkki +16 \\
x^2+4x &= 16 && \ppalkki +4\\
x^2+4x+4 &= 20 &&  \text{muistikaava: $ x^2+4x+4= (x+2)^2$} \\
(x+2)^2 &= 20  && \text{ neliöjuuri } (20 > 0) \\
x+2 &= \pm \sqrt{20} && \ppalkki -2\\
x &= -2 \pm \sqrt{20} && \ppalkki \text{sievennetään vastaus} \\
x &= -2 \pm 2\sqrt{5}.
\end{align*}
\end{esimerkki}

\begin{esimerkki}
Ratkaistaan yhtälö $4x^2-4x-5=0$.

Toisen ja ensimmäisen asteen termit saadaan neliöstä $(2x-1)^2=4x^2-4x+1$. Rakennetaan se yhtälön vasemmalle puolelle:
\begin{align*}
4x^2-4x-5 &= 0 && \ppalkki +6 \\
4x^2-4x+1 &= 6 &&\text{muistikaava: $  4x^2-4x+1 = (2x-1)^2$} \\
(2x-1)^2 &= 6 \\
2x-1 &= \pm \sqrt{6} && \ppalkki +1\\
2x &= 1 \pm \sqrt{6} && \ppalkki : 2\\
x &= \frac{1 \pm \sqrt{6}}{2}.
\end{align*}
\end{esimerkki}

% \textbf{Esimerkki 5.} \\
% Ratkaistaan yhtälö $x^2+4x-16=0$.
% \begin{align*}
% x^2+4x-16&=0 \ \ \ \ \ &&\ppalkki +20 \\
% x^2+4x+4&=20 \ \ \ \ \ &&\ppalkki a^2+2ab+b^2=(a+b)^2 \\
% (x+2)^2&=20 \ \ \ \ \ &&\ppalkki \sqrt[]{} \\
% x+2 &= \pm \sqrt[]{20} \\
% x&=-2 \pm \sqrt[]{20} \\
% x&=-2 \pm 2 \sqrt[]{5} \\
% \end{align*}
% \textbf{Esimerkki 6.} \\
% Ratkaise yhtälö $16x^2-64x+2=0$. \\
% \textbf{Ratkaisu}
% \begin{align*}
% 16x^2-16x+2&=0 \ \ \ \ \ &&\ppalkki +2 \\
% 16x^2-16x+4&=2 \ \ \ \ \ &&\ppalkki a^2-2ab+b^2=(a-b)^2 \\
% (4x+2)^2&=2 \ \ \ \ \ &&\ppalkki \sqrt[]{} \\
% 4x+2&=\pm \sqrt[]{2} \\
% 4x&=-2 \pm \sqrt[]{2} \\
% x&=-\frac{1}{2} \pm \frac{\sqrt[]{2}}{4} \\
% x&=-\frac{1}{2} \pm \frac{\sqrt[]{2}}{ 2\sqrt[]{2}\sqrt[]{2}} \\
% x&=-\frac{1}{2} \pm \frac{1}{2 \sqrt[]{2}} \\
% \end{align*}
%
% Ratkaisutapaa, jossa toisen asteen yhtälö täydennetään lisäämällä tai vähentämällä termejä binomin neliöksi, kutsutaan neliöksi täydentämiseksi.
%
% Toisen asteen yhtälö voidaan aina ratkaista neliöön täydentämällä. Yleensä toisen asteen yhtälöt kuitenkin ratkaistaan käyttämällä suoraa kaavaa.
%
% Seuraavassa kappaleessa johdamme toisen asteen yhtälön ratkaisukaavan neliöksi täydentämistä käyttäen.

%\begin{esimerkki}
%Määritä polynomin $x^2+2x+2$ suurin/pienin arvo.
%    \begin{esimratk}
%	Polynomissa on kolme termiä, ja näiden vaikutusta polynomin arvoon on hankala tarkastella yhdessä. Muokataan polynomia ... ja neliöksi täydentäminen
%    \end{esimratk}
%\end{esimerkki}

\begin{tehtavasivu}

\subsubsection*{Opi perusteet}

\begin{tehtava}
    Ratkaise yhtälöt.
    \begin{alakohdat}
        \alakohta{$x^2 = 16$}
        \alakohta{$x^2 = - 16$}
        \alakohta{$x^2 - 13 = 0$}
        \alakohta{$3x^2 - 12 = 0$}

    \end{alakohdat}
    \begin{vastaus}
        \begin{alakohdat}
            \alakohta{$x=\pm 4$}
            \alakohta{Ei ratkaisuja.}
            \alakohta{$x = \pm \sqrt{13}$.}
            \alakohta{$x=\pm 2$}
        \end{alakohdat}
    \end{vastaus}
\end{tehtava}

\begin{tehtava}
    Ratkaise yhtälöt.
    \begin{alakohdat}
        \alakohta{$x(x-3)= 0$}
        \alakohta{$x^2 + 4x = 0$}
        \alakohta{$7x^2-3x = 0$}
    \end{alakohdat}
    \begin{vastaus}
        \begin{alakohdat}
            \alakohta{$x=0$ tai $x=3$}
            \alakohta{$x =0$ tai $x=-4$.}
            \alakohta{$x=0$ tai $x=\frac{3}{7}$}
        \end{alakohdat}
    \end{vastaus}
\end{tehtava}

\begin{tehtava}
    Kirjoita neliöksi tunnistamalla muistikaava
    \begin{alakohdat}
        \alakohta{$x^2 +2x +1$}
        \alakohta{$x^2 +6x +9$}
        \alakohta{$x^2 -4x +4$}
    \end{alakohdat}
    \begin{vastaus}
        \begin{alakohdat}
            \alakohta{$(x+1)^2$}
            \alakohta{$(x+3)^2$.}
            \alakohta{$(x-2)^2$}
        \end{alakohdat}
    \end{vastaus}
\end{tehtava}

\begin{tehtava}
    Ratkaise yhtälö täydentämällä neliöksi
    \begin{alakohdat}
        \alakohta{$x^2 -2x +1 = 4$}
        \alakohta{$x^2 +4x = 5$}
        \alakohta{$x^2 -3x + 10 = 0$}
    \end{alakohdat}
    \begin{vastaus}
        \begin{alakohdat}
            \alakohta{$x = 3$ tai $x= -1$. Neliöksi täydennettynä $(x-1)^2=4$}
            \alakohta{$x = -5$ tai $x = 1$. Neliöksi täydennettynä $(x+2)^2=9$}
            \alakohta{Ei ratkaisua. Neliöksi täydennettynä $(x-3)^2=-1$}
        \end{alakohdat}
    \end{vastaus}
\end{tehtava}

\subsubsection*{Hallitse kokonaisuus}

\begin{tehtava}
    Ratkaise seuraavat yhtälöt.
    \begin{alakohdat}
        \alakohta{$x^2 - 100 = 0$}
        \alakohta{$x^2 + 100 = 0$}
       \alakohta{$x^2 - 10 = 0$}
%        \alakohta{$x^2 + 10 = 0$}
        \alakohta{$-x^2 - 25 = 0$}
%        \alakohta{$-x^2 + 25 = 0$}
        \alakohta{$2x^2 - 98 = 0$}
%        \alakohta{$2x^2 + 98 = 0$}
    \end{alakohdat}
    \begin{vastaus}
        \begin{alakohdat}
            \alakohta{$x=\pm10$}
            \alakohta{Ei ratkaisuja.}
            \alakohta{$x=\pm\sqrt{10}$}
%            \alakohta{Ei ratkaisuja.}
            \alakohta{Ei ratkaisuja.}
%            \alakohta{$x=\pm5$}
            \alakohta{$x=\pm7$}
%            \alakohta{Ei ratkaisuja.}
        \end{alakohdat}
    \end{vastaus}
\end{tehtava}

\begin{tehtava}
    Ratkaise seuraavat yhtälöt.
    \begin{alakohdat}
        \alakohta{$x^2 - 72x = 0$}
%        \alakohta{$x^2 + 72x = 0$}
%        \alakohta{$x^2 - 56x = 0$}
        \alakohta{$x^2 + 56x = 0$}
        \alakohta{$-x^2 - 13x = 0$}
%        \alakohta{$-x^2 + 13x = 0$}
%        \alakohta{$2x^2 - 43x = 0$}
        \alakohta{$2x^2 + 43x = 0$}
    \end{alakohdat}
    \begin{vastaus}
        \begin{alakohdat}
            \alakohta{$x=0$ tai $x=72$}
%            \alakohta{$x=-72$ tai $x=0$}
%            \alakohta{$x=0$ tai $x=56$}
            \alakohta{$x=-56$ tai $x=0$}
            \alakohta{$x=-13$ tai $x=0$}
%            \alakohta{$x=0$ tai $x=13$}
%            \alakohta{$x=0$ tai $x=21,5$}
            \alakohta{$x=-21,5$ tai $x=0$}
        \end{alakohdat}
    \end{vastaus}
\end{tehtava}

\begin{tehtava}
    Ratkaise seuraavat yhtälöt.
    \begin{alakohdat}
        \alakohta{$x^2 - 36 = 0$}
        \alakohta{$x^2 - 85x = 0$}
        \alakohta{$x^2 + 11x = -6x$}
        \alakohta{$x^2 + 10x = -4x^2$}
    \end{alakohdat}
    \begin{vastaus}
        \begin{alakohdat}
            \alakohta{$x=\pm6$}
            \alakohta{$x=0$ tai $x=85$}
            \alakohta{$x=0$ tai $x=-17$}
            \alakohta{$x=0$ tai $x=-2$}
        \end{alakohdat}
    \end{vastaus}
\end{tehtava}

\begin{tehtava}
    Ratkaise yhtälö täydentämällä neliöksi.
    \begin{alakohdat}
        \alakohta{$x^2 -6x +9 = 1$}
        \alakohta{$x^2 +2x +4 = 0$}
        \alakohta{$x^2 -4x - 7 = 0$}
        \alakohta{$x^2 +x = \frac{3}{4}$}
        \alakohta{$2x^2 +3x -2 = 0$}
    \end{alakohdat}
    \begin{vastaus}
        \begin{alakohdat}
            \alakohta{$x = 4$ tai $x= 2$. Neliöksi täydennettynä $(x-3)^2=1$.}
            \alakohta{Ei ratkaisua. Neliöksi täydennettynä $(x+1)^2=-3$}
            \alakohta{$x = 2 \pm \sqrt{11}$ Neliöksi täydennettynä $(x-2)^2=11$.}
            \alakohta{$x = \frac{1}{2}$ tai $x=-1\frac{1}{2}$ Neliöksi täydennettynä $(x+\frac{1}{2})^2=1$.}
            \alakohta{$x = -2$ tai $x = \frac{1}{2}$. 
            Neliöksi täydennettynä $(x+\frac{3}{4})^2=\frac{25}{16}$.}
        \end{alakohdat}
    \end{vastaus}
\end{tehtava}

\begin{tehtava}
    Toisen asteen yhtälön vakiotermi on 4 ja sen ratkaisut ovat 2 ja 3. Mikä yhtälö on kyseessä?

    \begin{vastaus}
		Olkoon yhtälö muotoa $ax^2+bx+4=0$. \\      
      Muodostetaan yhtälöpari:
      \[
        \left\{
          \begin{aligned}
            a\cdot 2^2 + b\cdot 2 + 4 &= 0 \\
            a\cdot 3^2 + b\cdot 3 + 4 &= 0
          \end{aligned}
        \right.
      \]
      
      Yhtälöparin ratkaisuna saadaan $a=\frac23$ ja $b=-3\frac13$. Yhtälö on siis $\frac{2}{3}x^2-3\frac{1}{3}x+4=0$.
      
      Vastauksen voi saada myös ilmaisemalla polynomiyhtälön tekijämuodossa $a(x-2)(x-3)=0$. Tässä tarvitaan kuitenkin juurten ja tekijöiden välistä yhteyttä, joka opetetaan vasta myöhemmin tässä kirjassa.
    \end{vastaus}
\end{tehtava}

\subsubsection*{Lisää tehtäviä}

\begin{tehtava}
    Ratkaise yhtälöt käyttämällä tulon nollasääntöä.
    \begin{alakohdat}
        \alakohta{$(x^2-1)(x-7)=0$}
        \alakohta{$(x^2-9)(x^2-16)=0$}
        \alakohta{$(x-4)=x(x-4)$}
    \end{alakohdat}
    \begin{vastaus}
        \begin{alakohdat}
            \alakohta{$x=-1$, $x=1$ tai $x=7$}
            \alakohta{$x=-4$, $x=-3$, $x=3$ tai $x=4$}
            \alakohta{$x=1$ tai $x=4$}
        \end{alakohdat}
    \end{vastaus}
\end{tehtava}

\begin{tehtava}
    Ratkaise seuraavat yhtälöt.
    \begin{alakohdat}
        \alakohta{$x^2 - 9 = 0$}
        \alakohta{$2x^2 + 8 = 0$}
        \alakohta{$-x^2 + 11 = -5$}
        \alakohta{$3 - x^2 = -1 + 3x^2$}
    \end{alakohdat}
    \begin{vastaus}
        \begin{alakohdat}
            \alakohta{$x=\pm3$}
            \alakohta{Ei ratkaisuja}
            \alakohta{$x=\pm4$}
            \alakohta{$x=\pm1$}
        \end{alakohdat}
    \end{vastaus}
\end{tehtava}

\begin{tehtava}
    Ratkaise yhtälöt.
    \begin{alakohdat}
        \alakohta{$x^2(4x^2-1)^2 = 0 $}
        \alakohta{$-x^4(3x-1)^2 = 0$}
    \end{alakohdat}
    \begin{vastaus}
        \begin{alakohdat}
            \alakohta{$x=0$, $x= \frac{1}{2}$ tai $x= -\frac{1}{2}$}
            \alakohta{$x=0$ tai $x= \frac{1}{3}$}
        \end{alakohdat}
    \end{vastaus}
\end{tehtava}

\begin{tehtava}
    Ratkaise seuraavat yhtälöt.
    \begin{alakohdat}
        \alakohta{$x^2 - 3x = 0$}
        \alakohta{$10x + 2x^2 = 0$}
        \alakohta{$2x^2 - x^3 = 0$}
        \alakohta{$-3x^2 + 8x = -2x$}
    \end{alakohdat}
    \begin{vastaus}
        \begin{alakohdat}
            \alakohta{$x=0$ tai $x=3$}
            \alakohta{$x=0$ tai $x=-5$}
            \alakohta{$x=0$ tai $x=2$}
            \alakohta{$x=0$ tai $x=\frac{10}{3}$}
        \end{alakohdat}
    \end{vastaus}
\end{tehtava}

%\begin{tehtava}
%    Elokuvassa \emph{Dredd} pudotetaan ihmisiä kuolemaan noin $1$ kilometrin korkeudesta. Ennen pudotusta heille annetaan huumausainetta, joka hidastaa aikakäsityksen $1$ prosenttiin normaalista. Vapaassa pudotuksessa pudottu matka ajanhetkellä $t$ on $\frac{1}{2} gt^2$, jossa $g$ on putoamiskiihtyvyytenä tunnettu vakio, jolle voimme tässä hyvin käyttää arviota $g \approx 10\frac{\text{m}}{\text{s}^2}$.
%    \begin{alakohdat}
%    \alakohta{Olettaen, että huumausaineen vaikutus kestää koko putoamisen ajan, kuinka pitkältä aika pudotuksesta kuolemaan \textbf{uhrista} tuntuu? (Oleta annetut arvot tarkoiksi ja muodosta relevantti toisen asteen yhtälö.)}
%    \alakohta{Mikä menee fataalisti pieleen, jos a)-kohdan laskee suoraan kuvatulla tavalla?}
%    \end{alakohdat}
%    \begin{vastaus}
%        \begin{alakohdat}
%            \alakohta{Vastaukseksi saadaan $1414 \, \text{s} = 23 \, \text{min} \, 34 \, \text{s}$. Käytännössä hyvä vastaustarkkuus voisi olla esimerkiksi $25 \, \text{min}$.}
%            \alakohta{Tehtävä ei huomioi ilmanvastusta. Ihminen saavuttaa korkeimmillaan rajanopeuden $v_{raja} \approx 55\frac{\text{m}}{\text{s}}$. Tehtävän mallissa putoavan ihmisen nopeus nousee $v_{max} \approx 141\frac{\text{m}}{\text{s}}$ asti. Todellisuudessa putoaminen siis kestää vieläkin kauemmin.}
%        \end{alakohdat}
%    \end{vastaus}
%\end{tehtava}


%\begin{tehtava}
%Ollessaan leirikoulussa Lapissa lukiolaisryhmä saapuu järvelle ja havaitsee, että järven halkaisijan suuntainen maiseman poikkileikkaus on likimain paraabelin $\frac{1}{2500}x^2-\frac{1}{5}x$ muotoinen, jos $x$-akseli on on vedenpinnan taso ja yksikkönä on metri. Kuinka pitkä matka vastarannalle on?
%\begin{vastaus}
%500 metriä.
%\end{vastaus}
%\end{tehtava}

\end{tehtavasivu}

	\section{Toisen asteen yhtälön ratkaisukaava}

\qrlinkki{http://opetus.tv/maa/maa2/toisen-asteen-yhtalon-ratkaisukaava/}{Opetus.tv: \emph{toisen asteen ratkaisukaava} (9:03, 11:06 ja 10:09)}

Edellisessä kappaleessa opittiin, että toisen asteen yhtälö voidaan aina ratkaista täydentämällä se neliöksi.
Neliöksi täydentämistä käytetään kuitenkin harvoin, sillä saman ajatuksen voi ilmaista valmiina kaavana.
Johdetaan seuraavassa toisen asteen yhtälön ratkaisukaava. \\ \\

Lähdetään liikkeelle täydellisestä toisen asteen yhtälöstä $ax^2+bx+c=0$. Kuten edellisen kappaleen esimerkeissä, haluamme muokata sen muotoon, josta se on helppo täydentää binomin neliöksi. Huomataan binomin neliön muistikaavan sovelluksena, että $(\alpha x + \beta)^2 = \alpha^2 x^2 + 2\alpha\beta x +\beta^2$; verrataan tätä toisen asteen yhtälöön:
\begin{align*}
ax^2+bx+c &= 0 \\
\alpha^2 x^2 + 2\alpha\beta x +\beta^2 &= \text{?}
\end{align*}
(Tässä käytetään kreikkalaisia kirjaimia sekaannuksen välttämiseksi, sillä $a$ muistikaavassa $(a + b)^2$ ei välttämättä ole sama kuin toisen asteen yhtälön kerroin $a$.)

Nähdään että yhtälöt muistuttavat hieman toisiaan. Haluamme kuitenkin toisen asteen yhtälön vasemman puolen samanlaiseen muotoon kuin alemman:
\begin{align*}
ax^2+bx+c&=0 &&\textnormal{\footnotesize{vähennetään puolittain}} \ c \\
ax^2+bx &= -c &&\textnormal{\footnotesize{kerrotaan puolittain termillä}} \ 4a\\
4a^2 x^2 + 4abx &= -4ac \\
(2a)^2 x^2 +4abx &= -4ac 
\end{align*}
%\begin{align*}
%ax^2+bx+c&=0 &&\textnormal{\footnotesize{kerrotaan molemmat puolet termillä}} \ 4a \\
%4a \cdot ax^2+4a \cdot bx + 4a \cdot c&=0 \\
%4a^2x^2+4abx+4ac&=0 &&\textnormal{\footnotesize{vähennetään puolittain termi}} \ 4ac  \\
%4a^2x^2+4abx&=-4ac
%\end{align*}
Nyt voidaan täydentää vasen puoli neliöksi, mistä saaddaan esimerkkien tapaan ratkaisu $x$:n suhteen.
\begin{align*}
(2a)^2 x^2 +4abx &=-4ac &&\textnormal{\footnotesize{lisätään puolittain termi}} \ b^2 \\
(2a)^2 x^2 +2\cdot 2a\cdot bx +b^2&=b^2-4ac &&\textnormal{\footnotesize{neliö:}} \ (2a)^2 x^2 +2\cdot 2a\cdot bx+b^2=(2ax+b)^2 \\
(2ax+b)^2&=b^2-4ac &&\textnormal{\footnotesize{otetaan puolittain neliöjuuri, jos}} \ b^2 -4ac \geq 0 \\
2ax+b&= \pm \sqrt[]{b^2-4ac} &&\textnormal{\footnotesize{vähennetään puolittain termi}} \ b \\
2ax&=-b \pm \sqrt[]{b^2-4ac} &&\textnormal{\footnotesize{jaetaan puolittain termillä}} \ 2a \neq 0 \\
x&= \frac{-b \pm \sqrt[]{b^2-4ac}}{2a}
\end{align*}
Toisen asteen yhtälön ratkaisukaava on siis \[x= \frac{-b \pm \sqrt[]{b^2-4ac}}{2a}\] oletuksella, että $b^2-4ac \geq 0$. Oletus tarvitaan, koska negatiivisille luvuille ei ole reaaliluvuilla määriteltyä neliöjuurta.\\
\laatikko{\textbf{Toisen asteen yhtälön ratkaisukaava} \\
Yhtälön
$ax^2+bx+c=0$, missä $a \neq 0$ ja $b^2 - 4ac \geq 0$, reaaliset ratkaisut ovat
muotoa \\
\[ x=\frac{-b \pm \sqrt{b^2-4ac}}{2a}.\]
}
\begin{esimerkki}
Ratkaise yhtälö $x^2-8x+16=0$. \\
\begin{esimratk}
Yhtälö on muodossa $ax^2+bx+c=0$:
\begin{align*}
\underbrace{1}_{=a}x^2 +\underbrace{(-8)}_{=b}x+\underbrace{16}_{=c}=0
\end{align*}
Sijoitetaan vakioiden $a=1$, $b=-8$ ja $c=16$ arvot toisen asteen yhtälön
ratkaisukaavaan.
\begin{align*}
x&=\frac{-(-8)\pm \sqrt[]{(-8)^2-4\cdot 1 \cdot 16}}{2 \cdot 1} \\
x&=\frac{8 \pm \sqrt{64- 64}}{2} \\
x&=\frac{8 \pm 0}{2} \\
x&=4
\end{align*}
\end{esimratk}
\begin{esimvast}
$x=4$.
\end{esimvast}
\end{esimerkki}

\begin{esimerkki}
Ratkaise yhtälö $15x^2+24x=-10$.
\begin{esimratk}
Siirretään kaikki termit samalle puolelle yhtälöä:
\begin{align*}
\underbrace{15}_{=a}x^2+\underbrace{24}_{=b}x+\underbrace{10}_{=c}=0
\end{align*}
Sijoitetaan  $a=15$, $b=24$ ja $c=10$ toisen asteen yhtälön ratkaisukaavaan.
\begin{align*}
x&=\frac{-24 \pm \sqrt[]{24^2-4 \cdot 15 \cdot 10}}{2 \cdot 15} \\
x&=\frac{-24 \pm \sqrt[]{576-600}}{30} \\
x&=\frac{-24 \pm \sqrt[]{-24}}{30}
\end{align*}
Koska juurrettava on negatiivinen ($-24<0$), niin yhtälöllä ei ole reaalilukuratkaisuja.
\end{esimratk}
\begin{esimvast}
Ei ratkaisua.
\end{esimvast}
\end{esimerkki}

\begin{esimerkki}
Ratkaise yhtälö $x^2+2x-3=0$.
\begin{esimratk}
Sijoitetaan  $a=1$, $b=2$ ja $c=-3$ ratkaisukaavaan.
\begin{align*}
x&=\frac{-2 \pm \sqrt[]{2^2-4 \cdot 1 \cdot (-3)}}{2 \cdot 1} \\
x&=\frac{-2 \pm \sqrt[]{4+12}}{2} \\
x&=\frac{-2 \pm \sqrt[]{16}}{2} \\
x&=\frac{-2 \pm 4}{2} \\
x&=-1 \pm 2 \\
x&= -1+2 \text{ tai } x = -1-2 \\
x&=1 \text{ tai } x=-3 
\end{align*}
\end{esimratk}
\begin{esimvast}
$x=1$ tai $x=-3$.
\end{esimvast}
\end{esimerkki}

\begin{esimerkki}
Kahden luvun erotus on $7$ ja tulo 330. Mitkä luvut ovat kyseessä?
\begin{esimratk}
Olkoon pienempi luvuista $x$, jolloin suurempi on $x+7$. Saadaan yhtälö
\begin{align*}
x(x+7) &= 330 && \ppalkki -330 \\
x^2+7x-330 &= 0 && \ppalkki \text{ 2. asteen yhtälön ratkaisukaava}\\
x&=\frac{-7 \pm \sqrt{7^2-4\cdot 1 \cdot 330}}{2\cdot 1} \\
x&=\frac{-7 \pm \sqrt{1\,369}}{2}  &&	\ppalkki \sqrt{1\,369} = 37\\
x&=\frac{-7 \pm 37}{2}  \\
x&=\frac{-7 + 37}{2}\  \text{ tai } \  x = \frac{-7 - 37}{2}  \\
x&=15\  \text{ tai } \ x = -21
\end{align*}
Koska $x$ oli pienempi alkuperäisistä luvuista, suurempi on $15+7=22$ tai
$-21+7=-14$.
\end{esimratk}
\begin{esimvast}
Luvut ovat $15$ ja $22$ tai $-21$ ja $-14$.
\end{esimvast}
\end{esimerkki}

%\begin{esimerkki}
%Ratkaistaan yhtälö $-\sqrt{2}x+\frac{1}{2}=x^2$.
%\end{esimerkki}

%Yleinen toisen asteen yhtälö on muotoa $ax^2+bx+c=0$.
%Kerrotaan yhtälön molemmat puolet vakiolla $4a$: $4a^2x^2+4abx+4ac=0$.
%Siirretään termi $4ac$ toiselle puolelle: $4a^2x^2+4abx=-4ac$.
%Pyritään täydentämään vasen puoli neliöksi.
%Lisätään puolittain termi $b^2$: $4a^2x^2+4abx+b^2=b^2-4ac$.
%Havaitaan vasemmalla puolella neliö: $(2ax+b)^2=b^2-4ac$.
%Otetaan puolittain neliöjuuri: $2ax+b=\pm\sqrt{b^2-4ac}$.
%Vähennetään puolittain termi $b$: $2ax=-b\pm\sqrt{b^2-4ac}$.
%Jaetaan puolittain vakiolla $2a$: $x=\frac{-b\pm\sqrt{b^2-4ac}}{2a}$.

\begin{tehtavasivu}

\subsubsection*{Opi perusteet}

\begin{tehtava}
    Ratkaise
    \begin{alakohdat}
        \alakohta{$x^2 + 6x + 5 = 0$}
        \alakohta{$x^2 - 2x - 3 = 0$}
        \alakohta{$2x^2+3x+5= 0$.}
    \end{alakohdat}
    \begin{vastaus}
        \begin{alakohdat}
            \alakohta{$x = -5 \tai x = -1$}
            \alakohta{$x = 3 \tai x = -1$}
            \alakohta{Ei ratkaisuja.}
        \end{alakohdat}
    \end{vastaus}
\end{tehtava}

\begin{tehtava}
    Ratkaise
    \begin{alakohdat}
        \alakohta{$9x^2 - 12x + 4 = 0$}
        \alakohta{$2x+x^2 = -4$}
        \alakohta{$x^2+3x=5$}
        \alakohta{$3x^2 - 13x + 50 = -2x^2 + 17x + 5$.}
    \end{alakohdat}
    \begin{vastaus}
        \begin{alakohdat}
            \alakohta{$x = \dfrac{2}{3}$}
            \alakohta{Ei ratkaisua}
            \alakohta{$x = \frac{-3 \pm \sqrt{29}}{2}$}
            \alakohta{$x = 3$}
        \end{alakohdat}
    \end{vastaus}
\end{tehtava}

\begin{tehtava}
    Ratkaise
    \begin{alakohdat}
        \alakohta{$9x^2 - 15x + 6 = 0$}
        \alakohta{$x^2 + 23x = 0$}
        \alakohta{$4x^2 - 64 = 0$}
        \alakohta{$6x^2 + 1 = -18x$.}
    \end{alakohdat}
    \begin{vastaus}
        \begin{alakohdat}
            \alakohta{$x = 1 \tai x = \frac{2}{3}$}
            \alakohta{$x = 0 \tai x = -23$}
            \alakohta{$x = 4 \tai x = -4$}
            \alakohta{$x = \frac{-9 \pm 5\sqrt{3}}{6}$}
        \end{alakohdat}
    \end{vastaus}
\end{tehtava}

\begin{tehtava}
    Suorakulmaisen muotoisen alueen piiri on $34$\,m ja pinta-ala $60$\,m$^2$. Selvitä alueen mitat.
    \begin{vastaus}
		Alueen toinen sivu on $5$\,m ja toinen $12$\,m.
    \end{vastaus}
\end{tehtava}

\subsubsection*{Hallitse kokonaisuus}

\begin{tehtava}
    Ratkaise
    \begin{alakohdat}
		\alakohta{$-x^2 + 4x + 7 = 0$}
		\alakohta{$x^2 - 13x + 1 = 0$}
		\alakohta{$4x^2 - 3x - 5 = 0$}
		\alakohta{$\frac{5}{6} x^2 + \frac{4}{7} x - 1 = 0$.}
    \end{alakohdat}
    \begin{vastaus}
        \begin{alakohdat}
			\alakohta{$x = 2\pm \sqrt{11}$}
			\alakohta{$x = \frac{13\pm \sqrt{165}}{2}$}
			\alakohta{$x = \frac{3\pm \sqrt{89}}{8}$}
			\alakohta{$x = \frac{-12\pm \sqrt{1614}}{35}$}
        \end{alakohdat}
    \end{vastaus}
\end{tehtava}

\begin{tehtava}
    Kahden luvun summa on $8$ ja tulo $15$. Määritä luvut.
    \begin{vastaus}
		Luvut ovat $3$ ja $5$.
    \end{vastaus}
\end{tehtava}

\begin{tehtava}
    Suorakulmion muotoisen talon mitat ovat $6,0\,\text{m} \times 10,0$\,m. Talon kivijalan ympärille halutaan levittää tasalevyinen sorakerros. Kuinka leveä kerros saadaan, kun soraa riittää $20$\,m$^2$ alalle?
    \begin{vastaus}
		$58$\,cm levyinen
    \end{vastaus}
\end{tehtava}

\begin{tehtava}
	Jalkapallo potkaistiin ilmaan tasaiselta kentältä. Sivusta katsottuna pallon lentorata oli paraabeli, jonka yhtälö oli muotoa
	$$y=-x^2+15x-36,$$
	missä $x$ on etäisyys metreinä kenttää pitkin mitattuna ja $y$ korkeus kentän pinnasta.
Kuinka pitkan matkan pallon lensi?
	\begin{vastaus}
		Paraabelin nollakohdat ovat $x=3$ ja $x=12$, joten pallo lensi $12-3 = 9$ metriä. 
	\end{vastaus}
\end{tehtava}

\begin{tehtava}
Ratkaise yhtälö $(4t+1)x^2-8tx+(4t-1)=0$ vakion $t$ kaikilla reaaliarvoilla.
	\begin{vastaus} \begin{tabular}{l}
		Kun $t=-\frac{1}{4}$, toisen asteen termin kerroin on $0$, ja ainoa ratkaisu on $x = 1$  \\
		Kun $t \neq \frac{1}{4}$ kyseessä on toisen asteen yhtälö ja ratkaisu on $x= 1$ tai $x=\frac{4t-1}{4t+1}$. 	
		\end{tabular}
    \end{vastaus}
\end{tehtava}

\begin{tehtava}
	Viljami sijoittaa $1\,000$\,€ korkorahastoon, jossa korko lisätään pääomaan vuosittain. Rahasto perii aina koronmaksun yhteydessä $20$ euron vuosittaisen hoitomaksun, joka vähennetään summasta koronlisäyksen jälkeen. Viljami laskee, että hän saisi yhteensä $2,4\%$ lisäyksen pääomaansa kahden vuoden aikana.
        \begin{alakohdat}
            \alakohta{Mikä on rahaston korkoprosentti? Ilmoita tarkka arvo ja likiarvo mielekkäällä tarkkuudella.}
            \alakohta{Paljonko rahaa Viljamin pitäisi sijoittaa, että hänen sijoituksensa kasvaisi yhteensä $5,0\%$ kahden vuoden aikana?}
        \end{alakohdat}
	\begin{vastaus}
	    \begin{alakohdat}
		\alakohta{
		Merkitään korkokerrointa $x$:llä.
		$$(1\,000x -20)x-20=1,024\cdot 1\,000$$
		$$x = \dfrac{1+\sqrt{10\,441}}{100} \approx 1.03181211580212$$
		Vastaus: $3,2\%$
		}
		\alakohta{
		Merkitään Viljamin sijoittamaa summaa $a$:lla.
		\begin{flalign*}
		(ax -20)x-20 &= 1,050a\\
		a(x^2 -1,050) &= 20x+20\\
		a &= \dfrac{20x-20}{x^2-1,050} \approx 2\,776,41224
		\end{flalign*}
		Vastaus: $2\,800$\,€
		}
	    \end{alakohdat}
	\end{vastaus}
\end{tehtava}

\begin{tehtava}
    Kun kappaleen kiihtyvyys $a$ on vakio, pätee $x = v_0t + \dfrac{1}{2}at^2$ ja $v = v_0 + at$, missä $x$ on kuljettu matka, $v_0$ alkunopeus, $v$ loppunopeus ja $t$ aika.
		\begin{alakohdat}
            \alakohta{Kivi heitetään suoraan alas Olympiastadionin tornista (korkeus $72$ metriä) nopeudella $3,0$\,m/s. Kuinka monen sekunnin kuluttua kivi osuu maahan? Putoamiskiihtyvyys on noin $10$\,m/$\text{s}^2$}
            \alakohta{Bussin nopeus on $20$\,m/s. Bussi pysähtyy jarruttamalla tasaisesti. Se pysähtyy $10$ sekunnissa. Laske jarrutusmatka.}
        \end{alakohdat}
    \begin{vastaus}
        \begin{alakohdat}
            \alakohta{Jarrutusmatka on $100$ metriä.}
            \alakohta{Noin $3,5$ sekunnin kuluttua. (Todellisuudessa putoamisessa kestää kauemmin, sillä lasku ei huomioi ilmanvastusta.)}
        \end{alakohdat}
    \end{vastaus}
\end{tehtava}

\begin{tehtava}
	Johda neliöksi täydentämällä ratkaisukaava yhtälölle
	\[ x^2 +px+q=0. \]
	Tarkista sijoittamalla tavanomaiseen ratkaisukaavaan.
	\begin{vastaus}
		$x=\frac{-p \pm \sqrt{p^2-4q}}{2}$.
	\end{vastaus}
\end{tehtava}

\begin{tehtava} % toisen asteen yhtälö
On olemassa viisi peräkkäistä positiivista kokonaislukua, joista kolmen ensimmäisen neliöiden summa on yhtä suuri kuin kahden jälkimmäisen neliöiden summa. Mitkä luvut ovat kyseessä?
    \begin{vastaus}
		$10^2+11^2+12^2 = 13^2 + 14^2$.
    	(Jos negatiivisetkin luvut sallittaisiin, $(-2)^2+(-1)^2+0^2 = 1^2 + 2^2$ kävisi myös.) Löytyykö vastaava $4 + 3$ luvun sarja? Entä pidempi?
    \end{vastaus}
\end{tehtava}

\begin{tehtava}
	$\star$ Ratkaise yhtälö $(x^2-2)^6=(x^2+4x+4)^3$.
	\begin{vastaus}
		$x=-1$, $x=0$ tai $x=\frac{1 \pm \sqrt{17}}{2}$
	\end{vastaus}
\end{tehtava}

\subsubsection*{Lisää tehtäviä}

\begin{tehtava}
    Ratkaise yhtälöt.
    \begin{alakohdat}
        \alakohta{$x^2+3x+2=0$}
        \alakohta{$2x^2+5x-12=0$}
        \alakohta{$3x^2-7x-20=0$}
        \alakohta{$x^2+3x-5=0$}
        \alakohta{$x^2+5x-24=0$}
    \end{alakohdat}
    \begin{vastaus}
        \begin{alakohdat}
            \alakohta{$x=-2$ tai $x=-1$}
            \alakohta{$x=3/2$ tai $x=-4$}
            \alakohta{$x=4$ tai $x=-5/3$}
            \alakohta{$x=\frac{3\pm\sqrt{29}}{2}$}
            \alakohta{$x=3$ tai $x=-8$}
        \end{alakohdat}
    \end{vastaus}
\end{tehtava}

\begin{tehtava}
    Ratkaise yhtälöt.
    \begin{alakohdat}
        \alakohta{$x^2+3x-5=4x+8$}
        \alakohta{$8x^2-5x+1=-36$}
        \alakohta{$-3x^2-4x+2=-5x^2+3$}
        \alakohta{$-3x^2+4x+13=-5x^2+10x+9$}
    \end{alakohdat}
    \begin{vastaus}
        \begin{alakohdat}
            \alakohta{$\frac{1\pm\sqrt{53}}{2}$}
            \alakohta{Ei ratkaisua reaalilukujen joukossa.}
            \alakohta{$1\pm\frac{\sqrt{6}}{2}$}
            \alakohta{$x=1$ tai $x=2$}
        \end{alakohdat}
    \end{vastaus}
\end{tehtava}

\begin{tehtava}
    Ratkaise
    \begin{alakohdat}
		\alakohta{$-\frac{5}{7} x^2 + \frac{4}{11} x - \frac{1}{2} = 0$}
		\alakohta{$\frac{2}{3} x^2 - \frac{18}{5} x + \frac{3}{10} = 0$}
	\end{alakohdat}
    \begin{vastaus}
        \begin{alakohdat}
			\alakohta{Ei ratkaisuja.}
			\alakohta{$\frac{27 \pm \sqrt{684}}{10} = \frac{27 \pm 6 \sqrt{19}}{10}$}
        \end{alakohdat}
    \end{vastaus}
\end{tehtava}

\begin{tehtava} % toisen asteen yhtälö
Neliön muotoisen taulun sivu on $36$\,cm. Taululle tehdään tasalevyinen kehys, jonka nurkat on pyöristettu neljännesympyrän muotoisiksi. Kuinka leveä kehys on, kun sen pinta-ala on puolet taulun pinta-alasta? Ympyrän pinta-ala on $\pi r^2$.
    \begin{vastaus}
     $7,7$\,cm
    \end{vastaus}
\end{tehtava}

\begin{tehtava}
(K93/T5) Ratkaise yhtälö
        $\frac{2x+a^2-3a}{x-1}=a$ vakion $a$ kaikilla reaaliarvoilla.
\begin{vastaus}
        \begin{enumerate}
         \item{$x=a$, jos $a \neq 2$ ja $a \neq 1$}
         \item{$x\neq 1$, jos $a=2$}
         \item{ei ratkaisua, jos $a=1$}
        \end{enumerate}
    \end{vastaus}
\end{tehtava}

\begin{tehtava}
    Kultaisessa leikkauksessa jana on jaettu siten, että pidemmän osan suhde lyhyempään on sama kuin koko janan suhde pidempään osaan. Tämä suhde ei riipu koko janan pituudesta ja sitä merkitään yleensä kreikkalaisella aakkosella fii eli $\varphi$. Kultaista leikkausta on taiteessa kautta aikojen pidetty ''jumalallisena suhteena''. %varupphi?
		\begin{alakohdat}
            \alakohta{Laske kultaiseen leikkauksen suhteen $\varphi$ tarkka arvo ja likiarvo.}
            \alakohta{Napa jakaa ihmisvartalon pituussuunnassa suunnilleen kultaisen leikkauksen suhteessa. Millä korkeudella napa on $170$\,cm pitkällä ihmisellä?}
        \end{alakohdat}
    \begin{vastaus}
        \begin{alakohdat}
            \alakohta{$ \varphi = \dfrac{\sqrt{5}+1}{2} \approx 1,618$}
            \alakohta{Noin $105$\,cm korkeudella.}
        \end{alakohdat}
    \end{vastaus}
\end{tehtava}

\begin{tehtava}
(K96/T2b) Yhtälössä $x^2-2ax+2a-1=0$ korvataan luku $a$ luvulla $a+1$. Miten muuttuvat yhtälön juuret?
\begin{vastaus}
     Toinen kasvaa kahdella ja toinen ei muutu.
    \end{vastaus}
\end{tehtava}

\begin{tehtava} % toisen asteen yhtälö
Ratkaise yhtälö $x - 3 = \frac{1}{x}$.
    \begin{vastaus}
    $x =\frac{3 \pm \sqrt{13}}{2}$
    \end{vastaus}
\end{tehtava}

\begin{tehtava} %esimerkki tehtävä, malli sijoituksesta
	$\star$ Ratkaise yhtälö $(x^3-2)^2+x^3-2=2$.
	\begin{vastaus}
		Kirjoitetaan yhtälö muotoon $(x^3-2)^2+(x^3-2)-2=0$ ja sovelletaan ratkaisukaavaa niin, että ratkaistaan pelkän $x$ sijaan yhtälöstä lauseke $x^3-2$. Näin saadaan $x^3-2=1$ tai $x^3-2=-2$, joista voidaan edelleen ratkaista $x=\sqrt[3]{3}$ tai $x=0$.
	\end{vastaus}
\end{tehtava}

\end{tehtavasivu}


	\section{Diskriminantti}

\qrlinkki{http://opetus.tv/maa/maa2/diskriminantti/}{Opetus.tv: \emph{diskriminantti 2. asteen yhtälölle} (7:56 ja 8:30)}

\begin{esimerkki}
    Ratkaistaan toisen asteen yhtälö $3x^2-5x+10=0$.\\
    
%    \begin{align*}
%        \underbrace{3}_{=a}x^2\underbrace{-5}_{=b}x+\underbrace{10}_{=c}=0
%    \end{align*}
    Sijoitetaan $a=3$, $b=-5$ ja $c=10$ toisen asteen yhtälön ratkaisukaavaan $x=\frac{-b \pm \sqrt[]{b^2-4ac}}{2a}$.
    \begin{align*}
        x &=\frac{-(-5) \pm \sqrt[]{(-5)^2-4\cdot 3 \cdot 10}}{2 \cdot 3} \\
        x &=-\frac{5 \pm \sqrt[]{25-120}}{6} \\
          x &=-\frac{5 \pm \sqrt[]{-95}}{6} 
    \end{align*}
    Koska juurrettava on negatiivinen,
	 yhtälöllä ei ole ratkaisuja.
\end{esimerkki}

Edellisen esimerkin tulokseen päästäisiin nopeamminkin.
Koska ratkaisukaavassa esiintyvä lauseke $b^2-4ac$ on negatiivinen,
ratkaisua ei ole.

Lauseketta $b^2-4ac$ kutsutaan \termi{diskriminantti}{diskriminantiksi}. Sen arvo kertoo yhtälön ratkaisujen lukumäärän. Jos $D<0$, ratkaisuja ei ole, sillä negatiivisella luvulla ei ole neliöjuurta. Jos $D=0$, neliöjuuren arvoksikin tulee $0$ ja ratkaisuja saadaan vain yksi ($\pm 0 = 0$).
Jos $D>0$, neliöjuuri saa positiivisen arvon ja ratkaisuja on kaksi.

Diskriminantin avulla voidaan siis tutkia yhtälön ratkaisujen lukumäärää ilman että
yhtälöä tarvitsee ratkaista.

\newpage
\laatikko{Toisen asteen yhtälön $ax^2+bx+c=0$ ratkaisujen lukumäärä
voidaan laskea diskriminantin $D=b^2-4ac$ avulla seuraavasti:
\begin{itemize}
\item
Jos $D<0$, yhtälöllä ei ole reaalisia ratkaisuja.
\item
Jos $D=0$, yhtälöllä on tasan yksi reaalinen ratkaisu.
\item
Jos $D>0$, yhtälöllä on kaksi erisuurta reaaliratkaisua.
\end{itemize}
}
Tapauksessa $D=0$ yhtälön ainoaa ratkaisua kutsutaan
sen \termi{kaksoisjuuri}{kaksoisjuureksi}.

\begin{esimerkki}
\ \\
\parbox{4.5cm}{
\begin{kuvaajapohja}{1}{-1}{3}{-1}{3}
  \kuvaaja{2*x**2-2*x+1}{}{blue}
\end{kuvaajapohja}
}
\parbox{6cm}{$2x^2-2x+1=0$:\\$D=(-2)^2-4 \cdot 2 \cdot 1=4-8=-4$, eli $D <0$. Ei reaalisia ratkaisuja.}
\\
\parbox{4.5cm}{
\begin{kuvaajapohja}{1}{-1}{3}{-1}{3}
  \kuvaaja{x**2-2*x+1}{}{blue}
\end{kuvaajapohja}
}
\parbox{6cm}{$x^2-2x+1=0$:\\$D=(-2)^2-4 \cdot 1 \cdot 1=4-4=0$, eli $D = 0$. Yksi reaaliratkaisu.}
\\
\parbox{4.5cm}{
\begin{kuvaajapohja}{1}{-1}{3}{-2}{2}
  \kuvaaja{2*x**2-4*x+1}{}{blue}
\end{kuvaajapohja}
}
\parbox{6cm}{$2x^2-4x+1=0$:\\$D=(-4)^2-4 \cdot 2 \cdot 1=16-8=8$, eli $D > 0$. Kaksi eri reaaliratkaisua.}
\end{esimerkki}

\newpage

\begin{esimerkki}
Selvitetään, onko yhtälöllä $x^2+x+2=0$ ratkaisuja.

Tutkitaan diskriminanttia.
\[D=1^2-4\cdot 1 \cdot 2 = 1-8 = -7\]
Koska $D<0$, yhtälöllä ei ole ratkaisuja.

Jos yhtälön ratkaisemista yrittäisi ratkaisukaavan avulla, tulisi
neliöjuuren alle negatiivinen luku.
\end{esimerkki}

\begin{esimerkki}
Millä $k$:n arvolla yhtälöllä $9x^2+kx+1$ on tasan yksi ratkaisu?

Jotta ratkaisuja olisi tasan yksi, on diskriminantin oltava 0.
\begin{align*}
D &= 0\\
k^2-4\cdot 9\cdot 1 &= 0\\
k^2-36 &= 0\\
k^2 &= 36\\
k &= \pm 6
\end{align*}
Yhtälöllä on täsmälleen yksi ratkaisu, kun $k=-6$ tai $k=6$.
\end{esimerkki}

\begin{tehtavasivu}

\paragraph*{Opi perusteet}

\begin{tehtava}
	Laske diskriminanttien arvot.
	Kuinka monta ratkaisua yhtälöillä on?
	\begin{alakohdat}
		\alakohta{$3x^2+4x+1=0$}
		\alakohta{$x^2+2x+5=0$}
		\alakohta{$3x^2-6x+3=0$}
		\alakohta{$x^2-7x-40=0$}
	\end{alakohdat}
	\begin{vastaus}
		\begin{alakohdat}
			\alakohta{$D=4$, eli kaksi ratkaisua.}
			\alakohta{$D=-1$, eli ei ratkaisuja.}
			\alakohta{$D=0$, eli $1$ ratkaisu}
			\alakohta{$D=209$, kaksi ratkaisua}
		\end{alakohdat}
	\end{vastaus}
\end{tehtava}


\begin{tehtava}
	Kuinka monta ratkaisua yhtälöillä on?
	\begin{alakohdat}
		\alakohta{$12x^2+12x-4=0$}
		\alakohta{$12x^2+12x+4=0$}
	\end{alakohdat}
	\begin{vastaus}
		\begin{alakohdat}
			\alakohta{Kaksi. $D=12^2-4 \cdot 12 \cdot (-4) = 336 >0$}
			\alakohta{Ei yhtään. $D=12^2-4 \cdot 12 \cdot 4 = -48 <0$}
		\end{alakohdat}
	\end{vastaus}
\end{tehtava}

\begin{tehtava}
	Tulkitse polynomifunktion lauseketta: Onko kyseessä ylös- vai alaspäin aukeava paraabeli?
	Kuinka monta nollakohtaa funktiolla on?
	\begin{alakohdat}
		\alakohta{$P(x)=-3x^2+9x-5$}
		\alakohta{$Q(y)=5y^2-2y+1$}
		\alakohta{$R(z)=z^2-7z-40$}
		\alakohta{$S(w)=3w^2-6w+3$}
	\end{alakohdat}
	\begin{vastaus}
	Nollakohtien määrä voidaan päätellä diskriminantin arvosta.
		\begin{alakohdat}
			\alakohta{alaspäin, 2 nollakohtaa}
			\alakohta{ylöspäin, ei yhtään nollakohtaa}
			\alakohta{ylöspäin,2 nollakohtaa}
			\alakohta{ylöspäin, 1 nollakohta}
		\end{alakohdat}

	\end{vastaus}


\end{tehtava}


\begin{tehtava}
	Millä luvuilla $c$ yhtälöllä $x^2+5x+c = 0$ ei ole ratkaisua?
	\begin{vastaus}
		 $c> \frac{25}{4} =6,25$. (Halutaan $D < 0$, eli $5^2-4\cdot 1 \cdot c <0$.)
	\end{vastaus}
\end{tehtava}

\paragraph*{Hallitse kokonaisuus}

\begin{tehtava}
	Kuinka monta ratkaisua yhtälöillä on?
	\begin{alakohdat}
		\alakohta{$10x^2-8x-35=14x-10$}
		\alakohta{$-6x^2+15x-59=5x+17$}
%		\alakohta{$7x^2-6x+2=10$}
		\alakohta{$3x^2+7x=2x^2+x-9$}
	\end{alakohdat}
	\begin{vastaus}
		\begin{alakohdat}
			\alakohta{Kaksi. $D=(-22)^2-4 \cdot 10 \cdot (-25) = 1484 >0$}
			\alakohta{Nolla. $D=10^2-4\cdot (-6) \cdot (-76) = -1724 <0$}
%			\alakohta{Kaksi. $D=(-6)^2-4\cdot 7\cdot (-8) 260 > 0$}
			\alakohta{Yksi. $D=6^2-4\cdot 1 \cdot 9 = 0$}
		\end{alakohdat}
	\end{vastaus}
\end{tehtava}

\begin{tehtava}
	Millä vakion $a$ arvoilla yhtälöllä \\ $(2a-1)x^2+(a+1)x+3=0$ on 
	täsmälleen yksi juuri?
	\begin{vastaus}
		Sopivat $a$:n arvot ovat $\frac{1}{2}$, $11+6\sqrt{3}$ ja $11-6\sqrt{3}$.
	\end{vastaus}
\end{tehtava}

\begin{tehtava}
	Mitä voit sanoa ratkaisujen lukumäärästä vaillinaisten yhtälöiden $ax^2+c=0$  ja $ax^2+bx=0$  tapauksessa? Oletetaan, että $a$,$b$,$c \neq 0$.
	\begin{vastaus}
		\begin{description}
			\item[$ax^2+c=0$] Joko kaksi ratkaisua tai ei yhtään ratkaisua. ($D \neq 0$)
			\item[$ax^2+bx=0$] Aina kaksi ratkaisua. ($D > 0$)
		\end{description}
	\end{vastaus}
\end{tehtava}

\begin{tehtava}
	$ \star $ Osoita, että diskriminantti on $0$ jos ja vain jos yhtälö voidaan esittää muodossa \\ $(c_1 x+ c_2)^2=0$, missä $c_1$ ja $c_2$ ovat reaalilukuja.
	\begin{vastaus}
		Suunta "$\Rightarrow$": $(c_1 x+ c_2)^2=0 \Leftrightarrow c_1^2 x^2 + 2c_1 c_2 x+ c_2^2 =0 \Rightarrow
		D=(2 c_1 c_2)^2-4 c_1^2 c_2^2 =4 c_1^2 c_2^2 -4 c_1^2 c_2^2 =0$ \\
		Suunta "$\Leftarrow$": $D=0 \Leftrightarrow b^2-4ac=0 \Leftrightarrow b^2=4ac \Leftrightarrow c=\frac{b^2}{4a} \Rightarrow ax^2+bx+\frac{b^2}{4a}=0 \Leftrightarrow 4a^2x^2+4abx+b^2=0 \Leftrightarrow (2ax+b)^2=0$
	\end{vastaus}
\end{tehtava}

\paragraph*{Lisää tehtäviä}

\begin{tehtava}
	Kuinka monta ratkaisua yhtälöillä on?
	\begin{alakohdat}
		\alakohta{$9x^2+12x-4=0$}
		\alakohta{$5x^2+4x-10=0$}
		\alakohta{$3x^2-12x+12=0$}
		\alakohta{$5x^2+10x-30=0$}
	\end{alakohdat}
	\begin{vastaus}
		\begin{alakohdat}
			\alakohta{Kaksi. $D=12^2-4 \cdot 9 \cdot (-4) = 288 >0$}
			\alakohta{Kaksi. $D=4^2-4\cdot 5 \cdot (-10) = 216 >0$}
			\alakohta{Yksi. $D=(-12)^2-4\cdot 3\cdot 12 =0$}
			\alakohta{Kaksi. $D=10^2-4\cdot 5 \cdot (-30) = 700 >0$}
		\end{alakohdat}
	\end{vastaus}
\end{tehtava}

\begin{tehtava}
	Millä vakion $k$ arvoilla yhtälöllä \\ $-x^2-x-k = 0$ on ratkaisuja?
	\begin{vastaus}
		Pitää olla $D=(-1)^2-4 \cdot (-1) \cdot (-k) \geq 0$. Siis $k \leq \frac{1}{4}$.
	\end{vastaus}
\end{tehtava}

\begin{tehtava}
	Millä vakion $a$ arvoilla yhtälöllä \\ $ax^2+x=ax-5$ on 
	täsmälleen yksi ratkaisu?
	\begin{vastaus}
		$a =11 \pm 2\sqrt{30}$.
	\end{vastaus}
\end{tehtava}

\begin{tehtava}
	Kuinka monta ratkaisua yhtälöillä on vakion $a$ eri arvoilla?
	\begin{alakohdat}
		\alakohta{$x^2+6x+a+1=0$}
		\alakohta{$ax^2+4x-1=0$}
	\end{alakohdat}
	\begin{vastaus}
		\begin{alakohdat}
			\alakohta{Kaksi, kun $a<8$, yksi, kun $a=8$, ja nolla, kun $a>8$.}
			\alakohta{Kaksi, kun $-4<a$ ja $a\neq 0$; yksi, kun $a=-4$ tai $a=0$, ja nolla, kun $a<-4$.}
		\end{alakohdat}
	\end{vastaus}
\end{tehtava}

\end{tehtavasivu}


	\section{Toisen asteen epäyhtälö}

\qrlinkki{http://opetus.tv/maa/maa2/toisen-asteen-epayhtalo/}{Opetus.tv: \emph{toisen asteen epäyhtälö} (9:20 ja 10:19)}

%\begin{esimerkki}
%\missingfigure{johdantoesimerkki esim. lämpötilatarkastelu tjsp}
%\begin{itemize}
%\item{Milloin lämpötila on suurempi kuin nolla?}
%\item{Milloin lämpötila on pienempi kuin nolla?}
%\item{Milloin lämpötilan merkki voi vaihtua?} 
%\end{itemize}
%\end{esimerkki}

%\subsection*{Toisen asteen epäyhtälön ratkaiseminen kuvaajan avulla}

Toisen asteen epäyhtälön voi ratkaista tutkimalla toisen asteen polynomin kulkua.

\begin{esimerkki}
Ratkaise epäyhtälö $x^2-5>0$.

\textbf{Ratkaisu:}
Tarkastellaan polynomifunktiota $f(x)=x^2-5$. Tehtävänanto voidaan nyt muotoilla uudelleen. On selvitettävä, millä muuttujan $x$ arvoilla polynomifunktion $f(x)=x^2-5$ arvot ovat positiivisia.
%Funktion $f$ arvot ovat suurempia kuin nolla niillä muuttujan $x$ arvoilla, joilla pätee $x^2-5>0$.

\begin{kuvaajapohja}[\kuvaajaAsetusEiRuudukkoa\kuvaajaAsetusEiLukuja]{0.6}{-5}{5}{-6}{5}
\kuvaajakohtaarvo{-3}{4}{$-3$}{$f(-3)$}
\kuvaajakohtaarvo{1}{-4}{$1$}{$f(1)$}
\kuvaaja{x**2-5}{$f(x)=x^2-5$}{black}
\end{kuvaajapohja}

\begin{itemize}
\item Funktion arvot ovat positiivisia, kun funktion kuvaaja on $x$-akselin yläpuolella.
\item Funktion arvot ovat negatiivisia, kun funktion kuvaaja on $x$-akselin alapuolella.
\end{itemize}
%\begin{lukusuora}{-2.5}{2.5}{5}
%\lukusuoraparaabeli{-1}{1}{-1}
%\lukusuoraalanimi{0}{$-$}
%\lukusuoranimi{-1.8}{$+$}
%\lukusuoranimi{1.8}{$+$}
%\end{lukusuora}

Kahden vierekkäisen nollakohdan välissä funktion kuvaaja on joko kokonaan $x$-akselin alapuolella tai kokonaan $x$-akselin yläpuolella. Tällöin funktion arvot ovat aina joko pelkästään positiivisia tai pelkästään negatiivisia.

%Ratkaistaan funktion $f$ nollakohdat, koska niissä kohdissa funktion arvojen merkki voi vaihtua.
Funktion käyttäytymisestä saadaan siis paljon tietoa etsimällä sen nollakohdat:
\begin{align*}
f(x)&=0 \\
x^2-5&=0 \\
x^2&=5 \\
x&=\pm \sqrt{5}
\end{align*}

Nyt tiedetään, että funktio leikkaa $x$-akselin kohdissa $x=-\sqrt{5}$ ja $x=\sqrt{5}$. Kuvasta nähdään, että funktion kuvaaja on $x$-akselin yläpuolella, kun $x<-\sqrt{5}$ tai $x>\sqrt{5}$. Siten funktion arvot ovat positiivisia, kun $x<-\sqrt{5}$ tai $x>\sqrt{5}$.

\begin{lukusuora}{-2.5}{2.5}{5}
\lukusuoraparaabeli{-1}{1}{-1}
\lukusuoraalanimi{0}{$-$}
\lukusuoranimi{-1.8}{$+$}
\lukusuoranimi{1.8}{$+$}
\lukusuorapienipiste{-1}{\hspace{4mm}-\!$\sqrt{5}$}
\lukusuorapienipiste{1}{$\sqrt{5}$\hspace{3mm}}
\end{lukusuora}

\textbf{Vastaus:} $x<-\sqrt{5}$ tai $x>\sqrt{5}$.

\end{esimerkki}

Toisen asteen polynomifunktion $f(x)=ax^2+bx+c$ arvot voivat vaihtaa merkkiään vain funktion nollakohdissa. Jos halutaan tietää, milloin toisen asteen polynomifunktion arvot ovat positiivisia tai negatiivisia, niin
\begin{itemize}
\item[1.] Ratkaistaan funktion $f(x)=ax^2+bx+c$ nollakohdat eli etsitään ne muuttujan $x$ arvot, joilla $ax^2+bx+c=0$.
\item[2.] Hahmotellaan funktion kuvaajan aukeamissuunta ja merkitään kuvaajaan funktion nollakohdat.
\item[3.] Päätellään hahmotelmasta milloin funktion arvot ovat positiivisia ja milloin negatiivisia.
\end{itemize} 

Toisen asteen epäyhtälöitä voi toki ratkaista myös ilman paraabelien ominaisuuksiin vetoamista, mutta se ei ole sen helpompaa. Tästä lisää luvussa \ref{kork_ast}.

\begin{esimerkki} 
Ratkaise epäyhtälö $x^2-6<-5x$.
 
\textbf{Ratkaisu:}
Muutetaan ensin epäyhtälö muotoon $x^2+5x-6<0$ lisäämällä molemmille puolille termi $5x$. Nyt tehtävä voidaan ratkaista tutkimalla, millä muuttujan arvoilla funktion $f(x)=x^2+5x-6$ arvot ovat negatiivisia.
 
1. Ratkaistaan polynomifunktion $f(x)=x^2+5x-6$ nollakohdat.
\begin{align*}
f(x)&=0 & \\
x^2+5x-6&=0 \ \  \ \ \ & || \text{ 2. asteen yhtälön ratkaisukaava} \\ 
x&=\frac{-5 \pm \sqrt[]{(5)^2-4 \cdot 1 \cdot(-6)}}{2 \cdot 1} & \\
x&=\frac{-5 \pm \sqrt[]{25+24}}{2} & \\
x&=\frac{-5 \pm \sqrt[]{49}}{2} & \\
x&=\frac{-5 \pm 7}{2} & \\
x&=-6 \text{ tai } x=1 &
\end{align*}
2. Hahmotellaan polynomifunktion kuvaaja. Se aukeaa ylöspäin, koska toisen
asteen termin $x^2$ kerroin on positiivinen.
 
\begin{lukusuora}{-2.5}{2.5}{5}
\lukusuoraparaabeli{-1}{1}{-1}
\lukusuoraalanimi{0}{$-$}
\lukusuoranimi{-1.8}{$+$}
\lukusuoranimi{1.8}{$+$}
\lukusuorapienipiste{-1}{\hspace{3mm}-6}
\lukusuorapienipiste{1}{1\hspace{1mm}}
\end{lukusuora}
 
3.  Kuvaajasta voidaan päätellä, että $f(x)<0$, kun $-6 < x < 1$.
 
\begin{esimvast} 
Alkuperäinen epäyhtälö toteutuu, kun $-6 < x <1$.  
\end{esimvast}
\end{esimerkki}

\begin{esimerkki}
Ratkaise epäyhtälö $5x^2+13x+3<0$.

\textbf{Ratkaisu:}

Tutkitaan funktiota $f(x)=5x^2+13x+3$. Nyt on selvitettävä, millä muuttujan $x$ arvoilla funktion arvot ovat negatiivisia.

1. Ratkaistaan funktion nollakohdat, jotta tiedetään, missä kohdissa kuvaaja leikkaa $x$-akselin.
\begin{align*}
f(x)&=0 \\
5x^2+13x+3&=0 & || \text{ 2. asteen yhtälön ratkaisukaava}  \\
x&=\frac{-13 \pm \sqrt[]{13^2-4 \cdot 5 \cdot 3}}{2 \cdot 5} & \\
x&=\frac{-13 \pm \sqrt[]{169-60}}{10} & \\
x&=\frac{-13 \pm \sqrt[]{109}}{10} & 
\end{align*}

2. Hahmotellaan funktion kuvaaja.
Kuvaaja on ylöspäin aukeava paraabeli, koska toisen asteen termin kerroin $5$ on positiivinen. 

\begin{lukusuora}{-2.5}{2.5}{5}
\lukusuoraparaabeli{-1}{1}{-1}
\lukusuoraalanimi{0}{$-$}
\lukusuoranimi{-1.8}{$+$}
\lukusuoranimi{1.8}{$+$}
\lukusuorapienipiste{-1}{}
\lukusuorapienipiste{1}{}
\end{lukusuora}

3. Kuvaajasta voidaan päätellä, että $f(x)<0$, kun $\frac{-13 - \sqrt[]{109}}{10}<x< \frac{-13 + \sqrt[]{109}}{10}$.

\textbf{Vastaus:}
$\frac{-13 - \sqrt[]{109}}{10}<x< \frac{-13 + \sqrt[]{109}}{10}$

\end{esimerkki}

\begin{esimerkki}
Mikko rakentaa suorakulmion muotoista kaniaitausta lemmikkikanilleen. Aitaus rajoittuu yhdeltä sivulta Mikon taloon. Hänellä on yhteensä $14$ metriä kaniverkkoa. Miten Mikon pitää valita aitauksensa mitat, jotta kaniaitauksen koko on vähintään $12$ neliömetriä?

%fixme Tähän esimerkkiin olisi hyvä saada kuva.

\textbf{Ratkaisu}

Olkoon talon suuntaisen sivun pituus $y$ ja kahden päätysivun pituus $x$. Koska tiedämme, että $x+x+y=14$, saadaan tästä ratkaistua talon suuntaisen sivun pituudeksi $y=14 - 2x$.
%Sivujen pituuksille pitää päteä
%\begin{align*}
%x&>0 \ \ \ \ \ \ \text{ ja } \ \ \ \ \ y>0 \\
%x&>0 \ \ \ \ \ \ \text{ ja } \ \ \ \ \ 14-2x>0 \\ 
%x&>0 \ \ \ \ \ \ \text{ ja } \ \ \ \ \ 14>2x \\ 
%x&>0 \ \ \ \ \ \ \text{ ja } \ \ \ \ \ 7>x \\
%\end{align*} 
Aitauksen pinta-alaa kuvaa funktio $A(x)=x(14-2x)=14x-2x^2=-2x^2-14x$. Halutaan, että $A(x)>12$, josta saadaan epäyhtälö $-2x^2+14x>12$. Tämä epäyhtälö saadaan vielä muotoon $-2x^2+14x-12>0$.

Nyt ratkaistavana on epäyhtälö $-2x^2+14x-12>0$. Tutkitaan funktiota $f(x)=-2x^2+14x-12$ ja selvitetään, milloin sen arvot ovat positiivisia.

Selvitetään ensin funktion nollakohdat:
\begin{align*}
-2x^2+14x-12&=0 \\
x&=\frac{-14 \pm \sqrt[]{14^2-4 \cdot (-2) \cdot (-12)}}{2 \cdot (-2)} \\
x&=\frac{-14 \pm \sqrt[]{196-96}}{-4} \\
x&=\frac{-14 \pm 10}{-4} \\
x&=\frac{-24}{-4} \quad \text{tai} \quad x=\frac{-4}{-4} \\
x&=6 \quad \text{tai} \quad x=1
\end{align*}

Hahmotellaan sitten funktion $f(x)=-2x^2+14x-12$ kuvaaja. Koska 2. asteen termin kerroin $-2$ on negatiivinen, on kyseessä alaspäin aukeava paraabeli. Hahmotellaan sen kuvaaja:

\begin{lukusuora}{-2.5}{2.5}{5}
\lukusuoraparaabeli{-1}{1}{1}
\lukusuoranimi{0}{$+$}
\lukusuoraalanimi{-1.8}{$-$}
\lukusuoraalanimi{1.8}{$-$}
\lukusuorapienipiste{-1}{1\hspace{1mm}}
\lukusuorapienipiste{1}{\hspace{1mm}6}
\end{lukusuora}

Kuvasta nähdään, että funktion arvot ovat positiivisia, kun $1<x<6$. Siten
epäyhtälö $-2x^2+14x-12$ toteutuu, kun $1<x<6$. Tämä tarkoittaa, että aitauksen ala on suurempi kuin 12 neliömetriä, kun $1<x<6$.

\textbf{Vastaus:} Olkoon $y$ aitauksen talonsuuntainen sivun ja $x$ päätysivun pituus. Tällöin $1<x<6$ ja $y=14-2x$. Toisin sanottuna, Mikko valitsee päätysivun pituudeksi ($x$) jotakin väliltä 1-6~m, jolloin talonsuuntaisen sivun pituus ($y$) metreinä saadaan sijoittamalla $x$ lausekkeeseen $y=14-2x$.
\end{esimerkki}

% \subsection*{Toisen asteen epäyhtälön ratkaiseminen tekijöihin jakamisen avulla}
% \begin{esimerkki} 
% Ratkaise epäyhtälö $x^2+10x>0$.
% 
% \textbf{Ratkaisu:}
% 
% \begin{align*}
% x^2+10x&=0 &||\text{ tulon nollasääntö} \\
% x(x+10)&=0 & \\
% x&=0 \quad \text{tai} \quad x+10=0 & \\
% x&=0 \quad \text{tai} \quad x=-10 &
% \end{align*}
% Koska kahden positiivisen tai negatiivisen luvun tulo on positiivinen, niin
% polynomi $x^2+10x=x(x+10)$ saa positiivisen arvon, kun 
% \begin{itemize}
% \item{$x>0$ ja $x+10>0$} \\tai \\
% \item{$x<0$ ja $x+10<0$} 
% \end{itemize}
% Koska negatiivisen ja positiivisen luvun tulo on negatiivinen, niin
% polynomi $x^2+10x=x(x+10)$ saa negatiivisen arvon, kun
% \begin{itemize}
% \item{$x>0$ ja $x+10<0$} \\ tai \\
% \item{$x<0$ ja $x+10>0$} \\
% \end{itemize}
% 
% \begin{center}
% \begin{merkkikaavio}{2}
% \merkkikaavioKohta{$-10$}
% \merkkikaavioKohta{$0$}
% 
% 	\merkkikaavioFunktio{$x$}
% 	\merkkikaavioMerkki{$-$}
% 	\merkkikaavioMerkki{$-$}
% 	\merkkikaavioMerkki{$+$}
% \merkkikaavioUusirivi
% 	\merkkikaavioFunktio{$x+10$}
% 	\merkkikaavioMerkki{$-$}
% 	\merkkikaavioMerkki{$+$}
% 	\merkkikaavioMerkki{$+$}
% \merkkikaavioUusiriviKaksoisviiva
% 	\merkkikaavioFunktio{$x(x+10)$}
% 	\merkkikaavioMerkki{$+$}
% 	\merkkikaavioMerkki{$-$}
% 	\merkkikaavioMerkki{$+$}
% \end{merkkikaavio}
% \end{center}
% 
% Tästä huomataan, että epäyhtälö $x^2+10x>0$ on tosi, kun $x>0$ tai $x<-10$. 
% \end{esimerkki} 
% 
% Toinen tapa ratkaista toisen asteen epäyhtälö on 
% \begin{itemize}
% \item{jakaa polynomi tekijöihinsä }
% \item{tutkia millä muuttujan $x$ arvoilla kahden lausekkeen tulo on positiivinen ja milloin negatiivinen}
% \end{itemize}
% 
% \begin{esimerkki}
% Ratkaise epäyhtälö $x^2-6x+9 \leq 0$. \\ \\
% \textbf{Ratkaisu:} \\
% Ratkaistaan millä muuttujan $x$ arvolla polynomi saa arvon nolla.
% \begin{align*}
% x^2-6x+9&=0 \\
% x&=\frac{-(-6) \pm \sqrt[]{(-6)^2-4 \cdot 1 \cdot 9}}{2 \cdot 1} \\
% x&=-3
% \end{align*}
% Yhtälöllä $x^2-6x+9=0$ on kaksoisjuuri $x=-3$. 
% Polynomi saadaan siis muotoon $x^2-6x+9=(x-3)^2$.
% 
% \begin{center}
% \begin{merkkikaavio}{1}
% \merkkikaavioKohta{$3$}
% 
% 	\merkkikaavioFunktio{$x-3$}
% 	\merkkikaavioMerkki{$-$}
% 	\merkkikaavioMerkki{$+$}
% \merkkikaavioUusirivi
% 	\merkkikaavioFunktio{$x-3$}
% 	\merkkikaavioMerkki{$-$}
% 	\merkkikaavioMerkki{$+$}
% \merkkikaavioUusirivi
% 	\merkkikaavioFunktio{$(x-3)(x-3)$}
% 	\merkkikaavioMerkki{$+$}
% 	\merkkikaavioMerkki{$+$}
% \end{merkkikaavio}
% \end{center}
% 
% Kaaviosta saadaan pääteltyä, että $x^2-6x+9=(x-3)^2$ on positiivinen kun $x \neq -3$. Siis epäyhtälö $x^2-6x+9 \leq 0$ toteutuu, kun $x=-3$ jolloin polynomi saa arvon nolla. 
% \end{esimerkki}
% \begin{esimerkki}
% Ratkaise epäyhtälö $2x^2+2x-12<0$. 
% 1. Ratkaistaan millä muuttujan $x$ arvolla polynomi $2x^2+2x-12$ saa arvon nolla.
% \begin{align*}
% 2x^2+2x-12&=0 \\
% x&=\frac{-2 \pm \sqrt[]{2^2-4 \cdot 2 \cdot (-12)}}{2 \cdot 2} \\
% x&=\frac{-2 \pm \sqrt[]{4+96}}{4} \\
% x&=\frac{-2 \pm 10}{4} \\
% x&=-3 \text{ tai } x = 2
% \end{align*}
% 2. Polynomi saadaan muotoon $2x^2+2x-12=2(x-2)(x+3)$.  \\
% 3. Piirretään merkkikaavio:
% \begin{center}
% \begin{merkkikaavio}{2}
% \merkkikaavioKohta{$-3$}
% \merkkikaavioKohta{$2$}
% 
% 
% 	\merkkikaavioFunktio{$2(x-2)$}
% 	\merkkikaavioMerkki{$-$}
% 	\merkkikaavioMerkki{$-$}
% 	\merkkikaavioMerkki{$+$}
% \merkkikaavioUusirivi
% 	\merkkikaavioFunktio{$x+3$}
% 	\merkkikaavioMerkki{$-$}
% 	\merkkikaavioMerkki{$+$}
% 	\merkkikaavioMerkki{$+$}
% \merkkikaavioUusiriviKaksoisviiva
% 	\merkkikaavioFunktio{$2(x-2)(x+3)$}
% 	\merkkikaavioMerkki{$+$}
% 	\merkkikaavioMerkki{$-$}
% 	\merkkikaavioMerkki{$+$}
% \end{merkkikaavio}
% \end{center}
% 
% 4. Huomataan, että polynomi $2(x-2)(x+3)$ on negatiivinen, kun $-3<x<2$. 
% \end{esimerkki}

\begin{tehtavasivu}

\subsubsection*{Opi perusteet}

\begin{tehtava}
    Ratkaise seuraavat epäyhtälöt.
    \begin{alakohdat}
        \alakohta{$x^2-4<0$}
        \alakohta{$x^2+x \geq 0$}
        \alakohta{$x^2+1<0$}
        \alakohta{$x^2+5x+6>0$}
    \end{alakohdat}
    \begin{vastaus}
        \begin{alakohdat}
            \alakohta{$-2<x<2$}
            \alakohta{$x \geq 0$ tai $x \leq -1$}
            \alakohta{ei ratkaisuja}
            \alakohta{$x >-2$ tai $x < -3$}
        \end{alakohdat}
    \end{vastaus}
\end{tehtava}

\begin{tehtava}
    Ratkaise seuraavat epäyhtälöt.
    \begin{alakohdat}
        \alakohta{$-x^2+5x+8>0$}
        \alakohta{$-2x^2+6x+9>0$}
        \alakohta{$4x^2-8x-72<0$}
%        \alakohta{$5x^2+10x-73<0$}
    \end{alakohdat}
    \begin{vastaus}
        \begin{alakohdat}
            \alakohta{$\frac{5-\sqrt{57}}{2}  < x < \frac{5+\sqrt{57}}{2}$}
            \alakohta{$\frac{3-3\sqrt{3}}{2}<x<\frac{3+3\sqrt{3}}{2} $}
            \alakohta{$1-\sqrt{19} < x < 1+\sqrt{19}$}
%            \alakohta{$-1+\frac{1}{5} \sqrt{390} > x > -1-\frac{1}{5} \sqrt{390}$}
        \end{alakohdat}
    \end{vastaus}
\end{tehtava}

\begin{tehtava}
    Ratkaise seuraavat epäyhtälöt.
    \begin{alakohdat}
        \alakohta{$4x^2+13x\geq 0$}
        \alakohta{$4x^2-776\geq 0$}
        \alakohta{$7x^2-8x\leq 0$}
        \alakohta{$11x^2-12\leq 0$}
    \end{alakohdat}
    \begin{vastaus}
        \begin{alakohdat}
            \alakohta{$x \geq 0$ tai $x \leq -3,25$}
            \alakohta{$x \geq \sqrt{194}$ tai $x \leq -\sqrt{194}$}
            \alakohta{$0,875 \geq x \geq 0$}
            \alakohta{$2\sqrt{\frac{3}{11}} \geq x \geq -2\sqrt{\frac{3}{11}}$}
        \end{alakohdat}
    \end{vastaus}
\end{tehtava}


%\begin{tehtava}
%  \begin{enumerate}[a)]
%    \item Ratkaise funktion $2x^2 - 5x - 3$ nollakohdat
%    \item Millä muuttujan arvoilla edellisen kohdan funktio $2x^2 - 5x - 3$ saa positiivisia arvoja?
%    \item Onko em. funktiolla globaali ääriarvo (minimi tai maksimi), ja jos on, missä kohtaa funktio saa tämän arvon? Mikä on funktion arvo silloin?
%  \end{enumerate}
%
%  \begin{vastaus}
%    \begin{enumerate}[a)]
%      \item $x = 1.2$ tai $x = -0.2$
%      \item Kun $x<-0,2$ tai $x>1,2$
%      \item Koska neliötermin kerroin a on positiivinen (2), funktiolla on globaali minimi (mutta ei ylärajaa). Symmetrian vuoksi minimi on nollakohtien puolivälissä kohdassa 0.5, jossa funktio saa siis pienimmän arvonsa -5.
%    \end{enumerate}
%  \end{vastaus}
%\end{tehtava}


\subsubsection*{Hallitse kokonaisuus}

\begin{tehtava}
  Tutki, millä muuttujan x arvoilla seuraavat funktiot saavat positiivisia arvoja.
  \begin{alakohdat}
    \alakohta{$f(x)=x^2 - 4$}
    \alakohta{$g(x)=-x^2 - 2x + 3$}
    \alakohta{$h(x)=x^2 + 2x + 5$}
    \alakohta{$i(x)=-x^2 - 1$}
  \end{alakohdat}

  \begin{vastaus}
    \begin{alakohdat}
      \alakohta{$x \leq -2$ tai $x \geq 2$}
      \alakohta{$-3 \geq x \leq 1$}
      \alakohta{Kaikilla $x$:n arvoilla.}
      \alakohta{Ei millään $x$:n arvoilla.}
    \end{alakohdat}
  \end{vastaus}
\end{tehtava}

\begin{tehtava}
(K93/T3b) Autoilijan työmatkan kesto $t$ riippuu liikennevirrasta $m$ kaavan $t=0,01m^2+0,03m+18$ mukaisesti, missä $t$ on ajoaika minuutteina ja $m$ liikenteen mittauspisteen minuutissa ohittavien autojen määrä. Kuinka suuri saa liikennevirta enintään olla, jotta autoilijan työmatka kestäisi enintään puoli tuntia?
\begin{vastaus}
        $33$ autoa/minuutti
    \end{vastaus}
\end{tehtava}

\begin{tehtava}
	Osoita, että funktio $f(x)=x^2-6x+10 $ saa vain positiivisia arvoja.
    \begin{vastaus}
	Funktiolla ei ole nollakohtia ($D<0$) ja sen kuvaaja on ylöspäin aukeava paraabeli.
    Helpompi tapa: $f(x)= (x-3)^2+1 >0.$
    \end{vastaus}
\end{tehtava}

\begin{tehtava}
	Millä vakion $t$ arvoilla yhtälöllä \\ $6x^2+2tx+2t=0$ on kaksi juurta?
	\begin{vastaus}
		Juuria on kaksi, kun $D=4t^2-48t>0$. Tämä toteutuu, kun $t < 0$ tai $t > 12$.
	\end{vastaus}
\end{tehtava}

\begin{tehtava}
Jäätelökioskin päivittäiset kiinteät kulut ovat $400$ euroa. Jokainen jäätelö maksaa kauppiaalle $0,50$ euroa. Kun jäätelön myyntihinta on $x$ euroa, sitä myydään $1\,000 - 200x$ kappaletta. 
\begin{alakohdat}
\alakohta{Millä myyntihinnoilla jäätelön myynti on kannattavaa?}
\alakohta{Millä myyntihinnalla saadaan suurin tuotto? Kuinka suuri?}
\end{alakohdat}
    \begin{vastaus}
		\begin{alakohdat}
		\alakohta{$1,00$ \euro \ $<$ myyntihinta $<$ $4,5$ \euro.}
		\alakohta{$2,75$ \euro, jolloin voitto on $612,50$ \euro. } 
		% Epäilyttävän hyvä bisnes
		\end{alakohdat}
    \end{vastaus}
\end{tehtava}

\begin{tehtava}
    Ratkaise epäyhtälö $ax^2+(a+1)x+1 > 0$ vapaan parametrin $a$ funktiona.
    \begin{vastaus}
        $a < 0$: $-\frac{1}{a} > x > -1$ \\ $a = 0$: $x > -1$ \\ $1 > a > 0$: $x > -1$ tai $x < -\frac{1}{a}$ \\ $a = 1$: $x \neq -1$ \\ $a > 1$: $x \in \rr$
    \end{vastaus}
\end{tehtava}

% $ax^2+x+5 < 0$
% $x^2+x+a \leq 0$

\begin{tehtava}
    Ratkaise seuraava epäyhtälö:
    $$p+(3-(2p-6))^2<\dfrac{2+9p}{3}(22+2)$$
    \begin{vastaus}
        $$p<\dfrac{-167-\sqrt{205\,705}}{372} \qquad\text{tai}\qquad p>\dfrac{-167+\sqrt{205\,705}}{372}$$
    \end{vastaus}
\end{tehtava}

\subsubsection*{Lisää tehtäviä}

\begin{tehtava}
Ilmaan heitetyn pallon korkeus metreinä on $h=20t-5t^2$, missä $t$ on aika sekunteina. Kuinka kauan pallo on yli $10$ metrin korkeudella maasta?
    \begin{vastaus}
	$2,8$\,s
    \end{vastaus}
\end{tehtava}

\begin{tehtava}
	Millä vakion $r$ arvoilla yhtälöllä $rx^2-rx-1 = 0$ ei ole ratkaisua?
	\begin{vastaus}
		Pitää olla $D=(-r)^2-4 \cdot r \cdot (-1)=r^2+4r<0$. Siis $-4 < r < 0$.
	\end{vastaus}
\end{tehtava}

\begin{tehtava}
    Ratkaise epäyhtälö $a^2x^2+ax-2 \geq 0$ vapaan parametrin $a$ funktiona.
    \begin{vastaus}
        $a < 0$: $x \geq -\frac{2}{a}$ tai $x \leq \frac{1}{a}$ \\ $a = 0$: ei ratkaisuja \\ $a > 0$: $x \geq \frac{1}{a}$ tai $x \leq -\frac{2}{a}$
    \end{vastaus}
\end{tehtava}

\end{tehtavasivu}

	\section{Polynomien jakolause}

\qrlinkki{http://opetus.tv/maa/maa2/polynomi-tekijoihin-nollakohtien-avulla/}{Opetus.tv: \emph{polynomin jakaminen tekijöihin nollakohtiensa avulla} (6:48)}

Polynomin $P(x)=(x-2)(x-3)$ nollakohdat ovat tulon nollasäännön nojalla $x=2$ ja $x=3$. Nollakohdat voi siis nähdä tekijöihin jaetusta polynomista suoraan. Tämä  yhteys toimii myös toisin päin: nollakohtien avulla voi selvittää polynomin tekijät.

Yleisesti on voimassa polynomien jakolause:

\laatikko{\textbf{Polynomien jakolause} \\
Jos $x=b$ on polynomin nollakohta, $x-b$ on polynomin tekijä.}

Polynomien jakolauseen todistus jätetään kurssiin 12.
%Polynomien jakolauseen todistus on hahmoteltu liitteessä
%\ref{tod:poljako}.

\begin{esimerkki}
Jaa tekijöihin polynomi $P(x)=x^2-3x+2$.
\begin{esimratk}
Ratkaistaan ensin polynomin nollakohdat.
\begin{align*}
x^2-3x+2&=0 \\
x&=\frac{-(-3) \pm \sqrt[]{(-3)^2-4 \cdot 1 \cdot 2}}{2 \cdot 1} \\
x&=\frac{3 \pm \sqrt[]{9-8}}{2} \\
x&=\frac{3 \pm 1}{2} \\
x&=1 \textrm{ tai } x = 2.
\end{align*}
Nollakohtien perusteella $(x-1)$ ja $(x-2)$ ovat polynomin $P(x)$ tekijöitä.
Tarkistetaan:
 $(x-1)(x-2)=x^2-2x-x+2= x^2-3x+2$.
\end{esimratk}
\begin{esimvast}
$x^2-3x+2 = (x-1)(x-2)$.
\end{esimvast}
\end{esimerkki}

Ensimmäisen asteen tekijöiden lisäksi saatetaan tarvita vakiokerroin:

\begin{esimerkki}
Jaa polynomi $-2x^2-x+1$ tekijöihinsä.
\begin{esimratk}
Ratkaistaan nollakohdat:
\begin{align*}
-2x^2-x+1&=0 \\
x&=\frac{-(-1) \pm \sqrt[]{(-1)^2-4 \cdot (-2) \cdot 1}}{2 \cdot (-2)} \\
x&=\frac{1 \pm \sqrt[]{1+8}}{-4} \\
x&=\frac{1 \pm 3}{-4} \\
x&=-1 \textrm{ tai } x = \frac{1}{2}.
\end{align*}
Jakolauseen mukaan $(x-\frac{1}{2})$ ja $(x-(-1))$ eli$(x+1)$ ovat kyseisen polynomin tekijöitä.
Ne keskenään kertomalla ei kuitenkaan saada oikeaa tulosta:
$$\left(x-\frac{1}{2}\right)(x+1)=x^2+\frac{1}{2}x-\frac{1}{2}.$$
Puuttuu vielä korkeimman asteen termin kerroin $-2$. Sillä kertomalla saadaan alkuperäinen polynomi:
$-2\left(x-\frac{1}{2}\right)(x+1)=-2x^2-x+1$.
\end{esimratk}
\begin{esimvast}
$-2x^2-x+1 = -2(x-\frac{1}{2})(x+1)$.
\end{esimvast}
\end{esimerkki}

\subsubsection*{Toisen asteen polynomin tekijät}

Polynomien jakolauseen mukaan

\laatikko{Jos toisen asteen polynomin $ax^2+bx+c$ nollakohdat ovat $x_1$ ja $x_2$,
\[ ax^2+bx+c=a(x-x_1)(x-x_2). \]}
Huomaa, että kerroin $a$ on edellisessä yhtälössä kummallakin puolella sama.
(Muuten korkeimman asteen termit eivät täsmää.)

\begin{esimerkki}
Jaetaan tekijöihin $P(x)=2x^2 + 4x-30$. \\
Ratkaistaan nollakohdat yhtälöstä $$2x^2 + 4x-30=0$$ toisen asteen yhtälön ratkaisukaavalla.
Nollakohdat ovat $x_1=3$ ja $x_2=-5$. Saadaan siis
$$P(x)= 2(x-3)(x-(-5)) = 2(x-3)(x+5).$$
\end{esimerkki}

\begin{esimerkki}
Toisen asteen polynomin $P$ nollakohdat ovat $x=1$ ja $x=3$. Lisäksi $P(2)=3$.
Määritä $P(x)$.
\begin{esimratk}
Koska polynomin nollakohdat ovat $x_1=1$ ja $x_2=3$, polynomi on muotoa
\begin{align*} P(x)=a(x-1)(x-3). \end{align*}
Lisäksi tiedetään $P(2)=3$, joten saadaan
\begin{align*}
P(2) = a(2-1)(2-3) &= 3 \\
	a \cdot 1 \cdot (-1) &= 3	&& \ppalkki : (-1) \\
	a = -3.
\end{align*}
\end{esimratk}
\begin{esimvast}
$P(x)=-3(x-1)(x-3)$.
\end{esimvast}
\end{esimerkki}

Jos toisen asteen polynomilla on vain yksi nollakohta, kyseessä on niin sanottu kaksinkertainen juuri. Voidaan tulkita, että nollakohdat $x_1$ ja $x_2$ ovat yhtäsuuret. Tällöin tekijöiksi saadaan $a(x-x_1)(x-x_1)=a(x-x_1)^2$.

\begin{esimerkki}
Jaa tekijöihin $P(x)=2x^2-4x+2$.
\begin{esimratk}
Ratkaistaan nollakohdat:
\begin{align*}
2x^2-4x+2 &= 0	\\
x &= \frac{4\pm \sqrt{(-4)^2-4\cdot 2 \cdot 2}}{2\cdot 2} \\
x &= \frac{4 \pm 0}{4} = 1.
\end{align*}
Yhtälöllä on vain yksi ratkaisu, joten se on kaksoisjuuri.
Polynomi voidaan siis jakaa tekijöihin seuraavasti: \\ $P(x)=2(x-1)(x-1)=2(x-1)^2$. 
\end{esimratk}
\begin{esimvast}
$P(x)=2(x-1)^2$.
\end{esimvast}
\end{esimerkki}


Jos toisen asteen polynomilla ei ole nollakohtia, sitä ei voi jakaa ensimmäisen asteen tekijöihin. (Sillä ensimmäisen asteen tekijällä on aina nollakohta.) Esimerkiksi
polynomia $$x^2+4$$ ei voi jakaa tekijöihin.


\begin{tehtavasivu}

\paragraph*{Opi perusteet}

\begin{tehtava}
	Ratkaise yhtälöt, ja jaa vastaavat polynomifunktiot (joiden nollakohdat olet juuri laskenut) tekijöihinsä.
	\begin{alakohdat}
		\alakohta{$x^2+6x-6=0$}
		\alakohta{$-8x^2+10x-2=0$}
		\alakohta{$x^2-4x+4=0$}
	\end{alakohdat}
	\begin{vastaus}
		\begin{alakohdat}
			\alakohta{nollakohdat $x=1$ ja $x=-6$, $P(x)=(x-1)(x+6)$}
			\alakohta{nollakohdat $x= \frac {1}{4}$ ja $x=1$, $P(x)=-2(4x-1)(x-1)$}
				\alakohta{nollakohta $x=2$, $P(x)=(x-2)^2$}
		\end{alakohdat}
	\end{vastaus}
\end{tehtava}

\begin{tehtava}
Jaa tekijöihin
\begin{alakohdat}
\alakohta{$x^2-2x-15$}
\alakohta{$\frac{1}{3}x^2-2x+3$}
\alakohta{$4x^2+2x+2$}
\end{alakohdat}
\begin{vastaus}
\begin{alakohdat}
\alakohta{$(x-5)(x+3)$}
\alakohta{$\frac{1}{3}(x-3)^2$}
\alakohta{$2(2x^2+x+1)$, ei jakaudu 1. asteen tekijöihin.}
\end{alakohdat}
\end{vastaus}
\end{tehtava}

\begin{tehtava}
    Millä seuraavista polynomeista on yhteisiä tekijöitä?

    \begin{kuvaajapohja}{1.5}{-1.5}{2.5}{-3.5}{1.5}
	\kuvaaja{(x-1)*(x+1)}{$P(x)$}{red}
	\kuvaaja{2*(x-2)*(x+0.5)}{$Q(x)$}{blue}
	\kuvaaja{0.25*(x-2)*(x-1)*(x+1)}{$R(x)$}{black}
	\kuvaaja{-(x-0.25)*(x-1.5)*(x+0.75)}{$S(x)$}{black}
    \end{kuvaajapohja}
    \begin{vastaus}
	$P(x)$:llä ja $R(x)$:llä on kaksi yhteistä tekijää, koska on kaksi kohtaa, jossa molemmat saavat arvon nolla. Vastaavasti $Q(x)$:llä ja $R(x)$:llä
	on yksi yhteinen tekijä. $S(x)$:llä ei ole yhteisiä tekijöitä minkään muun polynomin kanssa.
    \end{vastaus}
\end{tehtava}


\paragraph*{Hallitse kokonaisuus}

\begin{tehtava}
    Jaa tekijöihin helpoimmalla tavalla
    \begin{alakohdat}
        \alakohta{$x^2-9$}
        \alakohta{$4x^2-4x+1$}
        \alakohta{$4x^2-4x-8$}
    \end{alakohdat}
    \begin{vastaus}
    	\begin{alakohdat}
        \alakohta{$(x+3)(x-3)$ (muistikaava)}
        \alakohta{$(2x-1)^2$ (muistikaava)}
        \alakohta{$4(x-2)(x+1)$}
        \end{alakohdat}
    \end{vastaus}
\end{tehtava}

\begin{tehtava}
    Toisen asteen polynomille $P$ pätee \\ $P(-3)=P(4)=0$ ja $P(1)=12$. Ratkaise $P(x)$.
    \begin{vastaus}
        $P(x)=-(x+3)(x-4)=-x^2+x+12$
    \end{vastaus}
\end{tehtava}

\begin{tehtava}
    Osoita, että jos toisen asteen polynomin toisen asteen termin kerroin on 1, niin sen vakiotermi on yhtä suuri kuin sen nollakohtien tulo.
    \begin{vastaus}
        Kirjoitetaan polynomi tekijämuodossa ja kerrotaan auki: $(x-a)(x-b)=x^2-2(a+b)x+ab$. Nyt syntyneen polynomin vakiotermi on $ab$.
    \end{vastaus}
\end{tehtava}


\begin{tehtava}
    Toisen asteen polynomille $P$ pätee $P(0)=P(5)=3$ ja $P(3)=-27$. Ratkaise $P(x)$.
    \begin{vastaus}
        $P(x)=5x(x-5)+3=5x^2-25x+3$
    \end{vastaus}
\end{tehtava}


\begin{tehtava}
    Paraabelin kuvaajia katsomalla voidaan huomata, että paraabelin huippu löytyy aina nollakohtien puolivälistä. (Tarkempi perustelu saadaan esimerkiksi kurssilla 4.)
    Huipun $x$-koordinaatti on siis nollakohtien $x$-koordinaattien keskiarvo.
    \begin{alakohdat}
        \alakohta{Etsi paraabelin $-10x^2+5x+5$ huipun $x$- ja $y$-koordinaatit.}
        \alakohta{Johda lauseke yleisen paraabelin $ax^2+bx+c$ huipun $x$-koordinaatille.}
        % myös y-koordinaattia voisi kysyä
      
    \end{alakohdat}
    \begin{vastaus}
        \begin{alakohdat}
            \alakohta{$x=\frac14$ ja $y=5\frac58$.}
            \alakohta{$x=-\frac{b}{2a}$} % ja $y=c-\frac{b^2}{4a}$
        \end{alakohdat}
    \end{vastaus}
\end{tehtava}

\begin{tehtava}
    $\star $ Tutki, onko seuraava väite totta vai ei: Jos polynomilla on
    nollakohta $x=1$,
	sen kerrointen summa on $0$. 
    \begin{vastaus}
        Totta, sillä polynomin $P(x)$ kerrointen summa on $P(1)$. (Koska $1^n=1$.)
    \end{vastaus}
\end{tehtava}

\paragraph*{Lisää tehtäviä}

\begin{tehtava}
    Jaa tekijöihin.
    \begin{alakohdat}
        \alakohta{$4x^2 +4x +1$}
        \alakohta{$4x^2 +4x +4$}
        \alakohta{$9-x^2$}
    \end{alakohdat}
    \begin{vastaus}
        \begin{alakohdat}
        \alakohta{$(2x+1)^2$}
        \alakohta{$4(x^2 +x +1)$}
        \alakohta{$(3+x)(3-x)$}
        \end{alakohdat}
    \end{vastaus}
\end{tehtava}

\begin{tehtava}
	Jaa tekijöihin.
	\begin{alakohdat}
		\alakohta{$x^2-x-6$} % \{[ \star ]\}
		\alakohta{$(x-4)^2-9$} %  \{[ \star ]\}
	\end{alakohdat}
	\begin{vastaus}
		\begin{alakohdat}
			\alakohta{$(x-3)(x+2)$}
			\alakohta{$(x-1)(x-7)$}
		\end{alakohdat}
	\end{vastaus}
\end{tehtava}

\begin{tehtava}
Jaa tekijöihin
\begin{alakohdat}
\alakohta{$5x^2+10x-15$}
\alakohta{$12x^2-10x+2$}
\alakohta{$-8x^2+8x+6$}
\end{alakohdat}
\begin{vastaus}
\begin{alakohdat}
\alakohta{$5(x+3)(x-1)$}
\alakohta{$2(2x-1)(3x-1)$}
\alakohta{$-2(2x-3)(2x+1)$}
\end{alakohdat}
\end{vastaus}
\end{tehtava}

\begin{tehtava}
    Jaa tekijöihin
    \begin{alakohdat}
        \alakohta{$-10x^2+5x+5$}
        \alakohta{$8x^3-12x^2+4x$}
    \end{alakohdat}
    \begin{vastaus}
    	\begin{alakohdat}
        \alakohta{$-5(2x+1)(x-1)$}
        \alakohta{$(4x)(2x-1)(x-1)$}
        \end{alakohdat}
    \end{vastaus}
\end{tehtava}

\begin{tehtava}
    Kolmannen asteen polynomille $P$ pätee $P(-1)=P(2)=P(3)=0$ ja $P(1)=-8$. Ratkaise $P(x)$.
    \begin{vastaus}
        $P(x)=-2(x+1)(x-2)(x-3)=-2x^3+8x^2-2x-2$
    \end{vastaus}
\end{tehtava}

\begin{tehtava}
   Määritä toisen asteen yhtälö jonka juuret ovat yhtälön $ x^2+7x+49 =0 $ 
 \begin{alakohdat}
    	\alakohta{juurien käänteisluvut}
        \alakohta{juurien neliöiden käänteisluvut}
    \end{alakohdat}
    \begin{vastaus}
        \begin{alakohdat}
            \alakohta{$k(49x^2+7x+1)=0$}
            \alakohta{$k(7x^2 \pm x\sqrt{7} +1)=0$}
        \end{alakohdat}
    \end{vastaus}
\end{tehtava}

\end{tehtavasivu}


\osa{Korkeampi aste}

	\section{Korkeamman asteen polynomifunktio}

\qrlinkki{http://opetus.tv/maa/maa2/n-asteinen-polynomifunktio/}{Opetus.tv: \emph{N-asteinen polynomifunktio} (10:40)}

Kaikki paraabelit ovat samanlaisia, mutta korkeamman asteen polynomien kuvaajat eivät ole. Yleisesti pätee kuitenkin seuraava tulos:

\laatikko[Polynomien ominaisuuksia]{
\begin{itemize}
\item Asteen $n$ polynomilla on korkeintaan $n$ nollakohtaa.
\item Jos polynomin aste on pariton, sillä on vähintään yksi nollakohta.
% ??
%\item Kaikki polynomit voidaan jakaa tekijöihin, jotka ovat korkeintaan toista astetta. 
\end{itemize}}

Todistetaan näistä ensimmäinen tulos:

\begin{todistus}
Jos $n$ asteen polynomilla $P$ olisi yli $n$ eri nollakohtaa, sillä olisi yli $n$ ensimmäisen asteen tekijää. Polynomin $P$ aste olisi siis yli $n$, mikä on ristiriita. Nollakohtia on siis korkeintaan $n$.
\end{todistus}

Toista tulosta ei todisteta tässä täsmällisesti, mutta valoitetaan asiaa esimerkin kautta. Tutkitaan esimerkiksi polynomia 
$$P(x)=x^3+bx^2+cx+d.$$
Polynomia on kätevintä tarkastella muodossa, jossa $x^3$ on otettu yhteiseksi tekijäksi:
$$P(x) = x^3\left(1+\frac{a}{x}+\frac{b}{x^2}+\frac{d}{x^3}\right)$$
Kun $x$ on hyvin suuri positiivinen tai hyvin pieni negatiivinen luku,
termit $\frac{b}{x}$, $\frac{c}{x^2}$ ja $\frac{d}{x^3}$ ovat hyvin pieniä, eli
\begin{align*}
P(x)&= x^3\left(1+\frac{b}{x}+\frac{c}{x}+\frac{d}{x^3}\right) \\
	& \approx  x^3\left(1+0+0+0\right) = x^3
\end{align*}
Voidaan siis päätellä, että riippumatta kertoimista $b$, $c$, $d$ polynomin $P$ arvo on positiivinen, kun $x$ on suuri positiivinen luku ja negatiivinen, kun $x$ on pieni negatiivinen luku.

Koska $P$ saa sekä positiivisia että negatiivia arvoja, sillä on jossakin niiden välissä nollakohta. (Tämän takaa jatkuvuus, josta lisää kurssilla 7.) Esimerkin mukaisilla kolmannen asteen polynomeilla on siis aina nollakohta. Yleisesti pätee, että kaikilla paritonasteisille polynomeilla on ainakin yksi nollakohta.

\begin{esimerkki} Kolmannen asteen polynomilla on 1--3 nollakohtaa.

\begin{lukusuora}{-2.5}{3}{3.6}
\lukusuoraisobbox
\lukusuorakuvaaja{(x**3-x-1)/2}
\lukusuorapienipiste{1.32472}{}
\end{lukusuora}
\begin{lukusuora}{-2.8}{2.5}{3.6}
\lukusuoraisobbox
\lukusuorakuvaaja{(x**3-x+0.3849)/2}
\lukusuorapienipiste{-1.1547}{}
\lukusuorapienipiste{0.577028}{}
\end{lukusuora}
\begin{lukusuora}{-2}{2}{3.6}
\lukusuoraisobbox
\lukusuorakuvaaja{1.4*(x**3-x)}
\lukusuorapienipiste{-1}{}
\lukusuorapienipiste{0}{}
\lukusuorapienipiste{1}{}
\end{lukusuora}
\end{esimerkki}


\begin{esimerkki} Neljännen asteen polynomilla on $0--4$ nollakohtaa.
\begin{lukusuora}{-4}{4}{3.6}
\lukusuorabboxy{-0.5}{1.5}
\lukusuorakuvaaja{(x**4-5*x**2+12)/14}
\end{lukusuora}
\begin{lukusuora}{-4}{4}{3.6}
\lukusuorabboxy{-0.5}{1.5}
\lukusuorakuvaaja{(x**4-5*x**2+3*x+11.2)/14}
\lukusuorapienipiste{-1.71394}{}
\end{lukusuora}
\begin{lukusuora}{-4}{4}{3.6}
\lukusuorabboxy{-0.5}{1.5}
\lukusuorakuvaaja{(x**4-5*x**2-3)/14}
\lukusuorapienipiste{2.354}{}
\lukusuorapienipiste{-2.354}{}
\end{lukusuora}

\begin{lukusuora}{-4}{4}{3.6}
\lukusuorabboxy{-0.5}{1.5}
\lukusuorakuvaaja{(x**4-5*x**2+3*x+1.75842)/14}
\lukusuorapienipiste{-2.43622}{}
\lukusuorapienipiste{-0.367327}{}
\lukusuorapienipiste{1.402}{}
\end{lukusuora}
\begin{lukusuora}{-2.8}{2.8}{3.6}
\lukusuorabboxy{-0.5}{1.5}
\lukusuorakuvaaja{0.6*(x+1.5)*(x+0.5)*(x-0.5)*(x-1.5)}
\lukusuorapienipiste{1.5}{}
\lukusuorapienipiste{0.5}{}
\lukusuorapienipiste{-1.5}{}
\lukusuorapienipiste{-0.5}{}
\end{lukusuora}
\end{esimerkki}

\begin{tehtavasivu}

\subsubsection*{Opi perusteet}

\begin{tehtava}
    Anna esimerkki
    \begin{alakohdat}
	\alakohta{neljännen asteen polynomista, jolla on neljä nollakohtaa}
	\alakohta{kolmannen asteen polynomista, jolla on kaksi nollakohtaa}
	\alakohta{neljännen asteen polynomista, jolla on yksi nollakohta}
	\alakohta{neljänen asteen polynomista, jolla ei ole nollakohtia}
\end{alakohdat}
    \begin{vastaus}
	Esimerkiksi    
    \begin{alakohdat}
	\alakohta{$P(x)=x(x-1)(x-2)(x-3)$ (nollakohdat $x= 0$, $x = 1$, $x=2$ ja $x=3$)}
	\alakohta{$P(x)=x^2(x-1)$ (nollakohdat $x= 0$ ja $x = 1$)}
	\alakohta{$Q(x)=x^4$ (nollakohta $x=0$)}
	\alakohta{$R(x)=x^4+1$}
\end{alakohdat}
    \end{vastaus}
\end{tehtava}

\begin{tehtava}
Alla on polynomin $P(x)$ kuvaaja. \\
\begin{kuvaajapohja}{1}{-3}{3}{-2}{5}
  \kuvaaja{-x**5+3*x**3+2}{$P(x)$}{black}
\end{kuvaajapohja} \\
Kaikki polynomin nollakohdat näkyvät kuvaajassa.
\begin{alakohdat}
\alakohta{Mitä voidaan sanoa polynomin $P$ asteesta?}
\alakohta{Mikä on polynomin $P$ vakiotermi?}
\end{alakohdat}
\begin{vastaus}
\begin{alakohdat}
\alakohta{Nollakohtia on kolme, joten polynomin aste on vähintään $3$. (Itse asiassa todellinen aste on $5$, mutta sitä on vaikea päätellä silmämääräisesti kuvaajasta.)}
\alakohta{Vakiotermi on 2, koska $P(0)=2$}
\end{alakohdat}
\end{vastaus}
\end{tehtava}

\subsubsection*{Hallitse kokonaisuus}

\begin{tehtava}
    Mikä on se kolmannen asteen polynomi, jonka nollakohdat ovat $x=2$, $x=-1$ ja $x=3$, ja jonka vakiotermi on $3$?
    \begin{vastaus}
        $P(x)=\frac{1}{2}(x-2)(x+1)(x-3)$
    \end{vastaus}
\end{tehtava}

\begin{tehtava}
    Kolmannen asteen polynomille $P$ pätee $P(1)=P(2)=P(3)=-2$ ja $P(0)=16$. Ratkaise $P(x)$.
    \begin{vastaus}
        $P(x)=-3(x-1)(x-2)(x-3)-2=-3x^3+18x^2-33x+18$
    \end{vastaus}
\end{tehtava}

\begin{tehtava}
   	$\star$ Osoita, että jos $n$ asteen polynomeilla $P(x)$ ja $Q(x)$ on $n+1$ yhteistä pistettä, ne ovat sama polynomi.
    \begin{vastaus}
        Tarkastele polynomien erotuksen nollakohtia.
    \end{vastaus}
\end{tehtava}

\end{tehtavasivu}

	\section{Korkeamman asteen yhtälö}

\qrlinkki{http://opetus.tv/maa/maa2/n-asteinen-polynomiyhtalo/}{Opetus.tv: \emph{N-asteinen polynomiyhtälö} (8:38, 6:20 ja 15:53)}

Jos toisen asteen polynomiyhtälöllä on reaalilukuratkaisuja, ne löytyvät ratkaisukaavalla. Myös kolmannen ja neljännen asteen yhtälöille on olemassa ratkaisukaavat. Ne ovat kuitenkin niin monimutkaisia, että ne eivät juuri sovellu käsin laskettaviksi, eikä niitä siksi esitellä tässä. Korkeamman kuin neljännen asteen yhtälöille sen sijaan ei edes ole olemassa yleistä ratkaisukaavaa. Tämän osoitti norjalainen matemaatikko Niels Henrik Abel vuonna 1823.

Korkeamman kuin toisen asteen yhtälöt ratkaistaan käytännössä yleensä tietokoneen avulla. Niille yhtälöille, joille ei ole ratkaisukaavaa, ratkaiseminen onnistuu vain numeerisesti eli likiarvoja käyttäen. Joissain erikoistapauksissa korkeamman asteen yhtälön ratkaiseminen onnistuu nopeasti myös käsin. Seuraavassa tarkastellaan tällaisia yksinkertaisia erikoistapauksia.

\subsection*{Potenssiyhtälö}

Jos yhtälössä esiintyy vain yhtä muuttujan $x$ potenssia, yhtälö ratkeaa lopulta potenssi vastaavan juuren avulla.

\begin{esimerkki}
Ratkaise yhtälö $x^5+7=0$.

	\begin{esimratk}
\begin{align*}
x^5+7 &= 0 && \ppalkki -7 \\
x^5 &= -7 && \ppalkki \sqrt[5]{ \ } \\
x &= \sqrt[5]{-7} \\
x &= -\sqrt[5]{7}
\end{align*} 
	\end{esimratk}

	\begin{esimvast} $x = -\sqrt[5]{7}$.
	\end{esimvast}
\end{esimerkki}

\subsection*{Tekijöihinjako}

Polynomiyhtälöitä voidaan toisinaan ratkaista jakamalla polynomi tekijöihin.

%Jos polynomiyhtälössä $P(x) = 0$ polynomi $P(x)$ voidaan jakaa tekijöihin, ratkaisu saadaan etsimällä näiden tekijöiden nollakohdat.
%Esimerkiksi vakiotermittömässä polynomissa voidaan ottaa muuttuja yhteiseksi tekijäksi ja riittää ratkaista yhtä pienemmän asteen %polynomiyhtälö.

\begin{esimerkki}
Ratkaise yhtälö $x^3 - 3x^2 + x = 0$.

\begin{esimratk}
Polynomissa $x^3 - 3x^2 + x$ ei ole vakiotermiä. Voidaan siis ottaa yhteiseksi tekijäksi $x$, jolloin polynomi tulee muotoon $x(x^2 - 3x + 1)$. 

Nyt yhtälön ratkaiseminen voidaan aloittaa soveltamalla tulon nollasääntöä:
\begin{align*}
x^3 - 3x^2 + x & =0 \\
x(x^2 - 3x + 1) & =0 \\
x=0 \quad & \text{tai} \quad x^2 - 3x + 1 = 0 \\
\end{align*}

Jäljelle jäävään toisen asteen yhtälöön $x^2 - 3x + 1 = 0$ voidaan käyttää ratkaisukaavaa:
\[
x =\frac{3\pm\sqrt{3^2-4\cdot 1\cdot 1}}{2\cdot 1}=\frac{3\pm \sqrt{5}}{2}.
\]
\end{esimratk}

\begin{esimvast} $x=0$ tai $x=\dfrac{3\pm \sqrt{5}}{2}$
\end{esimvast}
\end{esimerkki}

Jos polynomi siis osataan jakaa tekijöihin, yhtälön voi ratkaista tulon nollasäännön avulla.
Sopiva tekijöihinjako voi kuitenkin olla vaikea löytää. Tarkastellaan vielä toista esimerkkiä.
%Aiemmin esitellyn polynomien jakolauseen mukaan kaikki polynomit voidaan jakaa tekijöihin, jotka ovat korkeintaan toista astetta.
%Periaatteessa tällä tavalla voidaan siis ratkaista kaikki polynomiyhtälöt. Tekijöihin jakaminen on kuitenkin yleensä vaikeaa.

\begin{esimerkki}
Ratkaise yhtälö $x^3-17x^2-x+17 = 0$.

\begin{esimratk}
Yhtälön vasemmalla puolella olevan polynomin voi jakaa tekijöihin ryhmittelemällä:

\begin{align*}
x^3-17x^2-x+17=x^2(x-17)+(-1)(x-17)=(x^2-1)(x-17).
\end{align*}

Nyt yhtälö ratkeaa tulon nollasäännöllä:
\begin{align*}
x^3-17x^2&-x+17=0 \\
(x^2-1)&(x-17)=0 \\
x^2-1 = 0 \quad &\text{tai} \quad x - 17 = 0 \\
x^2 = 1 \quad &\text{tai} \quad x = 17 \\
x =\pm 1 \quad &\text{tai} \quad x = 17 \\
\end{align*}
\end{esimratk}

\begin{esimvast}
$x = 17$, $x = 1$ tai $x=-1$
\end{esimvast}
\end{esimerkki}

\subsection*{Sijoitukset}

Joskus yhtälöt ratkeavat, kun niihin sijoitetaan jokin apumuuttuja. Tällöin puhutaan muuttujanvaihdosta.

%Esimerkiksi muotoa $ax^4+bx^2+c=0$ olevissa yhtälöissä huomataan, että merkitsemällä lauseketta $x^2$ kirjaimella $y$, yhtälö voidaan kirjoittaa muotoon $ay^2+by+c=0$. Uudesta yhtälöstä voidaan ratkaista $y$ toisen asteen yhtälön ratkaisukaavalla, ja sijoituksesta $y = x^2$ voidaan ratkaista $x$.

\begin{esimerkki}
Ratkaise yhtälö $2x^4+14x^2-36=0$.

\begin{esimratk}
Koska $x^4=(x^2)^2$, ratkaistava yhtälö voidaan kirjoittaa myös muodossa $2(x^2)^2+14x^2-36=0$. Kun nyt sijoitetaan lausekkeen $x^2$ paikalle $y$, eli merkitään $y=x^2$, saadaankin muuttujan $y$ yhtälö
\[
2y^2+14y-36=0.
\]
Tämä on toisen asteen yhtälö, joka osataan ratkaista esimerkiksi ratkaisukaavalla. Tällä tavoin saadaan
\[
y=\frac{-14\pm\sqrt{14^2-4\cdot 2\cdot(-36)}}{2\cdot 2}=\frac{-14\pm 22}{4}.
\]
Ratkaisut ovat siis $y=2$ ja $y=-9$.

On kuitenkin vielä selvitettävä alkuperäisen muuttujan $x$ arvot. Koska $x^2=y$, saadaan $y$:lle löydetyistä arvoista yhtälöt $x^2=2$ ja $x^2=-9$. Reaaliluvun neliö ei kuitenkaan voi olla negatiivinen, joten ainoat ratkaisut ovat yhtälön $x^2 = 2$ ratkaisut. Ne ovat $x=\pm\sqrt{2}$.
\end{esimratk}
	\begin{esimvast}
$x=\sqrt{2}$ tai $x=-\sqrt{2}$
	\end{esimvast}
\end{esimerkki}

Muotoa $ax^4+bx^2+c=0$ oleva yhtälö (eli ns. bikvadraattinen yhtälö) voidaan aina ratkaista sijoittamalla $y=x^2$. Yleisemmin muotoa $ax^{2n}+bx^n+c=0$ olevat yhtälöt voidaan ratkaista sijoituksella $y = x^n$.

\begin{esimerkki}
Ratkaise yhtälö $x^{10}+x^5=2$.

\begin{esimratk}
Muutetaan yhtälö muotoon $(x^5)^2+x^5-2=0$ ja tehdään sijoitus $y = x^5$. Nyt yhtälö saa muodon $y^2+y-2 = 0$. Toisen asteen yhtälön ratkaisukaavalla saadaan yhtälön ratkaisuiksi $y = -2$ ja $y = 1$.

Nyt alkuperäisen yhtälön ratkaisut saadaan yhtälöistä $x^5=-2$ ja $x^5=1$. Siten ratkaisut ovat $x = \sqrt[5]{-2}$ ja $x = 1$.
\end{esimratk}
	\begin{esimvast}
$x = \sqrt[5]{-2}$ tai $x = 1$
	\end{esimvast}
\end{esimerkki}

\begin{tehtavasivu}

\paragraph*{Opi perusteet}

\begin{tehtava}
    Ratkaise yhtälöt.
    \begin{alakohdat}
        \alakohta{$x(x+2)(x+17)=0$}
        \alakohta{$x^9  = -512$}
        \alakohta{$x^6 - 7x^7 = 0$}
    \end{alakohdat}
    \begin{vastaus}
        \begin{alakohdat}
            \alakohta{$x = 0$ tai $x=-2$ tai $x=17$}
            \alakohta{$x = - 2$}
            \alakohta{$x = 0$ tai $x=\frac{1}{7}$}
        \end{alakohdat}
    \end{vastaus}
\end{tehtava}

\begin{tehtava}
    Ratkaise yhtälöt.
    \begin{alakohdat}
        \alakohta{$x^3-5x^2+6x=0$}
        \alakohta{$x^4 - 16 = 0$}
        \alakohta{$x^6 - x^4 = 0$}
    \end{alakohdat}
    \begin{vastaus}
        \begin{alakohdat}
            \alakohta{$x = 0$ tai $x=2$ tai $x=3$}
            \alakohta{$x = \pm 2$}
            \alakohta{$x = 0$ tai $x=\pm 1$}
        \end{alakohdat}
    \end{vastaus}
\end{tehtava}

\begin{tehtava}
    Ratkaise yhtälöt.
    \begin{alakohdat}
        \alakohta{$x^4 - 2x^2 - 24 = 0$}
        \alakohta{$x^4 - 4x^2 - 5 = 0$}
        \alakohta{$x^4 - 8x^2 + 15 = 0$}
    \end{alakohdat}
    \begin{vastaus}
        \begin{alakohdat}
            \alakohta{$x = \pm\sqrt{6}$}
            \alakohta{$x = \pm\sqrt{5}$}
            \alakohta{$x = \pm\sqrt{3}$ tai $\pm\sqrt{5}$}
        \end{alakohdat}
    \end{vastaus}
\end{tehtava}

\paragraph*{Hallitse kokonaisuus}

\begin{tehtava}
Ratkaise yhtälöt.
\begin{alakohdat}
\alakohta{$x^5-3x^4+2x^3=0$}
\alakohta{$x^4+5x^3-x^2-5x=0$}
\alakohta{$x^3-4x^2-4x+16=0$}
\end{alakohdat}
\begin{vastaus}
\begin{alakohdat}
\alakohta{$x=0$ tai $x=1$ tai $x=2$}
\alakohta{$x=0$ tai $x=-5$ tai $x= \pm 1$}
\alakohta{$x=4$ tai $x= \pm 2$}
\end{alakohdat}
\end{vastaus}
\end{tehtava}

\begin{tehtava}
    Ratkaise yhtälöt.
    \begin{alakohdat}
        \alakohta{$x^4 - 16 = 0$}
        \alakohta{$2x^4 = 8x^2$}
        \alakohta{$x^6 - 2x^3 = 3$}
        \alakohta{$x^{100} - 2x^{50} + 1 = 0$}
    \end{alakohdat}
    \begin{vastaus}
        \begin{alakohdat}
            \alakohta{$x = \pm2$}
            \alakohta{$x = 0$ tai $x=\pm2$}
            \alakohta{$x = \sqrt[3]{3}$ tai $x= -1$}
            \alakohta{$x = \pm1$}
        \end{alakohdat}
    \end{vastaus}
\end{tehtava}

\begin{tehtava} % Korkeamman asteen yhtälö
Kun kolme peräkkäistä kokonaislukua kerrotaan keskenään, ja tuloon lisätään keskimmäinen luku, tulos on $15$ kertaa keskimmäisen luvun neliö. Mitkä luvut ovat kyseessä?
    \begin{vastaus}
	Luvut ovat $-1, 0$ ja $1$ tai $14, 15$ ja $16$.
    \end{vastaus}
\end{tehtava}

\begin{tehtava}
	Ratkaise yhtälö $x^{627} - 6x^{514} + 5x^{401} = 0$.
	\begin{vastaus}
		$x = 0$, $x = 1$ tai $x = \sqrt[113]{5}$
	\end{vastaus}
\end{tehtava}

\begin{tehtava}
    Ratkaise yhtälöt.
    \begin{alakohdat}
        \alakohta{$x^5 - 2x^3 + x = 0$}
        \alakohta{$x^8 + 4x^4 = 5x^6$}
    \end{alakohdat}
    \begin{vastaus}
        \begin{alakohdat}
        	\alakohta{$x = 0$ tai $x = \pm1$}
        	\alakohta{$x = 0$ tai $x = \pm1$ tai $x = \pm2$}
        \end{alakohdat}
    \end{vastaus}
\end{tehtava}

\begin{tehtava}
 	Ratkaise yhtälö $5^{x^3+4x^2+x}=1$. 
	\begin{vastaus}
	$x=0$ tai $x=-2 + \sqrt[]{3}$ tai $x=-2 - \sqrt[]{3}$
	\end{vastaus}
\end{tehtava}

\begin{tehtava}
	$ \star $ Ratkaise yhtälö $(x+\frac{1}{x})^2-x-\frac{1}{x}-6 = 0$.
	\begin{vastaus}
		$x = -1$, $x = \frac{3\pm \sqrt{5}}{2}$
	\end{vastaus}
\end{tehtava}

\begin{tehtava}
	$ \star $ Ratkaise yhtälö $2^x-1=\frac{12}{2^x}$
	\begin{vastaus}
	$x=2$
	\end{vastaus}
\end{tehtava}

\paragraph*{Lisää tehtäviä}

\begin{tehtava}
    Ratkaise yhtälöt.
    \begin{alakohdat}
        \alakohta{$x^8 - 1 = 0$}
        \alakohta{$x^8 - x^4 = 0$}
        \alakohta{$x^8 - x^4 - 1 = 0$}
    \end{alakohdat}
    \begin{vastaus}
        \begin{alakohdat}
            \alakohta{$x = \pm\sqrt{1}$}
            \alakohta{$x = 0$ tai $x = \pm\sqrt{1}$}
            \alakohta{$x = \pm\sqrt[4]{\frac{1+\sqrt{5}}{2}} (= \pm\sqrt[4]{\upvarphi})$ (luku $\upvarphi$ on kultaisena leikkauksena tunnettu vakio)}
        \end{alakohdat}
    \end{vastaus}
\end{tehtava}

\begin{tehtava}
    Ratkaise yhtälöt.
    \begin{alakohdat}
        \alakohta{$x^4 + 7x^3 = 0$}
        \alakohta{$2x^3 - 16x^2 + 32x = 0$}
        \alakohta{$x^6 + 6x^5 = -9x^4$}
        \alakohta{$x^3 - 2x^5 = 0$}
    \end{alakohdat}
    \begin{vastaus}
        \begin{alakohdat}
        	\alakohta{$x = 0$ tai $x = -7$}
        	\alakohta{$x = 0$ tai $x = 4$}
        	\alakohta{$x = 0$ tai $x = -3$}
            \alakohta{$x = 0$ tai $x = \pm\dfrac{1}{\sqrt{2}}$}
        \end{alakohdat}
    \end{vastaus}
\end{tehtava}

\begin{tehtava}
$\star$ (K94/T2a) Polynomin $P(x)=ax^3-31x^2+1$ eräs nollakohta on $x=1$. Määritä $a$ ja ratkaise tämän jälkeen yhtälö $P(x)=0$.
\begin{vastaus}
      $a=30$. Yhtälön ratkaisut ovat $1$, $\frac{1}{5}$ ja $-\frac{1}{6}$. (Vihje: kirjoita kolmannen asteen polynomi mielivaltaisen, tuntemattoman toisen asteen polynomin ja binomin $x-1$ tulona jaavaa tulosta sulkeet. Vertaa kolmannen asteen polynomin kertoimiin.)
    \end{vastaus}
\end{tehtava}

\end{tehtavasivu}


	\section{Korkeamman asteen epäyhtälö}
\label{kork_ast}

\qrlinkki{http://opetus.tv/maa/maa2/n-asteinen-polynomiepayhtalo/}{Opetus.tv: \emph{N-asteinen polynomiepäyhtälö} (16:19 ja 12:00)}

Korkeamman asteen epäyhtälö, kuten epäyhtälö
\[
x^3 -6x \leq x^2
\]
voinaan ratkaista siirtämällä kaikki termit epäyhtälön toiselle puolelle ja tutkimalla syntyvän polynomin merkkiä:
\begin{align*}
x^3-6x & \leq x^2 & &\ppalkki -x^2 \\
\underbrace{x^3-x^2-6x}_{P(x)} &\leq 0. &&
\end{align*}
Polynomin $P(x)$ merkin selvittämiseksi ratkaistaan sen nollakohdat:
\begin{align*}
    x^3 - x^2-6x &= 0 & &\ppalkki \text{$x$ yhteiseksi tekijäksi} \\
    x(x^2 -x -6) &= 0 & &\ppalkki \text{tulon nollasääntö} \\
    x = 0 \quad \text{tai} \quad & x^2 -x -6 = 0 & &\ppalkki \text{ratkaisukaava} \\
    x= 0 \quad \text{tai} \quad & x=\frac{-(-1) \pm \sqrt{(-1)^2-4\cdot 1 \cdot (-6)}}{2\cdot 1} && \\
    x = -2 \quad \text{tai} \quad & x = 3. &&
\end{align*}
Polynomin $P$ nollakohdat ovat siis $0$, $-2$ ja $3$. Tästä voidaan jatkaa kahdella eri tavalla.

\textbf{Tapa 1: Tekijöihin jako.}

Jaetaan polynomi tekijöihin nollakohtien avulla:
\[
P(x) = x^3 - x^2-6x = x(x^2-x-6) = x(x+2)(x-3).
\]
Tutkitaan kunkin tulon tekijän merkkiä:
\begin{align*}
    x+2>0 & \quad \text{kun} \quad x > -2\\
    x-3>0 & \quad \text{kun} \quad x > 3\\
    x>0 & \quad \text{kun} \quad x > 0.
\end{align*}

Kootaan tulokset \termi{merkkikaavio}{merkkikaavioon}. Merkitään ensin
lukusuoralle polynomin $P$ nollakohdat. Nämä kolme nollakohtaa jakavat
lukusuoran neljään osaan. Taulukoidaan lukusuoran alle kunkin tekijän
merkki kullakin välillä.
\begin{center}
    \begin{merkkikaavio}{3}
        \merkkikaavioKohta{$-2$}
        \merkkikaavioKohta{$0$}
        \merkkikaavioKohta{$3$}

        \merkkikaavioFunktio{$x+2$}
        \merkkikaavioMerkki{$-$}
        \merkkikaavioMerkki{$+$}
        \merkkikaavioMerkki{$+$}
        \merkkikaavioMerkki{$+$}

        \merkkikaavioUusirivi
        \merkkikaavioFunktio{$x-3$}
        \merkkikaavioMerkki{$-$}
        \merkkikaavioMerkki{$-$}
        \merkkikaavioMerkki{$-$}
        \merkkikaavioMerkki{$+$}

        \merkkikaavioUusirivi
        \merkkikaavioFunktio{$x$}
        \merkkikaavioMerkki{$-$}
        \merkkikaavioMerkki{$-$}
        \merkkikaavioMerkki{$+$}
        \merkkikaavioMerkki{$+$}

        \merkkikaavioUusiriviKaksoisviiva
        \merkkikaavioFunktio{$x(x+2)(x-3)$}
        \merkkikaavioMerkki{$-$}
        \merkkikaavioMerkki{$+$}
        \merkkikaavioMerkki{$-$}
        \merkkikaavioMerkki{$+$}
    \end{merkkikaavio}
\end{center}
Merkkikaavion alin rivi saadaan tulon merkkisäännöstä: kolmen negatiivisen luvun tulo on negatiivinen, yhden positiivisen ja kahden negatiivisen tulo positiivinen ja
niin edelleen.
Kaavion viimeiseltä riviltä voidaan nyt lukea vastaus alkuperäiseen kysymykseen: $x^3-x^2-6 \leq 0$, kun $x\leq -2$ tai $0\leq x \leq 3$.

\textbf{Tapa 2: Testipisteet.}

Polynomit ovat jatkuvia funktioita. (Jatkuvuutta käsitellään tarkemmin vasta kurssilla 7.)
Intuitiivisesti jatkuvuudessa on kyse siitä, että funktion kuvaaja on yhtenäinen viiva.
Jatkuvuudesta seuraa, että polynomi ei voi vaihtaa merkkiä kulkematta nollakohdan kautta.
Polynomin merkin nollakohtien välillä saa selville laskemalla funktion
arvon jossakin väliltä otetussa testipisteessä.

Esimerkissä nollakohdat olivat $-2$, $0$ ja $3$. Valitaan niiden välistä ja
ympäriltä testipisteiksi vaikkapa $x=-3$, $x=-1$, $x=1$ ja $x=4$. Tarkistetaan funktion merkki kussakin pisteessä:

\begin{tabular}{c|c|l|c}
Väli & Testipiste & $f(x)=x^3-x^2-6x$ & Funktion merkki \\
\hline
$x < -2$ & $x = -3$ & $(-3)^3 -(-3)^2 - 6(-3) = -18$ & $-$ \\
$-2 <x < 0$ & $x = -1$ & $(-1)^3 -(-1)^2 - 6(-1) =4$ & $+$ \\
$0 <x < 3$ & $x = 1$ & $1^3 -1^2 - 6\cdot 1 =  -6$ & $-$ \\
$3 <x $ & $x = 4$ & $4^3 -4^2 - 6\cdot 4 = 24$ & $+$
\end{tabular}

Vastaukseksi saadaan sama kuin edellä: $x^3-x^2-6 \leq 0$, kun $x\leq -2$ tai $0\leq x \leq 3$.

\begin{esimerkki}
Ratkaise epäyhtälö $x^2-x^4 \leq 0$.
\begin{esimratk}
Jaetaan ensin tekijöihin ja ratkaistaan nollakohdat.
\begin{align*}
x^2+x^4 &=x^2(1-x^2) && \ppalkki \text{ muistikaava }\\
&= x^2(1-x)(1+x) 
\end{align*}
Nollakohdat ovat $x=0$, $x=1$ ja $x=-1$. Tehdään merkkikaavio:
\begin{center}
    \begin{merkkikaavio}{3}
        \merkkikaavioKohta{$-1$}
        \merkkikaavioKohta{$0$}
        \merkkikaavioKohta{$1$}

        \merkkikaavioFunktio{$x^2$}
        \merkkikaavioMerkki{$+$}
        \merkkikaavioMerkki{$+$}
        \merkkikaavioMerkki{$+$}
        \merkkikaavioMerkki{$+$}

        \merkkikaavioUusirivi
        \merkkikaavioFunktio{$1-x$}
        \merkkikaavioMerkki{$+$}
        \merkkikaavioMerkki{$+$}
        \merkkikaavioMerkki{$+$}
        \merkkikaavioMerkki{$-$}

        \merkkikaavioUusirivi
        \merkkikaavioFunktio{$1+x$}
        \merkkikaavioMerkki{$-$}
        \merkkikaavioMerkki{$+$}
        \merkkikaavioMerkki{$+$}
        \merkkikaavioMerkki{$+$}

        \merkkikaavioUusiriviKaksoisviiva
        \merkkikaavioFunktio{$x^2(1-x)(1+x)$}
        \merkkikaavioMerkki{$-$}
        \merkkikaavioMerkki{$+$}
        \merkkikaavioMerkki{$+$}
        \merkkikaavioMerkki{$-$}
    \end{merkkikaavio}
\end{center}
Merkkikaaviosta voidaan lukea, että $x^2(1-x)(1+x) <0$, kun
$x < -1$ tai $x >1$. Lisäksi $x^2(1-x)(1+x)=0$, kun $x=-1$,
$x=0$ tai $x=1$.
\end{esimratk}
\begin{esimvast}
$x \leq -1$, $x=0$ tai $x \geq 1$.
\end{esimvast}
\end{esimerkki}

\begin{tehtavasivu}

\subsubsection*{Opi perusteet}

\begin{tehtava}
    Ratkaise
    \begin{alakohdat}
        \alakohta{$(x-1)(x-2)(x-3) \le 0$}
        \alakohta{$(x-1)(x-2)(x-3) > 0$}
        \alakohta{$-3(x-1)(x-2)(x-3) > 0$.}
    \end{alakohdat}
    \begin{vastaus}
        \begin{alakohdat}
            \alakohta{$x \le 1$ tai $2 \le x \le 3$}
            \alakohta{$1 < x < 2$ tai $x>3$}
            \alakohta{$x < 1$ tai $2<x<3$}
        \end{alakohdat}
    \end{vastaus}
\end{tehtava}

\begin{tehtava}
    Ratkaise $x^3-x^2<0$.
    \begin{vastaus}
        $x<0$ tai $0<x<1$
    \end{vastaus}
\end{tehtava}

\begin{tehtava}
    Ratkaise $x^4 \le 1$.
    \begin{vastaus}
        $-1 \le x \le 1$
    \end{vastaus}
\end{tehtava}

\subsubsection*{Hallitse kokonaisuus}

%neliö epänegatiivinen
\begin{tehtava}
    Ratkaise $(2x^3+4x^2-5x+7)^2 < 0$.
    \begin{vastaus}
        Ei ratkaisuja.
    \end{vastaus}
\end{tehtava}

%bikvadraattinen
\begin{tehtava}
    Ratkaise $x^4-3x^2-18 \le 0$.
    \begin{vastaus}
        $-\sqrt{6}\le x \le \sqrt{6}$
    \end{vastaus}
\end{tehtava}

\begin{tehtava} % Korkeamman asteen epäyhtälö
Olkoon $a > 0$. Millä muuttujan $x$ arvoilla funktion
$P(x)=x^3-ax$ arvot ovat positiivisia?
    \begin{vastaus}
	$P(x)>0$ kun $x > \sqrt{a}$ tai $-\sqrt{a}<x<0$.
    \end{vastaus}
\end{tehtava}

\begin{tehtava}
    Koska tulolla ja osamäärällä on sama merkkisääntö, merkkikaavioita
	voidaan käyttää myös osamääriin. Ratkaise epäyhtälöt
    \begin{alakohdat}
        \alakohta{$\frac{(x+3)(x-2)}{x-5} \le 0$}
        \alakohta{$x \geq \frac{1}{x}$}
    \end{alakohdat}
    \begin{vastaus}
        \begin{alakohdat}
            \alakohta{$x \le -3$ tai $2 \le x < 5$}
            \alakohta{$-1 \leq x < 0$ tai $x \geq 1$}
    	\end{alakohdat}
    \end{vastaus}
\end{tehtava}


%yhteinen tekijä x^3, binomikaava käänteisesti
\begin{tehtava}
    Ratkaise $4x^5+9 x^3 \le 12 x^4$.
    \begin{vastaus}
        $x\le0$ tai $x=\frac{3}{2}$
    \end{vastaus}
\end{tehtava}

\begin{tehtava}
Ratkaise $(x^5-2)(x^8-1) >0$
\begin{vastaus}
$x > \sqrt[5]{2}$ tai $-1<x<1$
\end{vastaus}
\end{tehtava}

\begin{tehtava}
Epäyhtälöiden ratkaisut/todistukset perustuvat usein tietoon, että epänegatiivisten lukujen summa on epänegatiivinen ja nolla jos, ja vain jos kaikki yhteenlaskettavat ovat nollia.

\begin{alakohdat}
\alakohta{Ratkaise $x^6 + x^2+1 > 0$}
\alakohta{Ratkaise $x^{10} + (x-1)^{10} < 0$}
\alakohta{Todista, että kaikilla reaaliluvuilla $x$ ja $y$
\[
(xy-1)^2+(x^2-y^2)^4+(xy-x-y+1)^6 \geq 0
\]
ja että epäyhtälössä vallitsee yhtäsuuruus jos, ja vain jos $x = y = 1$}
\end{alakohdat}

\begin{vastaus}
\begin{alakohdat}
\alakohta{$x \in \rr$}
\alakohta{Epäyhtälöllä ei ole ratkaisuja}
\alakohta{Vinkki: Käytä tehtävänannon havaintoa ja tutki, millä $x$:n ja $y$:n arvoilla summattavat saavat arvon 0}
\end{alakohdat}
\end{vastaus}
\end{tehtava}

\subsubsection*{Lisää tehtäviä}

\begin{tehtava} % Korkeamman asteen epäyhtälö
Ratkaise epäyhtälöt
		\begin{alakohdat}
		\alakohta{$x^3 + 2x^2-15x  > 0$  }
		\alakohta{$x^3-2x^2+x \leq 0$  }
		\end{alakohdat}
    \begin{vastaus}
		\begin{alakohdat}
		\alakohta{$-5<x<0$ tai $3 < x$}
		\alakohta{$x<0$ tai $x = 1$}
		\end{alakohdat}
    \end{vastaus}
\end{tehtava}





% x yhteinen tekijä ja sij. y=x^5
\begin{tehtava}
    Ratkaise $x+2x^6+x^{11}<0$.
    \begin{vastaus}
        $x<-1$ tai $ -1<x<0$
    \end{vastaus}
\end{tehtava}

\end{tehtavasivu}


\nosa{Kertausosio}
   \nluku{LIITE_testaatietosi}{Testaa tietosi!}
   \nluku{LIITE_lisatehtavia}{Kertaustehtäviä}
   \nluku{LIITE_harjoituskokeita}{Harjoituskokeita}

Seuraavissa harjoituskokeissa on kussakin kahdeksan tehtävää, jotka on suunniteltu niin, että jokaisesta (alakohtineen) saa korkeintaan kuusi pistettä. %tulevaisuudessa täydet ratkaisut pisteytysohjeineen %tarkista pisteytykset % puuttuu kuvia vastauksista

\subsection*{Harjoituskoe 1} %prosenttilaskuja pitäisi ympätä...

\begin{tehtava}
	\begin{alakohdat}
	\alakohta{Perustele, onko $t^{\pi}+1$ polynomi. ($1$\,p.)}
	\alakohta{Trinomi kerrotaan toisella trinomilla. Kuinka monta termiä syntyneeseen uuteen polynomiin tulee (ennen mahdollista sieventämistä)? ($0,5$\,p.)}
	\alakohta{Mikä on tulomuodossa esitetyn polynomin $(x^2+1)^{10}(1-x)^2$ aste? ($0,5$\,p.)}
	\alakohta{Mainitse jokin luku, joka kuuluu reaalilukuvälille $]1,966;1,967[$. ($0,5$\,p.)}
	\alakohta{Mikä on korkeimman asteen termin kerroin polynomiyhtälössä $9\,001k^2-\frac{1}{4}k-\sqrt{2}k^3=-k^2+k^3$? ($0,5$\,p.)}
	\alakohta{Polynomi voidaan esittää tulomuodossa $(x-2)x(x+3)$. Mikä on kyseisen polynomin suurin nollakohta? ($1$\,p.)}
	\alakohta{Minkälainen (muoto \& mahdollinen suunta) kuvaaja on polynomifunktiolla $P$, jonka arvot lasketaan yhtälöllä $P(t)=-100-\frac{1}{2}t$? ($1$\,p.)}
	\alakohta{Jaa tekijöihin polynomi $x^2+2x+1$. ($1$\,p.)}
	\end{alakohdat}
 			\begin{vastaus}
 				\begin{alakohdat}
	\alakohta{Ei ole polynomi, sillä muuttujan $t$ eksponentti ei ole luonnollinen luku.}
	\alakohta{Yhdeksän}
	\alakohta{$22$}
	\alakohta{Esimerkiksi $1,9665$ (reuna-arvot eivät kuulu välille!)}
	\alakohta{$1+\sqrt{2}$}
	\alakohta{$x=2$}
	\alakohta{laskeva suora}
	\alakohta{Suoraan muistikaavalla saadaan $x^2+2x+1=(x+1)^2$.}
	\end{alakohdat}
 			\end{vastaus}
 \end{tehtava}
 
 \begin{tehtava}
	\begin{alakohdat}
\alakohta{Näytä, että polynomiyhtälöt $(t+1)^2=t+1$ ja $t^2=-t$ ovat samat.}
\alakohta{Osoita, että $\sqrt{11+4\sqrt{6}}=\sqrt{3}+2\sqrt{2}$.} %esimerkkitehtävä+harjoituksia!
	\end{alakohdat}
	\begin{vastaus}
		\begin{alakohdat}
\alakohta{Avaa ensimmäisestä yhtälöstä sulut, kumoa ykköset ja yhdistä ensimmäisen asteen termit oikealle -- totea yhtäläisyys}
	\alakohta{Korottamalla $\sqrt{3}+2\sqrt{2}$ neliöön ja käyttämällä muistikaavoja saadaan $11+4\sqrt{6}$.}
	\end{alakohdat}
	\end{vastaus}
\end{tehtava}

\begin{tehtava}
Ratkaise epäyhtälöistä $x$.
	\begin{alakohdat}
	\alakohta{$\frac{2x}{3}-1\geq \frac{3}{2}x$}
	\alakohta{$(\frac{1}{2}x-1)^2<1-(x-1)^2$}
	\end{alakohdat}
	\begin{vastaus}
		\begin{alakohdat}
		\alakohta{$x\leq -\frac{6}{5}$}
	\alakohta{$\frac{2}{5}<x<2$}
	\end{alakohdat}
	\end{vastaus}
\end{tehtava}

\begin{tehtava}
Ratkaise yhtälöistä tuntematon reaaliluku $z$.
	\begin{alakohdat}
	\alakohta{$(2+\frac{\sqrt{2}}{2})z^2-z=\sqrt{2}z^3$}
	\alakohta{$z^4-2z^2+1=0$}
	\end{alakohdat}
		\begin{vastaus}
	\begin{alakohdat}
	\alakohta{$z=0$ tai $z=\frac{1}{2}$ tai $z=\sqrt{2}$}
	\alakohta{$z=-1$ tai $z=1$}
	\end{alakohdat}
		\end{vastaus}
\end{tehtava}

\begin{tehtava}
Millä parametrin $k$ arvoilla yhtälöllä $kx^2-(k+1)x+1=0$ on kaksi erisuurta reaalijuurta?
	\begin{vastaus}
Kaikilla paitsi $k=0$
	\end{vastaus}
\end{tehtava}

\begin{tehtava} %tästä pari perustehtävää ja kertaustehtäävää, myös suhteellisena, eikä tunnetuilla absoluuttisilla + kuinka paljon rahaa on tilillä kolmen vuoden kuluttua ilman lisätlalletusta?
Sofie tallettaa $1\,000$ euroa uudelle pankkitilille vuoden alussa ja uudelleen saman verran vuoden kuluttua. Pankki maksaa talletukselle vuosikorkoa siten, että kahden vuoden kuluttua ensimmäisestä talletuksen tilillä on rahaa $2\,100$ euroa. Kuinka suuri on pankin tarjoama korkokanta prosentin kymmenesosan tarkkuudella? Inflaatiota ja korkotuotoista maksettavaa lähdeveroa ei oteta huomioon. Muita tilitapahtumia ei ole.
	\begin{vastaus}
	Jos pankin korkokerroin on $x=(1+\frac{p}{100})$, niin vuoden kuluttua tilillä on rahaa $1\,000x$. Kahden vuoden kuluttua tilillä on rahaa $((1\,000x)+1\,000)x)$, mikä sisältää molemmat talletukset ja kaksi vuosikorkoa. Saadaan yhtälö $((1\,000x)+1\,000)x)=2\,100$ eli sievennettynä ja uudelleen järjesteltynä $1\,000x^2+1\,000x-2\,100=0$. Ratkaisukaavalla tai laskimella saadaan kaksi ratkaisua, joista hyväksytään vain positiivinen: $x=\frac{1}{10}(\sqrt{235}-5)\approx1,033$. Pankin tarjoama vuosikorkokanta on siten $3,3$\,\%.
	\end{vastaus}
\end{tehtava}

\begin{tehtava}
Mikä on funktion $f:\mathbb{R}\rightarrow \mathbb{R}$, pienin arvo, kun funktion arvot määritellään kaavalla $f(x)=3x^2-12x+6$?
	\begin{vastaus}
	$-6$ (vetoaminen paraabelin symmetrisyyteen, että pohja on nollakohtien puolivälissä, tai vastaus nähdään neliöön täydennetystä lausekkeesta perustellen, että reaaliluvun neliö voi olla vähintään nolla)
	\end{vastaus}
\end{tehtava}

\begin{tehtava}
	\begin{alakohdat}
	\alakohta{Osoita, että kolmannen asteen polynomi $x^3 \pm y^3$ voidaan esittää tulomuodossa $(x \pm y)(x^2 \mp xy+y^2)$. ($2$\,p.)}
	\alakohta{Ratkaise epäyhtälö $z^3-8\geq z-2$. ($4$\,p.)}
	\end{alakohdat}
	\begin{vastaus}
	\begin{alakohdat}
	\alakohta{Purkamalla sulkeet lausekkeesta $(x \pm y)(x^2 \mp xy+y^2)$ saadaan vaadittu $x^3 \pm y^3$. (Huomaa, toisiaan vastaavat etumerkit!)}
	\alakohta{Kirjoitetaan epäyhtälö $z^3-8\geq z-2$ a-kohdan kaavan avulla muodossa $(z-2)(z^2+2z+4) \geq z-2$. Epäyhtälöstä voidaan jakaa tekijä $z-2$ pois, mutta sitä varten tutkitaan erikseen tilanteet $z=2$, $z<2$,  ja $z>2$.
	
	Jos $z=2$, niin alkuperäinen epäyhtälö yksinkertaistuu muotoon $0\geq0$, mikä pitää paikkansa.
	
	Jos $z>2$, niin $z-2>0$, eli lauseke $z-2$ on positiivinen. Jakamalla epäyhtälö puolittain kyseisellä lausekkeella säilyttää suuruusjärjestyksen, eli epäyhtälömerkkiä ei tarvitse kääntää. Siis epäyhtälöstä $(z-2)(z^2+2z+4) \geq z-2$ on yhtäpitävä epäyhtälön $z^2+2z+4 \geq 1$ kanssa. Tästä saadaan edelleen toisen asteen polynomiepäyhtälö $z^2+2z+3 \geq 0$. Vasemman puolen polynomin diskrimanttia tutkimalla selviää, että kyseisellä polynomilla ei ole nollakohtia. Koska lisäksi kyseisen polynomin kuvaaja on ylöspäin aukeava paraabeli, voidaan päätellä, että polynomi $z^2+2z+3$ saa vain positiivisia arvoja. Epäyhtälö $z^2+2z+3 \geq 0$ pitää siis paikkansa kaikilla kahta suuremmilla muuttujan $z$ arvoilla.

Jos $z<2$, epäyhtälö $(z-2)(z^2+2z+4) \geq z-2$ muuttuu muotoon $z^2+2z+4 \leq 1$ eli $z^2+2z+3 \leq 0$. Aiemmin jo todettiin, että vasemman puolen polynomi on aina positiivinen, joten epäyhtälö ei pidä paikkaansa millään kahta pienemmillä $z$:n arvoilla.

Yhdistämällä osatulokset päädytään päätelmään, että alkuperäisen epäyhtälön $z^3-8\geq z-2$ ratkaisu on $z\geq 2$.
	}
	\end{alakohdat}
	\end{vastaus}
\end{tehtava}
\newpage
\subsection*{Harjoituskoe 2}

\begin{tehtava}
	\begin{alakohdat}
	\alakohta{Perustele, onko $\frac{1}{t^2}+\pi$ polynomi. ($1$\,p.)}
	\alakohta{Binomi kerrotaan trinomilla. Kuinka monta termiä syntyneeseen uuteen polynomiin tulee (ennen mahdollista sieventämistä)? ($0,5$\,p.)}
	\alakohta{Mikä on tulomuodossa esitetyn polynomin $(x^3+1)^5(1-x)^3$ aste? ($0,5$\,p.)}
	\alakohta{Esitä ehto $x\in ]-\pi,42]$ kaksoisepäyhtälönä. ($0,5$\,p.)}
	\alakohta{Jaa polynomi $y^2-2y+1$ tekijöihin. ($1$\,p.)}
	\alakohta{Sievennä $\frac{x^2-10^{102}}{x-10^{51}}$ ($1$\,p.)}
	\alakohta{Minkälainen (muoto \& mahdollinen suunta) kuvaaja on polynomifunktiolla $P$, jonka arvot lasketaan yhtälöllä $P(t)=12,34^{56}-789t^2$? ($0,5$\,p.)}
	\alakohta{Polynomi voidaan esittää tulomuodossa $(x-32)x(x+23)$. Mikä on kyseisen polynomin pienin nollakohta? ($0,5$\,p.)}
	\alakohta{Kuinka monta nollakohtaa voi korkeintaan olla kuudennen asteen polynomilla? ($0,5$\,p.)}

	\end{alakohdat}
 			\begin{vastaus}
 				\begin{alakohdat}
	\alakohta{Ei ole, sillä muuttujan $t$ eksponentti ($-2$) ei ole luonnollinen luku.}
	\alakohta{kuusi}
	\alakohta{$18$}
	\alakohta{$\pi < x \leq 42$}
	\alakohta{Suoraan muistikaavalla saadaan $y^2-2y+1=(y-1)^2$.}
	\alakohta{$x+10^{51}$}
	\alakohta{alaspäin avautuva paraabeli}
	\alakohta{$x=-23$}
	\alakohta{kuusi}
	\end{alakohdat}
 			\end{vastaus}
 \end{tehtava}
 
 \begin{tehtava}
 Ratkaise yhtälöistä reaalinen tuntematon $t$.
 	\begin{alakohdat}
 	\alakohta{$(9-t^2)(t-1)t=0$}
 	 \alakohta{$(2t^2-1)(t-1)=1$}
 	\end{alakohdat}
 	\begin{vastaus}
 	\begin{alakohdat}
 	 		\alakohta{$t=\pm 3$, $t=0$ tai $t=1$}
 		\alakohta{$t=0$ tai $t=\frac{1}{2}(1-\sqrt{3})$ tai $xt\frac{1}{2}(1+\sqrt{3})$}
 	\end{alakohdat}
 	\end{vastaus}
 \end{tehtava}

\begin{tehtava}
Millä parametrin $a$ arvoilla yhtälöllä $x^2-4ax-a=0$ on täsmälleen yksi reaaliratkaisu?
	\begin{vastaus}
	Diskriminantin lauseke on $16a^2+4a$, ja yhtälöllä on yksi reaalijuuri (kaksoisjuuri) jos ja vain jos diskriminantin arvo on nolla. Ratkaistaan siis yhtälö $16a^2+4a=0$.
	\begin{align*}
	16a^2+4a&=0 \\
	4a(4a+1)&=0 \\
	\end{align*}
	Eli $4a=0$ tai $4a+1=0$. Ensimmäisestä yhtälöstä saadaan ratkaistua $a=0$, ja toisesta $a=-\frac{1}{4}$.
	\end{vastaus}
\end{tehtava}

\begin{tehtava}
Reaaliluvuista $a$ ja $b$ tiedetään, että $a<b$. Osoita kyseistä epäyhtälöä muokkaamalla, että lukujen $a$ ja $b$ aritmeettinen keskiarvo on todellakin lukujen $a$ ja $b$ välissä.
	\begin{vastaus}
	Osoitetaan ensin, että keskiarvo $\frac{a+b}{2}$ on suurempi kuin $a$:
	\begin{align*}
	a&<b && ||+a \\
	2a&<a+b && ||:2 \\
	a&<\frac{a+b}{2} && 
	\end{align*}
	
	Osoitetaan sitten, että keskiarvo on pienempi kuin $b$:
	\begin{align*}
	a&<b && ||+b \\
	a+b&<2b && ||:2 \\
	\frac{a+b}{2}<b &&
	\end{align*}
	
	Koska keskiarvo on suurempi kuin $a$ mutta pienempi kuin $b$, on se $a$:n ja $b$:n välissä. (Todistuksen kannalta ei ole väliä, minkä merkkisiä $a$ ja $b$ ovat.) 
	\end{vastaus}
\end{tehtava}

\begin{tehtava}
Määritä pienin kokonaisluku, joka toteuttaa epäyhtälön $\frac{1}{4}n^4<240+n^2$.
	\begin{vastaus}
	$n=-5$
	\end{vastaus}
\end{tehtava}

\begin{tehtava} %PALJON LISÄÄ NÄITÄ TEHTÄVIÄ KIRJAAN JA ESIMERKKEJÄ ('' on myytävä näin monta,jotta pääsee omilleen, millä hinalla möi...'')
Karkkikauppias ostaa joka kuukausi irtokarkkeja tukusta hintaan $10$\,€/kg ja myy niitä tavallisesti asiakkailleen hintaan $12,9$\,€/kg. Makeisten ostamisen lisäksi kauppiaalla menee joka kuukausi $500$ euroa niin sanottuihin kiinteisiin kustannuksiin. Kiinteät kustannukset ovat aina samat riippumatta myynnin määrästä. Kysynnän hän osaa arvioida niin hyvin, että mitään makeisia ei jää myymättä.
\begin{alakohdat}
	\alakohta{Merkitään myynnin kuukausittaista määrää kilogrammoina $q$:lla. Muodosta lausekelle niin sanotulle voittofunktiolle $P$ eli kuvaukselle myynnin määrästä tuottoon (euroina), josta on vähennetty kaikki myyntitoiminnan kustannukset. Funktion arvot määräävään lausekkeeseen ei tarvitse erikseen merkitä yksiköitä.}
	\alakohta{Kauppias on pitkän myyntikokemuksen perusteella huomannut, että kilogrammahinnan kasvattaminen eurolla vähentää aina kuukausittaista myyntiä $50$ kilogrammaa ja toisaalta kilogrammahinnan vähentäminen eurolla lisää aina kuukausittaista myyntiä samat $50$ kilogrammaa. Mikä tulisi asettaa kilohinnaksi, jotta $180$ kilogramman kuukausimyynnillä kauppias ei jäisi liiketoiminnassaan tappiolle?}
\end{alakohdat}
	\begin{vastaus}
		\begin{alakohdat}
	\alakohta{$P(q)=12,9q-10q-500=2,9q-500$}
	\alakohta{Merkataan kilogrammahinnan euromääräistä lisäystä $x$:llä. Nyt voittofunktion määrittelevä yhtälö on $P(q,x)=(12,9+x)(q-50x)-10(q-50x)-500$, eli uusi kilogrammahinta on $12,9+x$ ja myynnin määrä $q-50x$. Sijoitetaan tehtävänannossa annettu $q=180$, niin saadaan $P(180,x)=(12,9+x)(180-50x)-10(180-50x)-500$. Ottamalla tästä tai jo aiemmasta muodosta yhteinen tekijä saadaan lyhyempi muoto $P(180,x)=(12,9+x-10)(180-50x)-500$ eli $P(180,x)=(2,9+x)(180-50x)-500$ (vrt. a-kohdan sievennetty lauseke).
	
Liiketoiminta on tappiotonta, kun $P(x)\geq0$, missä $P(x)=0$ tarkoittaisi tilannetta, että kauppiaan tulot kattavat täsmälleen kaikki kustannukset. Ratkaisemme siis epäyhtälön $(2,9+x)(180-50x)-500\geq0$, joka sievennettynä on $-50x^2+35x+22\geq0$. Polynomin $-50x^2+35x+22$ nollakohdat ovat ratkaisukaavan tai laskimen ratkaisuohjelman perusteella $x=-0,4$ ja $x=1,1$. Koska polynomin korkeimman asteen termin kerroin on negatiivinen, olisi polynomin kuvaaja alaspäin aukeava paraabeli, ja siten kuvaajan perusteella voidaan päätellä, että funktio $P$ saa positiiviset arvonsa nollakohtiensa välissä. Nollakohtia $x=-0,4$ ja $x=1,1$ vastaavat kilohinnat ovat $12,9-(-0,4)=13,3$ ja $12,9-1,1=11,8$, molemmat yksikössä €/kg.

Jotta myyntitappiolta vältyttäisiin, kauppiaan tulee asettaa kilohinnaksi vähintään $11,8$\,€/kg ja korkeitaan $13,3$\,€/kg. (Näiden rajojen välillä tulee voittoa., rajojen ulkopuolella tappiota.)
}
		\end{alakohdat}
	\end{vastaus}
\end{tehtava}

\begin{tehtava}
	\begin{alakohdat}
\alakohta{Avaa sulut lausekkeesta $(a-b)^3$ ja sievennä tulos. ($2$\,p.)}
\alakohta{Ratkaise reaaliluku $x$ yhtälöstä $x^3-3x^2+3x=9$. ($4$\,p.)}
	\end{alakohdat}
	\begin{vastaus}
		\begin{alakohdat}
\alakohta{$a^3-3a^2b+3ab^2-b^3$}
\alakohta{Binomin kuutioksi täydentämällä saadaan yhtälö potenssiyhtälö $(x-1)^3=8$, jonka ratkaisuna $x=3$}
	\end{alakohdat}
	\end{vastaus}
\end{tehtava}

\begin{tehtava}
Muodosta muuttujan $a$ funktio $f$, joka antaa toisen asteen polynomin $ax^2+2x+2a$ nollakohtien etäisyyden. Määritä funktion $f$ laajin reaalinen määrittelyjoukko ja arvojoukko.
	\begin{vastaus}
	Nollakohtien erotusfunktion määrittelee yhtälö $f(a)=\frac{\sqrt{4-8a^2}}{a}$. Määrittelyjoukkoa rajoittavia tekijöitä ovat neliöjuuri ja jakolasku. Nimittäjän perusteella vaaditaan $a\neq 0$, neliöjuuren perusteella $4-8a^2\geq 0$. Epäyhtälöstä saadaan ratkaisuksi $-\frac{1}{\sqrt{2}}\leq a \leq \frac{1}{\sqrt{2}}$. Yhdistämällä tiedot saadaan määrittelyjoukoksi kaksiosainen reaalilukuväli $[-\frac{1}{\sqrt{2}},0[\cup ]0,\frac{1}{\sqrt{2}}]$. Välin voi myös ilmaista muodossa $[-\frac{1}{\sqrt{2}},\frac{1}{\sqrt{2}}] \setminus \lbrace 0 \rbrace$.
	
Arvojoukon voi päätellä asettamalla funktion arvoksi tuntematon $t$ ja ratkaisemalla yhtälöstä muuttuja $x$:
	$$\frac{\sqrt{4-8x^2}}{x}=t$$
	$$\sqrt{4-8x^2}=tx$$
	$$4-8x^2=(tx)^2=t^2x^2$$
	$$t^2x^2+8x^2=4$$
	$$(t^2+8)x^2=4$$
	$$x^2=\frac{4}{t^2+8}$$
	$$x=\pm \frac{2}{\sqrt{t^2+8}}$$
	
	Huomataan, että saadun lausekkeen perusteella ei ole mitään syytä rajoittaa $t$:n arvoja. Funktion arvojoukko on siis koko reaalilukujen joukko $\mathbb{R}$.
	\end{vastaus}
\end{tehtava}

\subsection*{Harjoituskoe 3}

\begin{tehtava}
	\begin{alakohdat}
	\alakohta{Perustele, onko $\sqrt{k}+\sqrt{2}$ polynomi. ($1$\,p.)}
	\alakohta{Binomi kerrotaan trinomilla. Kuinka monta termiä syntyneeseen uuteen polynomiin tulee (ennen mahdollista sieventämistä)? ($1$\,p.)}
	\alakohta{Mikä on tulomuodossa esitetyn polynomin $(x^3+1)^5(1-x)^3$ aste? ($1$\,p.)}
	\alakohta{Selitä, miksi kerrottaessa epäyhtälöä negatiivisella luvulla epäyhtälön merkki kääntyy. ($1$\,p.)}
		\alakohta{Esitä muuttujaa $x$ koskeva kaksoisepäyhtälönä esitetty ehto $-\frac{2}{3}<x\leq100$ joukko-opillisesti. ($1$\,p.)}
		\alakohta{Muuttujan $x$ arvot ovat suoraan verrannollisia $y$:n arvoihin. Perustele, voiko $x$:n ja $y$:n välinen yhteys olla polynomiaalinen. ($1$\,p.)}
	\end{alakohdat}
 			\begin{vastaus}
 				\begin{alakohdat}
	\alakohta{Ei ole, sillä muuttujan $k$ eksponentti $\frac{1}{2}$ ei ole luonnollinen luku.}
	\alakohta{kuusi}
	\alakohta{$18$}
	\alakohta{$\pi < x \leq 42$}
	\alakohta{$x\in ]\frac{2}{3},100]$}
	\alakohta{Jos $x\propto y$, niin voidaan kirjoittaa $x=ky$, missä $k$ on vakio. Yksiterminen lauseke $ky$ on polynomi.}
	\end{alakohdat}
 			\end{vastaus}
 \end{tehtava}
 
 \begin{tehtava}
Olkoot $a$, $b$, $c$, $d$ ja $n$ lukuja. Osoita, että $$(a^2+nb^2)(c^2+nd^2)=(ac+nbd)^2+n(ad-bc)^2.$$
	\begin{vastaus}
	Puretaan sulkeet auki ja täydennetään binomien neliöiksi.
		\begin{align*}
		(a^2+nb^2)(c^2+nd^2) \\
		&=a^2c^2+na^2d^2+nb^2c^2+n^2b^2d^2 \\
		&= a^2c^2+n^2b^2d^2+na^2d^2+nb^2c^2 \\
		&= a^2c^2+n^2b^2d^2+na^2d^2+nb^2c^2 +2nabcd-2nabcd \\
		&= a^2c^2+2nabcd+n^2b^2d^2+na^2d^2-2nabcd+nb^2c^2 \\
		&= (ac+nbd)^2+na^2d^2-2nabcd+nb^2c^2 \\
		&= (ac+nbd)^2+n(a^2d^2-2abcd+b^2c^2) \\
		&= (ac+nbd)^2+n(ad-bc)^2 \\
\end{align*}
Huom.! Voi myös ''lähteä lopusta'', eli purkaa lauseke $(ac+nbd)^2+n(ad-bc)^2$ ja sitten ryhmitellä vastaus tulomuotoon $(a^2+nb^2)(c^2+nd^2)$. Kumpikin käy.
	\end{vastaus}
\end{tehtava}

 \begin{tehtava}
Tykin ammuksen lentokorkeus $h$ ajanhetkellä $t$ (sekunteina) laukaisun jälkeen noudattaa likimain yhtälöä on $h(t)=h_0+v_0t-\frac{1}{2}gt^2$, missä $h_0$ on laukaisukorkeus metreinä, $v_0$ ylöspäin suuntautuva pystysuora lähtönopeus yksikössä m/s, ja $g$ on putoamiskiihtyvyys $9,81$\,m\,s$^{-2}$. Ammus poistuu laukaisussa tykin piipusta metrin korkeudelta ylöspäin suuntautuvalla nopeudella $15,5$\,m/s. Kuinka kauan kestää ennen kuin tykinkuula osuu maahan? Piirrä tilanteesta kuva.
	\begin{vastaus}
	Ammus osuu maahan noin $3,22$ sekunnin päästä. %FIXME: kuva!
	\end{vastaus}
\end{tehtava}

\begin{tehtava}
Millä vakion $t$ arvoilla yhtälöllä $tx^2+tx-6=0$ ei ole lainkaan reaalisia ratkaisuja?
	\begin{vastaus}
Ei millään $t$:n reaalilukuarvoilla.
	\end{vastaus}
\end{tehtava}

\begin{tehtava}
Tutkitaan polynomifunktiota $P$, jonka arvot määritellään kaavalla $P(x)=x^3-2x^2-x$. Millä $x$:n arvoilla polynomi saa positiivisia arvoja?
	\begin{vastaus}
	Kun $1-\sqrt{2}<x<0$ tai $x>1+\sqrt{2}$
	\end{vastaus}
\end{tehtava}

\begin{tehtava}
Ratkaise yhtälö $t^5-t^4=t^2-t^3$.
	\begin{vastaus}
$t^5-t^4=t^2-t^3$ vastaa yhtälöä $t^5-t^4+t^3-t^2=0$, jonka taasen saa yhteisen tekijän (useaan kertaan) ottamalla muotoon $t^2(t-1)(t^2+1)=0$, eli reaaliset nollakohdat ovat $t=0$ ja $t=1$.
	\end{vastaus}
\end{tehtava}

\begin{tehtava}
Ratkaise yhtälöstä $y^{2n}=y^n$ molemmat reaalimuuttujat $y$ ja $n$.
	\begin{vastaus}
Alkuperäinen yhtälö on yhtäpitävä yhtälön $y^{2n}-y^n=0$ kanssa. Sovelletaan tulon nollasääntöä:
	\begin{align*}
	y^{2n}-y^n&=0 \\
	(y^n)^2-y^n&=0 \\
	y^n \cdot y^n-y^n&=0 \\
	y^n(y^n-1)&=0\\
	\end{align*}
Eli joko $y^n=0$ tai $y^n-1=0$.	 Ensimmäinen yhtälö voi päteä vain, jos $y=0$. (Jos käsitellään $n$:n eri arvoja, niin eksponenttifunktio ei voi saada milloinkaan arvoa $0$.) Tällöin vaaditaan yksikäsitteisyyden vuoksi $n\neq0$, koska $0^0$ ei ole määritelty.

Toisesta yhtälöstä saadaan $y^n=1$, joka voi olla tosi vain, jos joko $n=0$ ja $y \neq 0$ (minkä tahansa nollasta poikeavan luvun nollas potenssi on arvoltaan yksi) tai sitten $y=1$ ja $n$ on mielivaltainen.

Yhtälön ratkaisut ovat siis:
$y=0$ ja $n \neq 0$ \\
$n=0$ ja $y \neq 0$ \\
$y=1$ ja mikä tahansa $n$
	\end{vastaus}
\end{tehtava}

\begin{tehtava}
Polynomin $Q$ muuttuja on $x$, ja polynomi voidaan esittää tulomuodossa $(ax-b)(\frac{2ab}{a^2+b^2}x-1)(bx-a)$, missä $a$ ja $b$ ovat reaalisia vakioita, joille pätee $0<a<b$. Perustele, mikä on funktion nollakohtien suuruusjärjestys.
		\begin{vastaus}
Lausekkeen perusteella nollakohdat ovat $\frac{b}{a}$,$\frac{a^2+b^2}{2ab}$ ja $\frac{a}{b}$. Koska $0<a<b$, niin $a$ on positiivinen, ja kaksoisepäyhtälö voidaan jakaa sillä niin, että järjestys säilyy (''merkkien suuntaa ei tarvitse kääntää''): $0<1<\frac{b}{a}$. Samoin voidaan tehdä $b$:llä: $0<\frac{a}{b}<1$. Vertailemalla nähdään, että selvästi $\frac{a}{b}<\frac{b}{a}$. Kolmas nollakohta on kahden muun aritmeettinen keskiarvo, joten se on varmasti niiden välissä.
		\end{vastaus}
\end{tehtava}
\newpage

\subsection*{Harjoituskoe 4}

\begin{tehtava}
	\begin{alakohdat}
	\alakohta{Perustele, onko $\frac{2t^0+5t}{3}$ polynomi. ($1$\,p.)}
	\alakohta{Trinomi kerrotaan toisella trinomilla. Kuinka monta termiä syntyneeseen uuteen polynomiin tulee (ennen mahdollista sieventämistä)? ($0,5$\,p.)}
	\alakohta{Mikä on tulomuodossa esitetyn polynomin $2(1-2y)^2(y^2+2)^2$ aste? ($0,5$\,p.)}
	\alakohta{Esitä muuttujaa $t$ koskeva ehto $t\leq \pi$ joukko-opillisesti. ($0,5$\,p.)}
	\alakohta{Polynomi voidaan esittää tulomuodossa $(x-2)x(x+3)$. Mikä on kyseisen polynomin suurin nollakohta? ($0,5$\,p.)}
		\alakohta{Sievennä lauseke $\frac{x^3-4x^2-4x}{x^2-2x}$. ($3$\,p.)}
	\end{alakohdat}
 			\begin{vastaus}
 				\begin{alakohdat}
	\alakohta{On, sillä sievennetyssä muodossa $\frac{2}{3}+\frac{5}{3}t$ on selvästi ensimmäisen asteen polynomi.}
	\alakohta{Yhdeksän}
	\alakohta{$6$}
	\alakohta{$t \in ]\infty,\pi]$}
	\alakohta{$x=2$}
	\alakohta{$x-2$}
	\end{alakohdat}
 			\end{vastaus}
 \end{tehtava}

\begin{tehtava}
Polynomifunktion $P$ arvot määritellään kaavalla $P(x)=(k^2+2k)x+2k$. Määritä vakio $k$ siten, että ehto $P(2)=0$ pätee. Piirrä ehdon täyttävän polynomifunktion kuvaaja.
	\begin{vastaus}
$P(2)=0 \Rightarrow$ $(k^2+2k)\cdot 2+2k=0$
\begin{align*}
k^2\cdot 2+2k\cdot 2+2k&=0 \\
2k^2+4k+2k&=0 \\
k^2+2k+k&=0 \\
(k+1)^2&=0 \\
k+1=&0 \\
k&=-1  \\
\end{align*}

Tällöin siis $P(x)=(1-2)x-2=-x-2$, jonka kuvaaja on laskeva suora. %FIXME: lisääkuva
	\end{vastaus}
\end{tehtava}

\begin{tehtava}
Millä vakion $a$ arvoilla yhtälöllä $x^2+ax-a=0$ on kaksoisjuuri? Mikä tämä kaksoisjuuri tällöin on?
	\begin{vastaus}
Diskriminantista $a^2+4a=0$, josta kaksi juurta: $a=0$ tai $a=-4$.

Tapaus $a=0$:
\begin{align*}
x^2+0x-0&=0 \\
x^2&=0 \\
x&=0 \\
\end{align*}
Tapaus $a=-4$:
\begin{align*}
x^2-4x+4&=0 \\
(x-2)^2&=0 \\
x-2&=0 \\
x&=2
\end{align*}
	\end{vastaus}
\end{tehtava}

\begin{tehtava} %tästä esimerkkejä...
Erääseen ulkomaalaiseen matkapuhelinliittymään on saatavilla kaksi maksusuunnitelmaa mobiilidatan käyttöön. Tarjouksen A mukaan maksat kuukaudessa vakiohinnan $40$ euroa riippumatta siitä, kuinka paljon dataa kulkee. Tarjouksen B mukaan kuukauden ensimmäiset $100$ megatavua maksaa $25$ euroa, minkä jälkeen ylimenevästä osasta maksetaan $0,12$ euroa megatavua kohden. Millä kuukausittaisilla datamäärillä tarjous A on tarjousta B edullisempi?
	\begin{vastaus}
	Merkataan $x$:llä kulkeneen datan määrää megatavuissa. Tällöin tarjouksen A hinta on $A(x)=40$ (hinta aina $40$ euroa, vakiofunktio) ja tarjouksen B hinta määritellään paloittain:
	$$B(x)=\begin{cases}
	25, & \mbox{kun } x\leq 100 \\
	0,12(x-100)+25, & \mbox{kun } x>100
	\end{cases}$$
Yhtäsuuruuden voi merkitä vaihtoehtoisesti myös jälkimmäisen lausekkeen ehtoon: $x\geq 100$. Jälkimmäisessä ehdossa on pelkän $x$:n sijaan $x-100$, jotta datamäärän lasku alkaisi nollasta, eikä heti sadan megatavun rajan ylitettyä tulisi $0,12\cdot 100=12$ euron lisälaskua.

Alle sadan megatavun käytössä tarjous A ei voi olla halvempi, koska $40>25$. Sitä suuremmalla käytöllä sen sijaan tämä on mahdollista. Ratkaistaan siis epäyhtälö $40<0,12(x-100)+25$, josta selviää, millä $x$:n arvoilla tarjous A on halvempi.
\begin{align*}
40&<0,12(x-100)+25 \\
15 &<0,12(x-100) \\
125&<x-100 \\
225&<x \\
\end{align*}
	Tarjous A on siis halvempi, kun dataa siirretään kuukaudessa yli 225 megatavua.
	\end{vastaus}
\end{tehtava}

\begin{tehtava}
Ratkaise $y$ yhtälöstä $\frac{1}{2}n^n y^2-n^{2n}y-n^{3n+1}=0$, kun $n=50$. Esitä vastaus kymmenpotenssimuodossa kolmen merkitsevän numeron tarkkuudella.
	\begin{vastaus}
	Sieventämällä saadaan, että ratkaisut ovat muotoa $y=n^n (1\pm \sqrt{1+2n})$. Sijoittamalla $n=50$ saadaan laskimesta kymmenpotenssimuodoiksi $-8,04\cdot 10^{85}$ ja $9,81\cdot 10^{85}$.
	\end{vastaus}
\end{tehtava}

\begin{tehtava} %tästä harjoituksia!
Ratkaise yhtälö $x^4-\frac{1}{2}x^3-4x^2+2x=0$.
	\begin{vastaus}	
$x^4-\frac{1}{2}x^3-4x^2+2x=x(x-\frac{1}{2})(x^2-4)$, eli $x=0$, $x=\frac{1}{2}$ tai $x=\pm 2$.
	\end{vastaus}
\end{tehtava}

\begin{tehtava}
Funktion $y$ arvot määritellään kaavalla $y(t)=\dfrac{t^2+\left(\frac{1}{3}+\pi\right)t-\frac{\pi}{3}}{t^2+\left(\frac{1}{3}-\pi\right)t-\frac{\pi}{3}}$.
	\begin{alakohdat}
	\alakohta{Mikä on funktion määrittelyjoukko?}
	\alakohta{Määritä ne $ty$-koordinaatiston pisteet, joissa funktion kuvaaja leikkaa jomman kumman koordinaattiakselin.}
	\end{alakohdat}
		\begin{vastaus}
	\begin{alakohdat}
	\alakohta{Reaalilukujen joukko poislukien $t=-\frac{1}{3}$ ja $t=\pi$}
	\alakohta{$y$-akseli leikkautuu, kun $t=0$, eli pisteessä $(0,\frac{pi}{3})$. $t$-akseli leikkautuu vain yhdessä pisteessä: $(-\pi, 0)$, koska määrittejoukko ei sisällä funktion lausekkeen osoittajan toista nollakohtaa $-\frac{1}{3}$.}
	\end{alakohdat}
		\end{vastaus}
\end{tehtava}

\begin{tehtava}
Johda toisen asteen yhtälön ratkaisukaava lähtien yhtälön normaalimuodosta $ax^2+bx+c=0$, missä $a, b$ ja $c$ ovat reaalisia vakioita, ja $a \neq 0$.
	\begin{vastaus}
	Katso luku Toisen asteen yhtälön ratkaisukaava. (Johdosta on olemassa erilaisia variantteja.)
	\end{vastaus}
\end{tehtava}

%\newpage

%\subsection*{Harjoituskoe 5}
%\begin{enumerate}
%\item Ratkaise yhtälöt.\\ a) $-x-5+2x=0$\\ b) $8x^2=-2x+4$\\ c) $(-3x)^2-36=0$
%\item Ratkaise epäyhtälöt.\\ a) $-2x^2-2>2$\\ b) $x^2-1\geq0$\\ c) $ax^2<bx$
%\item Kolme lohikäärmettä ja yksi kissa painaa saman verran kuin 10 kissaa ja 18 Jarkkoa. Muodosta yhtälö tilanteesta yhtälö ja ratkaise yhden lohikäärmeen paino.
%\item Montako juurta on yhtälöllä\\ a) $9x^2+3x+1=0$\\ b) $6x^2-3x=-2$\\ c) $3x^2-32$\\ d) $3x-30=0$
%\item Millä parametrin r arvoilla yhtälölle $rx^2-rx+1=0$ ei ole ratkaisuja.
%\item 
%\item 
%\item Ratkaise yhtälö $\frac{21x^2}{700}-\frac{7x}{1400}-\frac{14x}{2800}=0$ ilman laskinta.
%\end{enumerate}

%\begin{enumerate}
%\item Ratkaise yhtälöt.\\ a) $4x-1=0$\\ b) $x=-3x+2$\\ c) $2g-3g=g-8$
%\item Ratkaise yhtälöt.\\ a) $4x^2-1=0$\\ b) $x^3=-3x$\\ c) $2y^2=y-8$
%\item Ratkaise epäyhtälöt.\\ a) $1-\dfrac{1-x}{6}<x$\\ b) $(x+1)(x^2-2x-1)\geq0$\\ c) $\frac{x}{2}>\frac{x}{5}$
%\item Täydennä neliöksi. \\ a) $x^2+2x+1$\\ b) $9x^2-6x+1$\\ c) $3x+4x^2+\frac{9}{16}$
%\item Ratkaise $x$. \\ a) $\frac{x}{2}(x-1)=0$\\ b) $(3x^2-3)(3x^2+1)=0$\\ c) $(x-a)(x-b)=0$
%\item Ratkaise yhtälö. $(x^2-4x+4)^2=0$
%\item Millä $x$:n arvoilla polynomi $x^2-2x-3$ saa positiivisia arvoja?
%\item Ratkaise yhtälöt.\\ a) $x-5x=0$\\ b) $x^4-1=0$\\ c) $(x-1)(x+4) = x(x-5)$
%\item Ratkaise epäyhtälöt.\\ a) $x^2-8\geq0$\\ b) $x^2-8\geq(x-3)^2$\\ c) $x^2-6x+9\leq0$
%\item Kolmannen asteen polynomifunktiolle pätee $P(-1)=0$, $P(0)=0$ ja $P(1)=0$. Lisäksi $P(3)=3$. Määritä polynomi $P$.
%\item
%\item Laske. \\ a) $(3x+1)^2+5=6x$\\ b) $5x>15x$\\ c)$3x^2=(3x)^2+6$ 
%\item Lukujen $a$ ja $b$ erotus $a-b=2$ ja tulo $ab=4$, laske käänteislukujen erotus $\frac{1}{a}-\frac{1}{b}$.
%\item Millä vakion $c$ arvoilla polynomi $x^2+cx+c$ saa sekä positiivia että negatiivisia arvoja?
%\item Jaa polynomi tekijöihin.\\ a) $x^2+x-30$\\ b)  $x^2-x-30$\\ c)  $-2x^2+5x-3$ 
%\item Ratkaise epäyhtälöt. \\
%a) $2x^3 \geq x$ \\
%b) $y^2 \leq 3y -9 $

%\item Ratkaise yhtälöt.\\ a) $3x-5=6$\\ b) $20x-20=20$\\ c) $2x-5=\frac{5x+2}{2}$
%\item Ratkaise yhtälöt.\\ a) $-5x^2-5=0$\\ b) $8x^2=-2x$\\ c) $x^2+10=0$
%\item Ratkaise epäyhtälöt.\\ a) $-5x^2-5>0$\\ b) $x^2+10\geq0$\\ c) $x^2<x$
%\item Millä $h$:n arvoilla yhtälöllä $h^2x^2+hx+\frac{1}{4}=0$ on tasan yksi ratkaisu?
%\item Ratkaise yhtälö $x^2-6x=0$\\ a) tulon nollasäännöllä\\ b) toisen asteen yhtälön ratkaisukaavalla
%\item Ratkaise epäyhtälö $(\frac{x+1}{-4})x-3>1$
%\item Jaa polynomi tekijöihin.\\ a) $x^2-x-2$\\ b) $4x^2-2x-2$
%\end{enumerate}


\section{Tehtäviä ylioppilaskokeista}

%erotellaan selvästi oppimäärät toisistaan myös muissa kirjoissa! T: Joonas :)

\subsubsection*{Lyhyen oppimäärän tehtäviä}

% \begin{tehtava}
% (K2011/1a) \\ Ratkaise yhtälö $4x+(5x-4) = 12+3x$.
% \begin{vastaus}
% $x=\frac{16}{3} = 5\frac13$
% \end{vastaus}
% \end{tehtava}

\begin{tehtava}
(k2011/1b) Sievennä lauseke $x^2+x-(x^2-x)$.
\begin{vastaus}
$2x$
\end{vastaus}
\end{tehtava}

\begin{tehtava}
(k2011/2b) \\ Sievennä lauseke $(\sqrt{x}-1)^2+2\sqrt{x}$.
\begin{vastaus}
$x+1$
\end{vastaus}
\end{tehtava}

\begin{tehtava}
(s2011/1c) \\ Ratkaise yhtälö $x^2-3(x+3) = 3x-18$.
\begin{vastaus}
$x=3$
\end{vastaus}
\end{tehtava}

\begin{tehtava}
(k2012/1a) Ratkaise yhtälö $7x+3 = 31$.
\begin{vastaus}
$x = 4$
\end{vastaus}
\end{tehtava}

\begin{tehtava}
(s2012/1a) Ratkaise yhtälö $x^2-2x = 0$.
\begin{vastaus}
$x=0$ tai $x=2$
\end{vastaus}
\end{tehtava}

\begin{tehtava}
(k2013/2a) Millä muuttujan $x$ arvoilla $4x+17$ on suurempi kuin $2-x$?
\begin{vastaus}
$x>-3$
\end{vastaus}
\end{tehtava}

\begin{tehtava}
(k2013/2b) Ratkaise yhtälö $x^2+14x=-49$.
\begin{vastaus}
$x=-7$
\end{vastaus}
\end{tehtava}

\begin{tehtava}
(s2013/1a) Ratkaise yhtälö $(x-2)^2=4$.
\begin{vastaus}
$x=0$ tai $x=4$ 
\end{vastaus}
\end{tehtava}

\subsubsection*{Pitkän oppimäärän tehtäviä}

\begin{tehtava}
(K1988/3) Millä $a$:n arvoilla yhtälön $x^2+(3a+1)x+81=0$ juuret ovat reaaliset?
\begin{vastaus}
$a \leq -\frac{19}{3}$ tai $a \geq \frac{17}{3}$
\end{vastaus}
\end{tehtava}

\begin{tehtava}
(K2004/1c) Olkoon $f(x)=x^3+3x^2+x+1$ ja $g(x)=x^3+x^2-2x+3$. Ratkaise yhtälö $f(x)=g(x)$.
\begin{vastaus}
$x=-2$ tai $x=\frac{1}{2}$  
\end{vastaus}
\end{tehtava}

\begin{tehtava}
(S2004/1a) Ratkaise epäyhtälö $2x-3<3-2x$.
\begin{vastaus}
$x<\frac{3}{2}$ 
\end{vastaus}
\end{tehtava}

\begin{tehtava}
(S2004/1b) Ratkaise epäyhtälö $(x+1)^2 \leq 1$.
\begin{vastaus}
$-2 \leq x \leq 0$ 
\end{vastaus}
\end{tehtava}

\begin{tehtava}
(S2004/1c) Ratkaise epäyhtälö $x^3<x^2$.
\begin{vastaus}
$x<1$, $x\neq0$ 
\end{vastaus}
\end{tehtava}

\begin{tehtava}
(K2005/1a) Sievennä lauseke $\frac{x}{1-x}+\frac{x}{1+x}$.
\begin{vastaus}
$\frac{2x}{1-x^2}$ 
\end{vastaus}
\end{tehtava}

\begin{tehtava}
(K2005/1b) Ratkaise $x$ yhtälöstä $x^2-ax-a^2=0$.
\begin{vastaus}
$x= \frac{1}{2}a(1-\sqrt{5})$ tai $x= \frac{1}{2}a(1+\sqrt{5})$ 
\end{vastaus}
\end{tehtava}

\begin{tehtava}
(S2005/1b) Ratkaise reaalilukualueella yhtälö $x+2=\frac{1}{x-2}$.
\begin{vastaus}
$x=\pm \sqrt{5}$ 
\end{vastaus}
\end{tehtava}

\begin{tehtava}
(S2005/4) Millä $a$:n arvoilla funktio $f(x)=-x^2+ac+a-3$ saa vain negatiivisia arvoja?
\begin{vastaus}
$-6<a<2$ 
\end{vastaus}
\end{tehtava}

\begin{tehtava}
(S2006/1a) Sievennä lauseke $(1+x)^3-(1-x)^3$.
\begin{vastaus}
$2x^3+6x$ 
\end{vastaus}
\end{tehtava}

\begin{tehtava}
(S2006/1b) Ratkaise yhtälö $ \frac{x+1}{x}=\frac{x}{x+1}$.
\begin{vastaus}
$x=-\frac{1}{2}$ 
\end{vastaus}
\end{tehtava}

\begin{tehtava}
(K2007/1a) Ratkaise yhtälö $7x^2-6x=0$.
\begin{vastaus}
$x=0$ tai $x=\frac{6}{7}$
\end{vastaus}
\end{tehtava}

\begin{tehtava}
(S2007/1a) Ratkaise epäyhtälö $2-3x>4x$.
\begin{vastaus}
$x< \frac{2}{7} $
\end{vastaus}
\end{tehtava}

\begin{tehtava}
(K2008/1a) Ratkaise yhtälö $2x^2=x+1$.
\begin{vastaus}
$x=1$ tai $x=-\frac{1}{2}$
\end{vastaus}
\end{tehtava}

\begin{tehtava}
(S2008/1a) Ratkaise epäyhtälö $\frac{1}{2} - \frac{x}{3} > \frac{3}{4}$.
\begin{vastaus}
$x<-\frac{3}{4}$
\end{vastaus}
\end{tehtava}

% Erillisten murtolausekkeiden laventamista samannimisiksi -> ehkä enemmän MAA1-asiaa.
% \begin{tehtava}
% (S2008/1b) Sievennä lauseke $\frac{1}{x}-\frac{1}{x^2}+ \frac{1+x}{x^2}$.
% \begin{vastaus}
% $ \frac{2}{x}$
% \end{vastaus}
% \end{tehtava}

\begin{tehtava}
(S2008/3b) Ratkaise yhtälö $4x^3-5x^2=2x-3x^3$.
\begin{vastaus}
$x=-\frac{2}{7}$ tai $x=0$ tai $x=1$
\end{vastaus}
\end{tehtava}

\begin{tehtava}
(K2009/1b) Ratkaise epäyhtälö$(x-3)^2>(x-1)(x+1)$.
\begin{vastaus}
$x<\frac{5}{3}$
\end{vastaus}
\end{tehtava}

\begin{tehtava}
(S2009/1a) Ratkaise yhtälö $(x-2)(x-3)=6$. 
\begin{vastaus}
$x=0$ tai $x=5$
\end{vastaus}
\end{tehtava}

\begin{tehtava}
(S2009/1b) Ratkaise yhtälö $\frac{x}{x-3}-\frac{1}{x}=1$.
\begin{vastaus}
$x=-\frac{3}{2}$
\end{vastaus}
\end{tehtava}

\begin{tehtava}
(S2009/2a) Ratkaise epäyhtälö $6(x-1)+4 \geq 3(7x+1)$. 
\begin{vastaus}
$x \leq -\frac{1}{3}$
\end{vastaus}
\end{tehtava}

% 7. kurssin asiaa:
% \begin{tehtava}
% (S2009/8) Ratkaise epäyhtälö $\frac{-x^2+x+2}{x^3+2x^2-3x}>0$.
% \begin{vastaus}
% $x<-3$ tai $-1<x<0$ tai $1<x<2$
% \end{vastaus}
% \end{tehtava}

\begin{tehtava}
(K2010/1a) Ratkaise yhtälö $7x^7+6x^6=0$.
\begin{vastaus}
$x=0$ tai $x=-\frac{6}{7}$
\end{vastaus}
\end{tehtava}

\begin{tehtava}
(K2010/1b) Sievennä lauseke $(\sqrt{a}+1)^2-a-1$.
\begin{vastaus}
$2\sqrt{a}$
\end{vastaus}
\end{tehtava}

\begin{tehtava}
(K2010/1c) Millä $x$:n arvoilla pätee $\frac{3}{3-2x}<0$?
\begin{vastaus}
$x>\frac{3}{2}$
\end{vastaus}
\end{tehtava}

\begin{tehtava}
(K2010/3b) Määritä toisen asteen yhtälön $x^2+px+q=0$ kertoimet $p$ ja $q$, kun yhtälön juuret ovat $-2-\sqrt{6}$ ja $-2+\sqrt{6}$.
\begin{vastaus}
$p=4$, $q=-2$
\end{vastaus}
\end{tehtava}

\begin{tehtava}
(S2010/1a) Sievennä lauseke $(a+b)^2-(a-b)^2$.
\begin{vastaus}
$4ab$
\end{vastaus}
\end{tehtava}

\begin{tehtava}
(S2010/2a) Ratkaise epäyhtälö $x\sqrt{7}-3 \leq 4x$.
\begin{vastaus}
$x \geq \frac{3}{\sqrt{7}-4}$
\end{vastaus}
\end{tehtava}

\begin{tehtava}
(S2010/2c) Ratkaise yhtälö $x^4-3x^2-4=0$.
\begin{vastaus}
$x=2$ tai $x=-2$
\end{vastaus}
\end{tehtava}

% Murtolukujen jakokulmaa ei tule tässä kurssissa.
% \begin{tehtava}
% (S2010/12) Määritä $a$ siten, että polynomi $P(x)=2x^4-3x^3-7x^2+a$ on jaollinen binomilla $2x-1$. Määritä tätä $a$:n arvoa vastaavat yhtälön $P(x)=0$ juuret.
% \begin{vastaus}
% $a=2$. Yhtälön $P(x)=0$ juuret ovat $x=\frac{1}{2}$, $x=-1$, $x=1+\sqrt{3}$ ja $x=1-\sqrt{3}$.
% \end{vastaus}
% \end{tehtava}

\begin{tehtava}
(K2011/1b) Ratkaise epäyhtälö $x^2-2 \leq x$.
\begin{vastaus}
$-1 \leq x \leq 2$
\end{vastaus}
\end{tehtava}

\begin{tehtava}
  (S2011/3b) Ratkaise epäyhtälö $\frac{2x+1}{x-1} \geq 3$.
\begin{vastaus}
$1<x \leq 4$
\end{vastaus}
\end{tehtava}

\begin{tehtava}
(S2012/1a) \\ Ratkaise yhtälö $2(1-3x+3x^2) = 3(1+2x+2x^2)$.
\begin{vastaus}
$x=-\frac{1}{12}$
\end{vastaus}
\end{tehtava}

\begin{tehtava}
(S2013/1a) \\ Ratkaise yhtälö $x^2+6x=2x^2+9$.
\begin{vastaus}
$x=3$
\end{vastaus}
\end{tehtava}

\begin{tehtava}
(S2013/1b) \\ Ratkaise yhtälö $\frac{1+x}{1-x}=\frac{1-x^2}{1+x^2}$.
\begin{vastaus}
$x=0$ tai $x=-1$
\end{vastaus}
\end{tehtava}

\begin{tehtava}
(S2013/1c) \\ Esitä polynomi $x^2-9x+14$ ensimmäisen asteen polynomien tulona.
\begin{vastaus}
$(x-2)(x-7)$
\end{vastaus}
\end{tehtava}

\begin{tehtava}
(K2013/1b) \\ Ratkaise epäyhtälö $\frac{3}{5}x-\frac{7}{10} < -\frac{2}{15}x$.
\begin{vastaus}
$x<\frac{21}{22}$
\end{vastaus}
\end{tehtava}

\begin{tehtava}
(K2013/3a) \\ Laske lausekkeen $(\sqrt{a}+\sqrt{b})^2$ tarkka arvo, kun positiiviset luvut $a$ ja $b$ ovat toistensa käänteislukuja ja lukujen $a$ ja $b$ keskiarvo on $2$.
\begin{vastaus}
$6$
\end{vastaus}
\end{tehtava}

\begin{tehtava}
(K2013/*14a) \\ Jaa $P(x)=x^2+x-2$ ensimmäisen asteen tekijöihin. (2 p.)
\begin{vastaus}
$(x+2)(x-1)$
\end{vastaus}
\end{tehtava}

\begin{tehtava}
(K2013/*14b) \\ Olkoon $P(x)=x^2+x-2$. Määritä sellaiset vakiot $A$ ja $B$, että $\frac{1}{P(x)}=\frac{A}{x-1}+\frac{B}{x+2} $ kaikilla $x \geq 2$. (2 p.)
\begin{vastaus}
$A= \frac{1}{3}$ ja $B=- \frac{1}{3}$
\end{vastaus}
\end{tehtava}


% (S2011, 1b) Suorakulmaisen kolmion hypotenuusan pituus on 5 % Kurssi 3
%   ja toisen kateetin pituus 2. Laske toisen kateetin pituus.

  

%fixme: etsi lisää yo-tehtäviä, näitä on kyllä olemassa

\nosa{Lisämateriaalia}
   \nluku{LIITE_vaikeat}{Tavoittele valaistumista} %tähtitehtäviksi?
   \nluku{LIITE_paraabeli}{Paraabeli}

\Closesolutionfile{ans}
\vast