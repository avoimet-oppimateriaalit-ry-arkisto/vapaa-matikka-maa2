\Opensolutionfile{ans}[content/LIITE_vastaukset]

\providecommand{\lukufilter}[2]{#2} % ylikirjoitetaan kaanna_luku.sh -skriptistä.
\newcommand{\osa}[1]{\chapter{#1}} % osa
\newcommand{\nosa}[1]{\chapter*{#1} \addcontentsline{toc}{chapter}{#1}} %numeroimaton osa
\newcommand{\luku}[2]{\section{#2} \lukufilter{#1}{\input{content/TEORIA_#1} \input{content/TEHT_#1}}} % luku
\newcommand{\nluku}[2]{\section*{#2} \addcontentsline{toc}{section}{#2} \lukufilter{#1}{\input{content/#1}}} % numeroimaton luku
\newcommand{\vast}{\section*{Vastaukset} \addcontentsline{toc}{section}{Vastaukset} \begin{vastaussivu} \input{content/LIITE_vastaukset} \end{vastaussivu}}

\newpage
\nluku{LIITE_esipuhe}{Esipuhe}

\osa{Polynomi}
    \luku{polynomi}{Polynomi}
    \luku{polynomien_kertolasku}{Polynomeilla laskeminen}
	\luku{muistikaavat}{Muistikaavat}
	\luku{tulon_nollasaanto_ja_tulon_merkkisaanto}{Tulon nollasääntö \& tulon merkkisääntö}
	\luku{tekijoihinjako}{Muistikaavat}
	\luku{polynomifunktion_kuvaaja}{Polynomifunktion kuvaaja}

\osa{Ensimmäinen aste}
    \luku{epayhtalo}{Epäyhtälöiden teoriaa}
    \luku{ensimmaisen_asteen_yhtalo}{Kertausta: ensimmäisen asteen yhtälö}
    \luku{ensimmaisen_asteen_epayhtalo}{Ensimmäisen asteen epäyhtälö}

\osa{Toinen aste}
	\luku{paraabeli}{Toisen asteen polynomifunktio ja sen kuvaaja}
	\luku{toisen_asteen_yhtalo}{Toisen asteen yhtälö}
	\luku{toisen_asteen_yhtalon_ratkaisukaava}{Toisen asteen yhtälön ratkaisukaava}
	\luku{diskriminantti}{Diskriminantti}
	\luku{toisen_asteen_epayhtalo}{Toisen asteen epäyhtälö}
	\luku{polynomin_jakaminen_tekijoihin}{Polynomin jakaminen tekijöihin}
	
\osa{Korkeampi aste}
	\luku{korkeamman_asteen_polynomifunktio}{Korkeamman asteen polynomifunktio}
	\luku{korkeamman_asteen_yhtalot}{Korkeamman asteen yhtälöt}
	\luku{korkeamman_asteen_epayhtalot}{Korkeamman asteen epäyhtälöt}

\newpage
\nosa{Kertausosio}
   \nluku{LIITE_testaatietosi}{Testaa tietosi!}
   \nluku{LIITE_kertausteht}{Kertaustehtäviä}
   \nluku{LIITE_harjoituskokeita}{Harjoituskokeita}
   \nluku{LIITE_yokokeita}{Ylioppilaskoetehtäviä}

\newpage
\nosa{Lisämateriaalia}
   \nluku{LIITE_paraabeli}{Paraabeli}
   
\newpage
\Closesolutionfile{ans}
\vast