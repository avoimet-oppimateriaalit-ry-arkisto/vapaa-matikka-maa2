\Opensolutionfile{ans}[content/LIITE_vastaukset]

\providecommand{\lukufilter}[2]{#2} % ylikirjoitetaan kaanna_luku.sh -skriptistä.
\newcommand{\osa}[1]{\chapter{#1}} % osa
\newcommand{\nosa}[1]{\chapter*{#1} \addcontentsline{toc}{chapter}{#1}} %numeroimaton osa
\newcommand{\luku}[2]{\section{#2} \lukufilter{#1}{\input{content/TEORIA_#1} \input{content/TEHT_#1}}} % luku
\newcommand{\nluku}[2]{\section*{#2} \addcontentsline{toc}{section}{#2} \lukufilter{#1}{\input{content/#1}}} % numeroimaton luku
\newcommand{\vast}{\section*{Vastaukset} \addcontentsline{toc}{section}{Vastaukset} \begin{vastaussivu} \input{content/LIITE_vastaukset} \end{vastaussivu}}

\newpage
\nluku{LIITE_esipuhe}{Esipuhe}

\osa{Polynomi}
    \luku{polynomi}{Polynomi}
    \luku{polynomien_kertolasku}{Polynomeilla laskeminen}
	\luku{muistikaavat}{Muistikaavat}
	\section{Tulon nollasääntö ja tulon merkkisääntö}

\subsection*{Tulon merkkisääntö}

Pitkän matematiikan 1. kurssilla on esitetty seuraava sääntö kahden reaaliluvun tulolle:

\laatikko[Tulon merkkisääntö kahdelle tulon tekijälle]{
    \begin{itemize}
        \item Jos tulon tekijät ovat samanmerkkisiä, tulo on positiivinen.
        \begin{itemize}
	        \item Kahden positiivisen luvun tulo on positiivinen.
	        \item Kahden negatiivisen luvun tulo on positiivinen.
        \end{itemize}
        \item Jos tulon tekijät ovat erimerkkisiä, tulo on negatiivinen.
        \begin{itemize}
	        \item Positiivisen ja negatiivisen luvun tulo on negatiivinen.
        \end{itemize}
    \end{itemize}
}

Tulon merkkisäännöstä seuraa, että reaaliluvun neliö ei voi olla negatiivinen, koska kahden samanmerkkisen luvun tulo aina on positiivinen. Lyhyemmin ilmaistuna $x^2 \geq 0$, eli reaaliluvun neliö on aina \termi{epänegatiivinen}{epänegatiivinen}.

\begin{esimerkki}
Osoita, että funktio $P(x)=x^2+7$ saa vain positiivisia arvoja.
    \begin{esimratk}
	Koska $x^2 \geq 0$, lausekkeen $x^2+7$ arvo on vähintään $7$. Fuktio saa siis
	vain positiivisia arvoja.
    \end{esimratk}
\end{esimerkki}

\begin{esimerkki}
Mikä on funktion $f:\mathbb{R} \rightarrow \mathbb{R}, f(t)=-t^4+3$ suurin arvo?
    \begin{esimratk}
	Koska muuttuja $t$ on reaaliluku, ei sen parillinen potenssi voi olla negatiivinen, vaan $t^4$:n arvo on vähintään $0$. Tämän perusteella $-t^4$:n arvo on epäpositiivinen eli korkeintaan nolla. Jos $-t^4$:n suurin arvo on nolla, niin lisäämällä tähän luvun $3$, saadaan funktion $f(t)=-t^4+3$ suurimmaksi arvoksi $3$.
    \end{esimratk}
\end{esimerkki}


Tulon merkkisääntö yleistyy mille tahansa määrälle tulontekijöitä.
Mikäli tulossa on pariton $(1, 3, 5, \ldots)$ määrä negatiivisia tekijöitä, tulo on negatiivinen.
Muulloin tulo on positiivinen.

%Hyödynnämme merkkisääntöä myöhemmin, kun teemme epäyhtälöistä merkkikaavioita.

\newpage

\subsection*{Tulon nollasääntö}

Tulon merkkisäännöstä seuraa, että positiivisten ja negatiivisten lukujen tulo on aina positiivinen tai negatiivinen, ei koskaan nolla.
Jos siis tulo on $0$, tulon tekijöistä ainakin yhden täytyy olla $0$.
Toisaalta jos jokin tulon tekijöistä on $0$, myös tulo on automaattisesti $0$.
Nämä tiedot yhdistämällä saadaan tulon nollasääntö:

\laatikko[Tulon nollasääntö]{
	\begin{description}
		\item Jos jokin tulon tekijöistä on $0$, tulo on $0$.
		\item Jos tulo on $0$, ainakin yksi tulon tekijöistä on $0$.
	\end{description}
}

\begin{esimerkki}
Sievennä lauseke $(x^5-7)\cdot y \cdot 0\cdot(3a-5b)^2$.
    \begin{esimratk}
	Koska tulossa on tekijänä $0$, vastaus on $0$.
    \end{esimratk}
\end{esimerkki}

\begin{esimerkki} Ratkaistaan yhtälö $(x+5) \cdot x =0 $.
    \begin{align*}
        (x+5)\cdot x &=0 \quad \ppalkki \text{ tulon nollasääntö} \\
        x +5= 0 \text{ tai } x &=0 \\
        x= -5 \text{ tai } x &=0.
    \end{align*}
    Ratkaisuja on siis kaksi, $x= -5$ tai $x= 0$.
\end{esimerkki}

%\begin{esimerkki}
%	\[2(x+5)=0\]
%	Nyt tulon nollasäännön perusteella tiedetään, että $2=0$ tai $x+5=0$.
%	Koska selvästi $2\neq 0$, jää ainoaksi ratkaisuksi $x+5=0$ eli $x=-5$.
%\end{esimerkki}

% \begin{esimerkki} Ratkaistaan $y$ yhtälöstä
%     \[(x^5+5x+5)\cdot 0\cdot \sqrt{x^3-1} =y\]
%     Koska vasemmalla puolella yksi tulon tekijöistä on $0$, tiedämme, että tulo on $0$. Siis $y=0$.
% \end{esimerkki}

%TODO: onko alla oleva järkevä? mistä tietää, mitä muuttujia on tarkoitus ratkaista. minusta vain sekoittaa asioita T: Jokke
\begin{esimerkki} Ratkaise yhtälö
    \[xyz=0.\]
Tulon nollasäännön perusteella $x=0$, $y=0$ tai $z=0$. Nollia voi siis
olla 1--3 kappaletta.
\end{esimerkki}

\begin{tehtavasivu}

\paragraph*{Opi perusteet}

\begin{tehtava}
	Laske.
	\begin{alakohdat}
		\alakohta{$2 \cdot 3$}
		\alakohta{$2 \cdot (-3)$}
		\alakohta{$-2 \cdot 3$}
		\alakohta{$-2 \cdot (-3)$}
		\alakohta{$17 \cdot 666 \cdot 0 \cdot (-31)$}
	\end{alakohdat}
	
	\begin{vastaus}
		\begin{alakohdat}
			\alakohta{$6$}
			\alakohta{$-6$}
			\alakohta{$-6$}
			\alakohta{$6$}
			\alakohta{$0$}
		\end{alakohdat}
	\end{vastaus}
\end{tehtava}

\begin{tehtava}
	Olkoon $a>0$, $b<0$, ja $c=0$. Mitä voit päätellä tulon merkistä?
	\begin{alakohdat}
		\alakohta{$a \cdot a$}
		\alakohta{$b \cdot a$}
		\alakohta{$b \cdot b$}
		\alakohta{$b \cdot c$}
	\end{alakohdat}
	
	\begin{vastaus}
		\begin{alakohdat}
			\alakohta{tulo $>0$}
			\alakohta{tulo $<0$}
			\alakohta{tulo $>0$}
			\alakohta{tulo on $0$}
		\end{alakohdat}
	\end{vastaus}
\end{tehtava}


\begin{tehtava}
    Ratkaise seuraavat yhtälöt käyttämällä tulon nollasääntöä.
    \begin{alakohdat}
        \alakohta{$x(3+x)=0$}
        \alakohta{$(x-4)(x+3)=0$}
		\alakohta{$0=x^2(x-5)$}
    \end{alakohdat}
    \begin{vastaus}
        \begin{alakohdat}
            \alakohta{$x=0$ tai $x=-3$}
            \alakohta{$x=4$ tai $x=-3$}
            \alakohta{$x=0$ tai $x=5$}
        \end{alakohdat}
    \end{vastaus}
\end{tehtava}

\paragraph*{Hallitse kokonaisuus}

\begin{tehtava}
	Olkoon $a > 0$. Mitkä vaihtoehdoista $b>0$, $b<0$, $b=0$ ovat mahdollisia, jos
	tiedetään, että
	\begin{alakohdat}
		\alakohta{$a \cdot b > 0$}
		\alakohta{$a \cdot b \leq 0$}
		\alakohta{$b \cdot b > 0$}
		\alakohta{$b \cdot b < 0$?}
	\end{alakohdat}
	\begin{vastaus}
		\begin{alakohdat}
			\alakohta{$b>0$}
			\alakohta{$b < 0$ ja $b = 0$}
			\alakohta{$b>0$ ja $b<0$}
			\alakohta{Mikään vaihtoehto ei kelpaa.}
		\end{alakohdat}
	\end{vastaus}
\end{tehtava}

\begin{tehtava}
    Ratkaise seuraavat yhtälöt käyttämällä tulon nollasääntöä.
    \begin{alakohdat}
        \alakohta{$y(y+4)=0$}
        \alakohta{$(x-2)(x-1)(x+5)=0$}
        \alakohta{$(x+1)(x^2+2)=0$}
    \end{alakohdat}
    \begin{vastaus}
        \begin{alakohdat}
            \alakohta{$y=0$ tai $y=-4$}
            \alakohta{$x=2$, $x=1$ tai $x=-5$}
            \alakohta{$x=-1$}
        \end{alakohdat}
    \end{vastaus}
\end{tehtava}

\begin{tehtava} 
Osoita, että funktio $f(x)=x^4+3x^2+1$ saa vain positiivisia arvoja.
    \begin{vastaus}
     $x^4\geq 0$ ja $x^2 \geq 0$, joten $f(x) \geq 1$.
    \end{vastaus}
\end{tehtava}

\paragraph*{Lisää tehtäviä}

\begin{tehtava}
	Olkoon $a \geq 0$, $b \leq 0$, ja $c=0$. Mitä voit päätellä tulon merkistä?
	\begin{alakohdat}
		\alakohta{$a \cdot a$}
		\alakohta{$b \cdot a$}
		\alakohta{$b \cdot c$}
	\end{alakohdat}
	
	\begin{vastaus}
		\begin{alakohdat}
			\alakohta{tulo $\geq 0$}
			\alakohta{tulo $\leq 0$}
			\alakohta{tulo on $0$}
		\end{alakohdat}
	\end{vastaus}
\end{tehtava}

\begin{tehtava}
    Ratkaise seuraavat yhtälöt käyttämällä tulon nollasääntöä.
    \begin{alakohdat}
        \alakohta{$(\smiley{}+1)\cdot (t+1)=0$}
        \alakohta{$x(x-5)=0$}
        \alakohta{$(2w+2)^2=0$}
    \end{alakohdat}
    \begin{vastaus}
        \begin{alakohdat}
            \alakohta{$\smiley{}=-1$ tai $t=-1$,\qquad  Symboli $\smiley{}$ esittää jotain lukua, sillä muutoin laskutoimitukset eivät olisi mielekkäitä. Tehtävässä ei myöskään ole selvää, minkä muuttujan suhteen yhtälö pitäisi ratkaista. Siksi on ratkaistu molempien muuttujien suhteen.}
            \alakohta{$x=0$ tai $x=5$}
            \alakohta{$w=-1$}
        \end{alakohdat}
    \end{vastaus}
\end{tehtava}

\begin{tehtava}
    Sievennä seuraava lauseke: $(a-x)\cdot(b-x)\cdot(c-x)\cdot...\cdot(\mathring{a}-x)\cdot(\ddot{a}-x)\cdot(\ddot{o}-x)$.
    \begin{vastaus}
        Tulossa esiintyy tekijänä $(x-x)=0$. Niinpä tulon nollasäännön mukaan
        \begin{align*}
            &(a-x)\cdot(b-x)\cdot(c-x)\cdot...\cdot(x-x)\cdot...\cdot(\ddot{a}-x)\cdot(\ddot{o}-x) \\
            =&(a-x)\cdot(b-x)\cdot(c-x)\cdot...\cdot 0\cdot...\cdot(\ddot{a}-x)\cdot(\ddot{o}-x) \\
            =&0
        \end{align*}
    \end{vastaus}
\end{tehtava}



\begin{tehtava} %
$\star$ Osoita, että $x^2+\frac{1}{x^2}\geq 2$, kun $x \neq 0$.
    \begin{vastaus}
     Aloita tiedosta $\left(x-\frac{1}{x}\right)^2 \geq 0$ ja sievennä.
    \end{vastaus}
\end{tehtava}

\begin{tehtava} 
$\star$ Osoita, että kun $a \geq 0$ ja $b \geq 0$, pätee \\ $\frac{a+b}{2} \geq \sqrt{ab}$. Milloin yhtäsuuruus on voimassa?
    \begin{vastaus}
     Opastus: Aloita tiedosta $\left(\sqrt{a}-\sqrt{b}\right)^2 \geq 0$ ja sievennä. Yhtäsuuruus pätee, kun $a = b$.
    \end{vastaus}
\end{tehtava}

\end{tehtavasivu}

	\section{Tekijöihinjako}

%Pitkän matematiikan 1. kurssilla on käsitelty lukujen jakamista tekijöihin.

Luvun $12$ \termi{tekijä}{tekijät} luonnolisten lukujen joukossa ovat luvut $1$, $2$, $3$, $4$, $6$ ja $12$. Ne ovat lukuja, joista saadaan luku $12$ kertomalla ne jollain luonnollisella luvulla. Toisin sanottuna luku $12$ voidaan jakaa millä tahansa näistä luvuista ilman jakojäännöstä.

\begin{esimerkki}
Luku $12$ voidaan kirjoittaa tekijöidensä tulona monella eri tavalla. Esimerkiksi
\begin{align*}
&12 = 2 \cdot 6, \\
&12= 4 \cdot 3 \text{ tai} \\
&12= 2 \cdot 2 \cdot 3.
\end{align*}
\end{esimerkki}

\subsection{Polynomin jakaminen tekijöihin}

\qrlinkki{http://opetus.tv/maa/maa2/polynomin-jakaminen-tekijoihin/}{Opetus.tv: \emph{polynomin jakaminen tekijöihin} (9:50 ja 5:44)}

Polynomeja voidaan jakaa tekijöihin kuten lukujakin. Polynomien tapauksessa tekijöihinjako tarkoittaa polynomin esittämistä saman- tai pienempiasteisten polynomien tulona, \termi{tulomuoto}{tulomuodossa}. Aste on aina pienempi, ellei kyse ole pelkästä vakiokertoimen ottamisesta yhteiseksi tekijäksi.

\subsubsection*{Yhteinen tekijä}

Kun polynomin jokaisessa termissä on sama tekijä, se voidaan ottaa yhteiseksi tekijäksi.

\begin{esimerkki}
Jaa tekijöihin polynomi $10x^3-20x^2$.
\begin{align*}
& 10x^3-20x^2 &\emph{otetaan $10$ yhteiseksi tekijäksi}\\
=& 10(x^3-2x^2) &\emph{otetaan $x^2$ yhteiseksi tekijäksi} \\
=& 10x^2(x-2)  
\end{align*}
\end{esimerkki}

\begin{esimerkki}
Jaa tekijöihin \quad a) $x^3+x$ \quad b) $3x^2+6x.$
\begin{alakohdat}
    \alakohta{$x^3+x = x(x^2+1)$}
    \alakohta{$3x^2+6x = 3x(x+2)$.}
\end{alakohdat}
\end{esimerkki}

\subsubsection*{Muistikaavat}

\begin{esimerkki}
Jaa tekijöihin \quad a) $x^2-4$ \quad b) $x^2+8x+16.$
	\begin{esimratk}
Jaetaan polynomit tekijöihin hyödyntämällä muistikaavoja
\begin{alakohdat}
    \alakohta{$x^2-4 = x^2-2^2 = (x+2)(x-2)$}
    \alakohta{$x^2+8x+16 = x^2+ 2\cdot 4 \cdot x + 4^2 = (x+4)^2$}
\end{alakohdat}
	\end{esimratk}
\end{esimerkki}

\begin{esimerkki}
Jaa tekijöihin polynomi $5x^3-20x^2+20x$.
\begin{align*}
& 5x^3-20x^2+20x \ \ \ \ &\emph{otetaan $5$ yhteiseksi tekijäksi}\\
=& 5(x^3-4x^2+4x)&\emph{otetaan $x$ yhteiseksi tekijäksi}  \\
=& 5x(x^2-4x+4) &\emph{sovelletaan muistikaavaa} \\
=& 5x(x^2-2\cdot 2x+2^2)  \\
=& 5x(x-2)^2
\end{align*}
\end{esimerkki}

%On suositeltavaa tarkistaa itse, että yllä esitetyt tekijöihinjaot todella toimivat. Polynomien tekijöihinjaon toimivuus
%on helppoa tarkistaa -- täytyy vain laskea väitettyjen tekijöiden tulo ja katsoa, onko se alkuperäinen polynomi. Vaikka
%tarkistus onkin helppoa, tässä vaiheessa ei luultavasti vielä ole selvää, miten tekijöihinjaon voisi saada selville
%-- paitsi toisinaan arvaamalla, mutta tähän kysymykseen vastataan myöhemmin tällä kurssilla.

%Polynomien tekijöihinjako ei ole yksiselitteinen, mutta monesti hyödyllisintä on jakaa polynomi tekijöihin samoin kuin
%esimerkkitapauksissa eli niin, että ensimmäisenä on vakiotermi ja kaikissa muissa tekijäpolynomeissa korkeimman asteen
%termin kerroin on 1.
%
%Esimerkki selkeyttänee asiaa. Polynomi $6x^2+30x+36$ voidaan jakaa tekijöihin vaikkapa seuraavilla tavoilla:
%
%\begin{esimerkki}
%\qquad \\
%\begin{itemize}
%    \item $6(x+2)(x+3)$
%    \item $3(2x+4)(x+3)$
%    \item $3(x+2)(2x+6)$
%    \item $2(3x+6)(x+2)$
%    \item $(6x+12)(x+3)$
%    \item $(\frac12 x+1)(12x+36)$
%\end{itemize}
%\end{esimerkki}
%
%Kaikki nämä tavat ovat ''oikein,'' mutta lähes aina ensimmäinen muoto $6(x+2)(x+3)$ on kätevin.
%
%Toisinaan polynomeille voi löytää tekijöitä soveltamalla joitakin seuraavista keinoista:
%
%\begin{itemize}
%\item Otetaan korkeimman asteen termin kerroin yhteiseksi tekijäksi: \\
%$5x^4+3x^2+x-9 = 5(x^4+\frac{3}{5} x^2+\frac{1}{5} x-\frac{9}{5})$
%\item Otetaan $x$ tai sen potenssi yhteiseksi tekijäksi, jos mahdollista: \\
%$x^5+x^3+3x = x(x^4+x^2+3)$
%$x^7+x^6+5x^4+2x^2 = x^2(x^5+x^4+5x^2+2)$
%\item Sovelletaan muistikaavaa käänteisesti \\
%$x^2-5=x^2-\sqrt{5}^2=(x+\sqrt{5})(x-\sqrt{5})$ \\
%$x^2+8x+16=x^2+2\cdot 4x+4^2=(x+4)^2$ \\
%$x^2+x+\frac14=x^2+2\cdot \frac12 x+(\frac12)^2=(x+\frac12)^2$
%\end{itemize}

%Kaikkien polynomien tekijöihinjako ei kuitenkaan näillä menetelmillä onnistu. Myöhemmin tässä kirjassa opitaan, miten toisen asteen polynomin voidaan jakaa tekijöihin nollakohtiensa avulla.

\subsubsection*{Ryhmittely}

Seuraavassa esimerkissä tekijöihin jako on toteutettu termien ryhmittelyn avulla. Se on joissain tapauksissa näppärä tapa jakaa polynomi tekijöihin, mutta oikean ryhmittelyn keksimiseen ei ole helppoa sääntöä.

\begin{esimerkki}
Jaa tekijöihin $x^3+3x^2+x+3$.
\begin{align*}
x^3+3x^2+x+3 &=x^2(x+3)+1(x+3) \\ &=(x^2+1)(x+3)
\end{align*}
\end{esimerkki}

\begin{esimerkki}
Jaa tekijöihin $x^{11}+2x^{10}+3x+6$.
\begin{align*}
& x^{11}+2x^{10}+3x+6\\
& =x^{10}(x+2)+3(x+2)\\
&=(x^{10}+3)(x+2)
\end{align*}
\end{esimerkki}

\subsection{Yhtälön ratkaisu tekijöihin jakamalla}

Tulon nollasääntö on yksi tärkeimmistä syistä siihen, miksi polynomien tekijöihinjako on hyödyllistä.

Jos vaikkapa haluamme ratkaista yhtälön $2x^3-14x^2+32x-24=0$ ja satumme tietämään, että $2x^3-14x^2+32x-24=2(x-3)(x-2)^2$, voimme helposti päätellä, että polynomi saa arvon $0$ jos ja vain jos $x-3=0$ tai $x-2=0$. Yhtälön ainoat ratkaisut ovat siis $x=3$ ja $x=2$.

\begin{esimerkki}
Ratkaise yhtälö $x^3-2x^2=0$ tulon nollasäännön avulla.
\begin{esimratk}
\begin{align*}
x^3-2x^2 &= 0 && \ppalkki \text{ otetaan } x^2 \text{ yhteiseksi tekijäksi} \\
x^2\cdot(x-2) &= 0 && \ppalkki \text{ tulon nollasääntö} \\
x^2=0 \textrm{\quad tai}& \quad x-2=0 \\
x=0 \textrm{\quad tai}& \quad x=2 \\
\end{align*}
\end{esimratk}
\begin{esimvast}
$x=0$ tai $x=2$.
\end{esimvast}
\end{esimerkki}

%\begin{esimerkki}
%Ratkaise yhtälö $6x^3-36x^2+54x=0$ tulon nollasäännön avulla.
%\begin{esimratk}
%\begin{align*}
%6x^3-36x^2+54x &= 0 \\
%6(x^3-6x^2+9x) &= 0 && \ppalkki \text{ otetaan } 6 \text{ yhteiseksi tekijäksi} \\
%6x(x^2-6x+9) &= 0 && \ppalkki \text{ otetaan } x \text{ yhteiseksi tekijäksi} \\
%6x(x^2-2\cdot 3\cdot x+3^2)  &= 0 \\
%6x(x-3)^2 &= 0 & \\
%6x(x-3)(x-3) &= 0 && \ppalkki \text{ tulon nollasääntö}\\
%6x=0 \textrm{\quad tai}& \quad x-3=0 \\
%x=0 \textrm{\quad tai}& \quad x=3 \\
%\end{align*}
%\end{esimratk}
%\begin{esimvast}
%$x=0$ tai $x=3$.
%\end{esimvast}
%\end{esimerkki}

Myöhemmin tällä kurssilla esitellään polynomien jakolause, joka antaa syvällisemmän yhteyden polynomien nollakohtien ja tekijöiden välille.

\subsubsection*{Murtolausekkeiden sieventäminen}

Kun murtolausekkeen osoittaja ja nimittäjä jaetaan tekijöihinsä, kummassakin esiintyvät tekijät voidaan supistaa pois.

\begin{esimerkki}
    Sievennä \quad 
    a) $\dfrac{4x+2y}{6}$ \quad
    b)$\dfrac{x^2+x}{2x+2}$ \quad
    c) $\dfrac{x^2-16}{x+4}$
    \begin{esimratk}
        \begin{alakohdat}
            \alakohta{$\dfrac{4x+2y}{6}=\dfrac{2(2x+y)}{2\cdot 3}=\dfrac{2x+y}{3}$}
            \alakohta{$\dfrac{x^2+x}{2x+2}=\dfrac{x(x+1)}{2(x+1)}=\dfrac{x}{2}$}
            \alakohta{Käytetään muistikaavaa: $\dfrac{x^2-16}{x+4}=\dfrac{(x+4)(x-4)}{x+4} = x-4$.}
        \end{alakohdat}
    \end{esimratk}
    \begin{esimvast}
        a) $\dfrac{2x+y}{3}$ \quad
        b) $\dfrac{x}{2}$ \quad
        c) $x-4$.
    \end{esimvast}
\end{esimerkki}

\begin{tehtavasivu}

\paragraph*{Opi perusteet}

\begin{tehtava}
    Esitä tulona ottamalla yhteinen tekijä.
    \begin{alakohdat}
        \alakohta{$2x+6$}
        \alakohta{$x^2 -4x$}
        \alakohta{$3x^2 - 6x$}
    \end{alakohdat}
    \begin{vastaus}
        \begin{alakohdat}
        \alakohta{$2(x+3)$}
        \alakohta{$x(x-4)$}
        \alakohta{$3x(x-2)$}
        \end{alakohdat}
    \end{vastaus}
\end{tehtava}

\begin{tehtava}
    Jaa tekijöihin.
    \begin{alakohdat}
        \alakohta{$10a+5ab$}
        \alakohta{$x^4 -x^3$}
        \alakohta{$xy+x^2y$}
    \end{alakohdat}
    \begin{vastaus}
        \begin{alakohdat}
        \alakohta{$5a(2+b)$}
        \alakohta{$x^3(x-1)$}
        \alakohta{$xy(1+x)$}
        \end{alakohdat}
    \end{vastaus}
\end{tehtava}

\begin{tehtava}
    Sievennä.
    \begin{alakohdat}
        \alakohta{$\dfrac{3x-9}{3}$}
        \alakohta{$\dfrac{x^2-4x}{5x}$}
        \alakohta{$\dfrac{ab+a}{b^2+b}$}
    \end{alakohdat}
    \begin{vastaus}
        \begin{alakohdat}
        \alakohta{$x-3$}
        \alakohta{$\frac{x-4}{3}$}
        \alakohta{$\frac{a}{b}$}
        \end{alakohdat}
    \end{vastaus}
\end{tehtava}

\paragraph*{Hallitse kokonaisuus}

%\begin{tehtava}
%    Jaa tekijöihin.
%    \begin{alakohdat}
%    	\alakohta{$x^3 - x$}
%        \alakohta{$x^2 - x + \frac{1}{4}$}
%        \alakohta{$9-x^4$}
%    \end{alakohdat}
%    \begin{vastaus}
%        \begin{alakohdat}
%            \alakohta{$x(x-1)^2$}
%            \alakohta{$(x-\frac{1}{2})^2$}
%            \alakohta{$(3+x^2)(3-x^2)$}
%        \end{alakohdat}
%    \end{vastaus}
%\end{tehtava}


\begin{tehtava}
    Jaa tekijöihin.
    \begin{alakohdat}
        \alakohta{$-15x^5 +10y$}
        \alakohta{$x^3y^2 +x^2y^3$}
        \alakohta{$-4a^3 -2a^2 +2ab$}
    \end{alakohdat}
    \begin{vastaus}
        \begin{alakohdat}
        \alakohta{joko $5(-3x^5 +2y)$ tai $-5(3x^5 -2y)$}
        \alakohta{$x^2y^2(x+y)$}
        \alakohta{joko $2a(-2a^2 -a +b)$ tai $-2a(2a^2 +a -b)$}
        \end{alakohdat}
    \end{vastaus}
\end{tehtava}

\begin{tehtava}
    Jaa tekijöihin muistikaavojen avulla.
    \begin{alakohdat}
        \alakohta{$x^2+6x+9$}
        \alakohta{$y^2 - 2y+1$}
        \alakohta{$x^2 -25$}
    \end{alakohdat}
    \begin{vastaus}
        \begin{alakohdat}
        \alakohta{$(x+3)^2$}
        \alakohta{$(y-1)^2$}
        \alakohta{$(x-5)(x+5)$}
        \end{alakohdat}
    \end{vastaus}
\end{tehtava}

\begin{tehtava}
    Jaa tekijöihin ryhmittelemällä sopivasti.
    \begin{alakohdat}
        \alakohta{$x^3 +x^2 +x +1$}
        \alakohta{$a^3 +a^2b +2a +2b$}
        \alakohta{$8m^6-2m^4+4m^2-1$}
    \end{alakohdat}
    \begin{vastaus}
        \begin{alakohdat}
        \alakohta{$(x^2+1)(x+1)$}
        \alakohta{$(a^2+2)(a+b)$}
        \alakohta{$(2m^4 +1)(4m^2 -1)=(2m^4 +1)(2m+1)(2m-1)$}
        \end{alakohdat}
    \end{vastaus}
\end{tehtava}

\begin{tehtava}
    Sievennä.
    \begin{alakohdat}
        \alakohta{$\dfrac{x^3-2x^2}{2-x}$}
        \alakohta{$\dfrac{x^2+6x+9}{x^2+3x}$}
        \alakohta{$\dfrac{4-x^2}{x^2-2x}$}
    \end{alakohdat}
    \begin{vastaus}
        \begin{alakohdat}
        \alakohta{$-x^2$}
        \alakohta{$\frac{x+3}{x}$}
        \alakohta{$-\frac{x+2}{x}$}
        \end{alakohdat}
    \end{vastaus}
\end{tehtava}

\begin{tehtava}
    Jaa tekijöihin.
    \begin{alakohdat}
    	\alakohta{$x^2 -4$}
    	\alakohta{$x^2 -3$}
    	\alakohta{$5x^2 -3$}
		\alakohta{$16-x^4$}
    \end{alakohdat}
    \begin{vastaus}
        \begin{alakohdat}
            \alakohta{$(x+2)(x-2)$}
            \alakohta{$(x+\sqrt{3})(x-\sqrt{3})$}
            \alakohta{$(\sqrt{5}x+\sqrt{3})(\sqrt{5}x-\sqrt{3})$}
            \alakohta{$(4+x^2)(4-x^2)=(4+x^2)(2-x)(2+x)$}
        \end{alakohdat}
    \end{vastaus}
\end{tehtava}

\begin{tehtava}
	Jaa tekijöihin.
	\begin{alakohdat}
		\alakohta{$x^3-x$}
		\alakohta{$5ab+ b+10a+2$}
		\alakohta{$16x^2y^2+8xy+1$}
	\end{alakohdat}
	\begin{vastaus}
		\begin{alakohdat}
			\alakohta{$x(x+1)(x-1)$}
			\alakohta{$(5a+1)(b+2)$}
			\alakohta{$(4xy+1)^2$}
		\end{alakohdat}
	\end{vastaus}
\end{tehtava}

\begin{tehtava}
	Ratkaise yhtälö jakamalla tekijöihin.
	\begin{alakohdat}
		\alakohta{$x^2-16 = 0$}
		\alakohta{$x^2+7x = 0$}
		\alakohta{$x^2-6x+9 = 0$}
	\end{alakohdat}
	\begin{vastaus}
		\begin{alakohdat}
			\alakohta{$x=4$ tai $x=-4$}
			\alakohta{$x=0$ tai $x=-7$}
			\alakohta{$x=3$}
		\end{alakohdat}
	\end{vastaus}
\end{tehtava}

\begin{tehtava} 
Jaa tekijöihin \\ $(3x^2-7y^2+5)^2-(x^2-9y^2-5)^2$.
    \begin{vastaus}
		$8(x-2y)(x+2y)(x^2+y^2+7)$. \\
    Opastus: Älä kerro aluksi sulkuja auki vaan käytä heti muistikaavaa.
    \end{vastaus}
\end{tehtava}

\paragraph*{Lisää tehtäviä}

\begin{tehtava}
    Jaa tekijöihin muistikaavojen avulla.
    \begin{alakohdat}
        \alakohta{$x^2-4x+4$}
        \alakohta{$9y^2 + 6y+1$}
        \alakohta{$49-4x^2$}
    \end{alakohdat}
    \begin{vastaus}
        \begin{alakohdat}
        \alakohta{$(x-2)^2$}
        \alakohta{$(3y+1)^2$}
        \alakohta{$(7-2x)(7+2x)$}
        \end{alakohdat}
    \end{vastaus}
\end{tehtava}

\begin{tehtava}
	Ratkaise yhtälö jakamalla tekijöihin.
	\begin{alakohdat}
		\alakohta{$x^3-x^2 = 0$}
		\alakohta{$x^3+3x^2-4x-12 = 0$}
		\alakohta{$x^2-4x+4 = 4$}
	\end{alakohdat}
	\begin{vastaus}
		\begin{alakohdat}
			\alakohta{$x=0$ tai $x=1$}
			\alakohta{$x=-3$, $x=2$ tai $x=-2$}
			\alakohta{$x=0$ tai $x=4$}
		\end{alakohdat}
	\end{vastaus}
\end{tehtava}

\begin{tehtava}
	Ratkaise yhtälöt.
	\begin{alakohdat}
		\alakohta{$-x^4+4x^2=0$}
		\alakohta{$x^5-16x^3=0$}
	\end{alakohdat}
	\begin{vastaus}
		\begin{alakohdat}
			\alakohta{$x=-2$, $x=0$ tai $x=2$ (Tekijöihin jakamalla yhtälö sievenee muotoon $x^2(2+x)(2-x)=0$.)}
			\alakohta{$x=-4$, $x=0$ tai $x=4$ (Tekijöihin jakamalla yhtälö sievenee muotoon $x^3(x+4)(x-4)=0$.)}
		\end{alakohdat}
	\end{vastaus}
\end{tehtava}

\end{tehtavasivu}

	\luku{polynomifunktion_kuvaaja}{Polynomifunktion kuvaaja}

\osa{Ensimmäinen aste}
    \luku{epayhtalo}{Epäyhtälöiden teoriaa}
    \luku{ensimmaisen_asteen_yhtalo}{Kertausta: ensimmäisen asteen yhtälö}
    \luku{ensimmaisen_asteen_epayhtalo}{Ensimmäisen asteen epäyhtälö}

\osa{Toinen aste}
	\luku{paraabeli}{Toisen asteen polynomifunktio ja sen kuvaaja}
	\include{content/toisen_asteen_yhtalo}
	\section{Toisen asteen yhtälön ratkaisukaava}

\qrlinkki{http://opetus.tv/maa/maa2/toisen-asteen-yhtalon-ratkaisukaava/}{Opetus.tv: \emph{toisen asteen ratkaisukaava} (9:03, 11:06 ja 10:09)}

Edellisessä kappaleessa opittiin, että toisen asteen yhtälö voidaan aina ratkaista täydentämällä se neliöksi.
Neliöksi täydentämistä käytetään kuitenkin harvoin, sillä saman ajatuksen voi ilmaista valmiina kaavana.
Johdetaan seuraavassa toisen asteen yhtälön ratkaisukaava. \\ \\

Lähdetään liikkeelle täydellisestä toisen asteen yhtälöstä $ax^2+bx+c=0$.
\begin{align*}
ax^2+bx+c&=0 &&\textnormal{\footnotesize{kerrotaan molemmat puolet termillä}} \ 4a \\
4a \cdot ax^2+4a \cdot bx + 4a \cdot c&=0 \\
4a^2x^2+4abx+4ac&=0 &&\textnormal{\footnotesize{vähennetään puolittain termi}} \ 4ac  \\
4a^2x^2+4abx&=-4ac
\end{align*}
Täydennetään vasen puoli binomin neliöksi.
\begin{align*}
4a^2x^2+4abx&=-4ac &&\textnormal{\footnotesize{lisätään puolittain termi}} \ b^2 \\
4a^2x^2+4abx+b^2&=b^2-4ac &&\textnormal{\footnotesize{neliö:}} \ 4a^2x^2+4abx+b^2=(2ax+b)^2 \\
(2ax+b)^2&=b^2-4ac &&\textnormal{\footnotesize{otetaan puolittain neliöjuuri, jos}} \ b^2 \geq 4ac \\
2ax+b&= \pm \sqrt[]{b^2-4ac} &&\textnormal{\footnotesize{vähennetään puolittain termi}} \ b \\
2ax&=-b \pm \sqrt[]{b^2-4ac} &&\textnormal{\footnotesize{jaetaan puolittain termillä}} \ 2a \neq 0 \\
x&= \frac{-b \pm \sqrt[]{b^2-4ac}}{2a}
\end{align*}
Toisen asteen yhtälön ratkaisukaava on siis \[x= \frac{-b \pm \sqrt[]{b^2-4ac}}{2a}\] oletuksella, että $b^2 \geq 4ac$. Oletus tarvitaan, koska negatiivisille luvuille ei ole reaaliluvuilla määriteltyä neliöjuurta.\\
\laatikko{\textbf{Toisen asteen yhtälön ratkaisukaava} \\
Yhtälön
$ax^2+bx+c=0$, missä $a \neq 0$ ja $b^2 \geq 4ac$, reaaliset ratkaisut ovat
muotoa \\
\[ x=\frac{-b \pm \sqrt{b^2-4ac}}{2a}.\]
}
\begin{esimerkki}
Ratkaistaan yhtälö $x^2-8x+16=0$.
\begin{align*}
\underbrace{1}_{=a}x^2 +\underbrace{(-8)}_{=b}x+\underbrace{16}_{=c}=0
\end{align*}
Sijoitetaan vakioiden $a=1$, $b=-8$ ja $c=16$ arvot toisen asteen yhtälön
ratkaisukaavaan.
\begin{align*}
x&=\frac{-(-8)\pm \sqrt[]{(-8)^2-4\cdot 1 \cdot 16}}{2 \cdot 1} \\
x&=\frac{8 \pm \sqrt{64- 64}}{2} \\
x&=\frac{8 \pm 0}{2} \\
x&=4
\end{align*}
\end{esimerkki}

\begin{esimerkki}
Ratkaistaan yhtälö $15x^2+24x+10=0$.
\begin{align*}
\underbrace{15}_{=a}x^2+\underbrace{24}_{=b}x+\underbrace{10}_{=c}=0
\end{align*}
Sijoitetaan vakioiden $a=15$, $b=24$ ja $c=10$ arvot toisen asteen yhtälön ratkaisukaavaan.
\begin{align*}
x&=\frac{-24 \pm \sqrt[]{24^2-4 \cdot 15 \cdot 10}}{2 \cdot 15} \\
x&=\frac{-24 \pm \sqrt[]{576-600}}{30} \\
x&=\frac{-24 \pm \sqrt[]{-24}}{30}
\end{align*}
Koska juurrettava on negatiivinen ($-24<0$), niin yhtälöllä ei ole reaalilukuratkaisuja. \\
\end{esimerkki}

\begin{esimerkki}
Ratkaistaan yhtälö $x^2+2x-3=0$.
\begin{align*}
\underbrace{1}_{=a} \cdot x^2+\underbrace{2}_{=b}x\underbrace{-3}_{=c}=0
\end{align*}
Sijoitetaan vakioiden $a=1$, $b=2$ ja $c=-3$ arvot toisen asteen yhtälön ratkaisukaavaan.
\begin{align*}
x&=\frac{-2 \pm \sqrt[]{2^2-4 \cdot 1 \cdot (-3)}}{2 \cdot 1} \\
x&=\frac{-2 \pm \sqrt[]{4+12}}{2} \\
x&=\frac{-2 \pm \sqrt[]{16}}{2} \\
x&=\frac{-2 \pm 4}{2} \\
x&=-1 \pm 2 \\
x&=1 \text{ tai } x=-3 \\
\end{align*}
\end{esimerkki}

%\begin{esimerkki}
%Ratkaistaan yhtälö $-\sqrt{2}x+\frac{1}{2}=x^2$.
%\end{esimerkki}

%Yleinen toisen asteen yhtälö on muotoa $ax^2+bx+c=0$.
%Kerrotaan yhtälön molemmat puolet vakiolla $4a$: $4a^2x^2+4abx+4ac=0$.
%Siirretään termi $4ac$ toiselle puolelle: $4a^2x^2+4abx=-4ac$.
%Pyritään täydentämään vasen puoli neliöksi.
%Lisätään puolittain termi $b^2$: $4a^2x^2+4abx+b^2=b^2-4ac$.
%Havaitaan vasemmalla puolella neliö: $(2ax+b)^2=b^2-4ac$.
%Otetaan puolittain neliöjuuri: $2ax+b=\pm\sqrt{b^2-4ac}$.
%Vähennetään puolittain termi $b$: $2ax=-b\pm\sqrt{b^2-4ac}$.
%Jaetaan puolittain vakiolla $2a$: $x=\frac{-b\pm\sqrt{b^2-4ac}}{2a}$.

\begin{tehtavasivu}

\paragraph*{Opi perusteet}

\begin{tehtava}
    Ratkaise
    \begin{alakohdat}
        \alakohta{$x^2 - 2x - 3 = 0$}
        \alakohta{$-x^2 - 6x - 5 = 0$}
        \alakohta{$x + 2x^2 - 6= 0$}
        \alakohta{$1 + x + 3x^2= 0$.}
    \end{alakohdat}
    \begin{vastaus}
        \begin{alakohdat}
            \alakohta{$x = 3 \tai x = -1$}
            \alakohta{$x = -5 \tai x = -1$}
            \alakohta{$x = \frac{3}{2} \tai x = -2$}
            \alakohta{Ei ratkaisuja.}
        \end{alakohdat}
    \end{vastaus}
\end{tehtava}

\begin{tehtava}
    Ratkaise
    \begin{alakohdat}
        \alakohta{$9x^2 - 12x + 4 = 0$}
        \alakohta{$x^2 + 2x = -4$}
        \alakohta{$4x^2 = 12x - 8$}
        \alakohta{$3x^2 - 13x + 50 = -2x^2 + 17x + 5$.}
    \end{alakohdat}
    \begin{vastaus}
        \begin{alakohdat}
            \alakohta{$x = \dfrac{2}{3}$}
            \alakohta{Ei ratkaisua.}
            \alakohta{$x = 1$ tai $x = 2$}
            \alakohta{$x = 3$}
        \end{alakohdat}
    \end{vastaus}
\end{tehtava}

\begin{tehtava}
    Ratkaise
    \begin{alakohdat}
        \alakohta{$9x^2 - 15x + 6 = 0$}
        \alakohta{$x^2 + 23x = 0$}
        \alakohta{$4x^2 - 64 = 0$}
        \alakohta{$6x^2 + 18x + 1 = 0$.}
    \end{alakohdat}
    \begin{vastaus}
        \begin{alakohdat}
            \alakohta{$x = 1 \tai x = \frac{2}{3}$}
            \alakohta{$x = 0 \tai x = -23$}
            \alakohta{$x = 4 \tai x = -4$}
            \alakohta{$x = \frac{-9 \pm 5\sqrt{3}}{6}$}
        \end{alakohdat}
    \end{vastaus}
\end{tehtava}

\begin{tehtava}
    Suorakulmaisen muotoisen alueen piiri on $34$~m ja pinta-ala $60$~m$^2$. Selvitä alueen mitat.
    \begin{vastaus}
		Alueen toinen sivu on $5$ m ja toinen $12$ m.
    \end{vastaus}
\end{tehtava}

\paragraph*{Hallitse kokonaisuus}

\begin{tehtava}
    Ratkaise
    \begin{alakohdat}
		\alakohta{$-x^2 + 4x + 7 = 0$}
		\alakohta{$x^2 - 13x + 1 = 0$}
		\alakohta{$4x^2 - 3x - 5 = 0$}
		\alakohta{$\frac{5}{6} x^2 + \frac{4}{7} x - 1 = 0$.}
    \end{alakohdat}
    \begin{vastaus}
        \begin{alakohdat}
			\alakohta{$x = 2\pm \sqrt{11}$}
			\alakohta{$x = \frac{13\pm \sqrt{165}}{2}$}
			\alakohta{$x = \frac{3\pm \sqrt{89}}{8}$}
			\alakohta{$x = \frac{-12\pm \sqrt{1614}}{35}$}
        \end{alakohdat}
    \end{vastaus}
\end{tehtava}

\begin{tehtava}
    Kahden luvun summa on $8$ ja tulo $15$. Määritä luvut.
    \begin{vastaus}
		Luvut ovat $3$ ja $5$.
    \end{vastaus}
\end{tehtava}

\begin{tehtava}
    Suorakulmion muotoisen talon mitat ovat $6,0~\text{m} \times 10,0$~m.
	Talon kivijalan ympärille halutaan levittää tasalevyinen sorakerros. Kuinka 
    leveä kerros saadaan, kun soraa riittää 20~$\text{m}^2$ alalle?
    \begin{vastaus}
		$58$ cm levyinen
    \end{vastaus}
\end{tehtava}

\begin{tehtava}
	Sivusta katsottuna pallon lentorata on muotoa $y=-x^2+15x-36$. Oletetaan, että pallo heitettiin korkeudelta $y=0$. Laske
		\begin{alakohdat}
			\alakohta{heiton pituus}
			\alakohta{mistä heitettiin}
			\alakohta{mihin pallo laskeutui}
		\end{alakohdat}
	\begin{vastaus}
		\begin{alakohdat}
			\alakohta{9 yksikköä}
			\alakohta{kohdasta $x=3$}
			\alakohta{kohtaan $x=12$}
		\end{alakohdat}
	\end{vastaus}
\end{tehtava}

\begin{tehtava}
    Tasaisesti kiihtyvässä liikkeessä on voimassa kaavat $v = v_0 + at$ ja $s = v_0t + \dfrac{1}{2}at^2$, missä $v$ on loppunopeus, $v_0$ alkunopeus, $a$ kiihtyvyys, $t$ aika ja $s$ siirtymä.
		\begin{alakohdat}
            \alakohta{Auton nopeus on $20$~m/s. Auto pysäytetään jarruttamalla tasaisesti. Se pysähtyy $10$ sekunnissa. Laske jarrutusmatka.}
            \alakohta{Kivi heitetään suoraan alas $50$ metriä syvään rotkoon nopeudella $3,0$~m/s. Kuinka monen sekunnin kuluttua se kohtaa rotkon pohjan? Putoamiskiihtyvyys on
            noin $10$~m/$\text{s}^2$}
        \end{alakohdat}
    \begin{vastaus}
        \begin{alakohdat}
            \alakohta{Jarrutusmatka on $100$ metriä.}
            \alakohta{Noin $2,9$ sekunnin kuluttua.}
        \end{alakohdat}
    \end{vastaus}
\end{tehtava}


\begin{tehtava}
Ratkaise yhtälö $(4t+1)x^2-8tx+(4t-1)=0$ vakion $t$ kaikilla reaaliarvoilla.
	\begin{vastaus} \begin{tabular}{l}
		Kun $t=-\frac{1}{4}$, toisen asteen termin kerroin on $0$, ja ainoa ratkaisu on
	 $x = 1$  \\
		Kun $t \neq \frac{1}{4}$ kyseessä on toisen asteen yhtälö ja ratkaisu on $x= 1$ tai $x=\frac{4t-1}{4t+1}$. 	
		\end{tabular}
    \end{vastaus}
\end{tehtava}


\begin{tehtava}
	Viljami sijoittaa 1000 € korkorahastoon, jossa korko lisätään pääomaan vuosittain. Rahasto perii aina koronmaksun yhteydessä
	20 euron vuosittaisen hoitomaksun, joka vähennetään summasta koronlisäyksen jälkeen. Viljami laskee, että hän saisi yhteensä $2,4\%$ lisäyksen
	pääomaansa kahden vuoden aikana.
        \begin{alakohdat}
            \alakohta{Mikä on rahaston korkoprosentti? Ilmoita tarkka arvo ja likiarvo mielekkäällä tarkkuudella.}
            \alakohta{Paljonko rahaa Viljamin pitäisi sijoittaa, että hänen sijoituksensa kasvaisi yhteensä $5,0\%$ kahden vuoden aikana?}
        \end{alakohdat}
	\begin{vastaus}
	    \begin{alakohdat}
		\alakohta{
		Merkitään korkokerrointa $x$:llä.
		$$(1000x -20)x-20=1,024\cdot 1000$$
		$$x = \dfrac{1+\sqrt{10441}}{100} \approx 1.03181211580212$$
		Vastaus: $3,2\%$
		}
		\alakohta{
		Merkitään Viljamin sijoittamaa summaa $a$:lla.
		\begin{flalign*}
		(ax -20)x-20 &= 1,050a\\
		a(x^2 -1,050) &= 20x+20\\
		a &= \dfrac{20x-20}{x^2-1,050} \approx 2776,41224
		\end{flalign*}
		Vastaus: 2800 €
		}
	    \end{alakohdat}
	\end{vastaus}
\end{tehtava}


\begin{tehtava}
	Johda neliöksi täydentämällä ratkaisukaava yhtälölle
	\[ x^2 +px+q=0. \]
	Tarkista sijoittamalla tavanomaiseen ratkaisukaavaan.
	\begin{vastaus}
		$x=\frac{-p \pm \sqrt{p^2-4q}}{2}$.
	\end{vastaus}
\end{tehtava}

\begin{tehtava} % HANKALA! (Joo, tykkään. -Ville)
	$\star$ Ratkaise yhtälö $(x^2-2)^6=(x^2+4x+4)^3$.
	\begin{vastaus}
		$x=-1$, $x=0$ tai $x=\frac{1 \pm \sqrt{17}}{2}$
	\end{vastaus}
\end{tehtava}


\paragraph*{Lisää tehtäviä}

\begin{tehtava}
    Ratkaise seuraavat yhtälöt.
    \begin{alakohdat}
        \alakohta{$x^2+3x+2=0$}
        \alakohta{$2x^2+5x-12=0$}
        \alakohta{$3x^2-7x-20=0$}
        \alakohta{$x^2+3x-5=0$}
        \alakohta{$x^2+5x-24=0$}
    \end{alakohdat}
    \begin{vastaus}
        \begin{alakohdat}
            \alakohta{$x=-2$ tai $x=-1$}
            \alakohta{$x=3/2$ tai $x=-4$}
            \alakohta{$x=4$ tai $x=-5/3$}
            \alakohta{$x=\frac{3\pm\sqrt{29}}{2}$}
            \alakohta{$x=3$ tai $x=-8$}
        \end{alakohdat}
    \end{vastaus}
\end{tehtava}

\begin{tehtava}
    Ratkaise seuraavat yhtälöt.
    \begin{alakohdat}
        \alakohta{$x^2+3x-5=4x+8$}
        \alakohta{$8x^2-5x+1=-36$}
        \alakohta{$-3x^2-4x+2=-5x^2+3$}
        \alakohta{$-3x^2+4x+13=-5x^2+10x+9$}
    \end{alakohdat}
    \begin{vastaus}
        \begin{alakohdat}
            \alakohta{$\frac{1\pm\sqrt{53}}{2}$}
            \alakohta{Ei ratkaisua reaalilukujen joukossa.}
            \alakohta{$1\pm\frac{\sqrt{6}}{2}$}
            \alakohta{$x=1$ tai $x=2$}
        \end{alakohdat}
    \end{vastaus}
\end{tehtava}

\begin{tehtava}
    Ratkaise
    \begin{alakohdat}
		\alakohta{$-\frac{5}{7} x^2 + \frac{4}{11} x - \frac{1}{2} = 0$}
		\alakohta{$\frac{2}{3} x^2 - \frac{18}{5} x + \frac{3}{10} = 0$}
	\end{alakohdat}
    \begin{vastaus}
        \begin{alakohdat}
			\alakohta{Ei ratkaisuja.}
			\alakohta{$\frac{27 \pm \sqrt{684}}{10} = \frac{27 \pm 6 \sqrt{19}}{10}$}
        \end{alakohdat}
    \end{vastaus}
\end{tehtava}

\begin{tehtava}
(K93/T5) Ratkaise yhtälö
        $\frac{2x+a^2-3a}{x-1}=a$ vakion $a$ kaikilla reaaliarvoilla.
\begin{vastaus}
        \begin{enumerate}
         \item{$x=a$, jos $a \neq 2$ ja $a \neq 1$}
         \item{$x\neq 1$, jos $a=2$}
         \item{ei ratkaisua, jos $a=1$}
        \end{enumerate}
    \end{vastaus}
\end{tehtava}

\begin{tehtava}
    Kultaisessa leikkauksessa jana on jaettu siten, että pidemmän osan suhde lyhyempään on sama kuin koko janan suhde pidempään osaan. Tämä suhde ei riipu koko janan pituudesta ja sitä merkitään yleensä kreikkalaisella aakkosella fii eli $\varphi$. Kultaista leikkausta on taiteessa kautta aikojen pidetty ''jumalallisena suhteena''.
		\begin{alakohdat}
            \alakohta{Laske kultaiseen leikkauksen suhteen $\varphi$ tarkka arvo ja likiarvo.}
            \alakohta{Napa jakaa ihmisvartalon pituussuunnassa suunnilleen kultaisen leikkauksen suhteessa. Millä korkeudella napa on $170$~cm pitkällä ihmisellä?}
        \end{alakohdat}
    \begin{vastaus}
        \begin{alakohdat}
            \alakohta{$ \varphi = \dfrac{\sqrt{5}+1}{2} \approx 1,618$}
            \alakohta{Noin $105$ cm korkeudella.}
        \end{alakohdat}
    \end{vastaus}
\end{tehtava}

\begin{tehtava}
(K96/T2b) Yhtälössä $x^2-2ax+2a-1=0$ korvataan luku $a$ luvulla $a+1$. Miten muuttuvat yhtälön juuret?
\begin{vastaus}
     Toinen kasvaa kahdella ja toinen ei muutu.
    \end{vastaus}
\end{tehtava}

\begin{tehtava}
	$\star$ Ratkaise yhtälö $(x^3-2)^2+x^3-2=2$.
	\begin{vastaus}
		Kirjoitetaan yhtälö muotoon $(x^3-2)^2+(x^3-2)-2=0$ ja sovelletaan ratkaisukaavaa.
		Näin saadaan $x^3-2=1$ tai $x^3-2=-2$, joista voidaan edelleen ratkaista $x=\sqrt[3]{3}$ tai $x=0$.
	\end{vastaus}
\end{tehtava}

\end{tehtavasivu}

	\include{content/diskriminantti}
	\include{content/toisen_asteen_epayhtalo}
	\section{Polynomien jakolause}

\qrlinkki{http://opetus.tv/maa/maa2/polynomi-tekijoihin-nollakohtien-avulla/}{Opetus.tv: \emph{polynomin jakaminen tekijöihin nollakohtiensa avulla} (6:48)}

Polynomin $P(x)=(x-2)(x-3)$ nollakohdat ovat tulon nollasäännön nojalla $x=2$ ja $x=3$. Nollakohdat voi siis nähdä tekijöihin jaetusta polynomista suoraan. Tämä  yhteys toimii myös toisin päin: nollakohtien avulla voi selvittää polynomin tekijät.

Yleisesti on voimassa polynomien jakolause:

\laatikko{\textbf{Polynomien jakolause} \\
Jos $x=b$ on polynomin nollakohta, $x-b$ on polynomin tekijä.}

Polynomien jakolauseen todistus jätetään kurssiin 12.
%Polynomien jakolauseen todistus on hahmoteltu liitteessä
%\ref{tod:poljako}.

\begin{esimerkki}
Jaa tekijöihin polynomi $P(x)=x^2-3x+2$.
\begin{esimratk}
Ratkaistaan ensin polynomin nollakohdat.
\begin{align*}
x^2-3x+2&=0 \\
x&=\frac{-(-3) \pm \sqrt[]{(-3)^2-4 \cdot 1 \cdot 2}}{2 \cdot 1} \\
x&=\frac{3 \pm \sqrt[]{9-8}}{2} \\
x&=\frac{3 \pm 1}{2} \\
x&=1 \textrm{ tai } x = 2.
\end{align*}
Nollakohtien perusteella $(x-1)$ ja $(x-2)$ ovat polynomin $P(x)$ tekijöitä.
Tarkistetaan:
 $(x-1)(x-2)=x^2-2x-x+2= x^2-3x+2$.
\end{esimratk}
\begin{esimvast}
$x^2-3x+2 = (x-1)(x-2)$.
\end{esimvast}
\end{esimerkki}

Ensimmäisen asteen tekijöiden lisäksi saatetaan tarvita vakiokerroin:

\begin{esimerkki}
Jaa polynomi $-2x^2-x+1$ tekijöihinsä.
\begin{esimratk}
Ratkaistaan nollakohdat:
\begin{align*}
-2x^2-x+1&=0 \\
x&=\frac{-(-1) \pm \sqrt[]{(-1)^2-4 \cdot (-2) \cdot 1}}{2 \cdot (-2)} \\
x&=\frac{1 \pm \sqrt[]{1+8}}{-4} \\
x&=\frac{1 \pm 3}{-4} \\
x&=-1 \textrm{ tai } x = \frac{1}{2}.
\end{align*}
Jakolauseen mukaan $(x-\frac{1}{2})$ ja $(x-(-1))$ eli$(x+1)$ ovat kyseisen polynomin tekijöitä.
Ne keskenään kertomalla ei kuitenkaan saada oikeaa tulosta:
$$\left(x-\frac{1}{2}\right)(x+1)=x^2+\frac{1}{2}x-\frac{1}{2}.$$
Puuttuu vielä korkeimman asteen termin kerroin $-2$. Sillä kertomalla saadaan alkuperäinen polynomi:
$-2\left(x-\frac{1}{2}\right)(x+1)=-2x^2-x+1$.
\end{esimratk}
\begin{esimvast}
$-2x^2-x+1 = -2(x-\frac{1}{2})(x+1)$.
\end{esimvast}
\end{esimerkki}

\subsubsection*{Toisen asteen polynomin tekijät}

Polynomien jakolauseen mukaan

\laatikko{Jos toisen asteen polynomin $ax^2+bx+c$ nollakohdat ovat $x_1$ ja $x_2$,
\[ ax^2+bx+c=a(x-x_1)(x-x_2). \]}
Huomaa, että kerroin $a$ on edellisessä yhtälössä kummallakin puolella sama.
(Muuten korkeimman asteen termit eivät täsmää.)

\begin{esimerkki}
Jaetaan tekijöihin $P(x)=2x^2 + 4x-30$. \\
Ratkaistaan nollakohdat yhtälöstä $$2x^2 + 4x-30=0$$ toisen asteen yhtälön ratkaisukaavalla.
Nollakohdat ovat $x_1=3$ ja $x_2=-5$. Saadaan siis
$$P(x)= 2(x-3)(x-(-5)) = 2(x-3)(x+5).$$
\end{esimerkki}

\begin{esimerkki}
Toisen asteen polynomin $P$ nollakohdat ovat $x=1$ ja $x=3$. Lisäksi $P(2)=3$.
Määritä $P(x)$.
\begin{esimratk}
Koska polynomin nollakohdat ovat $x_1=1$ ja $x_2=3$, polynomi on muotoa
\begin{align*} P(x)=a(x-1)(x-3). \end{align*}
Lisäksi tiedetään $P(2)=3$, joten saadaan
\begin{align*}
P(2) = a(2-1)(2-3) &= 3 \\
	a \cdot 1 \cdot (-1) &= 3	&& \ppalkki : (-1) \\
	a = -3.
\end{align*}
\end{esimratk}
\begin{esimvast}
$P(x)=-3(x-1)(x-3)$.
\end{esimvast}
\end{esimerkki}

Jos toisen asteen polynomilla on vain yksi nollakohta, kyseessä on niin sanottu kaksinkertainen juuri. Voidaan tulkita, että nollakohdat $x_1$ ja $x_2$ ovat yhtäsuuret. Tällöin tekijöiksi saadaan $a(x-x_1)(x-x_1)=a(x-x_1)^2$.

\begin{esimerkki}
Jaa tekijöihin $P(x)=2x^2-4x+2$.
\begin{esimratk}
Ratkaistaan nollakohdat:
\begin{align*}
2x^2-4x+2 &= 0	\\
x &= \frac{4\pm \sqrt{(-4)^2-4\cdot 2 \cdot 2}}{2\cdot 2} \\
x &= \frac{4 \pm 0}{4} = 1.
\end{align*}
Yhtälöllä on vain yksi ratkaisu, joten se on kaksoisjuuri.
Polynomi voidaan siis jakaa tekijöihin seuraavasti: \\ $P(x)=2(x-1)(x-1)=2(x-1)^2$. 
\end{esimratk}
\begin{esimvast}
$P(x)=2(x-1)^2$.
\end{esimvast}
\end{esimerkki}

Jos toisen asteen polynomilla ei ole nollakohtia, sitä ei voi jakaa ensimmäisen asteen tekijöihin. (Sillä ensimmäisen asteen tekijällä on aina nollakohta.) Esimerkiksi polynomia $$x^2+4$$ ei voi jakaa tekijöihin. %kompleksiluvuila tämä kyllä onnistuu)


\begin{tehtavasivu}

\paragraph*{Opi perusteet}

\begin{tehtava}
	Ratkaise yhtälöt, ja jaa vastaavat polynomifunktiot (joiden nollakohdat olet juuri laskenut) tekijöihinsä.
	\begin{alakohdat}
		\alakohta{$x^2+6x-6=0$}
		\alakohta{$-8x^2+10x-2=0$}
		\alakohta{$x^2-4x+4=0$}
	\end{alakohdat}
	\begin{vastaus}
		\begin{alakohdat}
			\alakohta{nollakohdat $x=1$ ja $x=-6$, $P(x)=(x-1)(x+6)$}
			\alakohta{nollakohdat $x= \frac {1}{4}$ ja $x=1$, $P(x)=-2(4x-1)(x-1)$}
				\alakohta{nollakohta $x=2$, $P(x)=(x-2)^2$}
		\end{alakohdat}
	\end{vastaus}
\end{tehtava}

\begin{tehtava}
Jaa tekijöihin
\begin{alakohdat}
\alakohta{$x^2-2x-15$}
\alakohta{$\frac{1}{3}x^2-2x+3$}
\alakohta{$4x^2+2x+2$}
\end{alakohdat}
\begin{vastaus}
\begin{alakohdat}
\alakohta{$(x-5)(x+3)$}
\alakohta{$\frac{1}{3}(x-3)^2$}
\alakohta{$2(2x^2+x+1)$, ei jakaudu 1. asteen tekijöihin.}
\end{alakohdat}
\end{vastaus}
\end{tehtava}

\begin{tehtava}
    Millä seuraavista polynomeista on yhteisiä tekijöitä?

    \begin{kuvaajapohja}{1.5}{-1.5}{2.5}{-3.5}{1.5}
	\kuvaaja{(x-1)*(x+1)}{$P(x)$}{red}
	\kuvaaja{2*(x-2)*(x+0.5)}{$Q(x)$}{blue}
	\kuvaaja{0.25*(x-2)*(x-1)*(x+1)}{$R(x)$}{black}
	\kuvaaja{-(x-0.25)*(x-1.5)*(x+0.75)}{$S(x)$}{black}
    \end{kuvaajapohja}
    \begin{vastaus}
	$P(x)$:llä ja $R(x)$:llä on kaksi yhteistä tekijää, koska on kaksi kohtaa, jossa molemmat saavat arvon nolla. Vastaavasti $Q(x)$:llä ja $R(x)$:llä on yksi yhteinen tekijä. $S(x)$:llä ei ole yhteisiä tekijöitä minkään muun polynomin kanssa.
    \end{vastaus}
\end{tehtava}

\paragraph*{Hallitse kokonaisuus}

\begin{tehtava}
    Jaa tekijöihin helpoimmalla tavalla
    \begin{alakohdat}
        \alakohta{$x^2-9$}
        \alakohta{$4x^2-4x+1$}
        \alakohta{$4x^2-4x-8$}
    \end{alakohdat}
    \begin{vastaus}
    	\begin{alakohdat}
        \alakohta{$(x+3)(x-3)$ (muistikaava)}
        \alakohta{$(2x-1)^2$ (muistikaava)}
        \alakohta{$4(x-2)(x+1)$}
        \end{alakohdat}
    \end{vastaus}
\end{tehtava}

\begin{tehtava}
    Toisen asteen polynomille $P$ pätee $P(-3)=P(4)=0$ ja $P(1)=12$. Ratkaise $P(x)$.
    \begin{vastaus}
        $P(x)=-(x+3)(x-4)=-x^2+x+12$
    \end{vastaus}
\end{tehtava}

\begin{tehtava}
    Osoita, että jos toisen asteen polynomin toisen asteen termin kerroin on $1$, niin sen vakiotermi on yhtä suuri kuin sen nollakohtien tulo.
    \begin{vastaus}
        Kirjoitetaan polynomi tekijämuodossa ja kerrotaan auki: $(x-a)(x-b)=x^2-2(a+b)x+ab$. Nyt syntyneen polynomin vakiotermi on $ab$.
    \end{vastaus}
\end{tehtava}

\begin{tehtava}
    Toisen asteen polynomille $P$ pätee $P(0)=P(5)=3$ ja $P(3)=-27$. Ratkaise $P(x)$.
    \begin{vastaus}
        $P(x)=5x(x-5)+3=5x^2-25x+3$
    \end{vastaus}
\end{tehtava}

\begin{tehtava}
    Paraabelin kuvaajia katsomalla voidaan huomata, että paraabelin huippu löytyy aina nollakohtien puolivälistä. (Tarkempi perustelu saadaan esimerkiksi kurssilla MAA4.) Symmetrian nojalla huipun $x$-koordinaatti on siis nollakohtien $x$-koordinaattien keskiarvo.
    \begin{alakohdat}
        \alakohta{Etsi paraabelin $-10x^2+5x+5$ huipun $x$- ja $y$-koordinaatit.}
        \alakohta{Johda lauseke yleisen paraabelin $ax^2+bx+c$ huipun $x$-koordinaatille.}
        % myös y-koordinaattia voisi kysyä
    \end{alakohdat}
    \begin{vastaus}
        \begin{alakohdat}
            \alakohta{$x=\frac14$ ja $y=5\frac58$.}
            \alakohta{$x=-\frac{b}{2a}$} % ja $y=c-\frac{b^2}{4a}$
        \end{alakohdat}
    \end{vastaus}
\end{tehtava}

\paragraph*{Lisää tehtäviä}

\begin{tehtava}
    Jaa tekijöihin.
    \begin{alakohdat}
        \alakohta{$4x^2 +4x +1$}
        \alakohta{$4x^2 +4x +4$}
        \alakohta{$9-x^2$}
    \end{alakohdat}
    \begin{vastaus}
        \begin{alakohdat}
        \alakohta{$(2x+1)^2$}
        \alakohta{$4(x^2 +x +1)$}
        \alakohta{$(3+x)(3-x)$}
        \end{alakohdat}
    \end{vastaus}
\end{tehtava}

\begin{tehtava}
	Jaa tekijöihin.
	\begin{alakohdat}
		\alakohta{$x^2-x-6$} % \{[ \star ]\}
		\alakohta{$(x-4)^2-9$} %  \{[ \star ]\}
	\end{alakohdat}
	\begin{vastaus}
		\begin{alakohdat}
			\alakohta{$(x-3)(x+2)$}
			\alakohta{$(x-1)(x-7)$}
		\end{alakohdat}
	\end{vastaus}
\end{tehtava}

\begin{tehtava}
Jaa tekijöihin
\begin{alakohdat}
\alakohta{$5x^2+10x-15$}
\alakohta{$12x^2-10x+2$}
\alakohta{$-8x^2+8x+6$}
\end{alakohdat}
\begin{vastaus}
\begin{alakohdat}
\alakohta{$5(x+3)(x-1)$}
\alakohta{$2(2x-1)(3x-1)$}
\alakohta{$-2(2x-3)(2x+1)$}
\end{alakohdat}
\end{vastaus}
\end{tehtava}

\begin{tehtava}
    Jaa tekijöihin
    \begin{alakohdat}
        \alakohta{$-10x^2+5x+5$}
        \alakohta{$8x^3-12x^2+4x$}
    \end{alakohdat}
    \begin{vastaus}
    	\begin{alakohdat}
        \alakohta{$-5(2x+1)(x-1)$}
        \alakohta{$(4x)(2x-1)(x-1)$}
        \end{alakohdat}
    \end{vastaus}
\end{tehtava}

\begin{tehtava}
    Kolmannen asteen polynomille $P$ pätee $P(-1)=P(2)=P(3)=0$ ja $P(1)=-8$. Ratkaise $P(x)$.
    \begin{vastaus}
        $P(x)=-2(x+1)(x-2)(x-3)=-2x^3+8x^2-2x-2$
    \end{vastaus}
\end{tehtava}

\begin{tehtava}
   Määritä toisen asteen yhtälö jonka juuret ovat yhtälön $ x^2+7x+49 =0 $ 
 \begin{alakohdat}
    	\alakohta{juurien käänteisluvut}
        \alakohta{juurien neliöiden käänteisluvut}
    \end{alakohdat}
    \begin{vastaus}
        \begin{alakohdat}
            \alakohta{$k(49x^2+7x+1)=0$}
            \alakohta{$k(7x^2 \pm x\sqrt{7} +1)=0$}
        \end{alakohdat}
    \end{vastaus}
\end{tehtava}

\begin{tehtava}
    $\star $ Tutki, onko seuraava väite totta vai ei: Jos polynomilla on nollakohta $x=1$, sen kerrointen summa on $0$. 
    \begin{vastaus}
        Totta, sillä polynomin $P(x)$ kerrointen summa on $P(1)$. (Koska $1^n=1$.)
    \end{vastaus}
\end{tehtava}

\end{tehtavasivu}
	
\osa{Korkeampi aste}
	\section{$\star$ Korkeamman asteen polynomifunktio}

\qrlinkki{http://opetus.tv/maa/maa2/n-asteinen-polynomifunktio/}{Opetus.tv: \emph{N-asteinen polynomifunktio} (10:40)}

Kaikki paraabelit ovat samanlaisia, mutta korkeamman asteen polynomien kuvaajat
eivät ole. Yleisesti pätee kuitenkin seuraava
lause.


\laatikko[Polynomien ominaisuuksia]{
\begin{itemize}
\item Asteen $n$ polynomilla on korkeintaan $n$ nollakohtaa.
\item Jos polynomin aste on pariton, sillä on vähintään yksi nollakohta.
% ??
%\item Kaikki polynomit voidaan jakaa tekijöihin, jotka ovat korkeintaan toista astetta. 
\end{itemize}}

Todistetaan näistä ensimmäinen tulos:

\begin{todistus}
Jos $n$ asteen polynomilla $P$ olisi yli $n$ eri nollakohtaa, sillä olisi yli $n$ ensimmäisen asteen
tekijää. Polynomin $P$ aste olisi siis yli $n$, mikä on ristiriita. Nollakohtia on siis
korkeintaan $n$.
\end{todistus}

Toista tulosta ei todisteta tässä täsmällisesti, mutta valoitetaan asiaa esimerkin kautta. Tutkitaan esimerkiksi polynomia 
$$P(x)=x^3+bx^2+cx+d.$$
Polynomia on kätevintä tarkastella muodossa, jossa $x^3$ on otettu yhteiseksi tekijäksi:
$$P(x) = x^3\left(1+\frac{a}{x}+\frac{b}{x^2}+\frac{d}{x^3}\right)$$
Kun $x$ on hyvin suuri positiivinen tai hyvin pieni negatiivinen luku,
termit $\frac{b}{x}$, $\frac{c}{x^2}$ ja $\frac{d}{x^3}$ ovat hyvin pieniä, eli
\begin{align*}
P(x)&= x^3\left(1+\frac{b}{x}+\frac{c}{x}+\frac{d}{x^3}\right) \\
	& \approx  x^3\left(1+0+0+0\right) = x^3
\end{align*}
Voidaan siis päätellä, että riippumatta kertoimista $b$, $c$, $d$ polynomin $P$
arvo on positiivinen, kun $x$ on suuri positiivinen luku ja negatiivinen, kun
$x$ on pieni negatiivinen luku.

Koska $P$ saa sekä positiivisia että negatiivia arvoja, sillä on jossakin niiden
välissä nollakohta. (Tämän takaa jatkuvuus, josta lisää kurssilla 7.) Esimerkin mukaisilla kolmannen asteen polynomeilla on siis aina
nollakohta. Yleisesti pätee, että kaikilla paritonasteisille polynomeilla on ainakin yksi nollakohta.

\begin{esimerkki} Kolmannen asteen polynomilla on 1--3 nollakohtaa.

\begin{lukusuora}{-2.5}{3}{3.6}
\lukusuoraisobbox
\lukusuorakuvaaja{(x**3-x-1)/2}
\lukusuorapienipiste{1.32472}{}
\end{lukusuora}
\begin{lukusuora}{-2.8}{2.5}{3.6}
\lukusuoraisobbox
\lukusuorakuvaaja{(x**3-x+0.3849)/2}
\lukusuorapienipiste{-1.1547}{}
\lukusuorapienipiste{0.577028}{}
\end{lukusuora}
\begin{lukusuora}{-2}{2}{3.6}
\lukusuoraisobbox
\lukusuorakuvaaja{1.4*(x**3-x)}
\lukusuorapienipiste{-1}{}
\lukusuorapienipiste{0}{}
\lukusuorapienipiste{1}{}
\end{lukusuora}

\end{esimerkki}


\begin{esimerkki} Neljännen asteen polynomilla on 0--4 nollakohtaa.

\begin{lukusuora}{-4}{4}{3.6}
\lukusuorabboxy{-0.5}{1.5}
\lukusuorakuvaaja{(x**4-5*x**2+12)/14}
\end{lukusuora}
\begin{lukusuora}{-4}{4}{3.6}
\lukusuorabboxy{-0.5}{1.5}
\lukusuorakuvaaja{(x**4-5*x**2+3*x+11.2)/14}
\lukusuorapienipiste{-1.71394}{}
\end{lukusuora}
\begin{lukusuora}{-4}{4}{3.6}
\lukusuorabboxy{-0.5}{1.5}
\lukusuorakuvaaja{(x**4-5*x**2-3)/14}
\lukusuorapienipiste{2.354}{}
\lukusuorapienipiste{-2.354}{}
\end{lukusuora}

\begin{lukusuora}{-4}{4}{3.6}
\lukusuorabboxy{-0.5}{1.5}
\lukusuorakuvaaja{(x**4-5*x**2+3*x+1.75842)/14}
\lukusuorapienipiste{-2.43622}{}
\lukusuorapienipiste{-0.367327}{}
\lukusuorapienipiste{1.402}{}
\end{lukusuora}
\begin{lukusuora}{-2.8}{2.8}{3.6}
\lukusuorabboxy{-0.5}{1.5}
\lukusuorakuvaaja{0.6*(x+1.5)*(x+0.5)*(x-0.5)*(x-1.5)}
\lukusuorapienipiste{1.5}{}
\lukusuorapienipiste{0.5}{}
\lukusuorapienipiste{-1.5}{}
\lukusuorapienipiste{-0.5}{}
\end{lukusuora}

\end{esimerkki}

\begin{tehtavasivu}

\paragraph*{Opi perusteet}

\begin{tehtava}
    Anna esimerkki
    \begin{alakohdat}
	\alakohta{neljännen asteen polynomista, jolla on neljä nollakohtaa}
	\alakohta{kolmannen asteen polynomista, jolla on kaksi nollakohtaa}
	\alakohta{neljännen asteen polynomista, jolla on yksi nollakohta}
	\alakohta{neljänen asteen polynomista, jolla ei ole nollakohtia}
\end{alakohdat}
    \begin{vastaus}
	Esimerkiksi    
    \begin{alakohdat}
	\alakohta{$P(x)=x(x-1)(x-2)(x-3)$ (nollakohdat $x= 0$, $x = 1$, $x=2$ ja $x=3$)}
	\alakohta{$P(x)=x^2(x-1)$ (nollakohdat $x= 0$ ja $x = 1$)}
	\alakohta{$Q(x)=x^4$ (nollakohta $x=0$)}
	\alakohta{$R(x)=x^4+1$}
\end{alakohdat}
    \end{vastaus}
\end{tehtava}



\begin{tehtava}
Alla on polynomin $P(x)$ kuvaaja. \\
\begin{kuvaajapohja}{1}{-3}{3}{-2}{5}
  \kuvaaja{-x**5+3*x**3+2}{$P(x)$}{black}
\end{kuvaajapohja} \\
Kaikki polynomin nollakohdat näkyvät kuvaajassa.
\begin{alakohdat}
\alakohta{Mitä voidaan sanoa polynomin $P$ asteesta?}
\alakohta{Mikä on polynomin $P$ vakiotermi?}
\end{alakohdat}
\begin{vastaus}
\begin{alakohdat}
\alakohta{Nollakohtia on kolme, joten polynomin aste on vähintään 3. (Itse asiassa todellinen aste on 5, mutta sitä on vaikea päätellä silmämääräisesti kuvaajasta.)}
\alakohta{Vakiotermi on 2, koska $P(0)=2$}
\end{alakohdat}
\end{vastaus}
\end{tehtava}


\paragraph*{Hallitse kokonaisuus}

\begin{tehtava}
    Mikä on se kolmannen asteen polynomi, jonka nollakohdat ovat $x=2$, $x=-1$ ja $x=3$, ja jonka vakiotermi on $3$?
    \begin{vastaus}
        $P(x)=\frac{1}{2}(x-2)(x+1)(x-3)$
    \end{vastaus}
\end{tehtava}


\begin{tehtava}
    Kolmannen asteen polynomille $P$ pätee $P(1)=P(2)=P(3)=-2$ ja $P(0)=16$. Ratkaise $P(x)$.
    \begin{vastaus}
        $P(x)=-3(x-1)(x-2)(x-3)-2=-3x^3+18x^2-33x+18$
    \end{vastaus}
\end{tehtava}

\begin{tehtava}
$\star$   	Osoita, että jos $n$ asteen polynomeilla $P(x)$ ja $Q(x)$ on $n+1$ yhteistä pistettä, ne ovat sama polynomi.
    \begin{vastaus}
        Tarkastele polynomien erotuksen nollakohita.
    \end{vastaus}
\end{tehtava}

\end{tehtavasivu}
	\section{Korkeamman asteen yhtälöt}

\qrlinkki{http://opetus.tv/maa/maa2/n-asteinen-polynomiyhtalo/}{Opetus.tv: \emph{N-asteinen polynomiyhtälö} (8:38, 6:20 ja 15:53)}

Jos toisen asteen polynomiyhtälöllä on reaalilukuratkaisuja, ne löytyvät ratkaisukaavalla.
Myös kolmannen ja neljännen asteen yhtälöille on olemassa ratkaisukaavat.
Ne ovat kuitenkin niin monimutkaisia, että ne eivät juuri sovellu käsin laskettaviksi, eikä niitä siksi esitellä tässä.
Korkeamman kuin neljännen asteen yhtälöille sen sijaan ei edes ole olemassa yleistä ratkaisukaavaa.
Tämän osoitti norjalainen matemaatikko Niels Henrik Abel vuonna 1823.

Korkeamman kuin toisen asteen yhtälöt ratkaistaan käytännössä yleensä tietokoneen avulla.
Niille yhtälöille, joille ei ole ratkaisukaavaa, ratkaiseminen onnistuu vain numeerisesti eli likiarvoja käyttäen.
Joissain erikoistapauksissa korkeamman asteen yhtälön ratkaiseminen onnistuu myös käsin.
Seuraavassa tarkastellaan tällaisia erikoistapauksia.

\subsection*{Tekijöihinjako}

Polynomiyhtälöitä voidaan toisinaan ratkaista jakamalla polynomi tekijöihin.

%Jos polynomiyhtälössä $P(x) = 0$ polynomi $P(x)$ voidaan jakaa tekijöihin, ratkaisu saadaan etsimällä näiden tekijöiden nollakohdat.
%Esimerkiksi vakiotermittömässä polynomissa voidaan ottaa muuttuja yhteiseksi tekijäksi ja riittää ratkaista yhtä pienemmän asteen %polynomiyhtälö.

\begin{esimerkki}
Ratkaise yhtälö $x^3 - 3x^2 + x = 0$.

\begin{esimratk}
Polynomissa $x^3 - 3x^2 + x$ ei ole vakiotermiä. Voidaan siis ottaa yhteiseksi tekijäksi $x$, jolloin polynomi tulee muotoon $x(x^2 - 3x + 1)$. 

Nyt yhtälön ratkaiseminen voidaan aloittaa soveltamalla tulon nollasääntöä:
\begin{align*}
x^3 - 3x^2 + x & =0 \\
x(x^2 - 3x + 1) & =0 \\
x=0 \quad & \text{tai} \quad x^2 - 3x + 1 = 0 \\
\end{align*}

Jäljelle jäävään toisen asteen yhtälöön $x^2 - 3x + 1 = 0$ voidaan käyttää ratkaisukaavaa:
\[
x =\frac{3\pm\sqrt{3^2-4\cdot 1\cdot 1}}{2\cdot 1}=\frac{3\pm \sqrt{5}}{2}.
\]
\end{esimratk}

\begin{esimvast} $x=0$ tai $x=\dfrac{3\pm \sqrt{5}}{2}$
\end{esimvast}
\end{esimerkki}

Jos polynomi voidaan jakaa tekijöihin, sen ratkaiseminen helpottuu, koska voidaan soveltaa tulon nollasääntöä.
Sopiva tekijöihinjako on kuitenkin usein vaikea löytää. Tarkastellaan vielä toista esimerkkiä.
%Aiemmin esitellyn polynomien jakolauseen mukaan kaikki polynomit voidaan jakaa tekijöihin, jotka ovat korkeintaan toista astetta.
%Periaatteessa tällä tavalla voidaan siis ratkaista kaikki polynomiyhtälöt. Tekijöihin jakaminen on kuitenkin yleensä vaikeaa.

\begin{esimerkki}
Ratkaise yhtälö $x^3-17x^2-x+17 = 0$.

\begin{esimratk}
Yhtälön vasemmalla puolella olevan polynomin voi jakaa tekijöihin ryhmittelemällä:

\begin{align*}
x^3-17x^2-x+17=x^2(x-17)+(-1)(x-17)=(x^2-1)(x-17).
\end{align*}

Nyt yhtälö ratkeaa tulon nollasäännöllä:
\begin{align*}
x^3-17x^2&-x+17=0 \\
(x^2-1)&(x-17)=0 \\
x^2-1 = 0 \quad &\text{tai} \quad x - 17 = 0 \\
x^2 = 1 \quad &\text{tai} \quad x = 17 \\
x =\pm 1 \quad &\text{tai} \quad x = 17 \\
\end{align*}
\end{esimratk}

\begin{esimvast}
$x = 17$, $x = 1$ tai $x=-1$
\end{esimvast}
\end{esimerkki}

\subsection*{Sijoitukset}

Joskus yhtälöt ratkeavat, kun niihin sijoitetaan jokin apumuuttuja.
Tällöin puhutaan myös muuttujanvaihdosta.

%Esimerkiksi muotoa $ax^4+bx^2+c=0$ olevissa yhtälöissä huomataan, että merkitsemällä lauseketta $x^2$ kirjaimella $y$, yhtälö voidaan kirjoittaa muotoon $ay^2+by+c=0$. Uudesta yhtälöstä voidaan ratkaista $y$ toisen asteen yhtälön ratkaisukaavalla, ja sijoituksesta $y = x^2$ voidaan ratkaista $x$.

\begin{esimerkki}
Ratkaise yhtälö $2x^4+14x^2-36=0$.

\begin{esimratk}
Koska $x^4=(x^2)^2$, ratkaistava yhtälö voidaan kirjoittaa myös muodossa $2(x^2)^2+14x^2-36=0$.
Kun nyt sijoitetaan lausekkeen $x^2$ paikalle $y$, eli merkitään $y=x^2$, saadaankin muuttujan $y$ yhtälö
\[
2y^2+14y-36=0.
\]
Tämä on toisen asteen yhtälö, joka osataan ratkaista esimerkiksi ratkaisukaavalla.
Tällä tavoin saadaan
\[
y=\frac{-14\pm\sqrt{14^2-4\cdot 2\cdot(-36)}}{2\cdot 2}=\frac{-14\pm 22}{4}.
\]
Ratkaisut ovat siis $y=2$ ja $y=-9$.

On kuitenkin vielä selvitettävä alkuperäisen muuttujan $x$ arvot.
Koska $x^2=y$, saadaan $y$:lle löydetyistä arvoista yhtälöt $x^2=2$ ja $x^2=-9$.
Reaaliluvun neliö ei kuitenkaan voi olla negatiivinen, joten ainoat ratkaisut ovat yhtälön $x^2 = 2$ ratkaisut.
Ne ovat $x=\pm\sqrt{2}$.
\end{esimratk}

\begin{esimvast}
$x=2$ tai $x=-2$
\end{esimvast}
\end{esimerkki}

Muotoa $ax^4+bx^2+c=0$ oleva yhtälö (eli ns. bikvadraattinen yhtälö) voidaan aina ratkaista sijoittamalla $y=x^2$.
Yleisemmin muotoa $ax^{2n}+bx^n+c=0$ olevat yhtälöt voidaan ratkaista sijoituksella $y = x^n$.

\begin{esimerkki}
Ratkaise yhtälö $x^{10}+x^5=2$.

\begin{esimratk}
Muutetaan yhtälö muotoon $(x^5)^2+x^5-2=0$ ja tehdään sijoitus $y = x^5$.
Nyt yhtälö saa muodon $y^2+y-2 = 0$.
Toisen asteen yhtälön ratkaisukaavalla saadaan yhtälön ratkaisuiksi $y = -2$ ja $y = 1$.

Nyt alkuperäisen yhtälön ratkaisut saadaan yhtälöistä $x^5=-2$ ja $x^5=1$. Siten ratkaisut ovat $x = \sqrt[5]{-2}$ ja $x = 1$.
\end{esimratk}

\begin{esimvast}
$x = \sqrt[5]{-2}$ tai $x = 1$
\end{esimvast}

\end{esimerkki}

\begin{tehtavasivu}

\paragraph*{Opi perusteet}

\begin{tehtava}
    Ratkaise yhtälöt.
    \begin{alakohdat}
        \alakohta{$x^3-5x^2+6x=0$}
        \alakohta{$x^4 - 16 = 0$}
        \alakohta{$x^6 - x^4 = 0$}
    \end{alakohdat}
    \begin{vastaus}
        \begin{alakohdat}
            \alakohta{$x = 0$ tai $x=2$ tai $x=3$}
            \alakohta{$x = \pm 2$}
            \alakohta{$x = 0$ tai $x=\pm 1$}
        \end{alakohdat}
    \end{vastaus}
\end{tehtava}

\begin{tehtava}
    Ratkaise yhtälöt.
    \begin{alakohdat}
        \alakohta{$x^4 - 2x^2 - 24 = 0$}
        \alakohta{$x^4 - 4x^2 - 5 = 0$}
        \alakohta{$x^4 - 8x^2 + 15 = 0$}
    \end{alakohdat}
    \begin{vastaus}
        \begin{alakohdat}
            \alakohta{$x = \pm\sqrt{6}$}
            \alakohta{$x = \pm\sqrt{5}$}
            \alakohta{$x = \pm\sqrt{3}$ tai $\pm\sqrt{5}$}
        \end{alakohdat}
    \end{vastaus}
\end{tehtava}

\begin{tehtava}
    Ratkaise yhtälöt.
    \begin{alakohdat}
        \alakohta{$x^8 - 1 = 0$}
        \alakohta{$x^8 - x^4 = 0$}
        \alakohta{$x^8 - x^4 - 1 = 0$}
    \end{alakohdat}
    \begin{vastaus}
        \begin{alakohdat}
            \alakohta{$x = \pm\sqrt{1}$}
            \alakohta{$x = 0$ tai $x = \pm\sqrt{1}$}
            \alakohta{$x = \pm\sqrt[4]{\frac{1+\sqrt{5}}{2}} = \pm\sqrt[4]{\upvarphi}$ ($\upvarphi$ on kultaisena leikkauksena tunnettu vakio)}
        \end{alakohdat}
    \end{vastaus}
\end{tehtava}

\paragraph*{Hallitse kokonaisuus}

\begin{tehtava}
Ratkaise yhtälöt.
\begin{alakohdat}
\alakohta{$x^5-3x^4+2x^3=0$}
\alakohta{$x^4+5x^3-x^2-5x=0$}
\alakohta{$x^3-4x^2-4x+16=0$}
\end{alakohdat}
\begin{vastaus}
\begin{alakohdat}
\alakohta{$x=0$ tai $x=1$ tai $x=2$}
\alakohta{$x=0$ tai $x=-5$ tai $x= \pm 1$}
\alakohta{$x=4$ tai $x= \pm 2$}
\end{alakohdat}
\end{vastaus}
\end{tehtava}

\begin{tehtava}
    Ratkaise yhtälöt.
    \begin{alakohdat}
        \alakohta{$x^4 - 16 = 0$}
        \alakohta{$2x^4 = 8x^2$}
        \alakohta{$x^6 - 2x^3 = 3$}
        \alakohta{$x^{100} - 2x^{50} + 1 = 0$}
    \end{alakohdat}
    \begin{vastaus}
        \begin{alakohdat}
            \alakohta{$x = \pm2$}
            \alakohta{$x = 0$ tai $x=\pm2$}
            \alakohta{$x = \sqrt[3]{3}$ tai $x= -1$}
            \alakohta{$x = \pm1$}
        \end{alakohdat}
    \end{vastaus}
\end{tehtava}

\begin{tehtava}
	Ratkaise yhtälö $x^{627} - 6x^{514} + 5x^{401} = 0$.
	\begin{vastaus}
		$x = 0$, $x = 1$ tai $x = \sqrt[113]{5}$
	\end{vastaus}
\end{tehtava}


\begin{tehtava}
 	Ratkaise yhtälö $5^{x^3+4x^2+x}=1$. 
%	(K02/T2b) Ratkaise yhtälö $e^{x^3+4x^2+x}=1$. [$e$ on matemaattinen vakio, irrationaaliluku, jonka likiarvo on $2,718$.]
	\begin{vastaus}
	$x=0$ tai $x=-2 + \sqrt[]{3}$ tai $x=-2 - \sqrt[]{3}$
	\end{vastaus}
\end{tehtava}

\begin{tehtava}
	$ \star $ Ratkaise yhtälö $(x+\frac{1}{x})^2-x-\frac{1}{x}-6 = 0$.
	\begin{vastaus}
		$x = -1$, $x = \frac{3\pm \sqrt{5}}{2}$
	\end{vastaus}
\end{tehtava}

\begin{tehtava}
	$ \star $ Ratkaise yhtälö $2^x-1=\frac{12}{2^x}$
	\begin{vastaus}
	$x=2$
	\end{vastaus}
\end{tehtava}

\paragraph*{Lisää tehtäviä}

\begin{tehtava}
    Ratkaise yhtälöt.
    \begin{alakohdat}
        \alakohta{$x^4 + 7x^3 = 0$}
        \alakohta{$2x^3 - 16x^2 + 32x = 0$}
        \alakohta{$x^6 + 6x^5 = -9x^4$}
        \alakohta{$x^3 - 2x^5 = 0$}
    \end{alakohdat}
    \begin{vastaus}
        \begin{alakohdat}
        	\alakohta{$x = 0$ tai $x = -7$}
        	\alakohta{$x = 0$ tai $x = 4$}
        	\alakohta{$x = 0$ tai $x = -3$}
            \alakohta{$x = 0$ tai $x = \pm\dfrac{1}{\sqrt{2}}$}
        \end{alakohdat}
    \end{vastaus}
\end{tehtava}

\begin{tehtava}
    Ratkaise yhtälöt.
    \begin{alakohdat}
        \alakohta{$x^5 - 2x^3 + x = 0$}
        \alakohta{$x^8 + 4x^4 = 5x^6$}
    \end{alakohdat}
    \begin{vastaus}
        \begin{alakohdat}
        	\alakohta{$x = 0$ tai $x = \pm1$}
        	\alakohta{$x = 0$ tai $x = \pm1$ tai $x = \pm2$}
        \end{alakohdat}
    \end{vastaus}
\end{tehtava}

\begin{tehtava} % Korkeamman asteen yhtälö
Kun kolme peräkkäistä kokonaislukua kerrotaan keskenään, ja tuloon
lisätään keskimmäinen luku, tulos on 15 kertaa keskimmäisen luvun neliö.
Mitkä luvut ovat kyseessä?
    \begin{vastaus}
	Luvut ovat $-1, 0$ ja $1$ tai $14, 15$ ja $16$.
    \end{vastaus}
\end{tehtava}

\begin{tehtava} % Tämä kai tarvitsisi polynomien jakokulman?
(K94/T2a) Polynomin $P(x)=ax^3-31x^2+1$ eräs nollakohta on $x=1$. Määritä $a$ ja ratkaise tämän jälkeen yhtälö $P(x)=0$.
\begin{vastaus}
      $a=30$ yhtälön ratkaisut ovat $1$, $\frac{1}{5}$ ja $-\frac{1}{6}$.
    \end{vastaus}
\end{tehtava}

\end{tehtavasivu}

	\section{Korkeamman asteen epäyhtälöt}

\qrlinkki{http://opetus.tv/maa/maa2/n-asteinen-polynomiepayhtalo/}{Opetus.tv: \emph{N-asteinen polynomiepäyhtälö} (16:19 ja 12:00)}

Korkeamman asteen epäyhtälö, kuten epäyhtälö
\[
x^3 -6x \leq x^2
\]
ratkaistaan siirtämällä kaikki termit epäyhtälön toiselle puolelle ja tutkimalla syntyvän polynomin merkkiä:
\begin{align*}
x^3-6x & \leq x^2 & &\ppalkki -x^2 \\
\underbrace{x^3-x^2-6x}_{P(x)} &\leq 0. &&
\end{align*}
Polynomin $P(x)$ merkin selvittämiseksi ratkaistaan sen nollakohdat:
\begin{align*}
    x^3 - x^2-6x &= 0 & &\ppalkki \text{$x$ yhteiseksi tekijäksi} \\
    x(x^2 -x -6) &= 0 & &\ppalkki \text{tulon nollasääntö} \\
    x = 0 \quad \text{tai} \quad & x^2 -x -6 = 0 & &\ppalkki \text{ratkaisukaava} \\
    x= 0 \quad \text{tai} \quad & x=\frac{-(-1) \pm \sqrt{(-1)^2-4\cdot 1 \cdot (-6)}}{2\cdot 1} && \\
    x = -2 \quad \text{tai} \quad & x = 3. &&
\end{align*}
Polynomin $P$ nollakohdat ovat siis $0$, $-2$ ja $3$. Tästä voidaan jatkaa kahdella eri tavalla.

\textbf{Tapa 1: Tekijöihin jako.}

Jaetaan polynomi tekijöihin nollakohtien avulla:
\[
P(x) = x^3 - x^2-6x = x(x^2-x-6) = x(x+2)(x-3).
\]
Tutkitaan kunkin tulon tekijän merkkiä:
\begin{align*}
    x+2>0 & \quad \text{kun} \quad x > -2\\
    x-3>0 & \quad \text{kun} \quad x > 3\\
    x>0 & \quad \text{kun} \quad x > 0.
\end{align*}

Kootaan tulokset merkkikaavioon, jonka jokainen ruutu vastaa yhtä reaalilukuväliä lukusuoralla:
\begin{center}
    \begin{merkkikaavio}{3}
        \merkkikaavioKohta{$-2$}
        \merkkikaavioKohta{$0$}
        \merkkikaavioKohta{$3$}

        \merkkikaavioFunktio{$x+2$}
        \merkkikaavioMerkki{$-$}
        \merkkikaavioMerkki{$+$}
        \merkkikaavioMerkki{$+$}
        \merkkikaavioMerkki{$+$}

        \merkkikaavioUusirivi
        \merkkikaavioFunktio{$x-3$}
        \merkkikaavioMerkki{$-$}
        \merkkikaavioMerkki{$-$}
        \merkkikaavioMerkki{$-$}
        \merkkikaavioMerkki{$+$}

        \merkkikaavioUusirivi
        \merkkikaavioFunktio{$x$}
        \merkkikaavioMerkki{$-$}
        \merkkikaavioMerkki{$-$}
        \merkkikaavioMerkki{$+$}
        \merkkikaavioMerkki{$+$}

        \merkkikaavioUusiriviKaksoisviiva
        \merkkikaavioFunktio{$x(x+2)(x-3)$}
        \merkkikaavioMerkki{$-$}
        \merkkikaavioMerkki{$+$}
        \merkkikaavioMerkki{$-$}
        \merkkikaavioMerkki{$+$}
    \end{merkkikaavio}
\end{center}
Merkkikaavion alin rivi saadaan tulon merkkisäännöstä: kolmen negatiivisen luvun tulo on negatiivinen, yhden positiivisen ja kahden negatiivisen tulo positiivinen jne.
Kaavion viimeiseltä riviltä voidaan nyt lukea vastaus alkuperäiseen kysymykseen: $x^3-x^2-6 \leq 0$, kun $x\leq -2$ tai $0\leq x \leq 3$.

\textbf{Tapa 2: Testipisteet.}

Polynomit ovat jatkuvia funktioita. Jatkuvuutta käsitellään tarkemmin vasta kurssilla MAA7.
Intuitiivisesti jatkuvuudessa on kyse siitä, että funktion kuvaaja on yhtenäinen viiva.
Jatkuvuudesta seuraa, että polynomi ei voi vaihtaa merkkiä kulkematta nollakohdan kautta.
Polynomin merkin nollakohtien välillä saa selville tarkistamalla jonkin väliltä otetun testipisteen merkin.

Esimerkissä nollakohdat olivat $-2$, $0$ ja $3$. Valitaan niiden välistä ja
ympäriltä testipisteiksi vaikkapa $x=-3$, $x=-1$, $x=1$ ja $x=4$. Tarkistetaan funktion merkki kussakin pisteessä:

\begin{tabular}{c|c|l|c}
Väli & Testipiste & $f(x)=x^3-x^2-6x$ & Funktion merkki \\
\hline
$x < -2$ & $x = -3$ & $(-3)^3 -(-3)^2 - 6(-3) = -18$ & $-$ \\
$-2 <x < 0$ & $x = -1$ & $(-1)^3 -(-1)^2 - 6(-1) =4$ & $+$ \\
$0 <x < 3$ & $x = 1$ & $1^3 -1^2 - 6\cdot 1 =  -6$ & $-$ \\
$3 <x $ & $x = 4$ & $4^3 -4^2 - 6\cdot 4 = 24$ & $+$
\end{tabular}

Vastaukseksi saadaan sama kuin edellä: $x^3-x^2-6 \leq 0$, kun $x\leq -2$ tai $0\leq x \leq 3$.

\begin{tehtavasivu}

\paragraph*{Opi perusteet}

\begin{tehtava}
    Ratkaise
    \begin{alakohdat}
        \alakohta{$(x-1)(x-2)(x-3) \le 0$}
        \alakohta{$(x-1)(x-2)(x-3) > 0$}
        \alakohta{$3(x-1)(x-2)(x-3) > 0$.}
    \end{alakohdat}
    \begin{vastaus}
        \begin{alakohdat}
            \alakohta{$x \le 1$ tai $2 \le x \le 3$}
            \alakohta{$1 < x < 2$ tai $x>3$}
            \alakohta{$1 < x < 2$ tai $x>3$}
        \end{alakohdat}
    \end{vastaus}
\end{tehtava}

\begin{tehtava}
    Ratkaise $x^3-x^2<0$.
    \begin{vastaus}
        $x<0$ tai $0<x<1$
    \end{vastaus}
\end{tehtava}

\begin{tehtava}
    Ratkaise $x^4 \le 1$.
    \begin{vastaus}
        $-1 \le x \le 1$
    \end{vastaus}
\end{tehtava}

\paragraph*{Hallitse kokonaisuus}

\begin{tehtava}
Ratkaise $(x^5-2)(x^8-1) >0$
\begin{vastaus}
$x > \sqrt[5]{2}$ tai $-1<x<1$
\end{vastaus}
\end{tehtava}

%neliö epänegatiivinen
\begin{tehtava}
    Ratkaise $(2x^3+4x^2-5x+7)^2 < 0$.
    \begin{vastaus}
        Ei ratkaisuja.
    \end{vastaus}
\end{tehtava}

%bikvadraattinen
\begin{tehtava}
    Ratkaise $x^4-3x^2-18 \le 0$.
    \begin{vastaus}
        $-\sqrt{6}\le x \le \sqrt{6}$
    \end{vastaus}
\end{tehtava}

\begin{tehtava}
    Koska tulolla ja osamäärällä on sama merkkisääntö, merkkikaavioita
	voidaan käyttää myös osamääriin. Ratkaise epäyhtälöt
    \begin{alakohdat}
        \alakohta{$\frac{(x+3)(x-2)}{x-5} \le 0$}
        \alakohta{$x \geq \frac{1}{x}$}
    \end{alakohdat}
    \begin{vastaus}
        \begin{alakohdat}
            \alakohta{$x \le -3$ tai $2 \le x < 5$}
            \alakohta{$-1 \leq x < 0$ tai $x \geq 1$}
    	\end{alakohdat}
    \end{vastaus}
\end{tehtava}


%yhteinen tekijä x^3, binomikaava käänteisesti
\begin{tehtava}
    Ratkaise $4x^5+9 x^3 \le 12 x^4$.
    \begin{vastaus}
        $x\le0$ tai $x=\frac{3}{2}$
    \end{vastaus}
\end{tehtava}

\begin{tehtava}
Epäyhtälöiden ratkaisut/todistukset perustuvat usein tietoon, että epänegatiivisten lukujen summa on epänegatiivinen ja nolla jos, ja vain jos kaikki yhteenlaskettavat ovat nollia.

\begin{alakohdat}
\alakohta{Ratkaise $x^6 + x^2+1 > 0$}
\alakohta{Ratkaise $x^{10} + (x-1)^{10} < 0$}
\alakohta{Todista, että kaikilla reaaliluvuilla $x$ ja $y$
\[
(xy-1)^2+(x^2-y^2)^4+(xy-x-y+1)^6 \geq 0
\]
ja että epäyhtälössä vallitsee yhtäsuuruus jos, ja vain jos $x = y = 1$}
\end{alakohdat}

\begin{vastaus}
\begin{alakohdat}
\alakohta{$x \in \R$}
\alakohta{Epäyhtälöllä ei ole ratkaisuja}
\alakohta{Vinkki: Käytä tehtävänannon havaintoa ja tutki, millä $x$:n ja $y$:n arvoilla summattavat saavat arvon 0}
\end{alakohdat}
\end{vastaus}
\end{tehtava}

\paragraph*{Lisää tehtäviä}

% x yhteinen tekijä ja sij. y=x^5
\begin{tehtava}
    Ratkaise $x+2x^6+x^{11}<0$.
    \begin{vastaus}
        $x<-1$ tai $ -1<x<0$
    \end{vastaus}
\end{tehtava}

\end{tehtavasivu}


\newpage
\nosa{Kertausosio}
   \nluku{LIITE_testaatietosi}{Testaa tietosi!}
   \nluku{LIITE_kertausteht}{Kertaustehtäviä}
   \nluku{LIITE_harjoituskokeita}{Harjoituskokeita}
   \nluku{LIITE_yokokeita}{Ylioppilaskoetehtäviä}

\newpage
\nosa{Lisämateriaalia}
   \nluku{LIITE_paraabeli}{Paraabeli}
   
\newpage
\Closesolutionfile{ans}
\vast