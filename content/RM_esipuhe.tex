\section{Johdanto}

Polynomit ovat hyvin keskeisiä sovelletussa matematiikassa. Niin fysiikan, kemian, tietotekniikan kuin myös taloustieteiden parissa varsin monet laskutoimitukset pelkistyvät lopulta polynomien käsittelyksi. Esimerkiksi sellaiset tietotekniset sovellutukset kuten hahmon tunnistaminen kuvasta, äänen kompressointi (mp3) ja vaikkapa sikiön sydänäänien erottaminen perustuvat lopulta polynomien laskentaan. Itse asiassa polynomit ovat niin yleisiä matematiikassa, että riippumatta sovellusalueesta ne on hyvä tuntea perusteellisesti.

Pitkän matematiikan toisella kurssilla MAA2 Polynomifunktiot käsitellään polynomifunktioita, -yhtälöitä ja -epäyhtälöitä. Kurssilla syvennetään ensimmäisen kurssin asioita ja sovelletaan niitä polynomien maailmassa. Oppikirja on rakennettu siten, että aiheet esitellään lukujen alussa ja havainnollistetaan esimerkein. Tehtäviä on runsaasti ja niiden tarkoituksena on saada opiskelija sisäistämään opiskellut asiat ja siirtämään ne käytäntöön.


Tässä kirjassa käymme läpi opetussuunnitelman mukaiset keskeiset sisällöt, joita ovat
\begin{itemize}
\item polynomien tulo ja binomikaavat
\item polynomifunktio
\item toisen ja korkeamman asteen polynomiyhtälöt
\item toisen asteen yhtälön juurten lukumäärän tutkiminen
\item polynomiepäyhtälön ratkaiseminen
\end{itemize}

Opetussuunnitelman mukaiset kurssin keskeiset tavoitteet ovat, että opiskelija
\begin{itemize}
\item harjaantuu käsittelemään polynomifunktioita
\item oppii ratkaisemaan toisen asteen polynomiyhtälöitä ja tutkimaan ratkaisujen lukumäärää
\item oppii ratkaisemaan korkeamman asteen polynomiyhtälöitä, jotka voidaan ratkaista ilman polynomien jakolaskua
\item oppii ratkaisemaan yksinkertaisia polynomiepäyhtälöitä
\end{itemize}

Avoimet oppimateriaalit ry tuottaa ja julkaisee oppimateriaaleja ja kirjoja, jotka ovat kaikille vapaita käyttää. Vapaa matikka -sarja on suunnattu lukion pitkän matematiikan kursseille ja täyttää valtakunnallisen opetussuunnitelman vaatimukset.