\begin{tehtavasivu}

\subsubsection*{Opi perusteet}

\begin{tehtava}
    Ratkaise
    \begin{alakohdat}
        \alakohta{$(x-1)(x-2)(x-3) \le 0$}
        \alakohta{$(x-1)(x-2)(x-3) > 0$}
        \alakohta{$-3(x-1)(x-2)(x-3) > 0$.}
    \end{alakohdat}
    \begin{vastaus}
        \begin{alakohdat}
            \alakohta{$x \le 1$ tai $2 \le x \le 3$}
            \alakohta{$1 < x < 2$ tai $x>3$}
            \alakohta{$x < 1$ tai $2<x<3$}
        \end{alakohdat}
    \end{vastaus}
\end{tehtava}

\begin{tehtava}
    Ratkaise $x^3-x^2<0$.
    \begin{vastaus}
        $x<0$ tai $0<x<1$
    \end{vastaus}
\end{tehtava}

\begin{tehtava}
    Ratkaise $x^4 \le 1$.
    \begin{vastaus}
        $-1 \le x \le 1$
    \end{vastaus}
\end{tehtava}

\subsubsection*{Hallitse kokonaisuus}

%neliö epänegatiivinen
\begin{tehtava}
    Ratkaise $(2x^3+4x^2-5x+7)^2 < 0$.
    \begin{vastaus}
        Ei ratkaisuja.
    \end{vastaus}
\end{tehtava}

%bikvadraattinen
\begin{tehtava}
    Ratkaise $x^4-3x^2-18 \le 0$.
    \begin{vastaus}
        $-\sqrt{6}\le x \le \sqrt{6}$
    \end{vastaus}
\end{tehtava}

\begin{tehtava} % Korkeamman asteen epäyhtälö
Olkoon $a > 0$. Millä muuttujan $x$ arvoilla funktion
$P(x)=x^3-ax$ arvot ovat positiivisia?
    \begin{vastaus}
	$P(x)>0$ kun $x > \sqrt{a}$ tai $-\sqrt{a}<x<0$.
    \end{vastaus}
\end{tehtava}

\begin{tehtava}
    Koska tulolla ja osamäärällä on sama merkkisääntö, merkkikaavioita
	voidaan käyttää myös osamääriin. Ratkaise epäyhtälöt
    \begin{alakohdat}
        \alakohta{$\frac{(x+3)(x-2)}{x-5} \le 0$}
        \alakohta{$x \geq \frac{1}{x}$}
    \end{alakohdat}
    \begin{vastaus}
        \begin{alakohdat}
            \alakohta{$x \le -3$ tai $2 \le x < 5$}
            \alakohta{$-1 \leq x < 0$ tai $x \geq 1$}
    	\end{alakohdat}
    \end{vastaus}
\end{tehtava}


%yhteinen tekijä x^3, binomikaava käänteisesti
\begin{tehtava}
    Ratkaise $4x^5+9 x^3 \le 12 x^4$.
    \begin{vastaus}
        $x\le0$ tai $x=\frac{3}{2}$
    \end{vastaus}
\end{tehtava}

\begin{tehtava}
Ratkaise $(x^5-2)(x^8-1) >0$
\begin{vastaus}
$x > \sqrt[5]{2}$ tai $-1<x<1$
\end{vastaus}
\end{tehtava}

\begin{tehtava}
Epäyhtälöiden ratkaisut/todistukset perustuvat usein tietoon, että epänegatiivisten lukujen summa on epänegatiivinen ja nolla jos, ja vain jos kaikki yhteenlaskettavat ovat nollia.

\begin{alakohdat}
\alakohta{Ratkaise $x^6 + x^2+1 > 0$}
\alakohta{Ratkaise $x^{10} + (x-1)^{10} < 0$}
\alakohta{Todista, että kaikilla reaaliluvuilla $x$ ja $y$
\[
(xy-1)^2+(x^2-y^2)^4+(xy-x-y+1)^6 \geq 0
\]
ja että epäyhtälössä vallitsee yhtäsuuruus jos, ja vain jos $x = y = 1$}
\end{alakohdat}

\begin{vastaus}
\begin{alakohdat}
\alakohta{$x \in \rr$}
\alakohta{Epäyhtälöllä ei ole ratkaisuja}
\alakohta{Vinkki: Käytä tehtävänannon havaintoa ja tutki, millä $x$:n ja $y$:n arvoilla summattavat saavat arvon 0}
\end{alakohdat}
\end{vastaus}
\end{tehtava}

\subsubsection*{Lisää tehtäviä}

\begin{tehtava} % Korkeamman asteen epäyhtälö
Ratkaise epäyhtälöt
		\begin{alakohdat}
		\alakohta{$x^3 + 2x^2-15x  > 0$  }
		\alakohta{$x^3-2x^2+x \leq 0$  }
		\end{alakohdat}
    \begin{vastaus}
		\begin{alakohdat}
		\alakohta{$-5<x<0$ tai $3 < x$}
		\alakohta{$x<0$ tai $x = 1$}
		\end{alakohdat}
    \end{vastaus}
\end{tehtava}

% x yhteinen tekijä ja sij. y=x^5
\begin{tehtava}
    Ratkaise $x+2x^6+x^{11}<0$.
    \begin{vastaus}
        $x<-1$ tai $ -1<x<0$
    \end{vastaus}
\end{tehtava}

\end{tehtavasivu}