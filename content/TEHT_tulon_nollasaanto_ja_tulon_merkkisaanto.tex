\begin{tehtavasivu}

\subsubsection*{Opi perusteet}

\begin{tehtava}
	Laske.
	\alakohdat{
		§ $2 \cdot 3$
		§ $2 \cdot (-3)$
		§ $-2 \cdot 3$
		§ $-2 \cdot (-3)$
		§ $17 \cdot 666 \cdot 0 \cdot (-31)$
	}
	
	\begin{vastaus}
		\alakohdat{
			§ $6$
			§ $-6$
			§ $-6$
			§ $6$
			§ $0$
		}
	\end{vastaus}
\end{tehtava}

\begin{tehtava}
	Olkoon $a>0$, $b<0$, ja $c=0$. Mitä voit päätellä tulon merkistä?
	\alakohdat{
		§ $a \cdot a$
		§ $b \cdot a$
		§ $b \cdot b$
		§ $b \cdot c$
	}
	
	\begin{vastaus}
		\alakohdat{
			§ tulo $>0$
			§ tulo $<0$
			§ tulo $>0$
			§ tulo on $0$
		}
	\end{vastaus}
\end{tehtava}


\begin{tehtava}
    Ratkaise seuraavat yhtälöt käyttämällä tulon nollasääntöä.
    \alakohdat{
        § $x(3+x)=0$
        § $(x-4)(x+3)=0$
		§ $0=x^2(x-5)$
    }
    \begin{vastaus}
        \alakohdat{
            § $x=0$ tai $x=-3$
            § $x=4$ tai $x=-3$
            § $x=0$ tai $x=5$
        }
    \end{vastaus}
\end{tehtava}

\subsubsection*{Hallitse kokonaisuus}

\begin{tehtava}
	Olkoon $a > 0$. Mitkä vaihtoehdoista $b>0$, $b<0$ ja $b=0$ ovat mahdollisia, jos
	tiedetään, että
	\alakohdat{
		§ $a \cdot b > 0$
		§ $a \cdot b \leq 0$
		§ $b \cdot b > 0$
		§ $b \cdot b < 0$?
	}
	\begin{vastaus}
		\alakohdat{
			§ $b>0$
			§ $b < 0$ ja $b = 0$
			§ $b>0$ ja $b<0$
			§ Mikään vaihtoehto ei kelpaa.
		}
	\end{vastaus}
\end{tehtava}

\begin{tehtava}
    Ratkaise seuraavat yhtälöt käyttämällä tulon nollasääntöä.
    \alakohdat{
        § $y(y+4)=0$
        § $(x-2)(x-1)(x+5)=0$
        § $(x+1)(x^2+2)=0$
    }
    \begin{vastaus}
        \alakohdat{
            § $y=0$ tai $y=-4$
            § $x=2$, $x=1$ tai $x=-5$
            § $x=-1$
        }
    \end{vastaus}
\end{tehtava}

\begin{tehtava} 
Osoita, että funktio $f(x)=x^4+3x^2+1$ saa vain positiivisia arvoja.
    \begin{vastaus}
     $x^4\geq 0$ ja $x^2 \geq 0$, joten $f(x) \geq 1$.
    \end{vastaus}
\end{tehtava}

\subsubsection*{Lisää tehtäviä}

\begin{tehtava}
	Olkoon $a \geq 0$, $b \leq 0$, ja $c=0$. Mitä voit päätellä tulon merkistä?
	\alakohdat{
		§ $a \cdot a$
		§ $b \cdot a$
		§ $b \cdot c$
	}
	
	\begin{vastaus}
		\alakohdat{
			§ tulo $\geq 0$
			§ tulo $\leq 0$
			§ tulo on $0$
		}
	\end{vastaus}
\end{tehtava}

\begin{tehtava}
    Ratkaise seuraavat yhtälöt käyttämällä tulon nollasääntöä.
    \alakohdat{
        § $(g-2)\cdot (t+1)=0$
        § $x(x-5)=0$
        § $(2w+2)^2=0$
    }
    \begin{vastaus}
        \alakohdat{
            § $g=2$ tai $t=-1$. Tehtävässä ei ole selvää, minkä muuttujan suhteen yhtälö pitäisi ratkaista, joten se on ratkaistu molempien muuttujien suhteen.
            § $x=0$ tai $x=5$
            § $w=-1$
        }
    \end{vastaus}
\end{tehtava}

\begin{tehtava}
    Sievennä seuraava lauseke: $(a-x)\cdot(b-x)\cdot(c-x)\cdot...\cdot(\mathring{a}-x)\cdot(\ddot{a}-x)\cdot(\ddot{o}-x)$.
    \begin{vastaus}
        Tulossa esiintyy tekijänä $(x-x)=0$. Niinpä tulon nollasäännön mukaan
        \begin{align*}
            &(a-x)\cdot(b-x)\cdot(c-x)\cdot...\cdot(x-x)\cdot...\cdot(\ddot{a}-x)\cdot(\ddot{o}-x) \\
            =&(a-x)\cdot(b-x)\cdot(c-x)\cdot...\cdot 0\cdot...\cdot(\ddot{a}-x)\cdot(\ddot{o}-x) \\
            =&0
        \end{align*}
    \end{vastaus}
\end{tehtava}

%Laatinut V-P Kilpi 2013-11-10
\begin{tehtava}
Onko olemassa reaalilukua, jonka neliö on yhtä suuri kuin sen summa itsensä kanssa?
	\begin{vastaus}
On, esimerkiksi luku $2$. (Saadaan yhtälön $x^2=2x$ eräänä ratkaisuna.)
	\end{vastaus}
\end{tehtava}

\begin{tehtava}
$\star$ (YO 1971/3, lyhyt oppimäärä) Oletetaan, että $a>b>0$. Osoita, että $a^3-b^3>(a-b)^3$.
	\begin{vastaus}
	Binomin kuutio $(a-b)^3$ kehitettynä polynomiksi on $a^3-3a^2b+3ab^2-b^3$, joka järjesteltynä uudelleen on $a^3-b^3-3a^2b+3ab^2$. Nyt riittää osoittaa, että vertailukohteesta $a^3-b^3$ poikkeava osa $-3a^2b+3ab^2$ on negatiivinen. Jaetaan tekijöihin: $-3a^2b+3ab^2=3ab(-a+b)=3ab(b-a)$, mistä tulos seuraa, koska lähtöoletuksen perusteella $3ab$ on positiivinen ja $b-a$ negatiivinen, jolloin koko lausekekin on.  
	\end{vastaus}
\end{tehtava}

%tai vai vai vai joko tai, jos vai jos ja vain jos -tehtäviä :)
%potensseja tehtäviin!

\end{tehtavasivu}