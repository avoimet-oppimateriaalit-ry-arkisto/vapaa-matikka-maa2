\begin{tehtavasivu}

\subsubsection*{Opi perusteet}

\begin{tehtava}
  Aukeavatko seuraavat paraabelit ylös- vai alaspäin? %hnnnggh!
  \begin{alakohdat}
    \alakohta{$4x^2 + 100x - 3$}
    \alakohta{$-x^2 + 1\,337$}
    \alakohta{$5x^2 - 7x + 5$}
    \alakohta{$-6(-3x^2 + 5)$}
    \alakohta{$-13x(9 - 17x)$}
    \alakohta{$100(1-x^2)$}
  \end{alakohdat}

  \begin{vastaus}
    \begin{alakohdat}
      \alakohta{Ylös}
      \alakohta{Alas}
      \alakohta{Ylös}
      \alakohta{Ylös}
      \alakohta{Ylös}
      \alakohta{Alas}
    \end{alakohdat}
  \end{vastaus}
\end{tehtava}

\begin{tehtava}
Funktiot $P(x)$ ja $Q(x)$ ovat toisen asteen polynomeja.\\
\begin{kuvaajapohja}{1}{-2}{3}{-1}{3}
\kuvaaja{x*(2-x)}{$P(x)$}{black}
\kuvaaja{x**2+1}{$Q(x)$}{black}
\end{kuvaajapohja} \\
Päättele kuvaajan perusteella
\begin{alakohdat}
\alakohta{mihin suuntaan paraabelit aukeavat}
\alakohta{funktion $P$ nollakohdat}
\alakohta{yhtälön $Q(x)=2$ ratkaisu}
\alakohta{polynomin $Q(x)$ vakiotermi}
\end{alakohdat}

\begin{vastaus}
\begin{alakohdat}
\alakohta{$P$ alaspäin, $Q$ ylöspäin.}
\alakohta{nollakohdat: $x=0$ ja $x=2$}
\alakohta{$x=-1$ tai $x=1$}
\alakohta{$1$, sillä kun $x=0$, $Q(x)=1$.}
\end{alakohdat}
\end{vastaus}
\end{tehtava}

\subsubsection*{Hallitse kokonaisuus}

\begin{tehtava}
Kuvassa on funktion $P(x)=x^2$ kuvaaja.\\
\begin{kuvaajapohja}{1.5}{-1.5}{1.5}{-1}{3}
\kuvaaja{x**2}{$P(x)=x^2$}{black}
\end{kuvaajapohja} \\
Hahmottele kuvaajan avulla funktioiden
\begin{alakohdat}
\alakohta{$x^2-1$}
\alakohta{$2-x^2$}
\alakohta{$\frac{1}{2}x^2$}
\alakohta{$(x-2)^2$}
\end{alakohdat}
kuvaajat.
\begin{vastaus}
\begin{alakohdat}
\alakohta{
\begin{kuvaajapohja}{1}{-1.5}{1.5}{-2}{2}
\kuvaaja{x**2-1}{$P(x)=x^2-1$}{black}
\end{kuvaajapohja}}
\alakohta{
\begin{kuvaajapohja}{1}{-1.5}{1.5}{-1}{3}
\kuvaaja{2-x**2}{$P(x)=2-x^2$}{black}
\end{kuvaajapohja}}
\alakohta{
\begin{kuvaajapohja}{1}{-1.5}{1.5}{-1}{3}
\kuvaaja{0.5*x**2}{$P(x)=\frac12x^2$}{black}
\end{kuvaajapohja}}
\alakohta{
\begin{kuvaajapohja}{1}{-0.5}{3.5}{-1}{3}
\kuvaaja{(x-2)**2}{$P(x)=(x-2)^2$}{black}
\end{kuvaajapohja}}
\end{alakohdat}
\end{vastaus}
\end{tehtava}

\end{tehtavasivu}