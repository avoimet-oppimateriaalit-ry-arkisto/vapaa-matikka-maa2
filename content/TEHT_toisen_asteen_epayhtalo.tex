\begin{tehtavasivu}

\subsubsection*{Opi perusteet}

\begin{tehtava}
    Ratkaise seuraavat epäyhtälöt.
    \begin{alakohdat}
        \alakohta{$x^2-4<0$}
        \alakohta{$x^2+x \geq 0$}
        \alakohta{$x^2+1<0$}
        \alakohta{$x^2+5x+6>0$}
    \end{alakohdat}
    \begin{vastaus}
        \begin{alakohdat}
            \alakohta{$-2<x<2$}
            \alakohta{$x \geq 0$ tai $x \leq -1$}
            \alakohta{ei ratkaisuja}
            \alakohta{$x >-2$ tai $x < -3$}
        \end{alakohdat}
    \end{vastaus}
\end{tehtava}

\begin{tehtava}
    Ratkaise seuraavat epäyhtälöt.
    \begin{alakohdat}
        \alakohta{$-x^2+5x+8>0$}
        \alakohta{$-2x^2+6x+9>0$}
        \alakohta{$4x^2-8x-72<0$}
%        \alakohta{$5x^2+10x-73<0$}
    \end{alakohdat}
    \begin{vastaus}
        \begin{alakohdat}
            \alakohta{$\frac{5-\sqrt{57}}{2}  < x < \frac{5+\sqrt{57}}{2}$}
            \alakohta{$\frac{3-3\sqrt{3}}{2}<x<\frac{3+3\sqrt{3}}{2} $}
            \alakohta{$1-\sqrt{19} < x < 1+\sqrt{19}$}
%            \alakohta{$-1+\frac{1}{5} \sqrt{390} > x > -1-\frac{1}{5} \sqrt{390}$}
        \end{alakohdat}
    \end{vastaus}
\end{tehtava}

\begin{tehtava}
    Ratkaise seuraavat epäyhtälöt.
    \begin{alakohdat}
        \alakohta{$4x^2+13x\geq 0$}
        \alakohta{$4x^2-776\geq 0$}
        \alakohta{$7x^2-8x\leq 0$}
        \alakohta{$11x^2-12\leq 0$}
    \end{alakohdat}
    \begin{vastaus}
        \begin{alakohdat}
            \alakohta{$x \geq 0$ tai $x \leq -3,25$}
            \alakohta{$x \geq \sqrt{194}$ tai $x \leq -\sqrt{194}$}
            \alakohta{$0,875 \geq x \geq 0$}
            \alakohta{$2\sqrt{\frac{3}{11}} \geq x \geq -2\sqrt{\frac{3}{11}}$}
        \end{alakohdat}
    \end{vastaus}
\end{tehtava}


%\begin{tehtava}
%  \begin{enumerate}[a)]
%    \item Ratkaise funktion $2x^2 - 5x - 3$ nollakohdat
%    \item Millä muuttujan arvoilla edellisen kohdan funktio $2x^2 - 5x - 3$ saa positiivisia arvoja?
%    \item Onko em. funktiolla globaali ääriarvo (minimi tai maksimi), ja jos on, missä kohtaa funktio saa tämän arvon? Mikä on funktion arvo silloin?
%  \end{enumerate}
%
%  \begin{vastaus}
%    \begin{enumerate}[a)]
%      \item $x = 1.2$ tai $x = -0.2$
%      \item Kun $x<-0,2$ tai $x>1,2$
%      \item Koska neliötermin kerroin a on positiivinen (2), funktiolla on globaali minimi (mutta ei ylärajaa). Symmetrian vuoksi minimi on nollakohtien puolivälissä kohdassa 0.5, jossa funktio saa siis pienimmän arvonsa -5.
%    \end{enumerate}
%  \end{vastaus}
%\end{tehtava}


\subsubsection*{Hallitse kokonaisuus}

\begin{tehtava}
  Tutki, millä muuttujan x arvoilla seuraavat funktiot saavat positiivisia arvoja.
  \begin{alakohdat}
    \alakohta{$f(x)=x^2 - 4$}
    \alakohta{$g(x)=-x^2 - 2x + 3$}
    \alakohta{$h(x)=x^2 + 2x + 5$}
    \alakohta{$i(x)=-x^2 - 1$}
  \end{alakohdat}

  \begin{vastaus}
    \begin{alakohdat}
      \alakohta{$x \leq -2$ tai $x \geq 2$}
      \alakohta{$-3 \geq x \leq 1$}
      \alakohta{Kaikilla $x$:n arvoilla.}
      \alakohta{Ei millään $x$:n arvoilla.}
    \end{alakohdat}
  \end{vastaus}
\end{tehtava}

\begin{tehtava}
(K93/T3b) Autoilijan työmatkan kesto $t$ riippuu liikennevirrasta $m$ kaavan $t=0,01m^2+0,03m+18$ mukaisesti, missä $t$ on ajoaika minuutteina ja $m$ liikenteen mittauspisteen minuutissa ohittavien autojen määrä. Kuinka suuri saa liikennevirta enintään olla, jotta autoilijan työmatka kestäisi enintään puoli tuntia?
\begin{vastaus}
        $33$ autoa/minuutti
    \end{vastaus}
\end{tehtava}

\begin{tehtava}
	Osoita, että funktio $f(x)=x^2-6x+10 $ saa vain positiivisia arvoja.
    \begin{vastaus}
	Funktiolla ei ole nollakohtia ($D<0$) ja sen kuvaaja on ylöspäin aukeava paraabeli.
    Helpompi tapa: $f(x)= (x-3)^2+1 >0.$
    \end{vastaus}
\end{tehtava}

\begin{tehtava}
	Millä vakion $t$ arvoilla yhtälöllä \\ $6x^2+2tx+2t=0$ on kaksi juurta?
	\begin{vastaus}
		Juuria on kaksi, kun $D=4t^2-48t>0$. Tämä toteutuu, kun $t < 0$ tai $t > 12$.
	\end{vastaus}
\end{tehtava}

\begin{tehtava}
Jäätelökioskin päivittäiset kiinteät kulut ovat $400$ euroa. Jokainen jäätelö maksaa kauppiaalle $0,50$ euroa. Kun jäätelön myyntihinta on $x$ euroa, sitä myydään $1\,000 - 200x$ kappaletta. 
\begin{alakohdat}
\alakohta{Millä myyntihinnoilla jäätelön myynti on kannattavaa?}
\alakohta{Millä myyntihinnalla saadaan suurin tuotto? Kuinka suuri?}
\end{alakohdat}
    \begin{vastaus}
		\begin{alakohdat}
		\alakohta{$1,00$ \euro \ $<$ myyntihinta $<$ $4,5$ \euro.}
		\alakohta{$2,75$ \euro, jolloin voitto on $612,50$ \euro. } 
		% Epäilyttävän hyvä bisnes
		\end{alakohdat}
    \end{vastaus}
\end{tehtava}

\begin{tehtava}
    Ratkaise epäyhtälö $ax^2+(a+1)x+1 > 0$ vapaan parametrin $a$ funktiona.
    \begin{vastaus}
        $a < 0$: $-\frac{1}{a} > x > -1$ \\ $a = 0$: $x > -1$ \\ $1 > a > 0$: $x > -1$ tai $x < -\frac{1}{a}$ \\ $a = 1$: $x \neq -1$ \\ $a > 1$: $x \in \rr$
    \end{vastaus}
\end{tehtava}

% $ax^2+x+5 < 0$
% $x^2+x+a \leq 0$

\begin{tehtava}
    Ratkaise seuraava epäyhtälö:
    $$p+(3-(2p-6))^2<\dfrac{2+9p}{3}(22+2)$$
    \begin{vastaus}
        $$p<\dfrac{-167-\sqrt{205\,705}}{372} \qquad\text{tai}\qquad p>\dfrac{-167+\sqrt{205\,705}}{372}$$
    \end{vastaus}
\end{tehtava}

\subsubsection*{Lisää tehtäviä}

\begin{tehtava}
Ilmaan heitetyn pallon korkeus metreinä on $h=20t-5t^2$, missä $t$ on aika sekunteina. Kuinka kauan pallo on yli $10$ metrin korkeudella maasta?
    \begin{vastaus}
	$2,8$\,s
    \end{vastaus}
\end{tehtava}

\begin{tehtava}
	Millä vakion $r$ arvoilla yhtälöllä $rx^2-rx-1 = 0$ ei ole ratkaisua?
	\begin{vastaus}
		Pitää olla $D=(-r)^2-4 \cdot r \cdot (-1)=r^2+4r<0$. Siis $-4 < r < 0$.
	\end{vastaus}
\end{tehtava}

\begin{tehtava}
    Ratkaise epäyhtälö $a^2x^2+ax-2 \geq 0$ vapaan parametrin $a$ funktiona.
    \begin{vastaus}
        $a < 0$: $x \geq -\frac{2}{a}$ tai $x \leq \frac{1}{a}$ \\ $a = 0$: ei ratkaisuja \\ $a > 0$: $x \geq \frac{1}{a}$ tai $x \leq -\frac{2}{a}$
    \end{vastaus}
\end{tehtava}

\end{tehtavasivu}