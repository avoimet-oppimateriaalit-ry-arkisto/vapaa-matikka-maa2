\begin{tehtavasivu}

\subsubsection*{Opi perusteet}
%
%\begin{luoKuva}{teht1}
%lukusuora.pohja(-12, 12, 3, varaa_tila = False)
%lukusuora.kohta(0, "$0$")
%lukusuora.vali(-10, 10, False, False, "$-10$", "$10$")
%\end{luoKuva}
%
%\begin{luoKuva}{teht2}
%lukusuora.pohja(-11, 9, 3, varaa_tila = False)
%lukusuora.kohta(0, "$0$")
%lukusuora.vali(-9, 7, False, True, "$-9$", "$7$")
%\end{luoKuva}
%
%\begin{luoKuva}{teht3}
%lukusuora.pohja(-2, 12, 3, varaa_tila = False)
%lukusuora.kohta(0, "$0$")
%lukusuora.vali(5, None, True, False, "$5$", "$turha$")
%\end{luoKuva}
%
%\begin{luoKuva}{teht4}
%lukusuora.pohja(-1, 9, 3, varaa_tila = False)
%lukusuora.kohta(0, "$0$")
%lukusuora.vali(5, 7.5, True, True, "$5$", "$7,5$")
%\end{luoKuva}
%
%\begin{luoKuva}{teht5}
%lukusuora.pohja(-1, 7, 3, varaa_tila = False)
%lukusuora.kohta(0, "$0$")
%lukusuora.vali(1, 5, False, True, "$1$", "$5$")
%\end{luoKuva}

\begin{tehtava}
    Ratkaise seuraavat epäyhtälöt.
    \begin{alakohdat}
        \alakohta{$3x+6<4x$}
        \alakohta{$3x-6<2x+57$}
        \alakohta{$5y-2<12$}
        \alakohta{$3\leq y+9$}
        \alakohta{$z-5\geq-888$}
		\alakohta{$2z+5\leq 42z-995$}
    \end{alakohdat}
    \begin{vastaus}
        \begin{alakohdat}
            \alakohta{$x>6$}
            \alakohta{$x<63$}
            \alakohta{$y<2,8$}
            \alakohta{$y\geq -6$}
            \alakohta{$z\geq -883$}
			\alakohta{$z\geq 25$}
        \end{alakohdat}
    \end{vastaus}
\end{tehtava}

\begin{tehtava}
Maalipurkki sisältää $10$ litraa maalia. Maalin riittoisuus on noin $6\,\text{m}^2/\text{l}$. Talon ulkoseinän korkeus on $4,5$\,m. Ulkoseinälle tulevat laudat on maalattava kahteen kertaan. Riittääkö maali, jos maalattavan seinän pituus on
	\begin{alakohdat}
		\alakohta{$5$\,m}
		\alakohta{$10$\,m}
		\alakohta{Kuinka pitkälle seinälle yhden purkillisen sisältämä maali riittää?}
	\end{alakohdat}
	\begin{vastaus}
		\begin{alakohdat}
			\alakohta{riittää ($22,5~\text{m}^2 < 30~\text{m}^2$)}
			\alakohta{ei riitä ($45~\text{m}^2 > 30~\text{m}^2$)}
			\alakohta{noin $6,7$\,m seinälle}
		\end{alakohdat}

	\end{vastaus}
\end{tehtava}

\subsubsection*{Hallitse kokonaisuus}

\begin{tehtava}
    Ratkaise seuraavat yhtälöt tai epäyhtälöt.
    \begin{alakohdat}
        \alakohta{$-2r+6=0$}
        \alakohta{$-2r+6\leq 0$}
        \alakohta{$5y-2<y+6$}
        \alakohta{$8(x+2)\geq -5(5-x)+3$}
        \alakohta{$\frac{x+3}{2}+\frac{-2x+1}{3}>\frac{x-9}{4}$}
    \end{alakohdat}
    \begin{vastaus}
        \begin{alakohdat}
            \alakohta{$r=3$}
            \alakohta{$r\geq 3$}
            \alakohta{$y<2$}
            \alakohta{$x=-12\frac{2}{3}$}
            \alakohta{$x<9\frac{4}{5}$}
        \end{alakohdat}
    \end{vastaus}
\end{tehtava}

\begin{tehtava}
	Tietyn auton käyttövoimavero on $450$\,€/vuosi, ja keskimääräinen kulutus dieselöljyä käytettäessä on $5$ litraa/$100$\,km. Saman valmistajan vastaava bensiinikäyttöinen auto kuluttaa $8$ litraa/$100$\,km. Diesel maksaa $1,55$\,€/litra, ja bensiini maksaa $1,65$\,€/litra. Kun vain annetut tiedot huomioidaan, niin kuinka paljon esimerkin dieselajoneuvolla tulee vähintään ajaa vuodessa, jotta se on edullisempi? Autojen ostohintoja ei huomioida.
    \begin{vastaus}
        $8\,257$\,km
    \end{vastaus}
\end{tehtava}

\begin{tehtava}
    Ratkaise epäyhtälöt.
    \begin{alakohdat}
        \alakohta{$3x+6<2x\leq 9-x$}
        \alakohta{$3x+6<2x\leq 1+3x$}
    \end{alakohdat}
    \begin{vastaus}
        \begin{alakohdat}
            \alakohta{$x<-6$}
            \alakohta{ei ratkaisua}
        \end{alakohdat}
    \end{vastaus}
\end{tehtava}

\begin{tehtava}
	Millä $x$:n arvoilla luvut $2x - 5$, $-x$ ja $x + 4$ ovat erisuuria ja $2x - 5$ on luvuista
	\begin{alakohdat}
		\alakohta{suurin}
		\alakohta{toiseksi suurin}
		\alakohta{pienin?}
	\end{alakohdat}
	\begin{vastaus}
		\begin{alakohdat}
			\alakohta{$x > 9$}
			\alakohta{$\frac{5}{3} < x < 9$}
			\alakohta{$x < \frac{5}{3}$}
		\end{alakohdat}
	\end{vastaus}
\end{tehtava}

\subsubsection*{Lisää tehtäviä}

\begin{tehtava}
    Ratkaise seuraavat epäyhtälöt.
    \begin{alakohdat}
        \alakohta{$33x+2\geq 27x+6$}
        \alakohta{$3x-6\geq 4x-6$}
        \alakohta{$5y+5\geq 15$}
        \alakohta{$3y+2\geq 2y-1$}
        \alakohta{$z\geq 2z+1\,000$}
		\alakohta{$z-1\geq z+1$}
    \end{alakohdat}
    \begin{vastaus}
        \begin{alakohdat}
            \alakohta{$x\geq \frac{2}{3}$}
            \alakohta{$x\leq 0$}
            \alakohta{$y\geq 2$}
            \alakohta{$y\geq -3$}
            \alakohta{$z\leq -1000$}
			\alakohta{ei ratkaisuja}
        \end{alakohdat}
    \end{vastaus}
\end{tehtava}

\begin{tehtava}
Lukion päättötodistuksessa aineen arvosana määräytyy aineen pakollisten ja syventävien kurssien keskiarvosta pyöristettynä kokonaisluvuksi tavallisten sääntöjen mukaan. Opiskelija haluaa filosofian päättöarvosanakseen $7$ tai paremman. Opiskelija aikoo osallistua kolmelle filosofian kurssille. Kahden kurssin jälkeen hänen arvosanojensa keskiarvo on $6$. Mikä arvosana on opiskelijan vähintään saatava kolmannesta kurssista? Kurssit arvioidaan asteikolla 
$4$--$10$.
\begin{vastaus}
Muodostettava epäyhtälö on muotoa $\frac{2\cdot 6+n}{3}\geq 6,5$, missä $n \in {4,5,6,7,8,9,10}$, josta ratkaisuna saadaan $n\geq 7,5$. Opiskelija tarvitsee siis vähintään arvosanan $8$.
\end{vastaus}
\end{tehtava}

\begin{tehtava}
Ratkaise kaksoisepäyhtälö
\[ 2x-1 < x \leq 3+5x.  \]
    \begin{vastaus}
        $-\frac{3}{4} \leq x < 1$
    \end{vastaus}
\end{tehtava}

\end{tehtavasivu}