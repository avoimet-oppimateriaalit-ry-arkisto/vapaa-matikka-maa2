\section{Tehtäviä ylioppilaskokeista}

%erotellaan selvästi oppimäärät toisistaan myös muissa kirjoissa! T: Joonas :)

\subsubsection*{Lyhyen oppimäärän tehtäviä}



\begin{tehtava}
(s2013/1a) Ratkaise yhtälö $(x-2)^2=4$.
\begin{vastaus}
$x=0$ tai $x=4$ 
\end{vastaus}
\end{tehtava}

\begin{tehtava}
(k2013/2)
	\begin{alakohdat}
		\alakohta{ Millä muuttujan $x$ arvoilla $4x+17$ on suurempi kuin $2-x$?}
		\alakohta{Ratkaise yhtälö $x^2+14x=-49$.}
	\end{alakohdat}
\begin{vastaus}
	\begin{alakohdat}
		\alakohta{$x>-3$}
		\alakohta{$x=-7$}
	\end{alakohdat}
\end{vastaus}
\end{tehtava}̈́

\begin{tehtava}
(s2012/1a) Ratkaise yhtälö $x^2-2x = 0$.
\begin{vastaus}
$x=0$ tai $x=2$
\end{vastaus}
\end{tehtava}

\begin{tehtava}
(k2012/1a) Ratkaise yhtälö $7x+3 = 31$.
\begin{vastaus}
$x = 4$
\end{vastaus}
\end{tehtava}

\begin{tehtava}
(s2011/1c) \\ Ratkaise yhtälö $x^2-3(x+3) = 3x-18$.
\begin{vastaus}
$x=3$
\end{vastaus}
\end{tehtava}

% \begin{tehtava}
% (K2011/1a) \\ Ratkaise yhtälö $4x+(5x-4) = 12+3x$.
% \begin{vastaus}
% $x=\frac{16}{3} = 5\frac13$
% \end{vastaus}
% \end{tehtava}

\begin{tehtava}
(k2011/1b) Sievennä lauseke $x^2+x-(x^2-x)$.
\begin{vastaus}
$2x$
\end{vastaus}
\end{tehtava}

\begin{tehtava}
(k2011/2b) \\ Sievennä lauseke $(\sqrt{x}-1)^2+2\sqrt{x}$.
\begin{vastaus}
$x+1$
\end{vastaus}
\end{tehtava}


\subsubsection*{Pitkän oppimäärän tehtäviä}


\begin{tehtava}
(K2013/1b) \\ Ratkaise epäyhtälö $\frac{3}{5}x-\frac{7}{10} < -\frac{2}{15}x$.
\begin{vastaus}
$x<\frac{21}{22}$
\end{vastaus}
\end{tehtava}

\begin{tehtava}
(K2013/3a) \\ Laske lausekkeen $(\sqrt{a}+\sqrt{b})^2$ tarkka arvo, kun positiiviset luvut $a$ ja $b$ ovat toistensa käänteislukuja ja lukujen $a$ ja $b$ keskiarvo on $2$.
\begin{vastaus}
$6$
\end{vastaus}
\end{tehtava}

\begin{tehtava}
(K2013/*14a) \\
	\begin{alakohdat}
	\alakohta{ Jaa $P(x)=x^2+x-2$ ensimmäisen asteen tekijöihin. (2 p.)}
	\alakohta{  Olkoon $P(x)=x^2+x-2$. Määritä sellaiset vakiot $A$ ja $B$, että $\frac{1}{P(x)}=\frac{A}{x-1}+\frac{B}{x+2} $ kaikilla $x \geq 2$. (2 p.)}
	\end{alakohdat}
\begin{vastaus}
\begin{alakohdat}
	\alakohta{$(x+2)(x-1)$}
	\alakohta{$A= \frac{1}{3}$ ja $B=- \frac{1}{3}$}
	\end{alakohdat}
\end{vastaus}
\end{tehtava}


\begin{tehtava}
(S2013/1a) \\
\begin{alakohdat}
\alakohta{Ratkaise yhtälö $x^2+6x=2x^2+9$.}
\alakohta{Ratkaise yhtälö $\frac{1+x}{1-x}=\frac{1-x^2}{1+x^2}$.}
\alakohta{Esitä polynomi $x^2-9x+14$ ensimmäisen asteen polynomien tulona.}
\end{alakohdat}

\begin{vastaus}
	\begin{alakohdat}
		\alakohta{$x=3$}
		\alakohta{$x=0$ tai $x=-1$}
		\alakohta{$(x-2)(x-7)$}
	\end{alakohdat}
\end{vastaus}
\end{tehtava}


\begin{tehtava}
(S2012/1a) \\ Ratkaise yhtälö $2(1-3x+3x^2) = 3(1+2x+2x^2)$.
\begin{vastaus}
$x=-\frac{1}{12}$
\end{vastaus}
\end{tehtava}

% (S2011, 1b) Suorakulmaisen kolmion hypotenuusan pituus on 5 % Kurssi 3
%   ja toisen kateetin pituus 2. Laske toisen kateetin pituus.

\begin{tehtava}
  (S2011/3b) Ratkaise epäyhtälö $\frac{2x+1}{x-1} \geq 3$.
\begin{vastaus}
$1<x \leq 4$
\end{vastaus}
\end{tehtava}

\begin{tehtava}
(K2011/1b) Ratkaise epäyhtälö $x^2-2 \leq x$.
\begin{vastaus}
$-1 \leq x \leq 2$
\end{vastaus}
\end{tehtava}

\begin{tehtava}
(S2010/1a) Sievennä lauseke $(a+b)^2-(a-b)^2$.
\begin{vastaus}
$4ab$
\end{vastaus}
\end{tehtava}

\begin{tehtava}
(S2010/2a) Ratkaise epäyhtälö $x\sqrt{7}-3 \leq 4x$.
\begin{vastaus}
$x \geq \frac{3}{\sqrt{7}-4}$
\end{vastaus}
\end{tehtava}

\begin{tehtava}
(S2010/2c) Ratkaise yhtälö $x^4-3x^2-4=0$.
\begin{vastaus}
$x=2$ tai $x=-2$
\end{vastaus}
\end{tehtava}

% Murtolukujen jakokulmaa ei tule tässä kurssissa.
% \begin{tehtava}
% (S2010/12) Määritä $a$ siten, että polynomi $P(x)=2x^4-3x^3-7x^2+a$ on jaollinen binomilla $2x-1$. Määritä tätä $a$:n arvoa vastaavat yhtälön $P(x)=0$ juuret.
% \begin{vastaus}
% $a=2$. Yhtälön $P(x)=0$ juuret ovat $x=\frac{1}{2}$, $x=-1$, $x=1+\sqrt{3}$ ja $x=1-\sqrt{3}$.
% \end{vastaus}
% \end{tehtava}

\begin{tehtava}
(K2010/1a) Ratkaise yhtälö $7x^7+6x^6=0$.
\begin{vastaus}
$x=0$ tai $x=-\frac{6}{7}$
\end{vastaus}
\end{tehtava}

\begin{tehtava}
(K2010/1b) Sievennä lauseke $(\sqrt{a}+1)^2-a-1$.
\begin{vastaus}
$2\sqrt{a}$
\end{vastaus}
\end{tehtava}

\begin{tehtava}
(K2010/1c) Millä $x$:n arvoilla pätee $\frac{3}{3-2x}<0$?
\begin{vastaus}
$x>\frac{3}{2}$
\end{vastaus}
\end{tehtava}


\begin{tehtava}
(K2010/3b) Määritä toisen asteen yhtälön $x^2+px+q=0$ kertoimet $p$ ja $q$, kun yhtälön juuret ovat $-2-\sqrt{6}$ ja $-2+\sqrt{6}$.
\begin{vastaus}
$p=4$, $q=-2$
\end{vastaus}
\end{tehtava}

\begin{tehtava}
(S2009/1a) Ratkaise yhtälö $(x-2)(x-3)=6$. 
\begin{vastaus}
$x=0$ tai $x=5$
\end{vastaus}
\end{tehtava}

\begin{tehtava}
(S2009/1b) Ratkaise yhtälö $\frac{x}{x-3}-\frac{1}{x}=1$.
\begin{vastaus}
$x=-\frac{3}{2}$
\end{vastaus}
\end{tehtava}

\begin{tehtava}
(S2009/2a) Ratkaise epäyhtälö $6(x-1)+4 \geq 3(7x+1)$. 
\begin{vastaus}
$x \leq -\frac{1}{3}$
\end{vastaus}
\end{tehtava}

% 7. kurssin asiaa:
% \begin{tehtava}
% (S2009/8) Ratkaise epäyhtälö $\frac{-x^2+x+2}{x^3+2x^2-3x}>0$.
% \begin{vastaus}
% $x<-3$ tai $-1<x<0$ tai $1<x<2$
% \end{vastaus}
% \end{tehtava}

\begin{tehtava}
(K2009/1b) Ratkaise epäyhtälö$(x-3)^2>(x-1)(x+1)$.
\begin{vastaus}
$x<\frac{5}{3}$
\end{vastaus}
\end{tehtava}

\begin{tehtava}
(S2008/1a) Ratkaise epäyhtälö $\frac{1}{2} - \frac{x}{3} > \frac{3}{4}$.
\begin{vastaus}
$x<-\frac{3}{4}$
\end{vastaus}
\end{tehtava}

% Erillisten murtolausekkeiden laventamista samannimisiksi -> ehkä enemmän MAA1-asiaa.
% \begin{tehtava}
% (S2008/1b) Sievennä lauseke $\frac{1}{x}-\frac{1}{x^2}+ \frac{1+x}{x^2}$.
% \begin{vastaus}
% $ \frac{2}{x}$
% \end{vastaus}
% \end{tehtava}

\begin{tehtava}
(S2008/3b) Ratkaise yhtälö $4x^3-5x^2=2x-3x^3$.
\begin{vastaus}
$x=-\frac{2}{7}$ tai $x=0$ tai $x=1$
\end{vastaus}
\end{tehtava}


\begin{tehtava}
(K2008/1a) Ratkaise yhtälö $2x^2=x+1$.
\begin{vastaus}
$x=1$ tai $x=-\frac{1}{2}$
\end{vastaus}
\end{tehtava}

\begin{tehtava}
(S2007/1a) Ratkaise epäyhtälö $2-3x>4x$.
\begin{vastaus}
$x< \frac{2}{7} $
\end{vastaus}
\end{tehtava}

\begin{tehtava}
(K2007/1a) Ratkaise yhtälö $7x^2-6x=0$.
\begin{vastaus}
$x=0$ tai $x=\frac{6}{7}$
\end{vastaus}
\end{tehtava}

\begin{tehtava}
(S2006/1a) Sievennä lauseke $(1+x)^3-(1-x)^3$.
\begin{vastaus}
$2x^3+6x$ 
\end{vastaus}
\end{tehtava}

\begin{tehtava}
(S2006/1b) Ratkaise yhtälö $ \frac{x+1}{x}=\frac{x}{x+1}$.
\begin{vastaus}
$x=-\frac{1}{2}$ 
\end{vastaus}
\end{tehtava}

\begin{tehtava}
(S2005/1b) Ratkaise reaalilukualueella yhtälö $x+2=\frac{1}{x-2}$.
\begin{vastaus}
$x=\pm \sqrt{5}$ 
\end{vastaus}
\end{tehtava}

\begin{tehtava}
(S2005/4) Millä $a$:n arvoilla funktio $f(x)=-x^2+ac+a-3$ saa vain negatiivisia arvoja?
\begin{vastaus}
$-6<a<2$ 
\end{vastaus}
\end{tehtava}

\begin{tehtava}
(K2005/1a) Sievennä lauseke $\frac{x}{1-x}+\frac{x}{1+x}$.
\begin{vastaus}
$\frac{2x}{1-x^2}$ 
\end{vastaus}
\end{tehtava}

\begin{tehtava}
(K2005/1b) Ratkaise $x$ yhtälöstä $x^2-ax-a^2=0$.
\begin{vastaus}
$x= \frac{1}{2}a(1-\sqrt{5})$ tai $x= \frac{1}{2}a(1+\sqrt{5})$ 
\end{vastaus}
\end{tehtava}

\begin{tehtava}
(S2004/1a) Ratkaise epäyhtälö $2x-3<3-2x$.
\begin{vastaus}
$x<\frac{3}{2}$ 
\end{vastaus}
\end{tehtava}

\begin{tehtava}
(S2004/1b) Ratkaise epäyhtälö $(x+1)^2 \leq 1$.
\begin{vastaus}
$-2 \leq x \leq 0$ 
\end{vastaus}
\end{tehtava}

\begin{tehtava}
(S2004/1c) Ratkaise epäyhtälö $x^3<x^2$.
\begin{vastaus}
$x<1$, $x\neq0$ 
\end{vastaus}
\end{tehtava}

\begin{tehtava}
(K2004/1c) Olkoon $f(x)=x^3+3x^2+x+1$ ja $g(x)=x^3+x^2-2x+3$. Ratkaise yhtälö $f(x)=g(x)$.
\begin{vastaus}
$x=-2$ tai $x=\frac{1}{2}$  
\end{vastaus}
\end{tehtava}

\begin{tehtava}
(K1988/3) Millä $a$:n arvoilla yhtälön $x^2+(3a+1)x+81=0$ juuret ovat reaaliset?
\begin{vastaus}
$a \leq -\frac{19}{3}$ tai $a \geq \frac{17}{3}$
\end{vastaus}
\end{tehtava}

%fixme: etsi lisää yo-tehtäviä, näitä on kyllä olemassa