\section{Tulon nollasääntö ja tulon merkkisääntö}

\subsection*{Tulon merkkisääntö}

Pitkän matematiikan 1. kurssilla on esitetty seuraava sääntö kahden reaaliluvun tulolle:

\laatikko[Tulon merkkisääntö kahdelle tulon tekijälle]{
    \begin{itemize}
        \item Jos tulon tekijät ovat samanmerkkisiä, tulo on positiivinen.
        \begin{itemize}
	        \item Kahden positiivisen luvun tulo on positiivinen.
	        \item Kahden negatiivisen luvun tulo on positiivinen.
        \end{itemize}
        \item Jos tulon tekijät ovat erimerkkisiä, tulo on negatiivinen.
        \begin{itemize}
	        \item Positiivisen ja negatiivisen luvun tulo on negatiivinen.
        \end{itemize}
    \end{itemize}
}

Tulon merkkisäännöstä seuraa, että reaaliluvun neliö ei voi olla negatiivinen, koska kahden samanmerkkisen luvun tulo aina on positiivinen. Lyhyemmin ilmaistuna $x^2 \geq 0$, eli reaaliluvun neliö on aina \termi{epänegatiivinen}{epänegatiivinen}.

\begin{esimerkki}
Osoita, että funktio $P(x)=x^2+7$ saa vain positiivisia arvoja.
    \begin{esimratk}
	Koska $x^2 \geq 0$, lausekkeen $x^2+7$ arvo on vähintään $7$. Fuktio saa siis
	vain positiivisia arvoja.
    \end{esimratk}
\end{esimerkki}

\begin{esimerkki}
Mikä on funktion $f:\mathbb{R} \rightarrow \mathbb{R}, f(t)=-t^4+3$ suurin arvo?
    \begin{esimratk}
	Koska muuttuja $t$ on reaaliluku, ei sen parillinen potenssi voi olla negatiivinen, vaan $t^4$:n arvo on vähintään $0$. Tämän perusteella $-t^4$:n arvo on epäpositiivinen eli korkeintaan nolla. Jos $-t^4$:n suurin arvo on nolla, niin lisäämällä tähän luvun $3$, saadaan funktion $f(t)=-t^4+3$ suurimmaksi arvoksi $3$.
    \end{esimratk}
\end{esimerkki}


Tulon merkkisääntö yleistyy mille tahansa määrälle tulontekijöitä.
Mikäli tulossa on pariton $(1, 3, 5, \ldots)$ määrä negatiivisia tekijöitä, tulo on negatiivinen.
Muulloin tulo on positiivinen.

%Hyödynnämme merkkisääntöä myöhemmin, kun teemme epäyhtälöistä merkkikaavioita.

\newpage

\subsection*{Tulon nollasääntö}

Tulon merkkisäännöstä seuraa, että positiivisten ja negatiivisten lukujen tulo on aina positiivinen tai negatiivinen, ei koskaan nolla.
Jos siis tulo on $0$, tulon tekijöistä ainakin yhden täytyy olla $0$.
Toisaalta jos jokin tulon tekijöistä on $0$, myös tulo on automaattisesti $0$.
Nämä tiedot yhdistämällä saadaan tulon nollasääntö:

\laatikko[Tulon nollasääntö]{
	\begin{description}
		\item Jos jokin tulon tekijöistä on $0$, tulo on $0$.
		\item Jos tulo on $0$, ainakin yksi tulon tekijöistä on $0$.
	\end{description}
}

\begin{esimerkki}
Sievennä lauseke $(x^5-7)\cdot y \cdot 0\cdot(3a-5b)^2$.
    \begin{esimratk}
	Koska tulossa on tekijänä $0$, vastaus on $0$.
    \end{esimratk}
\end{esimerkki}

\begin{esimerkki} Ratkaistaan yhtälö $(x+5) \cdot x =0 $.
    \begin{align*}
        (x+5)\cdot x &=0 \quad \ppalkki \text{ tulon nollasääntö} \\
        x +5= 0 \text{ tai } x &=0 \\
        x= -5 \text{ tai } x &=0.
    \end{align*}
    Ratkaisuja on siis kaksi, $x= -5$ tai $x= 0$.
\end{esimerkki}

%\begin{esimerkki}
%	\[2(x+5)=0\]
%	Nyt tulon nollasäännön perusteella tiedetään, että $2=0$ tai $x+5=0$.
%	Koska selvästi $2\neq 0$, jää ainoaksi ratkaisuksi $x+5=0$ eli $x=-5$.
%\end{esimerkki}

% \begin{esimerkki} Ratkaistaan $y$ yhtälöstä
%     \[(x^5+5x+5)\cdot 0\cdot \sqrt{x^3-1} =y\]
%     Koska vasemmalla puolella yksi tulon tekijöistä on $0$, tiedämme, että tulo on $0$. Siis $y=0$.
% \end{esimerkki}

%TODO: onko alla oleva järkevä? mistä tietää, mitä muuttujia on tarkoitus ratkaista. minusta vain sekoittaa asioita T: Jokke
\begin{esimerkki} Ratkaise yhtälö
    \[xyz=0.\]
Tulon nollasäännön perusteella $x=0$, $y=0$ tai $z=0$. Nollia voi siis
olla 1--3 kappaletta.
\end{esimerkki}

\begin{tehtavasivu}

\paragraph*{Opi perusteet}

\begin{tehtava}
	Laske.
	\begin{alakohdat}
		\alakohta{$2 \cdot 3$}
		\alakohta{$2 \cdot (-3)$}
		\alakohta{$-2 \cdot 3$}
		\alakohta{$-2 \cdot (-3)$}
		\alakohta{$17 \cdot 666 \cdot 0 \cdot (-31)$}
	\end{alakohdat}
	
	\begin{vastaus}
		\begin{alakohdat}
			\alakohta{$6$}
			\alakohta{$-6$}
			\alakohta{$-6$}
			\alakohta{$6$}
			\alakohta{$0$}
		\end{alakohdat}
	\end{vastaus}
\end{tehtava}

\begin{tehtava}
	Olkoon $a>0$, $b<0$, ja $c=0$. Mitä voit päätellä tulon merkistä?
	\begin{alakohdat}
		\alakohta{$a \cdot a$}
		\alakohta{$b \cdot a$}
		\alakohta{$b \cdot b$}
		\alakohta{$b \cdot c$}
	\end{alakohdat}
	
	\begin{vastaus}
		\begin{alakohdat}
			\alakohta{tulo $>0$}
			\alakohta{tulo $<0$}
			\alakohta{tulo $>0$}
			\alakohta{tulo on $0$}
		\end{alakohdat}
	\end{vastaus}
\end{tehtava}


\begin{tehtava}
    Ratkaise seuraavat yhtälöt käyttämällä tulon nollasääntöä.
    \begin{alakohdat}
        \alakohta{$x(3+x)=0$}
        \alakohta{$(x-4)(x+3)=0$}
		\alakohta{$0=x^2(x-5)$}
    \end{alakohdat}
    \begin{vastaus}
        \begin{alakohdat}
            \alakohta{$x=0$ tai $x=-3$}
            \alakohta{$x=4$ tai $x=-3$}
            \alakohta{$x=0$ tai $x=5$}
        \end{alakohdat}
    \end{vastaus}
\end{tehtava}

\paragraph*{Hallitse kokonaisuus}

\begin{tehtava}
	Olkoon $a > 0$. Mitkä vaihtoehdoista $b>0$, $b<0$, $b=0$ ovat mahdollisia, jos
	tiedetään, että
	\begin{alakohdat}
		\alakohta{$a \cdot b > 0$}
		\alakohta{$a \cdot b \leq 0$}
		\alakohta{$b \cdot b > 0$}
		\alakohta{$b \cdot b < 0$?}
	\end{alakohdat}
	\begin{vastaus}
		\begin{alakohdat}
			\alakohta{$b>0$}
			\alakohta{$b < 0$ ja $b = 0$}
			\alakohta{$b>0$ ja $b<0$}
			\alakohta{Mikään vaihtoehto ei kelpaa.}
		\end{alakohdat}
	\end{vastaus}
\end{tehtava}

\begin{tehtava}
    Ratkaise seuraavat yhtälöt käyttämällä tulon nollasääntöä.
    \begin{alakohdat}
        \alakohta{$y(y+4)=0$}
        \alakohta{$(x-2)(x-1)(x+5)=0$}
        \alakohta{$(x+1)(x^2+2)=0$}
    \end{alakohdat}
    \begin{vastaus}
        \begin{alakohdat}
            \alakohta{$y=0$ tai $y=-4$}
            \alakohta{$x=2$, $x=1$ tai $x=-5$}
            \alakohta{$x=-1$}
        \end{alakohdat}
    \end{vastaus}
\end{tehtava}

\begin{tehtava} 
Osoita, että funktio $f(x)=x^4+3x^2+1$ saa vain positiivisia arvoja.
    \begin{vastaus}
     $x^4\geq 0$ ja $x^2 \geq 0$, joten $f(x) \geq 1$.
    \end{vastaus}
\end{tehtava}

\paragraph*{Lisää tehtäviä}

\begin{tehtava}
	Olkoon $a \geq 0$, $b \leq 0$, ja $c=0$. Mitä voit päätellä tulon merkistä?
	\begin{alakohdat}
		\alakohta{$a \cdot a$}
		\alakohta{$b \cdot a$}
		\alakohta{$b \cdot c$}
	\end{alakohdat}
	
	\begin{vastaus}
		\begin{alakohdat}
			\alakohta{tulo $\geq 0$}
			\alakohta{tulo $\leq 0$}
			\alakohta{tulo on $0$}
		\end{alakohdat}
	\end{vastaus}
\end{tehtava}

\begin{tehtava}
    Ratkaise seuraavat yhtälöt käyttämällä tulon nollasääntöä.
    \begin{alakohdat}
        \alakohta{$(\smiley{}+1)\cdot (t+1)=0$}
        \alakohta{$x(x-5)=0$}
        \alakohta{$(2w+2)^2=0$}
    \end{alakohdat}
    \begin{vastaus}
        \begin{alakohdat}
            \alakohta{$\smiley{}=-1$ tai $t=-1$,\qquad  Symboli $\smiley{}$ esittää jotain lukua, sillä muutoin laskutoimitukset eivät olisi mielekkäitä. Tehtävässä ei myöskään ole selvää, minkä muuttujan suhteen yhtälö pitäisi ratkaista. Siksi on ratkaistu molempien muuttujien suhteen.}
            \alakohta{$x=0$ tai $x=5$}
            \alakohta{$w=-1$}
        \end{alakohdat}
    \end{vastaus}
\end{tehtava}

\begin{tehtava}
    Sievennä seuraava lauseke: $(a-x)\cdot(b-x)\cdot(c-x)\cdot...\cdot(\mathring{a}-x)\cdot(\ddot{a}-x)\cdot(\ddot{o}-x)$.
    \begin{vastaus}
        Tulossa esiintyy tekijänä $(x-x)=0$. Niinpä tulon nollasäännön mukaan
        \begin{align*}
            &(a-x)\cdot(b-x)\cdot(c-x)\cdot...\cdot(x-x)\cdot...\cdot(\ddot{a}-x)\cdot(\ddot{o}-x) \\
            =&(a-x)\cdot(b-x)\cdot(c-x)\cdot...\cdot 0\cdot...\cdot(\ddot{a}-x)\cdot(\ddot{o}-x) \\
            =&0
        \end{align*}
    \end{vastaus}
\end{tehtava}



\begin{tehtava} %
$\star$ Osoita, että $x^2+\frac{1}{x^2}\geq 2$, kun $x \neq 0$.
    \begin{vastaus}
     Aloita tiedosta $\left(x-\frac{1}{x}\right)^2 \geq 0$ ja sievennä.
    \end{vastaus}
\end{tehtava}

\begin{tehtava} 
$\star$ Osoita, että kun $a \geq 0$ ja $b \geq 0$, pätee \\ $\frac{a+b}{2} \geq \sqrt{ab}$. Milloin yhtäsuuruus on voimassa?
    \begin{vastaus}
     Opastus: Aloita tiedosta $\left(\sqrt{a}-\sqrt{b}\right)^2 \geq 0$ ja sievennä. Yhtäsuuruus pätee, kun $a = b$.
    \end{vastaus}
\end{tehtava}

\end{tehtavasivu}
