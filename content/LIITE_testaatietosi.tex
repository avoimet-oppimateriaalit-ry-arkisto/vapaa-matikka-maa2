\section{Testaa tietosi!}

\subsection{Osaatko selittää?}

\begin{tehtava}
Miksi polynomin vakiotermin (tai vakiopolynomin) asteluku on nolla?
	\begin{tehtava}
	Koska mielivaltaisen vakion $a$ voidaan aina tulkita olevan kerroin muuttujan potenssille $x^0$.
	\end{tehtava}
\end{tehtava}

\begin{tehtava}
Miksi puhutaan vain termien summasta, vaikka polynomeissa voi esiintyä miinusmerkkejä?
	\begin{tehtava}
	Vähennyslasku määritellään $a-b=a+(-b)$, eli vastaluvun avulla kaikki vähenyslaskut voidaan tulkita yhteenlaskuina.
	\end{tehtava}
\end{tehtava}

\begin{tehtava}
Miten usean muuttujan polynomin asteluku lasketaan?
	\begin{tehtava}
	Niin kuin yhdenkin muuttujan polynomin tapauksessa, korkein termin aste eli muuttujan potenssin eksponentti määrää koko polynomin asteen. Jos termissä on kerrottuna eri muuttujia keskenään, termin asteluku on näiden muuttujien potenssien eksponenttien summa.
	\end{tehtava}
\end{tehtava}

\begin{tehtava}
Mitä tarkoitetaan tulon nollasäännöllä?
	\begin{tehtava}
\textit{Kahta} asiaa: toisaalta mikä tahansa tulo, jossa tekijänä on nolla, on arvoltaan nolla. Toisaalta, jos tulo on nolla, niin tiedetään, että varmasti ainakin yksi tulon tekijöistä on nolla.
	\end{tehtava}
\end{tehtava}

\begin{tehtava}
Kun polynomi jaetaan tekijöihin, mitä voit varmasti sanoa tekijäpolynomien asteista verrattuna alkuperäiseen polynomiin?
	\begin{tehtava}
Tekijät ovat alhaisempaa astetta kuin alkuperäinen polynomi, kunhan tekijänä ei ole vakiopolynomi.
	\end{tehtava}
\end{tehtava}

\begin{tehtava}
Miten selvität polynomin $(x^3-2x)^{15}$ asteen purkamatta sulkuja?
	\begin{tehtava}
a
	\end{tehtava}
\end{tehtava}

\begin{tehtava}
Miksi epäyhtälön ''suunta muuttuu'', kun se kerrotaan puolittain negatiivisella luvulla?
	\begin{tehtava}
a
	\end{tehtava}
\end{tehtava}


\subsection{Monivalinta}

Valitse monivalintatehtissä yksi paras vaihtoehto.

\begin{tehtava}
Mitkä ovat polynomin $x^4-x^2+\sqrt{2}x-1\,337$ termien kertoimet neljännestä asteesta pienempään luetellen?
	\begin{alakohdat}
	\alakohta{$1, -1, \sqrt{2}$ ja $-1\,337$}
	\alakohta{$1, 0, -1, \sqrt{2}$ ja $-1\,337$}
	\alakohta{$1, 0, -1, \sqrt{2}$ ja $1\,337$}
	\alakohta{$1, 1, \sqrt{2}$ ja $1\,337$}
	\end{alakohdat}
	\begin{vastaus}
	b) $1, 0, -1, \sqrt{2}$ ja $-1\,337$
	\end{vastaus}
\end{tehtava}

\begin{enumerate}



\item Monettako astetta on kahden muuttujan polynomi $x^5y-x^4y^3-715\,517y$?
\item Polynomifunktion $P(x)=-x^3-x^2-x-1$ arvo kohdassa $x=-1$?
\item Mikä polynomi saadaan, kun avataan sulut lausekkeesta $(2x-1)^2$?
\item Tulosta $ab$ tiedetään, että $a = 0$. Mikä seuraavista väitteistä pitää välttämätät paikkansa?
a) $ab = 0$
b) $b \geq a$
c) $b = 0$
d) $|b| \geq a$

\item Mitkä on funktion $f(x) = (x+3)(x-50)$ nollakohdat
\item Olkoon $f(x) = x^2 + 1$, ja $g(x) = x^2 - x$. Mitkä väitteistä pitää paikkaansa.
a) $f(x) + g(x) = 2x^2 - x + 1$
b) $f(x) \cdot g(x) = x^4-x^3+x^2-x$
c) $f(0) = g(0)$
d) $f(-1) = g(-1)$
\end{enumerate}

Pysyykö epäyhtälö yhtäpitävänä, jos
a) lisäämällä kummallekkin puolelle sama lauseke
b) kertomalla kummatkin puolet samalla luvulla
c) korotetaan kummatkin puolet potenssiin n
d) kerrotaan negatiivisellä luvulla ja käännetään epäyhtälön suunta.

\subsection{Tosi vai epätosi?}

Pitävätkö väitteet paikkansa?

\begin{tehtava}
	\begin{alakohdat}
	\alakohta{$\sqrt(2)$ on irrationaaliluku muttei transkendenttiluku.}
	\alakohta{Yhtälö $(3x+1)(\sqrt[3]{x}-1)=1$ ratkeaa tulon nollasäännön avulla.}
	\end{alakohdat}
	
	\begin{vastaus}
	\begin{alakohdat}
	\alakohta{Tosi}
	\alakohta{Epätosi}
	\end{alakohdat}
	\end{vastaus}
\end{tehtava}