\begin{Vastaus}{186}
        \begin{alakohdat}
            \alakohta{$x^2+4x+4$}
            \alakohta{$x^2-10x+25$}
            \alakohta{$b^2-16$}
            \alakohta{$x^2-2x+1$}
            \alakohta{$4x^2+4x+1$}
            \alakohta{$1-a^2$}
            \alakohta{$x^2-14x+49$}
            \alakohta{$y^2+2y+1$}
            \alakohta{$x^2+6xy+9y^2$}
            \alakohta{$9x^2-6x+1$}
            \alakohta{$4x^2-1$}
            \alakohta{$t^2-4t+4$}
            \alakohta{$4x^2+12x+9$}
            \alakohta{$a^2-4a+4$}
            \alakohta{$25x^2-4$}
            \alakohta{$9-c^2$}
            \alakohta{$100x^2-20x+1$}
            \alakohta{$4a^2+4ab+b^2$}
            \alakohta{$x^2+12x+36$}
        \end{alakohdat}
    
\end{Vastaus}
\begin{Vastaus}{187}
        \begin{alakohdat}
            \alakohta{$(1-a)(1+a)$}
            \alakohta{$(a-2)^2$}
            \alakohta{$(2x+1)^2$}
            \alakohta{$(x-7)^2$}
            \alakohta{$(10x-1)^2$}
            \alakohta{$(2a+b)^2$}
            \alakohta{$(y+1)^2$}
   		     \alakohta{$(x+2)^2$}
            \alakohta{$(2x+1)(2x-1)$}
             \alakohta{$(t-2)^2$}
            \alakohta{$(5x+2)(5x-2)$}
    	    \alakohta{$(b+4)(b-4)$}
 	       \alakohta{$(x-1)^2$}
            \alakohta{$(3-c)(3+c)$}
            \alakohta{$(x+6)^2$}
            \alakohta{$(2x+3)^2$}
  		    \alakohta{$(x-5)^2$}
            \alakohta{$(x-3y)^2$}
            \alakohta{$(3x-1)^2$}
         \end{alakohdat}
    
\end{Vastaus}
\begin{Vastaus}{188}
        \begin{alakohdat}
            \alakohta{$x^4-2x^2+1$}
            \alakohta{$a^{12}+6a^6b^3+9b^6$}
            \alakohta{$-(144-9x^2)=9x^2-144$}
            \alakohta{$x^2+2+\frac{1}{x^2}$}
         \end{alakohdat}
    
\end{Vastaus}
\begin{Vastaus}{189}
    \begin{alakohdat}
            \alakohta{Tutki lukujen $2+\sqrt{3}$ ja $2-\sqrt{3}$ tuloa.}
	        \alakohta{Yleisesti $\left(a+\sqrt{a^2-1}\right)^{-1}= a-\sqrt{a^2-1}$}
    \end{alakohdat}
    
\end{Vastaus}
\begin{Vastaus}{190}
     Aloita tiedosta $\left(x-\frac{1}{x}\right)^2 \geq 0$ ja sievennä.
    
\end{Vastaus}
\begin{Vastaus}{191}
     $x^4\geq 0$ ja $x^2 \geq 0$, joten $f(x) \geq 1$.
    
\end{Vastaus}
\begin{Vastaus}{192}
     Opastus: Aloita tiedosta $\left(\sqrt{a}-\sqrt{b}\right)^2 \geq 0$ ja sievennä. Yhtäsuuruus pätee, kun $a = b$.
    
\end{Vastaus}
\begin{Vastaus}{193}
     $7,7$~cm
    
\end{Vastaus}
\begin{Vastaus}{194}
    $x =\frac{3 \pm \sqrt{13}}{2}$
    
\end{Vastaus}
\begin{Vastaus}{195}
		$10^2+11^2+12^2 = 13^2 + 14^2$.
    	Jos negatiivisetkin luvut sallittaisiin, $(-2)^2+(-1)^2+0^2 = 1^2 + 2^2$ kävisi 			myös. Löytyykö vastaava $4 + 3$ luvun sarja? Entä pidempi?
    
\end{Vastaus}
\begin{Vastaus}{196}
		\begin{alakohdat}
		\alakohta{$1,00$ \euro \ $<$ myyntihinta $<$ $4,5$ \euro.}
		\alakohta{$2,75$ \euro, jolloin voitto on $612,50$ \euro. }
		% Epäilyttävän hyvä bisnes
		\end{alakohdat}
    
\end{Vastaus}
\begin{Vastaus}{197}
	Luvut ovat $-1, 0$ ja $1$ tai $14, 15$ ja $16$.
    
\end{Vastaus}
\begin{Vastaus}{198}
	$x=\frac{6}{\sqrt[5]{2}-1}$
    
\end{Vastaus}
\begin{Vastaus}{199}
	$x \approx 0,69$, $x \approx 1,86$ tai $x \approx 3,03$.
    
\end{Vastaus}
\begin{Vastaus}{200}
		\begin{alakohdat}
		\alakohta{$-5<x<0$ tai $3 < x$}
		\alakohta{$x<0$ tai $x = 1$}
		\end{alakohdat}
    
\end{Vastaus}
\begin{Vastaus}{201}
	Luvut, jotka ovat pienempiä kuin $-1$ ja luvut välillä $]0,1[$.
    
\end{Vastaus}
\begin{Vastaus}{202}
	$P(x)>0$ kun $x > \sqrt{a}$ tai $-sqrt{a}<x<0$.
    
\end{Vastaus}
