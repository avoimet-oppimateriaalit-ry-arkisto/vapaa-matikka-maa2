\section{Kertaustehtäviä}


\begin{tehtavasivu}


\begin{tehtava}
   	Muistikaavat on opittava ulkoa ja niiden käytön tulee automatisoitua.
	Laske siis nämä käyttäen muistikaavoja. Tavoite on kirjoittaa vastaus suoraan ilman välivaiheita. Jos se ei vielä onnistu, yritä selvitä yhdellä välivaiheella.
    \begin{alakohdat}
        \alakohta{$(x+2)^2$}
        \alakohta{$(x-5)^2$}
        \alakohta{$(b+4)(b-4)$}
        \alakohta{$(x-1)^2$}
            \alakohta{$(2x+1)^2$}
            \alakohta{$(1-a)(1+a)$}
            \alakohta{$(x-7)^2$}
            \alakohta{$(y+1)^2$}
            \alakohta{$(x-3y)^2$}
            \alakohta{$(3x-1)^2$}
            \alakohta{$(2x+1)(2x-1)$}
            \alakohta{$(t-2)^2$}
            \alakohta{$(2x+3)^2$}
            \alakohta{$(2-a)^2$}
            \alakohta{$(5x+2)(5x-2)$}
            \alakohta{$(3-c)(3+c)$}
            \alakohta{$(10x-1)^2$}
            \alakohta{$(2a+b)^2$}            
            \alakohta{$(x+6)^2$}                                                                                                                                                        
    \end{alakohdat}
    \begin{vastaus}
        \begin{alakohdat}
            \alakohta{$x^2+4x+4$}
            \alakohta{$x^2-10x+25$}
            \alakohta{$b^2-16$}
            \alakohta{$x^2-2x+1$}
            \alakohta{$4x^2+4x+1$}
            \alakohta{$1-a^2$}
            \alakohta{$x^2-14x+49$}
            \alakohta{$y^2+2y+1$}
            \alakohta{$x^2+6xy+9y^2$}
            \alakohta{$9x^2-6x+1$}
            \alakohta{$4x^2-1$}
            \alakohta{$t^2-4t+4$}
            \alakohta{$4x^2+12x+9$}
            \alakohta{$a^2-4a+4$}
            \alakohta{$25x^2-4$}
            \alakohta{$9-c^2$}                                                                                                                                    
            \alakohta{$100x^2-20x+1$}            
            \alakohta{$4a^2+4ab+b^2$}                                                                                                                                                        
            \alakohta{$x^2+12x+36$}                                                                                                                                                        
        \end{alakohdat}
    \end{vastaus}
\end{tehtava}


\begin{tehtava}
   	Muistikaavan mukaisen lausekkeen tunnistaminen on tärkeää.
Tunnista edellisessä tehtävässä laskemasi muistikaavat ja
   	esitä lausekkeet alkuperäisessä muodossaan tulona.
    \begin{alakohdat}
            \alakohta{$1-a^2$} 
	        \alakohta{$a^2-4a+4$}
            \alakohta{$4x^2+4x+1$}
            \alakohta{$x^2-14x+49$}
            \alakohta{$100x^2-20x+1$}            
            \alakohta{$4a^2+4ab+b^2$}                                                                                                                                                        
            \alakohta{$y^2+2y+1$}
            \alakohta{$x^2+4x+4$}
            \alakohta{$4x^2-1$}
            \alakohta{$t^2-4t+4$}
            \alakohta{$25x^2-4$}
            \alakohta{$b^2-16$}
            \alakohta{$x^2-2x+1$}
            \alakohta{$9-c^2$}                                                                                                                                    
            \alakohta{$x^2+12x+36$}                                                                                                                                                        
            \alakohta{$4x^2+12x+9$}            
            \alakohta{$x^2-10x+25$}
            \alakohta{$x^2+6xy+9y^2$}
            \alakohta{$9x^2-6x+1$}
    \end{alakohdat}
    \begin{vastaus}
        \begin{alakohdat}
            \alakohta{$(1-a)(1+a)$} 
            \alakohta{$(a-2)^2$}
            \alakohta{$(2x+1)^2$}
            \alakohta{$(x-7)^2$}
            \alakohta{$(10x-1)^2$}
            \alakohta{$(2a+b)^2$}   
            \alakohta{$(y+1)^2$}
   		     \alakohta{$(x+2)^2$}
            \alakohta{$(2x+1)(2x-1)$}
             \alakohta{$(t-2)^2$}
            \alakohta{$(5x+2)(5x-2)$}
    	    \alakohta{$(b+4)(b-4)$}
 	       \alakohta{$(x-1)^2$}
            \alakohta{$(3-c)(3+c)$}
            \alakohta{$(x+6)^2$}                                                                                                                                                         
            \alakohta{$(2x+3)^2$}            
  		    \alakohta{$(x-5)^2$}  
            \alakohta{$(x-3y)^2$}
            \alakohta{$(3x-1)^2$}
         \end{alakohdat}
    \end{vastaus}
\end{tehtava}

\newpage

\begin{tehtava}
Sievennä muistikaavan avulla
    \begin{alakohdat}
            \alakohta{$(x^2-1)^2$} 
	        \alakohta{$(a^6+3b^3)^2$}
            \alakohta{$(-12-3x)(12-3x)$}
            \alakohta{$(x+\frac{1}{x})^2$}
    \end{alakohdat}
    \begin{vastaus}
        \begin{alakohdat}
            \alakohta{$x^4-2x^2+1$} 
            \alakohta{$a^{12}+6a^6b^3+9b^6$}
            \alakohta{$-(144-9x^2)=9x^2-144$}
            \alakohta{$x^2+2+\frac{1}{x^2}$}
         \end{alakohdat}
    \end{vastaus}
\end{tehtava}

\begin{tehtava} % toisen asteen yhtälö
	Yllättäviä yhteyksiä:
    \begin{alakohdat}
            \alakohta{Perustele, että $\left(2+\sqrt{3}\right)^{-1}= 2-\sqrt{3}$.} 
	        \alakohta{$\star$ Miten tämä yleistyy?}
    \end{alakohdat}

    \begin{vastaus}
    \begin{alakohdat}
            \alakohta{Tutki lukujen $2+\sqrt{3}$ ja $2-\sqrt{3}$ tuloa.} 
	        \alakohta{Yleisesti $\left(a+\sqrt{a^2-1}\right)^{-1}= a-\sqrt{a^2-1}$}
    \end{alakohdat}
    \end{vastaus}
\end{tehtava}

\begin{tehtava} %Tämä voisi olla tulon merkkisäännön kohdalla?
Osoita, että $x^2+\frac{1}{x^2}\geq 2$, kun $x \neq 0$.
    \begin{vastaus}
     Aloita tiedosta $\left(x-\frac{1}{x}\right)^2 \geq 0$ ja sievennä.
    \end{vastaus}
\end{tehtava}

\begin{tehtava} %Tämä voisi olla tulon merkkisäännön kohdalla?
Osoita, että funktio $f(x)=x^4+3x^2+1$ saa vain positiivisia arvoja.
    \begin{vastaus}
     $x^4\geq 0$ ja $x^2 \geq 0$, joten $f(x) \geq 1$.
    \end{vastaus}
\end{tehtava}

\begin{tehtava} 
$\star$ Osoita, että kun $a \geq 0$ ja $b \geq 0$, pätee \\ $\frac{a+b}{2} \geq \sqrt{ab}$. Milloin yhtäsuuruus on voimassa?
    \begin{vastaus}
     Opastus: Aloita tiedosta $\left(\sqrt{a}-\sqrt{b}\right)^2 \geq 0$ ja sievennä. Yhtäsuuruus pätee, kun $a = b$.
    \end{vastaus}
\end{tehtava}

\paragraph*{Toisen asteen yhtälö}

\begin{tehtava} % toisen asteen yhtälö
Neliön muotoisen taulun sivu on 36 cm. Taululle tehdään tasalevyinen kehys, jonka
nurkat on pyöristettu neljännesympyrän muotoisiksi. Kuinka leveä kehys on, kun sen
pinta-ala on puolet taulun pinta-alasta? (ympyrän pinta-ala on $\pi r^2$.)
    \begin{vastaus}
     $7,7$~cm
    \end{vastaus}
\end{tehtava}

\begin{tehtava} % toisen asteen yhtälö
Ratkaise yhtälö $x - 3 = \frac{1}{x}$.
    \begin{vastaus}
    $x =\frac{3 \pm \sqrt{13}}{2}$
    \end{vastaus}
\end{tehtava}

\begin{tehtava} % toisen asteen yhtälö
On olemassa viisi peräkkäistä positiivista kokonaislukua, joista kolmen
ensimmäisen neliöiden summa on yhtä suuri kuin kahden jälkimmäisen
neliöiden summa. Mitkä luvut ovat kyseessä?
    \begin{vastaus}
		$10^2+11^2+12^2 = 13^2 + 14^2$.
    	Jos negatiivisetkin luvut sallittaisiin, $(-2)^2+(-1)^2+0^2 = 1^2 + 2^2$ kävisi 			myös. Löytyykö vastaava $4 + 3$ luvun sarja? Entä pidempi?
    \end{vastaus}
\end{tehtava}

\paragraph*{Toisen asteen epäyhtälö}

\begin{tehtava} % toisen asteen epäyhtälö
Jäätelökioskin päivittäiset kiinteät kulut ovat $400$ euroa. Jokainen jäätelö maksaa
kauppiaalle $0,50$ euroa. Kun jäätelön myyntihinta on $x$ euroa, sitä myydään
$1000 - 200x$ kappaletta. 
\begin{alakohdat}
\alakohta{Millä myyntihinnoilla jäätelön myynti on kannattavaa?}
\alakohta{Millä myyntihinnalla saadaan suurin tuotto? Kuinka suuri?}
\end{alakohdat}
    \begin{vastaus}
		\begin{alakohdat}
		\alakohta{$1,00$ \euro \ $<$ myyntihinta $<$ $4,5$ \euro.}
		\alakohta{$2,75$ \euro, jolloin voitto on $612,50$ \euro. } 
		% Epäilyttävän hyvä bisnes
		\end{alakohdat}
    \end{vastaus}
\end{tehtava}


\paragraph*{Korkeamman asteen yhtälö}

\begin{tehtava} % Korkeamman asteen yhtälö
Kun kolme peräkkäistä kokonaislukua kerrotaan keskenään, ja tuloon
lisätään keskimmäinen luku, tulos on 15 kertaa keskimmäisen luvun neliö.
Mitkä luvut ovat kyseessä?
    \begin{vastaus}
	Luvut ovat $-1, 0$ ja $1$ tai $14, 15$ ja $16$.
    \end{vastaus}
\end{tehtava}

\paragraph*{Korkeamman asteen epäyhtälö}

\begin{tehtava} % Korkeamman asteen epäyhtälö
Ratkaise epäyhtälöt
		\begin{alakohdat}
		\alakohta{$x^3 + 2x^2-15x  > 0$  }
		\alakohta{$x^3-2x^2+x \leq 0$  }
		\end{alakohdat}
    \begin{vastaus}
		\begin{alakohdat}
		\alakohta{$-5<x<0$ tai $3 < x$}
		\alakohta{$x<0$ tai $x = 1$}
		\end{alakohdat}
    \end{vastaus}
\end{tehtava}



\begin{tehtava} % Korkeamman asteen epäyhtälö
Olkoon $a > 0$. Millä muuttujan $x$ arvoilla funktion
$P(x)=x^3-ax$ arvot ovat positiivisia?
    \begin{vastaus}
	$P(x)>0$ kun $x > \sqrt{a}$ tai $-\sqrt{a}<x<0$.
    \end{vastaus}
\end{tehtava}

\newpage
\section*{Kertaustehtäviä}

\subsection*{Polynomi}

\begin{tehtava} 
Laske.
		\begin{alakohdat}
		\alakohta{$x+x+x$  }
		\alakohta{$x\cdot x \cdot x$  }
		\alakohta{$-5x^2-5x^2$  }
		\alakohta{$2x\cdot 7x$  }
		\alakohta{$\frac{1}{2}x^2\cdot(-6x^3)$  }						
		\end{alakohdat}
    \begin{vastaus}
		\begin{alakohdat}
		\alakohta{$3x$}
		\alakohta{$x^3$}
		\alakohta{$-10x^2$}
		\alakohta{$14x^2$}
		\alakohta{$-3x^5$}						
		\end{alakohdat}
    \end{vastaus}
\end{tehtava}

\begin{tehtava} 
Laske.
		\begin{alakohdat}
		\alakohta{$-2x^3+4x-x^2-x+5x^3$  }
		\alakohta{$x-3-(5-x)$  }
		\alakohta{$(x^2+5x+2)-(-x^2+3x-5)$  }
		\alakohta{$3x(x-7)$  }
		\alakohta{$-2x^2(3x^5-12x^2+1)$  }						
		\end{alakohdat}
    \begin{vastaus}
		\begin{alakohdat}
		\alakohta{$3x^3-x^2+3x$}
		\alakohta{$2x-8$}
		\alakohta{$2x^2+2x+7$}
		\alakohta{$3x^2-21x$}
		\alakohta{$-6x^7+24x^4-2x^2$}						
		\end{alakohdat}
    \end{vastaus}
\end{tehtava}

\begin{tehtava} 
Kerro sulut auki.
		\begin{alakohdat}
		\alakohta{$(x-2)(x+4)$  }
		\alakohta{$(3a+b)(a-b)$  }
		\alakohta{$(a+3)^2$  }
		\alakohta{$(x-1)^2$  }
		\alakohta{$(4x+1)(4x-1)$  }						
		\end{alakohdat}
    \begin{vastaus}
		\begin{alakohdat}
		\alakohta{$x^2+2x-8$}
		\alakohta{$3a^2-2ab-b^2$}
		\alakohta{$a^2+6a+9$}
		\alakohta{$x^2-2x+1$}
		\alakohta{$16x^2-1$}						
		\end{alakohdat}
    \end{vastaus}
\end{tehtava}

\begin{tehtava} 
Jaa tekijöihin.
		\begin{alakohdat}
		\alakohta{$4x^2+x$  }
		\alakohta{$5x^3y+10xy^2$  }
		\alakohta{$y^2-9$  }
		\alakohta{$x^2-4x+4$  }
		\alakohta{$x^4-5x^3+10x-50$  }						
		\end{alakohdat}
    \begin{vastaus}
		\begin{alakohdat}
		\alakohta{$x(4x+1)$}
		\alakohta{$5xy(x^2+2y)$}
		\alakohta{$(y+3)(y-3)$}
		\alakohta{$(x-2)^2$}
		\alakohta{$(x^3+10)(x-5)$}						
		\end{alakohdat}
    \end{vastaus}
\end{tehtava}

\begin{tehtava} 
Ratkaise yhtälö \\ $(x-3)(x+2)(x-1)=0$.
    \begin{vastaus}
		$x=3$ tai $x=-2$ tai $x=1$.
    \end{vastaus}
\end{tehtava}

\begin{tehtava} 
Miksi polynomi $x^6+3x^2+5$ ei voi saada negatiivisia arvoja?
    \begin{vastaus}
		Koska $x^6\geq 0$ ja $x^2 \geq 0$. (Parilliset potenssit.)
    \end{vastaus}
\end{tehtava}

\begin{tehtava} 
		\begin{alakohdat}
		\alakohta{Osoita oikeaksi kaavat \\
$a^3-b^3=(a-b)(a^2+ab+b^2)$ ja \\
$a^3+b^3=(a+b)(a^2-ab+b^2)$}
		\alakohta{Jaa $x^6-y^6$ neljään tekijään.}						
		\end{alakohdat}
    \begin{vastaus}
		\begin{alakohdat}
		\alakohta{Opastus: Kerro sulut auki.}
		\alakohta{$(x-y)(x^2+xy+y^2)(x+y)(x^2-xy+y^2)$}						
		\end{alakohdat}
    \end{vastaus}
\end{tehtava}

\begin{tehtava} 
Kuvassa on polynomin $P(x)$ kuvaaja.
\begin{kuvaajapohja}{0.8}{-3}{3}{-2}{4}
				\kuvaaja{x*(x-1)*(x+2)}{}{black}
\end{kuvaajapohja}

		\begin{alakohdat}
		\alakohta{Mitkä ovat polynomin nollakohdat?}
		\alakohta{Millä muuttujan $x$ arvoilla $P(x)>0$?}
		\alakohta{Kuinka monta ratkaisua yhtälöllä $P(x)=1$ on?}
		\end{alakohdat}
    \begin{vastaus}
		\begin{alakohdat}
		\alakohta{$x=-2$, $x=0$ ja $x=1$}
		\alakohta{$-2<x<0$ tai $1 < x$}
		\alakohta{Vähintään 3. (Kuvaajassa näkyvän alueen 
		ulkopuolella voisi olla lisää.)}
		\end{alakohdat}
    \end{vastaus}
\end{tehtava}

\begin{tehtava} %muistikaavat
Jaa tekijöihin \\ $(3x^2-7y^2+5)^2-(x^2-9y^2-5)^2$.
    \begin{vastaus}
		$8(x-2y)(x+2y)(x^2+y^2+7)$. \\
    Opastus: Älä kerro aluksi sulkuja auki vaan käytä heti muistikaavaa.
    \end{vastaus}
\end{tehtava}

\subsection*{Ensimmäinen aste}

\begin{tehtava} 
Ratkaise yhtälöt.
		\begin{alakohdat}
		\alakohta{$3x-5(x-2)=3-(-x)$ }
		\alakohta{$x(x-6) = x^2+5$  }
		\alakohta{$ \frac{2x}{3}-\frac{x-3}{4}=7$ }
		\end{alakohdat}
    \begin{vastaus}
		\begin{alakohdat}
		\alakohta{$x = \frac{7}{3}$}
		\alakohta{$x=-\frac{5}{6}$}
		\alakohta{$15$}			
		\end{alakohdat}
    \end{vastaus}
\end{tehtava}

\begin{tehtava} 
Mitä tarkoittaa
		\begin{alakohdat}
		\alakohta{$x \in [2,7]$ }
		\alakohta{$y \in ]-3,0]$  }
		\alakohta{$z \in ]-\infty, 5[$ ?}
		\end{alakohdat}
    \begin{vastaus}
		\begin{alakohdat}
		\alakohta{$2 \leq x \leq 7$}
		\alakohta{$-3 < y \leq 0$}
		\alakohta{$z < 5$}			
		\end{alakohdat}
    \end{vastaus}
\end{tehtava}

\begin{tehtava} 
Ratkaise epäyhtälöt.
		\begin{alakohdat}
		\alakohta{$-3x < 6$ }		
		\alakohta{$5x-2 > 7x+3$ }
		\alakohta{$ -2(x+3)  \leq x-(5-x)$  }
		\end{alakohdat}
    \begin{vastaus}
		\begin{alakohdat}
		\alakohta{$ x > -2$}	
		\alakohta{$x < -\frac{5}{2}=-2\frac{1}{2}$}
		\alakohta{$x \geq -\frac{1}{4}$}
		\end{alakohdat}
    \end{vastaus}
\end{tehtava}

\begin{tehtava} 
Polkupyörävuokraamo A laskuttaa pyörän vuokrasta $5$~\euro \ ja $2,5$~\euro \
jokaisesta täydestä tunnista. Vuokraamo B laskuttaa $11,50$~\euro \ ja $1,5$~\euro \ jokaisesta
täydestä tunnista. Kuinka monen tunnin vuokrassa vuokraamo B on edullisempi?
    \begin{vastaus}
	7 h tai sitä pidemmissä vuokrissa.
    \end{vastaus}
\end{tehtava}

\begin{tehtava} 
Ratkaise epäyhtälö \\
$ax \geq a-2x$ \\ parametrin $a$ kaikilla arvoilla.
    \begin{vastaus}
        $x \geq \frac{a}{a+2}$, kun $a > -2$ \\
        $x \leq \frac{a}{a+2}$, kun $a < -2$ \\
    $x \in \mathbb{R}$, kun $a = -2$ \\
	\end{vastaus}
\end{tehtava}

\subsection*{Toinen aste}

\begin{tehtava} 
Ratkaise yhtälöt.
		\begin{alakohdat}
		\alakohta{$x^2-13=0$ }
		\alakohta{$5x^+2x=0$  }
		\alakohta{$2x^2+5x-3=0$}
		\end{alakohdat}
    \begin{vastaus}
		\begin{alakohdat}
		\alakohta{$x= \pm \sqrt{13}$}
		\alakohta{$x=0$ tai $x=-\frac{2}{5}$}
		\alakohta{$x=-3$ tai $x= \frac{1}{2}$}			
		\end{alakohdat}
    \end{vastaus}
\end{tehtava}

\begin{tehtava} 
Ratkaise epäyhtälöt.
		\begin{alakohdat}
		\alakohta{$x^2-x-6<0$ }
		\alakohta{$x^2 \geq 5x$  }
		\end{alakohdat}
    \begin{vastaus}
		\begin{alakohdat}
		\alakohta{$ -2 < x < 3 $}
		\alakohta{$x \leq 0$ tai $x \geq 5$}
	\end{alakohdat}
    \end{vastaus}
\end{tehtava}

\begin{tehtava} 
Maijan ja Veeran ikien summa on 30. Ikien tulo on yli 186. Minkä ikäisiä tytöt voivat olla?
    \begin{vastaus}
	Vähintään 9, korkeintaan 21. Ratkeaa epäyhtälöstä $x(30-x)>186$.
	\end{vastaus}
\end{tehtava}

\begin{tehtava} 
Millä vakion $k$ arvolla yhtälöllä \\ $kx^2+kx=x-3$ on tasan yksi ratkaisu?
    \begin{vastaus}
		$k = 7 \pm 4 \sqrt{3}$. Diskriminantti on $D = (k-1)^2-4\cdot k \cdot 3$.
    \end{vastaus}
\end{tehtava}

\begin{tehtava} 
Polynomilla $P(x)=x^2-3x+c$ on tekijä $x+5$. Mikä on $c$?
    \begin{vastaus}
		$c=-40$
    \end{vastaus}
\end{tehtava}

\begin{tehtava} 
Suorakulmion $A$ sivut ovat $9$ ja $x^2+1$, suorakulmion $B$ sivut $5x+5$
ja $5-x$. Millä luvun $x$ arvoilla suorakulmion $A$ ala on suurempi?
    \begin{vastaus}
	$-1 < x < -\frac{4}{7}$ tai $2 < x < 5$. Huomioi, että suorakulmion $B$
    sivut ovat positiiviset vain, kun $-1<x<5$.
    \end{vastaus}
\end{tehtava}

\subsection*{Korkeampi aste}

\begin{tehtava} 
Ratkaise yhtälö $2x^7=5x^2$.
    \begin{vastaus}
		$x=0$ tai $x=\sqrt[5]{\frac{5}{2}}$
    \end{vastaus}
\end{tehtava}

\begin{tehtava} 
Anna esimerkki (jos mahdollista)
		\begin{alakohdat}
		\alakohta{4. asteen yhtälöstä, jolla ei ole ratkaisua }
		\alakohta{5. asteen yhtälöstä, jolla on tasan kaksi ratkaisua}
		\alakohta{3. asteen yhtälöstä, jolla on tasan neljä ratkaisua}
		\end{alakohdat}
    \begin{vastaus}
		\begin{alakohdat}
		\alakohta{esimerkiksi $x^4=-1$ }
		\alakohta{esimerkiksi $x^4(x+1)=0$ }
		\alakohta{mahdotonta}		
		\end{alakohdat}
    \end{vastaus}
\end{tehtava}

\begin{tehtava} 
Ratkaise yhtälö $x^4=x^2+6$.
    \begin{vastaus}
		$x=\pm \sqrt{3}$
    \end{vastaus}
\end{tehtava}

\begin{tehtava} 
Ratkaise yhtälö
$x^7=5x^5-x^6$.
     \begin{vastaus}
		$x=0$ tai $x\frac{-1 \pm \sqrt{21}}{2}$ 
    \end{vastaus}
\end{tehtava}

\begin{tehtava} % Korkeamman asteen epäyhtälö
Mitkä luvut ovat kuutiotaan suurempia?
    \begin{vastaus}
	Luvut, jotka ovat pienempiä kuin $-1$ ja luvut välillä $]0,1[$.
    \end{vastaus}
\end{tehtava}

\begin{tehtava} 
Ratkaise epäyhtälöt
		\begin{alakohdat}
		\alakohta{$x^4 < 5x $}
		\alakohta{$2x^3 \leq 3x-5x^2$}
		\end{alakohdat}
     \begin{vastaus}
		\begin{alakohdat}
		\alakohta{$0<x<\sqrt[3]{5}$}
		\alakohta{$x<-3$ tai $0<x\frac{1}{2}$}
		\end{alakohdat}
    \end{vastaus}
\end{tehtava}

\begin{tehtava} 
Kuution tilavuus (kuutiometreinä) on sama kuin sen särmien pituuksien summa (metreinä). Kuinka pitkä on kuution särmä?
    \begin{vastaus}
		$\sqrt{12}$~m $\approx 3,46$~m
    \end{vastaus}
\end{tehtava}

\subsection*{Sekalaisia}

\begin{tehtava} 
Sievennä:
		\begin{alakohdat}
			\alakohta{$(a^2-1)^2+(a^2+1)^2-2(a^4+1)$}
			\alakohta{$(x+y)^2-4xy$}
		\end{alakohdat}
	\begin{vastaus}
		\begin{alakohdat}
			\alakohta{$0$}
			\alakohta{$(x-y)^2$}
		\end{alakohdat}
    \end{vastaus}
\end{tehtava}

\begin{tehtava} 
Ratkaise yhtälö
$2x^5-(x+6)^5=0$.
    \begin{vastaus}
	$x=\frac{6}{\sqrt[5]{2}-1}$
    \end{vastaus}
\end{tehtava}

\begin{tehtava} 
Suorakulmion piiri on 34 ja lävistäjä 13. Ratkaise suorakulmion sivut.
    \begin{vastaus}
	Sivut ovat $5$ ja $12$.
    \end{vastaus}
\end{tehtava}

\begin{tehtava} % Korkeamman asteen yhtälö
Etsi yhtälön $x^5-5x^4+6x^3-1=0$ kaikkien kolmen ratkaisun likiarvot
laskimella tai tietokoneen avulla. Anna vastaukset
kahden desimaalin tarkkuudella.
    \begin{vastaus}
	$x \approx 0,69$, $x \approx 1,86$ tai $x \approx 3,03$.
    \end{vastaus}
\end{tehtava}


\end{tehtavasivu}

