\section{Polynomifunktion kuvaaja}
Polynomifunktiota voi
havainnollistaa koordinaatistoon piirretyn kuvaajan avulla:

%\begin{kuvaajapohja}{1.5}{-2}{2}{-3}{3}
%\kuvaaja{-x-1}{$P(x) = -x-1$}{red}
%\kuvaaja{x**2+x}{$Q(x) = x^2+x$}{blue}
%\kuvaaja{x**3-3*x-1}{$R(x) = x^3-3x-1$}{green}
%\end{kuvaajapohja}

% \begin{kuvaajapohja}{1.5}{-2}{2}{-3}{3}
% %\kuvaaja{-x-1}{$P(x) = -x-1$}{red}
% \kuvaaja{x**2+x}{$Q(x) = x^2+x$}{black}
% \kuvaaja{x**3-3*x-1}{$R(x) = x^3-3x-1$}{black}
% \end{kuvaajapohja}
% %

\begin{kuva}
kuvaaja.pohja(-2, 2, -3, 3, korkeus = 9, nimiX = '$x$')
kuvaaja.piirra("x**2+x", nimi = "$Q(x) = x^2+x$", kohta = 1, suunta = -45)
kuvaaja.piirra("x**3-3*x-1", nimi = "$R(x) = x^3-3x-1$", kohta = 1.8)
\end{kuva}

\subsection*{Kuvaajan piirtäminen}

\qrlinkki{http://opetus.tv/maa/maa2/suoran-piirtaminen/}{Opetus.tv: \emph{suoran piirtäminen} (5:47)}

Funktioiden kuvaajia voi piirtää tietokoneella, graafisella laskimella tai käsin. Kussakin tapauksessa periaate on sama: valitaan joitakin muuttujan arvoja, lasketaan funktion arvot ja merkitään pisteet $(x,y)$-koordinaatistoon. Tietokoneet ja laskimet laskevat funktion arvoja niin tiheään, että näyttää syntyvän yhtenäinen kuvaaja. Käsin piirrettäessä tyydytään muutamaan pisteeseen ja hahmotellaan kuvaaja niiden avulla.

% Esimerkin kuvat.
\begin{luoKuva}{esimkuva1}
kuvaaja.pohja(-4, 4, -3, 6, leveys = 3.5, nimiX = "$x$")
piste((-3, 5.5))
piste((-2, 2))
piste((-1, -0.5))
piste((0, -2))
piste((1, -2.5))
piste((2, -2))
piste((3, -0.5))
\end{luoKuva}
\begin{luoKuva}{esimkuva2}
kuvaaja.pohja(-4, 4, -3, 6, leveys = 3.5, nimiX = "$x$")
piste((-3, 5.5))
piste((-2, 2))
piste((-1, -0.5))
piste((0, -2))
piste((1, -2.5))
piste((2, -2))
piste((3, -0.5))
kuvaaja.piirra("0.5*x**2-x-2")
\end{luoKuva}

\begin{esimerkki}
Hahmotellaan polynomifunktion $f(x) = \dfrac{1}{2}x^2 - x - 2$ kuvaaja.
Lasketaan ensin joitakin funktion $f(x) = \dfrac{1}{2}x^2 - x - 2$ arvoja ja piirretään niitä vastaavat pisteet
koordinaatistoon. Lopuksi hahmotellaan kuvaaja, joka kulkee pisteiden kautta.

\begin{tabular}{c c c}
	\begin{tabular}{|c|r @{,} l|}
	\hline $x$ & \multicolumn{2}{c|}{$f(x)$} \\
	\hline
	-3 & 5&5 \\
	-2 & 2&0 \\
	-1 & -0&5 \\
	0 & -2&0 \\
	1 & -2&5 \\
	2 & -2&0 \\
	3 & -0&5 \\
	\hline
	\end{tabular}
	&
	\vcent{\naytaKuva{esimkuva1}}
	&
	\vcent{\naytaKuva{esimkuva2}}
\end{tabular}
% 
% \begin{tabular}{c c c}
% 	\begin{tabular}{|c|r @{,} l|}
% 	\hline $x$ & \multicolumn{2}{c|}{$f(x)$} \\
% 	\hline
% 	-3 & 5&5 \\
% 	-2 & 2&0 \\
% 	-1 & -0&5 \\
% 	0 & -2&0 \\
% 	1 & -2&5 \\
% 	2 & -2&0 \\
% 	3 & -0&5 \\
% 	\hline
% 	\end{tabular}
% 	&
% 	\vcent{\begin{kuvaajapohja}{0.6}{-4}{4}{-3}{6}
% 	\kuvaajapiste{-3}{5.5}
% 	\kuvaajapiste{-2}{2}
% 	\kuvaajapiste{-1}{-0.5}
% 	\kuvaajapiste{0}{-2}
% 	\kuvaajapiste{1}{-2.5}
% 	\kuvaajapiste{2}{-2}
% 	\kuvaajapiste{3}{-0.5}
% 	\end{kuvaajapohja}}
% 	&
% 	\vcent{\begin{kuvaajapohja}{0.6}{-4}{4}{-3}{6}
% 	\kuvaajapiste{-3}{5.5}
% 	\kuvaajapiste{-2}{2}
% 	\kuvaajapiste{-1}{-0.5}
% 	\kuvaajapiste{0}{-2}
% 	\kuvaajapiste{1}{-2.5}
% 	\kuvaajapiste{2}{-2}
% 	\kuvaajapiste{3}{-0.5}
% 	\kuvaaja{0.5*x**2-x-2}{}{black}
% 	\end{kuvaajapohja}}
% \end{tabular}

\end{esimerkki}

\subsection*{Kuvaajan tulkintaa}

%Ensimmäisen asteen polynomin kuvaaja on luonnollisesti aina suora. 
%miten niin luonnollisesti?

Kuvaajan avulla voidaan tehdä johtopäätöksiä funktion ominaisuuksista.
Esimerkiksi funktion arvoja voidaan lukea kuvaajasta.

\begin{esimerkki}
Seuraavassa on esitetty polynomifunktion $P(x)=-3x^2+2x+4$ kuvaaja.

\begin{kuvaajapohja}[\kuvaajaAsetusEiRuudukkoa]{0.7}{-3}{3}{-5}{5}
\kuvaajapiste{2}{-4}
\kuvaajakohtaarvo{2}{-4}{}{}
\kuvaaja{-3*x**2+2*x+4}{$P(x)$}{black}
\end{kuvaajapohja}

Kuvaajasta voi lukea funktion arvoja tai ainakin niiden likiarvoja. Kuvaajan perusteella näyttää siltä, että $P(2)=-4$. Näin todellakin on, sillä \\ $P(1)=-3\cdot 1^2+2\cdot 1+4=1$.
\end{esimerkki}


\begin{esimerkki}
Kuvaajasta ei välttämättä näe tarkkoja arvoja. Seuraavassa on esitetty erään polynomifunktion $P(x)$ kuvaaja. Kuvaajan perusteella näyttäisi siltä, että $P(1)=1$, mutta tarkkaa arvoa kuvaajasta ei voi päätellä.

\begin{kuva}
kuvaaja.pohja(-1.9, 1.7, -1.4, 2, leveys = 4, nimiX = "$x$", ruudukko = False)
kuvaaja.piirra("19./20*x**2+19./20*x-1", nimi = "$P(x)$", kohta = 1)
\end{kuva}
% \begin{kuvaajapohja}[\kuvaajaAsetusEiRuudukkoa]{1}{-2}{2}{-2}{2}
% \kuvaaja{19./20*x**2+19./20*x-1}{$P(x)$}{black}
% \end{kuvaajapohja}
 
Itse asiassa edellinen kuvaaja kuuluu funktiolle $P(x)=\dfrac{19}{20} x^2+\dfrac{19}{20} x-1$. Nyt tiedetään, että funktion $P(x)$ arvo kohdassa $x=1$ on
$$\dfrac{19}{20}\cdot 1^2+\dfrac{19}{20} \cdot 1-1=\dfrac{9}{10}$$
eikä $1$, kuten kuvaajan perusteella voisi luulla. Kuvaajasta ei siis voi lukea tarkkoja tietoja funktiosta.
\end{esimerkki}

\newpage

\subsubsection*{Nollakohta}

Funktion \termi{nollakohta}{nollakohta} on sellainen muuttujan arvo, jolla funktio saa arvon nolla. Esimerkiksi funktiolla $Q(x)=x^2-1$
on nollakohdat $-1$ ja $1$, sillä $Q(-1)=0$ ja $Q(1)=0$.

Funktion kuvaaja antaa tietoa nollakohdista. Niiden kohdalla kuvaaja leikkaa $x$-akselin.

\begin{esimerkki}
Funktion $P(x) = \dfrac{1}{3}x^2-4x+\dfrac{5}{2}$ kuvaajasta nähdään, että funktiolla on ainakin kaksi nollakohtaa. Toinen niistä on lähellä lukua $1$ ja toinen lukua $11$.

\begin{kuva}
kuvaaja.pohja(-5, 15, -10, 5, korkeus = 6, nimiX = "$x$")
piste((0.66146, 0))
piste((11.3385, 0))
kuvaaja.piirra("1./3*x**2-4*x+5./2", nimi = "$P(x) = \dfrac{1}{3}x^2-4x+\dfrac{5}{2}$", kohta = 11, suunta = -45)
\end{kuva}
% \begin{kuvaajapohja}{0.3}{-5}{15}{-10}{5}
% \kuvaajapiste{0.66146}{0}
% \kuvaajapiste{11.3385}{0}
% \kuvaaja{1./3*x**2-4*x+5./2}{$P(x) = \dfrac{1}{3}x^2-4x+\dfrac{5}{2}$}{black}
% \end{kuvaajapohja}
\end{esimerkki}

%Nollakohta tarkoittaa sitä annetun polynomin muuttujan arvoa, jolla koko
%polynomi saa arvon nolla. Kuvaajasta sen voi helposti lukea niinä kohtina,
%joissa kuvaaja leikkaa muuttujan koordinaattiakselin. Funktion $P(x)$ ja
%$xy$-koordinaatiston tapauksessa funktion nollakohdat ovat täsmälleen ne
%$x$-koordinaatit, joilla funktion kuvaaja leikkaa $x$-akselin.

%\subsection{Taylorin sarja}
%Eräs mielenkiintoinen ja hyvin tunnettu potenssisarja on Taylorin sarja.
%Se on päättymätön potenssisarja, jolla voidaan approksimoida muiden funktioiden
%arvoja.
%
%Yleisesti Taylorin sarjalla saadaan (rajatta derivoituvan) funktion $f$ arvo
%pisteessä $x_0$:
%
%\begin{align*}
%	f(x_0) = \sum\limits_{n=0}^\infty a_n(x-x_0)^n
%\end{align*}
%
%missä
%
%\begin{align*}
%a_n = \frac{f^n(x_0)}{n!}
%\end{align*}
%
%Koska sarja on äärettömän pitkä, sarjan arvoja edelleen arvioidaan Taylorin
%polynomilla, joka on muotoa
%
%\begin{align*}
%	P_k(x) = \sum\limits_{n=0}^k a_k(x-x_0)^k
%\end{align*}
%
%Polynomin avulla voidaan laskea esimerkiksi likiarvo funktiolle
%$(1-x)^{-1} = \frac{1}{1-x}$ pisteen a ympäristössä, kun $a \neg 1$:
%
%\begin{align*}
%	\frac{1}{1-x} \approx \frac{1}{1-a} + \frac{x-a}{(1-a)^2} +
%\frac{(x-a)^2}{(1-a)^3} + \frac{(x-a)^3}{(1-a)^4} ...
%\end{align*}
%
%\missingfigure{Funktion $(x-1)^-1$ kuvaaja}
%\missingfigure{Funktion $\frac{1}{1-a} + \frac{x-a}{(1-a)^2} +
%\frac{(x-a)^2}{(1-a)^3} +$ kuvaaja}

\begin{tehtavasivu}

\paragraph*{Opi perusteet}


\begin{tehtava}
\begin{kuva}
	kuvaaja.pohja(-2,4.5,-2,8,5,nimiX="$x$",nimiY="$P(x)$")
	kuvaaja.piirra("x**2-4*x+3")
\end{kuva}
Kuvassa on polynomifunktion $P$ kuvaajaa. Määritä on kuvaajan perusteella
\begin{alakohdat}
\alakohta{$P(-1)$}
\alakohta{$P(1)$}
\alakohta{$P(2)$}
\alakohta{polynomin $P$ nollakohdat}
\alakohta{millä $x$:n arvoilla $P(x)=3$.}
\end{alakohdat}
\begin{vastaus}
\begin{alakohdat}
\alakohta{$8$}
\alakohta{$0$}
\alakohta{$-1$}
\alakohta{Nollakohdat ovat $x=1$ ja $x=3$.}
\alakohta{Kun $x= 0$ tai $x=4$.}
\end{alakohdat}
\end{vastaus}
\end{tehtava}

\begin{tehtava}
    Hahmottele polynomien kuvaajat koordinaatistoon käsin. Laske testipisteitä tarpeen mukaan. Tarkista tietokoneella tai graafisella laskimella.
    \begin{alakohdat}
        \alakohta{$P(x) = x-2$}
        \alakohta{$Q(x) = 4-x^2$}
        \alakohta{$R(x) = \frac{1}{4}x^3$}
    \end{alakohdat}   
    \begin{vastaus}
    	\begin{alakohdat}
    	\alakohta{ \begin{kuvaajapohja}{0.4}{-4}{4}{-4}{4}
				\kuvaaja{2*x-2}{}{red}
			  \end{kuvaajapohja}}
    	\alakohta{ \begin{kuvaajapohja}{0.4}{-4}{4}{-3}{5}
				\kuvaaja{4-x**2}{}{red}
			  \end{kuvaajapohja}}
		\alakohta{ \begin{kuvaajapohja}{0.4}{-4}{4}{-4}{4}
				\kuvaaja{0.25*x**3}{}{red}
			  \end{kuvaajapohja}}
		\end{alakohdat}
    \end{vastaus}
\end{tehtava}

\paragraph*{Hallitse kokonaisuus}

\begin{tehtava}
Piirrä polynomifunktion kuvaaja ja päättele sen avulla polynomin nollakohdat.
Käytä graafista laskinta, tietokonetta tai piirrä käsin.
\begin{alakohdat}
\alakohta{$x^3+x^2+x+1$}
\alakohta{$x^2-6x+2$}
\end{alakohdat}
\begin{vastaus}
\begin{alakohdat}
\alakohta{$x=-1$}
\alakohta{$x \approx 0,35$ ja $x \approx 5,65$}
\end{alakohdat}
\end{vastaus}
\end{tehtava}

\begin{tehtava} $\star$
	Monia funktioita voidaan esittää likimääräisesti polynomeina (ns.
Taylorin polynomi). Esimerkiksi (kun $-1<x<1$)

	\begin{tabular}{lcll}
	$\frac{1}{1+x^2}$ &$\approx$ & $1-x^2+x^4-x^6+x^8-x^{10}$ \\
	$\sqrt{1+x}$ & $\approx $ & $ 1+\frac{x}{2}
	-\frac{x^2}{8}+\frac{x^3}{16}-\frac{5x^4}{128}$
	\end{tabular}

	Piirrä alkuperäinen funktio ja polynomi samaan kuvaajaan tietokoneella
tai graafisella laskimella. Kokeile, kuinka polynomin viimeisten termien pois
jättäminen vaikuttaa tarkkuuteen. Mitä havaitset? (Termejä voi laskea lisääkin,
mutta siihen ei puututa tässä.)

	\begin{vastaus}
		Mitä enemmän termejä, sitä parempi vastaavuus.
	\end{vastaus}
\end{tehtava}

\paragraph*{Lisää tehtäviä}

\begin{tehtava}
    Piirrä polynomien kuvaajat käsin.
    \begin{alakohdat}
        \alakohta{$x+4$}
        \alakohta{$2x-3$}
        \alakohta{$5$}
        \alakohta{$x^2+x-2$}
%        \alakohta{$x^3-x+3$}
    \end{alakohdat}   
    \begin{vastaus}
    	\begin{alakohdat}
    	\alakohta{ \begin{kuvaajapohja}{0.4}{-4}{4}{-1}{7}
				\kuvaaja{x+4}{}{red}
			  \end{kuvaajapohja}}
    	\alakohta{ \begin{kuvaajapohja}{0.4}{-4}{4}{-5}{3}
				\kuvaaja{2*x-3}{}{red}
			  \end{kuvaajapohja}}
    	\alakohta{ \begin{kuvaajapohja}{0.4}{-4}{4}{-1}{7}
				\kuvaaja{5}{}{red}
			  \end{kuvaajapohja}}
		\alakohta{ \begin{kuvaajapohja}{0.4}{-4}{4}{-3}{5}
				\kuvaaja{x**2+x-2}{}{red}
			  \end{kuvaajapohja}}
%		\alakohta{ \begin{kuvaajapohja}{0.4}{-4}{4}{-3}{5}
%				\kuvaaja{x**3-x+2}{}{red}
%			  \end{kuvaajapohja}}
		\end{alakohdat}
    \end{vastaus}
\end{tehtava}


\begin{tehtava}
    Piirrä polynomien kuvaajat tietokoneella tai laskimella.
    \begin{alakohdat}
        \alakohta{$x^2-6x+3$}
        \alakohta{$\frac{1}{5}x^3-x^2+2$}
        \alakohta{$-0,5x^4+0,25x^3+2,5x^2-1$}

    \end{alakohdat}
    \begin{vastaus}
    	\begin{alakohdat}
		\alakohta{ \begin{kuvaajapohja}{0.4}{-1}{7}{-7}{5}
				\kuvaaja{x**2-6*x+3}{}{red}
			  \end{kuvaajapohja}}
    	\alakohta{ \begin{kuvaajapohja}{0.4}{-2.5}{5.5}{-4}{4}
				\kuvaaja{0.2*x**3-x**2+2}{}{red}
			  \end{kuvaajapohja}}
		\alakohta{ \begin{kuvaajapohja}{0.4}{-3}{3}{-4}{4}
				\kuvaaja{-0.5*x**4+0.25*x**3+2.5*x**2-1}{}{red}
			  \end{kuvaajapohja}}
		\end{alakohdat}
    \end{vastaus}
\end{tehtava}

%\begin{tehtava}
%    Piirrä polynomien kuvaajat.
%    \begin{alakohdat}
%        \alakohta{$x+4$}
%        \alakohta{$2x-9$}
%        \alakohta{$5x+2$}
%        \alakohta{$6x+1$}
%    \end{alakohdat}
%    \begin{vastaus}
%    	\begin{alakohdat}
%        \alakohta{ \begin{kuvaajapohja}{0.4}{-4}{4}{-1}{7}
%				\kuvaaja{x+4}{}{red}
%			  \end{kuvaajapohja}}
%    	\alakohta{ \begin{kuvaajapohja}{0.4}{-2}{6}{-6}{2}
%				\kuvaaja{2*x-9}{}{red}
%			  \end{kuvaajapohja}}
%		\alakohta{ \begin{kuvaajapohja}{0.4}{-4}{4}{-2}{6}
%				\kuvaaja{5*x+2}{}{red}
%			  \end{kuvaajapohja}}
%		\alakohta{ \begin{kuvaajapohja}{0.4}{-4}{4}{-2}{6}
%				\kuvaaja{6*x+1}{}{red}
%			  \end{kuvaajapohja}}
%		\end{alakohdat}
%    \end{vastaus}
%\end{tehtava}
%
%\begin{tehtava}
%    Piirrä polynomien kuvaajat.
%    \begin{alakohdat}
%        \alakohta{$x^2-1$}
%        \alakohta{$2x^2$}
%        \alakohta{$4x^2+4$}
%        \alakohta{$x^2-6x+3$}
%    \end{alakohdat}
%    \begin{vastaus}
%    	\begin{alakohdat}
%        \alakohta{ \begin{kuvaajapohja}{0.4}{-4}{4}{-2}{6}
%				\kuvaaja{x+4}{}{red}
%			  \end{kuvaajapohja}}
%    	\alakohta{ \begin{kuvaajapohja}{0.4}{-4}{4}{-1}{7}
%				\kuvaaja{2*x-9}{}{red}
%			  \end{kuvaajapohja}}
%		\alakohta{ \begin{kuvaajapohja}{0.4}{-4}{4}{-1}{11}
%				\kuvaaja{5*x+2}{}{red}
%			  \end{kuvaajapohja}}
%		\alakohta{ \begin{kuvaajapohja}{0.4}{-1}{7}{-7}{5}
%				\kuvaaja{6*x+1}{}{red}
%			  \end{kuvaajapohja}}
%		\end{alakohdat}
%    \end{vastaus}
%\end{tehtava}

\end{tehtavasivu}


% tämä alla oleva omaksi filukseen! ja näihin ratkaisut! ... mutta miten?

\section{Testaa tietosi!}

\subsection*{Osaatko selittää?}

\begin{enumerate}

\item Miksi polynomin vakiotermin (tai vakiopolynomin) asteluku on nolla?
\item Miksi puhutaan vain termien summasta, vaikka polynomeissa voi esiintyä miinusmerkkejä?
\item Miten usean muuttujan polynomin asteluku lasketaan?
\item Mitä tarkoitetaan sillä, että luku on epänegatiivinen?
\item Mitä tarkoitetaan tulon nollasäännöllä?
\item Kun polynomi jaetaan tekijöihin, mitä voit varmasti sanoa tekijäpolynomien asteista verrattuna alkuperäiseen polynomiin?
\item Miten selvität polynomin $(x^3-2x)^{15}$ asteen purkamatta sulkuja?

%lisää MIKSI-tehtäviä!

\end{enumerate}

\subsection*{Kertauskysymyksiä}

Valitse yksi oikea vaihtoehto

\begin{enumerate}

\item Mitkä ovat polynomin $x^4-x^2+\sqrt{2}x-1337$ termien kertoimet neljännestä asteesta alaspäin luetellen? \\
a) $1, -1, \sqrt{2}$ ja $-1337$ \\
c) $1, 0, -1, \sqrt{2}$ ja $-1337$ \\
b) $1, 1, \sqrt{2}$ ja $1337$ \\


\item Monettako astetta on 2 muuttujan polynomi $x^5y-x^4y^3-715517y$?
\item Polynomifunktion $P(x)=-x^3-x^2-x-1$ arvo kohdassa $x=-1$?
\item Mikä polynomi saadaan, kun avataan sulut lausekkeesta $(2x-1)^2$?
\item Tulosta $ab$ tiedetään, että $a = 0$. Mikä seuraavista väitteistä pitää välttämätät paikkansa?
a) $ab = 0$
b) $b \geq a$
c) $b = 0$
d) $|b| \geq a$

\item Mitkä on funktion $f(x) = (x+3)(x-50)$ nollakohdat
\item Olkoon $f(x) = x^2 + 1$, ja $g(x) = x^2 - x$. Mitkä väitteistä pitää paikkaansa.
a) $f(x) + g(x) = 2x^2 - x + 1$
b) $f(x) \cdot g(x) = x^4-x^3+x^2-x$
c) $f(0) = g(0)$
d) $f(-1) = g(-1)$
\end{enumerate}

