Tässä on joitakin tehtäviä, jotka on arvioitu hauskoiksi ja hyödyllisiksi kaikkein osaavimmille opiskelijoille. Niiden parissa vietetty aika ei mene hukkaan, vaikkei tehtävä ratkeaisikaan.

\begin{tehtava}
    \begin{enumerate}[a)]
        \item Osoita, että jos polynomi $P$ jaetaan polynomilla $x-1$, niin jakojäännös on polynomin $P$ kertoimien summa.
        \item Päättele tästä, että kokonaisluku $n$ on jaollinen yhdeksällä vain jos sen kymmenjärjestelmäesityksen numeroiden summa on jaollinen yhdeksällä.
    \end{enumerate}
\end{tehtava}

\begin{tehtava} %Erittäin tärkeä ja hyödyllinen, mutta liekö vielä paikallaan... esitystä voisi ehkä muuttaa
Toisen asteen polynomi $P_1 = (ax-b)^2$ ($a \neq 0 $) on binomin neliönä aina epänegatiivinen. Tulon nollasäännön avulla nähdään, että sillä on tasan yksi nollakohta $x = \frac{b}{a}$. Tällöin polynomin diskriminantin arvo on 0, mikä voidaan nähdä myös suoraan laskemalla.

Polynomi $P_2 = (a_1x-b_1)^2+(a_2x-b_2)^2$ ($a_1,a_2 \neq $) on niin ikään aina epänegatiivinen, mutta sillä ei välttämättä ole nollakohtaa; se on epänegatiivisten termien summa, joka voi olla nolla vain, jos kaikkien termien neliöt ovat nollia. Koska jokainen binomin neliö voi saavuttaa nollan vain yhdessä pisteessä, ne voivat kaikki saavuttaa nollan korkeintaan yhdessä pisteessä. Mutta tällöin polynomin diskriminantin on oltava korkeintaan nolla. Muodosta polynomin $P_2$ diskriminantti, mikä epäyhtälö seuraa?

Saman päättelyn voi yleistää myös $n$:n binomin neliöiden summasta muodostetulle polynomille $P_n = (a_1x-b_1)^2+(a_2x-b_2)^2+\ldots+(a_nx-b_n)^2$. Vastaavalla päättelyllä polynomilla on korkeintaan yksi nollakohta, joten sen diskriminantti on epäpositiivinen. Laske $P_n$:n diskriminantti. Olet todistanut kuuluisan Cauchyn--Schwarzin epäyhtälön:
\[
(a_1^2+a_2^2+\ldots+a_n^2)(b_1^2+b_2^2+\ldots+b_n^2) \geq (a_1b_1+a_2b_2+\ldots+a_nb_n)^2
\]

\end{tehtava}