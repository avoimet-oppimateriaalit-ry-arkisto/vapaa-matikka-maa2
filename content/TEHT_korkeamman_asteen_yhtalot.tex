\begin{tehtavasivu}

\subsubsection*{Opi perusteet}

\begin{tehtava}
    Ratkaise yhtälöt.
    \begin{alakohdat}
        \alakohta{$x(x+2)(x+17)=0$}
        \alakohta{$x^9  = -512$}
        \alakohta{$x^6 - 7x^7 = 0$}
    \end{alakohdat}
    \begin{vastaus}
        \begin{alakohdat}
            \alakohta{$x = 0$ tai $x=-2$ tai $x=17$}
            \alakohta{$x = - 2$}
            \alakohta{$x = 0$ tai $x=\frac{1}{7}$}
        \end{alakohdat}
    \end{vastaus}
\end{tehtava}

\begin{tehtava}
    Ratkaise yhtälöt.
    \begin{alakohdat}
        \alakohta{$x^3-5x^2+6x=0$}
        \alakohta{$x^4 - 16 = 0$}
        \alakohta{$x^6 - x^4 = 0$}
    \end{alakohdat}
    \begin{vastaus}
        \begin{alakohdat}
            \alakohta{$x = 0$ tai $x=2$ tai $x=3$}
            \alakohta{$x = \pm 2$}
            \alakohta{$x = 0$ tai $x=\pm 1$}
        \end{alakohdat}
    \end{vastaus}
\end{tehtava}

\begin{tehtava}
    Ratkaise yhtälöt.
    \begin{alakohdat}
        \alakohta{$x^4 - 2x^2 - 24 = 0$}
        \alakohta{$x^4 - 4x^2 - 5 = 0$}
        \alakohta{$x^4 - 8x^2 + 15 = 0$}
    \end{alakohdat}
    \begin{vastaus}
        \begin{alakohdat}
            \alakohta{$x = \pm\sqrt{6}$}
            \alakohta{$x = \pm\sqrt{5}$}
            \alakohta{$x = \pm\sqrt{3}$ tai $\pm\sqrt{5}$}
        \end{alakohdat}
    \end{vastaus}
\end{tehtava}

\subsubsection*{Hallitse kokonaisuus}

\begin{tehtava}
Ratkaise yhtälöt.
\begin{alakohdat}
\alakohta{$x^5-3x^4+2x^3=0$}
\alakohta{$x^4+5x^3-x^2-5x=0$}
\alakohta{$x^3-4x^2-4x+16=0$}
\end{alakohdat}
\begin{vastaus}
\begin{alakohdat}
\alakohta{$x=0$ tai $x=1$ tai $x=2$}
\alakohta{$x=0$ tai $x=-5$ tai $x= \pm 1$}
\alakohta{$x=4$ tai $x= \pm 2$}
\end{alakohdat}
\end{vastaus}
\end{tehtava}

\begin{tehtava}
    Ratkaise yhtälöt.
    \begin{alakohdat}
        \alakohta{$x^4 - 16 = 0$}
        \alakohta{$2x^4 = 8x^2$}
        \alakohta{$x^6 - 2x^3 = 3$}
        \alakohta{$x^{100} - 2x^{50} + 1 = 0$}
    \end{alakohdat}
    \begin{vastaus}
        \begin{alakohdat}
            \alakohta{$x = \pm2$}
            \alakohta{$x = 0$ tai $x=\pm2$}
            \alakohta{$x = \sqrt[3]{3}$ tai $x= -1$}
            \alakohta{$x = \pm1$}
        \end{alakohdat}
    \end{vastaus}
\end{tehtava}

\begin{tehtava}
Kun kolme peräkkäistä kokonaislukua kerrotaan keskenään, ja tuloon lisätään keskimmäinen luku, tulos on $15$ kertaa keskimmäisen luvun neliö. Mitkä luvut ovat kyseessä?
    \begin{vastaus}
	Luvut ovat $-1, 0$ ja $1$ tai $14, 15$ ja $16$.
    \end{vastaus}
\end{tehtava}

\begin{tehtava}
	Ratkaise yhtälö $x^{627} - 6x^{514} + 5x^{401} = 0$.
	\begin{vastaus}
		$x = 0$, $x = 1$ tai $x = \sqrt[113]{5}$
	\end{vastaus}
\end{tehtava}

\begin{tehtava}
    Ratkaise yhtälöt.
    \begin{alakohdat}
        \alakohta{$x^5 - 2x^3 + x = 0$}
        \alakohta{$x^8 + 4x^4 = 5x^6$}
    \end{alakohdat}
    \begin{vastaus}
        \begin{alakohdat}
        	\alakohta{$x = 0$ tai $x = \pm1$}
        	\alakohta{$x = 0$ tai $x = \pm1$ tai $x = \pm2$}
        \end{alakohdat}
    \end{vastaus}
\end{tehtava}

\begin{tehtava}
 	Ratkaise yhtälö $5^{x^3+4x^2+x}=1$. 
	\begin{vastaus}
	$x=0$ tai $x=-2 + \sqrt[]{3}$ tai $x=-2 - \sqrt[]{3}$
	\end{vastaus}
\end{tehtava}

\subsubsection*{Lisää tehtäviä}

\begin{tehtava}
    Ratkaise yhtälöt.
    \begin{alakohdat}
        \alakohta{$x^8 - 1 = 0$}
        \alakohta{$x^8 - x^4 = 0$}
        \alakohta{$x^8 - x^4 - 1 = 0$}
    \end{alakohdat}
    \begin{vastaus}
        \begin{alakohdat}
            \alakohta{$x = \pm\sqrt{1}$}
            \alakohta{$x = 0$ tai $x = \pm\sqrt{1}$}
            \alakohta{$x = \pm\sqrt[4]{\frac{1+\sqrt{5}}{2}} (= \pm\sqrt[4]{\upvarphi})$ (luku $\upvarphi$ on kultaisena leikkauksena tunnettu vakio)}
        \end{alakohdat}
    \end{vastaus}
\end{tehtava}

\begin{tehtava}
Millä muuttujan $x$ arvoilla funktio $ f(x)=x^5$ saa saman arvon kuin funktio $ g(x)=x^4$?
\begin{vastaus}
Muuttujan arvoilla $0$ ja $1$
\end{vastaus}
\end{tehtava}

\begin{tehtava}
    Ratkaise yhtälöt.
    \begin{alakohdat}
        \alakohta{$x^4 + 7x^3 = 0$}
        \alakohta{$2x^3 - 16x^2 + 32x = 0$}
        \alakohta{$x^6 + 6x^5 = -9x^4$}
        \alakohta{$x^3 - 2x^5 = 0$}
    \end{alakohdat}
    \begin{vastaus}
        \begin{alakohdat}
        	\alakohta{$x = 0$ tai $x = -7$}
        	\alakohta{$x = 0$ tai $x = 4$}
        	\alakohta{$x = 0$ tai $x = -3$}
            \alakohta{$x = 0$ tai $x = \pm\dfrac{1}{\sqrt{2}}$}
        \end{alakohdat}
    \end{vastaus}
\end{tehtava}

\begin{tehtava}
$\star$ (K94/T2a) Polynomin $P(x)=ax^3-31x^2+1$ eräs nollakohta on $x=1$. Määritä $a$ ja ratkaise tämän jälkeen yhtälö $P(x)=0$.
\begin{vastaus}
      $a=30$. Yhtälön ratkaisut ovat $1$, $\frac{1}{5}$ ja $-\frac{1}{6}$. (Vihje: kirjoita kolmannen asteen polynomi mielivaltaisen, tuntemattoman toisen asteen polynomin ja binomin $x-1$ tulona ja avaa tulosta sulkeet. Vertaa kolmannen asteen polynomin kertoimiin.)
    \end{vastaus}
\end{tehtava}

\begin{tehtava}
	$ \star $ Ratkaise yhtälö $(x+\frac{1}{x})^2-x-\frac{1}{x}-6 = 0$.
	\begin{vastaus}
		$x = -1$, $x = \frac{3\pm \sqrt{5}}{2}$
	\end{vastaus}
\end{tehtava}

\begin{tehtava}
	$ \star $ Ratkaise yhtälö $2^x-1=\frac{12}{2^x}$
	\begin{vastaus}
	$x=2$
	\end{vastaus}
\end{tehtava}

\end{tehtavasivu}