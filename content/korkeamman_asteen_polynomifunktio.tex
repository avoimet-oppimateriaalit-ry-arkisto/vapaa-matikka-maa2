\section{Korkeamman asteen polynomifunktio}

\qrlinkki{http://opetus.tv/maa/maa2/n-asteinen-polynomifunktio/}{Opetus.tv: \emph{N-asteinen polynomifunktio} (10:40)}

Kaikki paraabelit ovat samanlaisia, mutta korkeamman asteen polynomien kuvaajat eivät ole. Yleisesti pätee kuitenkin seuraava tulos:

\laatikko[Polynomien ominaisuuksia]{
\begin{itemize}
\item Asteen $n$ polynomilla on korkeintaan $n$ nollakohtaa.
\item Jos polynomin aste on pariton, sillä on vähintään yksi nollakohta.
% ??
%\item Kaikki polynomit voidaan jakaa tekijöihin, jotka ovat korkeintaan toista astetta. 
\end{itemize}}

Todistetaan näistä ensimmäinen tulos:

\begin{todistus}
Jos $n$ asteen polynomilla $P$ olisi yli $n$ eri nollakohtaa, sillä olisi yli $n$ ensimmäisen asteen tekijää. Polynomin $P$ aste olisi siis yli $n$, mikä on ristiriita. Nollakohtia on siis korkeintaan $n$.
\end{todistus}

Toista tulosta ei todisteta tässä täsmällisesti, mutta valoitetaan asiaa esimerkin kautta. Tutkitaan esimerkiksi polynomia 
$$P(x)=x^3+bx^2+cx+d.$$
Polynomia on kätevintä tarkastella muodossa, jossa $x^3$ on otettu yhteiseksi tekijäksi:
$$P(x) = x^3\left(1+\frac{a}{x}+\frac{b}{x^2}+\frac{d}{x^3}\right)$$
Kun $x$ on hyvin suuri positiivinen tai hyvin pieni negatiivinen luku,
termit $\frac{b}{x}$, $\frac{c}{x^2}$ ja $\frac{d}{x^3}$ ovat hyvin pieniä, eli
\begin{align*}
P(x)&= x^3\left(1+\frac{b}{x}+\frac{c}{x}+\frac{d}{x^3}\right) \\
	& \approx  x^3\left(1+0+0+0\right) = x^3
\end{align*}
Voidaan siis päätellä, että riippumatta kertoimista $b$, $c$, $d$ polynomin $P$ arvo on positiivinen, kun $x$ on suuri positiivinen luku ja negatiivinen, kun $x$ on pieni negatiivinen luku.

Koska $P$ saa sekä positiivisia että negatiivia arvoja, sillä on jossakin niiden välissä nollakohta. (Tämän takaa jatkuvuus, josta lisää kurssilla 7.) Esimerkin mukaisilla kolmannen asteen polynomeilla on siis aina nollakohta. Yleisesti pätee, että kaikilla paritonasteisille polynomeilla on ainakin yksi nollakohta.

\begin{esimerkki} Kolmannen asteen polynomilla on 1--3 nollakohtaa.

\begin{lukusuora}{-2.5}{3}{3.6}
\lukusuoraisobbox
\lukusuorakuvaaja{(x**3-x-1)/2}
\lukusuorapienipiste{1.32472}{}
\end{lukusuora}
\begin{lukusuora}{-2.8}{2.5}{3.6}
\lukusuoraisobbox
\lukusuorakuvaaja{(x**3-x+0.3849)/2}
\lukusuorapienipiste{-1.1547}{}
\lukusuorapienipiste{0.577028}{}
\end{lukusuora}
\begin{lukusuora}{-2}{2}{3.6}
\lukusuoraisobbox
\lukusuorakuvaaja{1.4*(x**3-x)}
\lukusuorapienipiste{-1}{}
\lukusuorapienipiste{0}{}
\lukusuorapienipiste{1}{}
\end{lukusuora}
\end{esimerkki}


\begin{esimerkki} Neljännen asteen polynomilla on $0--4$ nollakohtaa.
\begin{lukusuora}{-4}{4}{3.6}
\lukusuorabboxy{-0.5}{1.5}
\lukusuorakuvaaja{(x**4-5*x**2+12)/14}
\end{lukusuora}
\begin{lukusuora}{-4}{4}{3.6}
\lukusuorabboxy{-0.5}{1.5}
\lukusuorakuvaaja{(x**4-5*x**2+3*x+11.2)/14}
\lukusuorapienipiste{-1.71394}{}
\end{lukusuora}
\begin{lukusuora}{-4}{4}{3.6}
\lukusuorabboxy{-0.5}{1.5}
\lukusuorakuvaaja{(x**4-5*x**2-3)/14}
\lukusuorapienipiste{2.354}{}
\lukusuorapienipiste{-2.354}{}
\end{lukusuora}

\begin{lukusuora}{-4}{4}{3.6}
\lukusuorabboxy{-0.5}{1.5}
\lukusuorakuvaaja{(x**4-5*x**2+3*x+1.75842)/14}
\lukusuorapienipiste{-2.43622}{}
\lukusuorapienipiste{-0.367327}{}
\lukusuorapienipiste{1.402}{}
\end{lukusuora}
\begin{lukusuora}{-2.8}{2.8}{3.6}
\lukusuorabboxy{-0.5}{1.5}
\lukusuorakuvaaja{0.6*(x+1.5)*(x+0.5)*(x-0.5)*(x-1.5)}
\lukusuorapienipiste{1.5}{}
\lukusuorapienipiste{0.5}{}
\lukusuorapienipiste{-1.5}{}
\lukusuorapienipiste{-0.5}{}
\end{lukusuora}
\end{esimerkki}

\begin{tehtavasivu}

\subsubsection*{Opi perusteet}

\begin{tehtava}
    Anna esimerkki
    \begin{alakohdat}
	\alakohta{neljännen asteen polynomista, jolla on neljä nollakohtaa}
	\alakohta{kolmannen asteen polynomista, jolla on kaksi nollakohtaa}
	\alakohta{neljännen asteen polynomista, jolla on yksi nollakohta}
	\alakohta{neljänen asteen polynomista, jolla ei ole nollakohtia}
\end{alakohdat}
    \begin{vastaus}
	Esimerkiksi    
    \begin{alakohdat}
	\alakohta{$P(x)=x(x-1)(x-2)(x-3)$ (nollakohdat $x= 0$, $x = 1$, $x=2$ ja $x=3$)}
	\alakohta{$P(x)=x^2(x-1)$ (nollakohdat $x= 0$ ja $x = 1$)}
	\alakohta{$Q(x)=x^4$ (nollakohta $x=0$)}
	\alakohta{$R(x)=x^4+1$}
\end{alakohdat}
    \end{vastaus}
\end{tehtava}

\begin{tehtava}
Alla on polynomin $P(x)$ kuvaaja. \\
\begin{kuvaajapohja}{1}{-3}{3}{-2}{5}
  \kuvaaja{-x**5+3*x**3+2}{$P(x)$}{black}
\end{kuvaajapohja} \\
Kaikki polynomin nollakohdat näkyvät kuvaajassa.
\begin{alakohdat}
\alakohta{Mitä voidaan sanoa polynomin $P$ asteesta?}
\alakohta{Mikä on polynomin $P$ vakiotermi?}
\end{alakohdat}
\begin{vastaus}
\begin{alakohdat}
\alakohta{Nollakohtia on kolme, joten polynomin aste on vähintään $3$. (Itse asiassa todellinen aste on $5$, mutta sitä on vaikea päätellä silmämääräisesti kuvaajasta.)}
\alakohta{Vakiotermi on 2, koska $P(0)=2$}
\end{alakohdat}
\end{vastaus}
\end{tehtava}

\subsubsection*{Hallitse kokonaisuus}

\begin{tehtava}
    Mikä on se kolmannen asteen polynomi, jonka nollakohdat ovat $x=2$, $x=-1$ ja $x=3$, ja jonka vakiotermi on $3$?
    \begin{vastaus}
        $P(x)=\frac{1}{2}(x-2)(x+1)(x-3)$
    \end{vastaus}
\end{tehtava}

\begin{tehtava}
    Kolmannen asteen polynomille $P$ pätee $P(1)=P(2)=P(3)=-2$ ja $P(0)=16$. Ratkaise $P(x)$.
    \begin{vastaus}
        $P(x)=-3(x-1)(x-2)(x-3)-2=-3x^3+18x^2-33x+18$
    \end{vastaus}
\end{tehtava}

\begin{tehtava}
   	$\star$ Osoita, että jos $n$ asteen polynomeilla $P(x)$ ja $Q(x)$ on $n+1$ yhteistä pistettä, ne ovat sama polynomi.
    \begin{vastaus}
        Tarkastele polynomien erotuksen nollakohtia.
    \end{vastaus}
\end{tehtava}

\end{tehtavasivu}