Seuraavissa harjoituskokeissa on kussakin kahdeksan tehtävää, jotka on suunniteltu niin, että jokaisesta (alakohtineen) saa korkeintaan kuusi pistettä. %tulevaisuudessa täydet ratkaisut pisteytysohjeineen

\subsection*{Harjoituskoe 1}

\begin{tehtava}
	\begin{alakohdat}
	\alakohta{Miksi $t^{\pi}+1$ ei ole polynomi? ($1$\,p.)}
	\alakohta{Trinomi kerrotaan toisella trinomilla. Kuinka monta termiä syntyneeseen uuteen polynomiin tulee (ennen mahdollista sieventämistä)? ($1$\,p.)}
	\alakohta{Mikä on tulomuodossa esitetyn polynomin $(x^2+1)^{10}(1-x)^2$ aste? ($1$\,p.)}
	\alakohta{Mainitse jokin luku, joka kuuluu reaalilukuvälille $]1,966;1,967[$. ($0,5$\,p.)}
	\alakohta{Mikä on korkeimman asteen termin kerroin polynomissa $9\,001k^2-\frac{1}{4}k-\sqrt{2}k^3-k^2+k^3$ ($1$\,p.)}
	\alakohta{Polynomi $P$ voidaan esittää tulomuodossa $(x-2)x(x+3)$. Mikä on kyseisen polynomin suurin nollakohta? ($0,5$\,p.)}
	\alakohta{Minkälainen (muoto \& mahdollinen suunta) kuvaaja on polynomifunktiolla $P$, jonka arvot lasketaan yhtälöllä $P(t)=-100-\frac{1}{2}t$? ($1$\,p.)}
	\end{alakohdat}
 			\begin{vastaus}
 				\begin{alakohdat}
	\alakohta{Muuttujan $t$ eksponentti ei ole luonnollinen luku.}
	\alakohta{Yhdeksän}
	\alakohta{$22$}
	\alakohta{Esimerkiksi $1,9665$ (reuna-arvot eivät kuulu välille!)}
	\alakohta{$1+\sqrt{2}$}
	\alakohta{$x=2$}
	\alakohta{laskeva suora}
	\end{alakohdat}
 			\end{vastaus}
 \end{tehtava}

\begin{tehtava}
	\begin{alakohdat}
	\alakohta{Sievennä lauseke $\frac{x^3-4x^2-4x}{x^2-2x}$.}
	\alakohta{Kirahvi kantaa suussaan karahvia ja kävelee vaa'alle, jonka toisella puolella on $7$ karahvia ja sattuu niin, että vaaka on tasapainossa. Mikä on karahvin paino kirahveina?}
	\alakohta{Ratkaise yhtälöstä tuntematon $a$: $(9-a^2)(a-1)a=0$}
	\end{alakohdat}
	\begin{vastaus}
		\begin{alakohdat}
			\alakohta{$x-2$}
			\alakohta{Yksi karahvi painaa yhden kuudesosakirahvin.}
			\alakohta{$a=\pm 3$, $a=1$ tai $1=0$.}
	\end{alakohdat}
	\end{vastaus}
\end{tehtava}

\begin{tehtava}
Ratkaise yhtälöistä tuntematon reaaliluku $z$.
	\begin{alakohdat}
	\alakohta{$(2+\frac{\sqrt{2}}{2})z^2-z=\sqrt{2}z^3$}
	\alakohta{$z^4-2z^2+1=0$}
	\end{alakohdat}
		\begin{vastaus}
	\begin{alakohdat}
	\alakohta{$z=0$ tai $z=\frac{1}{2}$ tai $z=\sqrt{2}$}
	\alakohta{$z=-1$ tai $z=1$}
	\end{alakohdat}
		\end{vastaus}
\end{tehtava}

\begin{tehtava}
	\begin{alakohdat}
	\alakohta{Ratkaise epäyhtälö $(\frac{1}{2}x-1)^2<1-(x-1)^2$.}
	\alakohta{Osoita, että $\sqrt{11+4\sqrt{6}}=\sqrt{3}+2\sqrt{2}$.} %esimerkkitehtävä+harjoituksia!
	\end{alakohdat}
	\begin{vastaus}
		\begin{alakohdat}
	\alakohta{$\frac{2}{5}<x<2$}
	\alakohta{Korottamalla $\sqrt{3}+2\sqrt{2}$ neliöön ja käyttämällä muistikaavoja saadaan $11+4\sqrt{6}$.}
	\end{alakohdat}
	\end{vastaus}
\end{tehtava}

\begin{tehtava}
Millä parametrin $k$ arvoilla yhtälöllä $kx^2-(k+1)x+1=0$ on kaksi erisuurta reaalijuurta?
	\begin{vastaus}
Kaikilla paitsi $k=0$
	\end{vastaus}
\end{tehtava}

\begin{tehtava} %tästä pari perustehtävää ja kertaustehtäävää, myös suhteellisena, eikä tunnetuilla absoluuttisilla + kuinka paljon rahaa on tilillä kolmen vuoden kuluttua ilman lisätlalletusta?
Sofie tallettaa $1\,000$ euroa uudelle pankkitilille vuoden alussa ja uudelleen saman verran vuoden kuluttua. Pankki maksaa talletukselle vuosikorkoa siten, että kahden vuoden kuluttua ensimmäisestä talletuksen tilillä on rahaa $2\,100$ euroa. Kuinka suuri on pankin tarjoama korkokanta prosentin kymmenesosan tarkkuudella? Korkotuotoista maksettavaa lähdeveroa ja inflaatiota ei oteta huomioon. Muita tilitapahtumia ei ole.
	\begin{vastaus}
	Jos pankin korkokerroin on $x=(1+\frac{p}{100})$, niin vuoden kuluttua tilillä on rahaa $1\,000x$. Kahden vuoden kuluttua tilillä on rahaa $((1\,000x)+1\,000)x)$, mikä sisältää molemmat talletukset ja kaksi vuosikorkoa. Saadaan yhtälö $((1\,000x)+1\,000)x)=2\,100$ eli sievennettynä ja uudelleen järjesteltynä $1\,000x^2+1\,000x-2\,100=0$. Ratkaisukaavalla tai laskimella saadaan kaksi ratkaisua, joista hyväksytään vain positiivinen: $x=\frac{1}{10}(\sqrt{235}-5)\approx1,033$. Pankin tarjoama vuosikorkokanta on siten $3,3$\,\%.
	\end{vastaus}
\end{tehtava}

\begin{tehtava}
Ratkaise $y$ yhtälöstä $\frac{1}{2}n^n y^2-n^{2n}y-n^{3n+1}=0$, kun $n=50$. Esitä vastaus kymmenpotenssimuodossa kolmen merkitsevän numeron tarkkuudella.
	\begin{vastaus}
	Sieventämällä saadaan, että ratkaisut ovat muotoa $y=n^2 (1\pm \sqrt{1+2n})$. Sijoittamalla $n=50$ saadaan laskimesta kymmenpotenssimuodoiksi $-8,04\cdot 10^{85}$ ja $9,81\cdot 10^{85}$.
	\end{vastaus}
\end{tehtava}

\begin{enumerate}
\item Ratkaise yhtälöt.\\ a) $4x-1=0$\\ b) $x=-3x+2$\\ c) $2g-3g=g-8$
\item Ratkaise yhtälöt.\\ a) $4x^2-1=0$\\ b) $x^3=-3x$\\ c) $2y^2=y-8$
\item Ratkaise epäyhtälöt.\\ a) $1-\dfrac{1-x}{6}<x$\\ b) $(x+1)(x^2-2x-1)\geq0$\\ c) $\frac{x}{2}>\frac{x}{5}$
\item Täydennä neliöksi. \\ a) $x^2+2x+1$\\ b) $9x^2-6x+1$\\ c) $3x+4x^2+\frac{9}{16}$
\item Ratkaise $x$. \\ a) $\frac{x}{2}(x-1)=0$\\ b) $(3x^2-3)(3x^2+1)=0$\\ c) $(x-a)(x-b)=0$
\item Ratkaise yhtälö. $(x^2-4x+4)^2=0$
\item Millä $x$:n arvoilla polynomi $x^2-2x-3$ saa positiivisia arvoja?
\end{enumerate}

\begin{tehtava}
Mikä on funktion $f$, $f(x)=3x^2-12x+6$ pienin arvo?
	\begin{vastaus}
	$-6$ (vetoaminen paraabelin symmetrisyyteen, että pohja on nollakohtien puolivälissä, tai vastaus nähdään neliöön täydennetystä lausekkeesta perustellen, että reaaliluvun neliö voi olla vähintään nolla)
	\end{vastaus}
\end{tehtava}
	
\newpage

\subsection*{Harjoituskoe 2}

\begin{tehtava}
	\begin{alakohdat}
	\alakohta{Miksi $\frac{1}{t^2}+\pi$ ei ole polynomi? ($1$\,p.)}
	\alakohta{Binomi kerrotaan trinomilla. Kuinka monta termiä syntyneeseen uuteen polynomiin tulee (ennen mahdollista sieventämistä)? ($1$\,p.)}
	\alakohta{Mikä on tulomuodossa esitetyn polynomin $(x^3+1)^5(1-x)^3$ aste? ($1$\,p.)}
	\alakohta{Esitä ehto $x\in ]-\pi,42]$ kaksoisepäyhtälönä. ($1$\,p.)}
	\alakohta{Jaa polynomi $y^3-2y^2+y$ tekijöihin. ($1$\,p.)}
	\alakohta{Sievennä $\frac{x^2-10^{102}}{x-10^{51}}$ ($1$\,p.)}
	\end{alakohdat}
 			\begin{vastaus}
 				\begin{alakohdat}
	\alakohta{Muuttujan $t$ eksponentti ($-2$) ei ole luonnollinen luku.}
	\alakohta{kuusi}
	\alakohta{$18$}
	\alakohta{$\pi < x \leq 42$}
	\alakohta{$y^3-2y^2+y=y(y-1)^2$}
	\alakohta{$x+10^{51}$}
	\end{alakohdat}
 			\end{vastaus}
 \end{tehtava}

\begin{tehtava}
	\begin{alakohdat}
%\alakohta{Näytä, että polynomiyhtälöt $(2t-1)^2=-2t+1$ ja $t^2=-t$ ovat yhtenevät.}
\alakohta{Ratkaise reaaliluku $x$ yhtälöstä $(2x^2-1)(x-1)=1$}
	\end{alakohdat}
	\begin{vastaus}
		\begin{alakohdat}
%\alakohta{Joko tee sijoitus $2t-1=x$, jolloin yhtälöt muuttuvat heti samanmuotoisiksi, tai puretaan sulut ja sievennetään:
%$$(2t-1)^2=-2t+1$$
%$$4t^2-4t+1+2t-1=0$$
%$$4t^2-2t=0$$
%$$2t^2-t=0$$
%$$•$$
%
%}
\alakohta{$x=0$ tai $x=\frac{1}{2}(1-\sqrt{3})$ tai $x=\frac{1}{2}(1+\sqrt{3})$}
	\end{alakohdat}
	\end{vastaus}
\end{tehtava}

\begin{enumerate}
\item Ratkaise yhtälöt.\\ a) $x-5x=0$\\ b) $x^4-1=0$\\ c) $(x-1)(x+4) = x(x-5)$
\item Ratkaise epäyhtälöt.\\ a) $x^2-8\geq0$\\ b) $x^2-8\geq(x-3)^2$\\ c) $x^2-6x+9\leq0$
\item Kolmannen asteen polynomifunktiolle pätee $P(-1)=0$, $P(0)=0$ ja $P(1)=0$. Lisäksi $P(3)=3$. Määritä polynomi $P$.
\item Millä vakion $t$ arvoilla yhtälöllä $tx^2+tx-6=0$ ei ole ratkaisuja?
\item Laske. \\ a) $(3x+1)^2+5=6x$\\ b) $5x>15x$\\ c)$3x^2=(3x)^2+6$ 
\item Lukujen $a$ ja $b$ erotus $a-b=2$ ja tulo $ab=4$, laske käänteislukujen erotus $\frac{1}{a}-\frac{1}{b}$.
\item Millä vakion $c$ arvoilla polynomi $x^2+cx+c$ saa sekä positiivia että negatiivisia arvoja?
\item Jaa polynomi tekijöihin.\\ a) $x^2+x-30$\\ b)  $x^2-x-30$\\ c)  $-2x^2+5x-3$ 
\end{enumerate}

\begin{tehtava}
	\begin{alakohdat}
\alakohta{Avaa sulut lausekkeesta $(a-b)^3$ ja sievennä tulos.}
\alakohta{Ratkaise reaaliluku $x$ yhtälöstä $x^3-3x^2+3x=9$}
	\end{alakohdat}
	\begin{vastaus}
		\begin{alakohdat}
\alakohta{$a^3-3a^2b+3ab^2-b^3$}
\alakohta{Binomin kuutioksi täydentämällä saadaan yhtälö potenssiyhtälö $(x-1)^3=8$, jonka ratkaisuna $x=3$}
	\end{alakohdat}
	\end{vastaus}
\end{tehtava}

\begin{tehtava}
Muodosta muuttujan $a$ funktio $f$, joka antaa toisen asteen polynomin $ax^2+2x+2a$ nollakohtien etäisyyden. Määritä funktion $f$ laajin reaalinen määrittelyjoukko ja arvojoukko.
	\begin{vastaus}
	Nollakohtien erotusfunktion määrittelee yhtälö $f(a)=\frac{\sqrt{4-8a^2}}{a}$. Määrittelyjoukkoa rajoittavia tekijöitä ovat neliöjuuri ja jakolasku. Nimittäjän perusteella vaaditaan $a\neq 0$, neliöjuuren perusteella $4-8a^2\geq 0$. Epäyhtälöstä saadaan kaksi ratkaisua: $-\frac{1}{\sqrt{2}}\leq a \leq \frac{1}{\sqrt{2}}$. Yhdistämällä tiedot saadaan määrittelyjoukoksi kaksiosainen reaalilukuväli $[-\frac{1}{\sqrt{2}},0[\cup ]0,\frac{1}{\sqrt{2}}]$.
	
	Arvojoukon voi päätellä asettamalla funktion arvoksi tuntematon $t$ ja ratkaisemalla yhtälöstä muuttuja $x$:
	$$\frac{\sqrt{4-8x^2}}{x}=t$$
	$$\sqrt{4-8x^2}=tx$$
	$$4-8x^2=(tx)^2=t^2x^2$$
	$$t^2x^2+8x^2=4$$
	$$(t^2+8)x^2=4$$
	$$x^2=\frac{4}{t^2+8}$$
	$$x=\pm \frac{2}{\sqrt{t^2+8}}$$
	
	Huomataan, että saadun lausekkeen perusteella ei ole mitään syytä rajoittaa $t$:n arvoja. Funktion arvojoukko on siis koko reaalilukujen joukko $\mathbb{R}$.
	\end{vastaus}
\end{tehtava}

\subsection*{Harjoituskoe 3}

\begin{tehtava}
	\begin{alakohdat}
	\alakohta{Miksi $\sqrt{k}+\sqrt{2}$ ei ole polynomi? ($1$\,p.)}
	\alakohta{Binomi kerrotaan trinomilla. Kuinka monta termiä syntyneeseen uuteen polynomiin tulee (ennen mahdollista sieventämistä)? ($1$\,p.)}
	\alakohta{Mikä on tulomuodossa esitetyn polynomin $(x^3+1)^5(1-x)^3$ aste? ($1$\,p.)}
	\alakohta{Selitä, miksi kerrottaessa epäyhtälöä negatiivisella luvulla epäyhtälön merkki kääntyy. ($1$\,p.)}
		\alakohta{Esitä muuttujaa $x$ koskeva kaksoisepälönä esitetty ehto $-\frac{2}{3}<x\leq100$ joukko-opillisesti. ($1$\,p.)
		\alakohta{Muuttujan $x$ arvot ovat suoraan verrannollisia $y$:n arvoihin. Perustele, voiko $x$ ja $y$ välinen yhteys olla polynomiaalinen. ($1$\,p.)}
	\end{alakohdat}
 			\begin{vastaus}
 				\begin{alakohdat}
	\alakohta{Muuttujan $k$ eksponentti ($\frac{1}{2}$) ei ole luonnollinen luku.}
	\alakohta{kuusi}
	\alakohta{$18$}
	\alakohta{$\pi < x \leq 42$}
	\alakohta{$x\in ]\frac{2}{3},100]$}
	\alakohta{Jos $x\propto y$, niin voidaan kirjoittaa $x=ky$, missä $k$ on vakio. Yksiterminen lauseke $ky$ on polynomi.}
	\end{alakohdat}
 			\end{vastaus}
 \end{tehtava}

%\begin{tehtava}
%Pitsaravintola myy pitsoja ... Muodosta funktio, joka esittää...
%\end{tehtava}

\begin{tehtava}
Millä $x$:n arvoilla polynomi $x^3-2x^2-x$ saa positiivisia arvoja?
	\begin{vastaus}
	Kun $1-\sqrt{2}<x<0$ tai $x>1+\sqrt{2}$
	\end{vastaus}
\end{tehtava}

\begin{tehtava}
Jaa polynomi $t^5-t^4+t^3-t^2$ tekijöihin. Kuinka monta reaalista nollakohtaa polynomilla on?
	\begin{vastaus}
	$t^5-t^4+t^3-t^2=t^2(t-1)(t^2+1)$, reaalisia nollakohtia on kaksi kappaletta
	\end{vastaus}
\end{tehtava}

\begin{tehtava}
Polynomin $Q$ muuttuja on $x$, ja polynomi voidaan esittää tulomuodossa $(ax-b)(\frac{2ab}{a^2+b^2}x-1)(bx-a)$, missä $a$ ja $b$ ovat reaalisia vakioita, joille pätee $0<a<b$. Perustele, mikä on funktion nollakohtien suuruusjärjestys.
		\begin{vastaus}
Lausekkeen perusteella nollakohdat ovat $\frac{b}{a}$,$\frac{a^2+b^2}{2ab}$ ja $\frac{a}{b}$. Koska $0<a<b$, niin $a$ on positiivinen, ja kaksoisepäyhtälö voidaan jakaa sillä niin, että järjestys säilyy (''merkkien suuntaa ei tarvitse kääntää''): $0<1<\frac{b}{a}$. Samoin voidaan tehdä $b$:llä: $0<\frac{a}{b}<1$. Vertailemalla nähdään, että selvästi $\frac{a}{b}<\frac{b}{a}$. Kolmas nollakohta on kahden muun aritmeettinen keskiarvo, joten se on varmasti niiden välissä.
%Jos ei sattumalta huomannut, että kolmas nollakohta on kahden muun aritmeettinen keskiarvo (jolloin se on varmasti niiden välissä), selviää sekin samoin:
%
%\begin{align*}
%0&<a&&<b &&&|\cdot a  \\
%0&<a^2&&<ab &&&|+b^2 \\
%b^2&<a^2+b^2&&<b^2+ab &&&|:2ab \\
%\frac{b^2}{2ab}&<\frac{a^2+b^2}{2ab}&&<\frac{b^2+ab}{2ab} &&&|\textrm{sievennetään} \\
%\frac{b}{2a}&<\frac{a^2+b^2}{2ab}&&<\frac{b+a}{2a} &&& \\
%\frac{1}{2} \frac{b}{a}&<\frac{a^2+b^2}{2ab}&&<\frac{b}{2a}+\frac{a}{2a} &&& \\
%\frac{1}{2} \frac{b}{a}&<\frac{a^2+b^2}{2ab}&&<\frac{1}{2} \frac{b}{a}+\frac{1}{2} &&& \\
%\end{align*}
		\end{vastaus}
\end{tehtava}
\newpage

\subsection*{Harjoituskoe 4}

\begin{enumerate}
\item Ratkaise epäyhtälöt. \\
a) $2x^3 \geq x$ \\
b) $y^2 \leq 3y -9 $

%osoita, että (neliöjuuässäsä, neliöönkorottamisella.)
%\item Millä vakion $a$ arvoilla yhtälöllä $x^2+ax+a=0$ on kaksoisjuuri?
\item Muodosta funktio, joka esittää...

\item .
\item Ratkaise yhtälöt.\\ a) $3x-5=6$\\ b) $20x-20=20$\\ c) $2x-5=\frac{5x+2}{2}$
\item Ratkaise yhtälöt.\\ a) $-5x^2-5=0$\\ b) $8x^2=-2x$\\ c) $x^2+10=0$
\item Ratkaise epäyhtälöt.\\ a) $-5x^2-5>0$\\ b) $x^2+10\geq0$\\ c) $x^2<x$
\item Millä $h$:n arvoilla yhtälöllä $h^2x^2+hx+\frac{1}{4}=0$ on tasan yksi ratkaisu?
\item Ratkaise yhtälö $x^2-6x=0$\\ a) tulon nollasäännöllä\\ b) toisen asteen yhtälön ratkaisukaavalla
\item Ratkaise epäyhtälö $(\frac{x+1}{-4})x-3>1$
\item Jaa polynomi tekijöihin.\\ a) $x^2-x-2$\\ b) $4x^2-2x-2$
\end{enumerate}

%Kuinka kaukana funktion ... kaksi nollakohtaa ovat toisistaan?

%Kinematiikkaa

\begin{tehtava}
Johda toisen asteen yhtälön ratkaisukaava lähtien yhtälön normaalimuodosta $ax^2+bx+c=0$, missä $a, b$ ja $c$ ovat reaalisia vakioita ja $a \neq 0$.
	\begin{vastaus}
	Katso luku Toisen asteen yhtälön ratkaisukaava. (On olemassa erilaisia variantteja.)
	\end{vastaus}
\end{tehtava}

\begin{tehtava}
Funktion $y$ arvot määritellään kaavalla $y(t)=\dfrac{t^2+\left(\frac{1}{3}+\pi\right)t-\frac{\pi}{3}}{t^2+\left(\frac{1}{3}-\pi\right)t-\frac{\pi}{3}}$.
	\begin{alakohdat}
	\alakohta{Mikä on funktion määrittelyjoukko?}
	\alakohta{Määritä ne $ty$-koordinaatiston pisteet, joissa funktion kuvaaja leikkaa jomman kumman koordinaattiakselin.}
	\end{alakohdat}
		\begin{vastaus}
	\begin{alakohdat}
	\alakohta{Reaalilukujen joukko poislukien $t=-\frac{1}{3}$ ja $t=\pi$}
	\alakohta{$y$-akseli leikkautuu, kun $t=0$, eli pisteessä $(0,\frac{pi}{3})$. $t$-akseli leikkautuu vain yhdessä pisteessä: $(-\pi, 0)$, koska määrittejoukko ei sisällä funktion lausekkeen osoittajan toista nollakohtaa $-\frac{1}{3}$.}
	\end{alakohdat}
		\end{vastaus}
\end{tehtava}
\newpage
%\subsection*{Harjoituskoe 5}
%\begin{enumerate}
%\item Ratkaise yhtälöt.\\ a) $-x-5+2x=0$\\ b) $8x^2=-2x+4$\\ c) $(-3x)^2-36=0$
%\item Ratkaise epäyhtälöt.\\ a) $-2x^2-2>2$\\ b) $x^2-1\geq0$\\ c) $ax^2<bx$
%\item Kolme lohikäärmettä ja yksi kissa painaa saman verran kuin 10 kissaa ja 18 Jarkkoa. Muodosta yhtälö tilanteesta yhtälö ja ratkaise yhden lohikäärmeen paino.
%\item Montako juurta on yhtälöllä\\ a) $9x^2+3x+1=0$\\ b) $6x^2-3x=-2$\\ c) $3x^2-32$\\ d) $3x-30=0$
%\item Millä parametrin r arvoilla yhtälölle $rx^2-rx+1=0$ ei ole ratkaisuja.
%\item 
%\item 
%\item Ratkaise yhtälö $\frac{21x^2}{700}-\frac{7x}{1400}-\frac{14x}{2800}=0$ ilman laskinta.
%\end{enumerate}