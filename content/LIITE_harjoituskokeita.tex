\subsection*{Harjoituskoe 1}

\begin{tehtava}
 a) Miksi $t^{pi}+1$ ei ole polynomi? \\ b) Trinomi kerrotaan toisella trinomilla. Kuinka monta termiä syntyneeseen uuteen polynomiin tulee (ennen mahdollista sieventämistä)? \\ c) ...
 			\begin{vastaus}
 			
 			\end{vastaus}
 \end{tehtava}

\begin{enumerate}
\item
\item Ratkaise yhtälöt.\\ a) $4x-1=0$\\ b) $x=-3x+2$\\ c) $2g-3g=g-8$
\item Ratkaise yhtälöt.\\ a) $4x^2-1=0$\\ b) $x^3=-3x$\\ c) $2y^2=y-8$
\item Ratkaise epäyhtälöt.\\ a) $1-\dfrac{1-x}{6}<x$\\ b) $(x+1)(x^2-2x-1)\geq0$\\ c) $\frac{x}{2}>\frac{x}{5}$
\item Millä parametrin $k$ arvoilla yhtälöllä $kx^2-(k+1)x+1=0$ on kaksi erisuurta reaalijuurta?
\item Täydennä neliöksi. \\ a) $x^2+2x+1$\\ b) $9x^2-6x+1$\\ c) $3x+4x^2+\frac{9}{16}$
\item Ratkaise $x$. \\ a) $\frac{x}{2}(x-1)=0$\\ b) $(3x^2-3)(3x^2+1)=0$\\ c) $(x-a)(x-b)=0$
\item Ratkaise yhtälö. $(x^2-4x+4)^2=0$
\item Millä $x$:n arvoilla polynomi $x^2-2x-3$ saa positiivisia arvoja?
\end{enumerate}

\subsection*{Harjoituskoe 2}

\begin{enumerate}
\item Ratkaise yhtälöt.\\ a) $x-5x=0$\\ b) $x^4-1=0$\\ c) $(x-1)(x+4) = x(x-5)$
\item Ratkaise epäyhtälöt.\\ a) $x^2-8\geq0$\\ b) $x^2-8\geq(x-3)^2$\\ c) $x^2-6x+9\leq0$
\item Kolmannen asteen polynomifunktiolle pätee $P(-1)=0$, $P(0)=0$ ja $P(1)=0$. Lisäksi $P(3)=3$. Määritä polynomi $P$.
\item Millä vakion $t$ arvoilla yhtälöllä $tx^2+tx-6=0$ ei ole ratkaisuja?
\item Laske. \\ a) $(3x+1)^2+5=6x$\\ b) $5x>15x$\\ c)$3x^2=(3x)^2+6$ 
\item Lukujen $a$ ja $b$ erotus $a-b=2$ ja tulo $ab=4$, laske käänteislukujen erotus $\frac{1}{a}-\frac{1}{b}$.
\item Millä vakion $c$ arvoilla polynomi $x^2+cx+c$ saa sekä positiivia että negatiivisia arvoja?
\item Jaa polynomi tekijöihin.\\ a) $x^2+x-30$\\ b)  $x^2-x-30$\\ c)  $-2x^2+5x-3$ 
\end{enumerate}


\subsection*{Harjoituskoe 3}

\begin{enumerate}
\item Jaa polynomi $t^5-t^4+t^3-t^2$tekijöihin. Kuinka monta reaalista nollakohtaa polynomilla on?
%\item Olkoon $a > 0$. Millä muuttujan $x$ arvoilla funktion $Q(x)=x3-ax^2$ arvot ovat epänegatiivisia?
%\item Millä vakion $r$ arvoilla yhtälöllä $rx^2+rx-1=0$ ei ole reaaliratkaisuja?
\item Muodosta funktio, joka esittää...
\end{enumerate}


\subsection*{Harjoituskoe 4}

\begin{enumerate}


\item Ratkaise epäyhtälöt. \\
a) $2x^3 \geq x$ \\
b) $y^2 \leq 3y -9 $

%\item Millä vakion $a$ arvoilla yhtälöllä $x^2+ax+a=0$ on kaksoisjuuri?
\item Muodosta funktio, joka esittää...

\item Johda toisen asteen yhtälön ratkaisukaava lähtien yhtälön normaalimuodosta $ax^2+bx+c=0$, missä $a, b$ ja $c$ ovat reaalisia vakioita ja $a \neq 0$.
\item Ratkaise yhtälöt.\\ a) $3x-5=6$\\ b) $20x-20=20$\\ c) $2x-5=\frac{5x+2}{2}$
\item Ratkaise yhtälöt.\\ a) $-5x^2-5=0$\\ b) $8x^2=-2x$\\ c) $x^2+10=0$
\item Ratkaise epäyhtälöt.\\ a) $-5x^2-5>0$\\ b) $x^2+10\geq0$\\ c) $x^2<x$
\item Millä $h$:n arvoilla yhtälöllä $h^2x^2+hx+\frac{1}{4}=0$ on tasan yksi ratkaisu?
\item $11,7$metriä pitkä lauta katkaistaan niin että toinen pala on $3,2$m pitempi kuin toinen. Minkä pituinen on lyhyempi pala?
\item Ratkaise yhtälö $x^2-6x=0$\\ a) tulon nollasäännöllä\\ b) toisen asteen yhtälön ratkaisukaavalla
\item Ratkaise epäyhtälö $(\frac{x+1}{-4})x-3>1$
\item Jaa polynomi tekijöihin.\\ a) $x^2-x-2$\\ b) $4x^2-2x-2$


\end{enumerate}

\subsection*{Harjoituskoe 5}
\begin{enumerate}
\item Ratkaise yhtälöt.\\ a) $-x-5+2x=0$\\ b) $8x^2=-2x+4$\\ c) $(-3x)^2-36=0$
\item Ratkaise epäyhtälöt.\\ a) $-2x^2-2>2$\\ b) $x^2-1\geq0$\\ c) $ax^2<bx$
\item Kolme lohikäärmettä ja yksi kissa painaa saman verran kuin 10 kissaa ja 18 Jarkkoa. Muodosta yhtälö tilanteesta yhtälö ja ratkaise yhden lohikäärmeen paino.
\item Montako juurta on yhtälöllä\\ a) $9x^2+3x+1=0$\\ b) $6x^2-3x=-2$\\ c) $3x^2-32$\\ d) $3x-30=0$
\item Millä parametrin r arvoilla yhtälölle $rx^2-rx+1=0$ ei ole ratkaisuja.
\item Kirahvi kantaa suussaan karahvia ja kävelee tasapaino vaa'alle jonka toisella puolella on 7 karahvia ja sattuu niin että vaaka on tasapainossa. Piirrä tilanteesta kuva, muodosta tilanteesta yhtälö ja ratkaise kirahvin paino karahveina.
\item Selitä miksi kerrottaessa epäyhtälöä negatiivisella luvulla epäyhtälön merkki kääntyy.
\item Ratkaise yhtälö $\frac{21x^2}{700}-\frac{7x}{1400}-\frac{14x}{2800}=0$ ilman laskinta.
\end{enumerate}