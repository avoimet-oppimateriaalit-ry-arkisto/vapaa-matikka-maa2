\qrlinkki{http://opetus.tv/maa/maa2/ensimmaisen-asteen-epayhtalo/}{Opetus.tv: \emph{ensimmäisen asteen epäyhtälö} ($14.55$ ja $8.21$)}

%Harjoittelemme nyt erityisesti 1. asteen epäyhtälöiden ratkaisemista -- toisen asteen ja sitä korkeampien polynomiepäyhtälöiden ratkaisemista käsitellään toisen asteen yhtälön käsittelyn jälkeen.

Samoin kuin yhtälöiden kohdalla, epäyhtälö pyritään muuttamaan niin yksinkertaiseen muotoon kuin mahdollista, jotta yksinkertaisesta tilanteesta nähdään välittömästi, mitkä luvut kuuluvat ratkaisuun ja mitkä eivät. Tuntemattomat pyritään yhdistämään, ja epäyhtälöä muokataan niin, että tuntematon saadaan yksin omalle puolelleen yhtälöä.

\begin{esimerkki}
Ratkaise epäyhtälö $2x+1 < 0$.
\begin{esimratk}
\begin{align*}
2x+1 &< 0 && \ppalkki -1 \\
2x &< -1 && \ppalkki :2 \\
x &< -\frac{1}{2}
\end{align*}
\end{esimratk}
\begin{esimvast}
$x < -\frac{1}{2}$. 
Ratkaisu voidaan esittää myös muodossa $x \in ]-\infty, -\frac{1}{2}[$.
\end{esimvast}

Ratkaisua voidaan tulkita graafisesti tutkimalla lausekkeeseen $2x+1$ liittyvää kuvaajaa:

\begin{kuva}
kuvaaja.pohja(-2, 2, -1, 3, korkeus = 5, nimiX = "$x$")
kuvaaja.piirra("2*x+1", nimi = "$f(x) = 2x + 1$", kohta = 0.75)
\end{kuva}

Alkuperäinen epäyhtälö $2x+1<0$ vaatii, että lausekkeen $2x+1$ arvo on negatiivinen. Yhtälön ratkaisu on mahdollista nähdä katsomalla kuvasta, millä kaikilla $x$:n arvoilla funktion $2x+1$ kuvaaja laskee vaaka-akselin alapuolelle. Tällöin funktio, siis toisaalta lauseke $2x+1$, saa negatiivisia arvoja.
\end{esimerkki}

Ensimmäisen asteen epäyhtälö ratkaistaan muuten kuin yhtälö, mutta kun kerrotaan tai jaetaan negatiivisella luvulla, epäyhtälömerkki kääntyy.

\begin{esimerkki}
Ratkaise epäyhtälö $x < 5x-8$.
\begin{esimratk}
\begin{align*}
x &< 5x-8 && \ppalkki -5x \\
-4x &< -8 && \ppalkki :(-4), \ \ 
\text{merkki kääntyy, } -4 < 0 \\
x &> 2 &&
\end{align*}
\end{esimratk}
\begin{esimvast}
$x > 2$.
\end{esimvast}
\end{esimerkki}

Merkki kääntyy yleisemmin aina, kun yhtälön molemmille puolille sovelletaan jotain vähenevää funktiota -- tästä lisää myöhemmilä kursseilla.

%esimerkki, jossa ei kokonaislukukertoimia >_>

\begin{esimerkki} Millä $w$:n arvoilla pätee
$-8w-(8-w) \geq \frac12 w+5$?
\begin{esimratk}
\begin{align*}
-8w-(8-w) &\geq \frac12 w+5 \\
-8w-8+w &\geq \frac12 w+5 \\
-7w-8 &\geq \frac12 w+5  \ \ \ \ \ && \ppalkki -\frac12 w \\
-\frac{15}{2} w-8 &\geq 5  \ \ \ \ \ && \ppalkki +8 \\
-\frac{15}{2} w &\geq 13  \ \ \ \ \ && \ppalkki :(-\frac{15}{2}) \\
w &\leq 13:(-\frac{15}{2}) \\
w &\leq -13\cdot \frac{2}{15} \\
w &\leq -\frac{26}{15} \\
\end{align*}
\end{esimratk}
\begin{esimvast}
$w \leq -\frac{26}{15}$
\end{esimvast}
\end{esimerkki}

\begin{esimerkki}
Ratkaise kaksoisepäyhtälö $1\leq q+7<-5q-2$.
\begin{esimratk}
Ratkaistavana on itse asiassa kaksi epäyhtälöä: $1\leq q+7$ ja $q+7<-5q-2$. Halutaan siis löytää ne $q$:n arvot, joilla molemmat epäyhtälöt pätevät.
\begin{align*}
1&\leq q+7 \ \ \ \ \ && \ppalkki -7 \\
-6&\leq q
\end{align*}
Vastaavasti toiselle yhtälölle:
\begin{align*}
q+7&<-5q-2  \ \ \ \ \ && \ppalkki +5q \\
6q+7&<-2 && \ppalkki -7 \\
6q&<-9 && \ppalkki :6 \\
q&< -\frac{9}{6} = -1\frac{1}{2} \\
\end{align*}
Saatiin siis ehdot $-6\leq q$ ja $q< -1\frac12$, jotka täytyy yhdistää.

\begin{tabular}{cl}
\begin{lukusuora}{-8}{2}{6} \lukusuoravalisa{-6}{}{$-6$}{} \lukusuorapystyviiva{0}{$0$} \end{lukusuora} & $-6\leq q$ \\
\begin{lukusuora}{-8}{2}{6} \lukusuoravaliaa{}{-1.5}{}{$-1\frac12$} \lukusuorapystyviiva{0}{$0$} \end{lukusuora} & $q< -1\frac12$ \\
\begin{lukusuora}{-8}{2}{6} \lukusuoravalisa{-6}{-1.5}{$-6$}{$-1\frac12$} \lukusuorapystyviiva{0}{$0$} \end{lukusuora} & $-6\leq q < -1\frac12$ \\
\end{tabular}

Ehdot ovat yhtä aikaa voimassa, kun $-6\leq q < -1\frac12$.
\end{esimratk}

\begin{esimvast}
 $-6\leq q < -1\frac12$.
\end{esimvast}
\end{esimerkki}