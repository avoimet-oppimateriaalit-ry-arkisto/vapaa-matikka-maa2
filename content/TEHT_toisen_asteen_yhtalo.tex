\begin{tehtavasivu}

\subsubsection*{Opi perusteet}

\begin{tehtava}
    Ratkaise yhtälöt.
    \begin{alakohdat}
        \alakohta{$x^2 = 16$}
        \alakohta{$x^2 = - 16$}
        \alakohta{$x^2 - 13 = 0$}
        \alakohta{$3x^2 - 12 = 0$}

    \end{alakohdat}
    \begin{vastaus}
        \begin{alakohdat}
            \alakohta{$x=\pm 4$}
            \alakohta{Ei ratkaisuja.}
            \alakohta{$x = \pm \sqrt{13}$.}
            \alakohta{$x=\pm 2$}
        \end{alakohdat}
    \end{vastaus}
\end{tehtava}

\begin{tehtava}
    Ratkaise yhtälöt.
    \begin{alakohdat}
        \alakohta{$x(x-3)= 0$}
        \alakohta{$x^2 + 4x = 0$}
        \alakohta{$7x^2-3x = 0$}
    \end{alakohdat}
    \begin{vastaus}
        \begin{alakohdat}
            \alakohta{$x=0$ tai $x=3$}
            \alakohta{$x =0$ tai $x=-4$.}
            \alakohta{$x=0$ tai $x=\frac{3}{7}$}
        \end{alakohdat}
    \end{vastaus}
\end{tehtava}

\begin{tehtava}
    Kirjoita neliöksi tunnistamalla muistikaava
    \begin{alakohdat}
        \alakohta{$x^2 +2x +1$}
        \alakohta{$x^2 +6x +9$}
        \alakohta{$x^2 -4x +4$}
    \end{alakohdat}
    \begin{vastaus}
        \begin{alakohdat}
            \alakohta{$(x+1)^2$}
            \alakohta{$(x+3)^2$.}
            \alakohta{$(x-2)^2$}
        \end{alakohdat}
    \end{vastaus}
\end{tehtava}

\begin{tehtava}
    Ratkaise yhtälö täydentämällä neliöksi
    \begin{alakohdat}
        \alakohta{$x^2 -2x +1 = 4$}
        \alakohta{$x^2 +4x = 5$}
        \alakohta{$x^2 -3x + 10 = 0$}
    \end{alakohdat}
    \begin{vastaus}
        \begin{alakohdat}
            \alakohta{$x = 3$ tai $x= -1$. Neliöksi täydennettynä $(x-1)^2=4$}
            \alakohta{$x = -5$ tai $x = 1$. Neliöksi täydennettynä $(x+2)^2=9$}
            \alakohta{Ei ratkaisua. Neliöksi täydennettynä $(x-3)^2=-1$}
        \end{alakohdat}
    \end{vastaus}
\end{tehtava}

\subsubsection*{Hallitse kokonaisuus}

\begin{tehtava}
    Ratkaise seuraavat yhtälöt.
    \begin{alakohdat}
        \alakohta{$x^2 - 100 = 0$}
        \alakohta{$x^2 + 100 = 0$}
       \alakohta{$x^2 - 10 = 0$}
%        \alakohta{$x^2 + 10 = 0$}
        \alakohta{$-x^2 - 25 = 0$}
%        \alakohta{$-x^2 + 25 = 0$}
        \alakohta{$2x^2 - 98 = 0$}
%        \alakohta{$2x^2 + 98 = 0$}
    \end{alakohdat}
    \begin{vastaus}
        \begin{alakohdat}
            \alakohta{$x=\pm10$}
            \alakohta{Ei ratkaisuja.}
            \alakohta{$x=\pm\sqrt{10}$}
%            \alakohta{Ei ratkaisuja.}
            \alakohta{Ei ratkaisuja.}
%            \alakohta{$x=\pm5$}
            \alakohta{$x=\pm7$}
%            \alakohta{Ei ratkaisuja.}
        \end{alakohdat}
    \end{vastaus}
\end{tehtava}

\begin{tehtava}
    Ratkaise seuraavat yhtälöt.
    \begin{alakohdat}
        \alakohta{$x^2 - 72x = 0$}
%        \alakohta{$x^2 + 72x = 0$}
%        \alakohta{$x^2 - 56x = 0$}
        \alakohta{$x^2 + 56x = 0$}
        \alakohta{$-x^2 - 13x = 0$}
%        \alakohta{$-x^2 + 13x = 0$}
%        \alakohta{$2x^2 - 43x = 0$}
        \alakohta{$2x^2 + 43x = 0$}
    \end{alakohdat}
    \begin{vastaus}
        \begin{alakohdat}
            \alakohta{$x=0$ tai $x=72$}
%            \alakohta{$x=-72$ tai $x=0$}
%            \alakohta{$x=0$ tai $x=56$}
            \alakohta{$x=-56$ tai $x=0$}
            \alakohta{$x=-13$ tai $x=0$}
%            \alakohta{$x=0$ tai $x=13$}
%            \alakohta{$x=0$ tai $x=21,5$}
            \alakohta{$x=-21,5$ tai $x=0$}
        \end{alakohdat}
    \end{vastaus}
\end{tehtava}

\begin{tehtava}
    Ratkaise seuraavat yhtälöt.
    \begin{alakohdat}
        \alakohta{$x^2 - 36 = 0$}
        \alakohta{$x^2 - 85x = 0$}
        \alakohta{$x^2 + 11x = -6x$}
        \alakohta{$x^2 + 10x = -4x^2$}
    \end{alakohdat}
    \begin{vastaus}
        \begin{alakohdat}
            \alakohta{$x=\pm6$}
            \alakohta{$x=0$ tai $x=85$}
            \alakohta{$x=0$ tai $x=-17$}
            \alakohta{$x=0$ tai $x=-2$}
        \end{alakohdat}
    \end{vastaus}
\end{tehtava}

\begin{tehtava}
    Ratkaise yhtälö täydentämällä neliöksi.
    \begin{alakohdat}
        \alakohta{$x^2 -6x +9 = 1$}
        \alakohta{$x^2 +2x +4 = 0$}
        \alakohta{$x^2 -4x - 7 = 0$}
        \alakohta{$x^2 +x = \frac{3}{4}$}
        \alakohta{$2x^2 +3x -2 = 0$}
    \end{alakohdat}
    \begin{vastaus}
        \begin{alakohdat}
            \alakohta{$x = 4$ tai $x= 2$. Neliöksi täydennettynä $(x-3)^2=1$.}
            \alakohta{Ei ratkaisua. Neliöksi täydennettynä $(x+1)^2=-3$}
            \alakohta{$x = 2 \pm \sqrt{11}$ Neliöksi täydennettynä $(x-2)^2=11$.}
            \alakohta{$x = \frac{1}{2}$ tai $x=-1\frac{1}{2}$ Neliöksi täydennettynä $(x+\frac{1}{2})^2=1$.}
            \alakohta{$x = -2$ tai $x = \frac{1}{2}$. 
            Neliöksi täydennettynä $(x+\frac{3}{4})^2=\frac{25}{16}$.}
        \end{alakohdat}
    \end{vastaus}
\end{tehtava}

\begin{tehtava}
    Toisen asteen yhtälön vakiotermi on 4 ja sen ratkaisut ovat 2 ja 3. Mikä yhtälö on kyseessä?

    \begin{vastaus}
		Olkoon yhtälö muotoa $ax^2+bx+4=0$. \\      
      Muodostetaan yhtälöpari:
      \[
        \left\{
          \begin{aligned}
            a\cdot 2^2 + b\cdot 2 + 4 &= 0 \\
            a\cdot 3^2 + b\cdot 3 + 4 &= 0
          \end{aligned}
        \right.
      \]
      
      Yhtälöparin ratkaisuna saadaan $a=\frac23$ ja $b=-3\frac13$. Yhtälö on siis $\frac{2}{3}x^2-3\frac{1}{3}x+4=0$.
      
      Vastauksen voi saada myös ilmaisemalla polynomiyhtälön tekijämuodossa $a(x-2)(x-3)=0$. Tässä tarvitaan kuitenkin juurten ja tekijöiden välistä yhteyttä, joka opetetaan vasta myöhemmin tässä kirjassa.
    \end{vastaus}
\end{tehtava}

\begin{tehtava}
Ollessaan leirikoulussa Lapissa lukiolaisryhmä saapuu järvelle ja havaitsee, että järven halkaisijan suuntainen maiseman poikkileikkaus on likimain paraabelin $\frac{1}{2\,500}x^2-\frac{1}{5}x$ muotoinen, jos $x$-akseli on on vedenpinnan taso ja yksikkönä on metri. Kuinka pitkä matka vastarannalle on?
\begin{vastaus}
$500$ metriä
\end{vastaus}
\end{tehtava}

\subsubsection*{Lisää tehtäviä}

\begin{tehtava}
    Ratkaise yhtälöt käyttämällä tulon nollasääntöä.
    \begin{alakohdat}
        \alakohta{$(x^2-1)(x-7)=0$}
        \alakohta{$(x^2-9)(x^2-16)=0$}
        \alakohta{$(x-4)=x(x-4)$}
    \end{alakohdat}
    \begin{vastaus}
        \begin{alakohdat}
            \alakohta{$x=-1$, $x=1$ tai $x=7$}
            \alakohta{$x=-4$, $x=-3$, $x=3$ tai $x=4$}
            \alakohta{$x=1$ tai $x=4$}
        \end{alakohdat}
    \end{vastaus}
\end{tehtava}

\begin{tehtava}
    Ratkaise seuraavat yhtälöt.
    \begin{alakohdat}
        \alakohta{$x^2 - 9 = 0$}
        \alakohta{$2x^2 + 8 = 0$}
        \alakohta{$-x^2 + 11 = -5$}
        \alakohta{$3 - x^2 = -1 + 3x^2$}
    \end{alakohdat}
    \begin{vastaus}
        \begin{alakohdat}
            \alakohta{$x=\pm3$}
            \alakohta{Ei ratkaisuja}
            \alakohta{$x=\pm4$}
            \alakohta{$x=\pm1$}
        \end{alakohdat}
    \end{vastaus}
\end{tehtava}

\begin{tehtava}
    Ratkaise yhtälöt.
    \begin{alakohdat}
        \alakohta{$x^2(4x^2-1)^2 = 0 $}
        \alakohta{$-x^4(3x-1)^2 = 0$}
    \end{alakohdat}
    \begin{vastaus}
        \begin{alakohdat}
            \alakohta{$x=0$, $x= \frac{1}{2}$ tai $x= -\frac{1}{2}$}
            \alakohta{$x=0$ tai $x= \frac{1}{3}$}
        \end{alakohdat}
    \end{vastaus}
\end{tehtava}

\begin{tehtava}
    Ratkaise seuraavat yhtälöt.
    \begin{alakohdat}
        \alakohta{$x^2 - 3x = 0$}
        \alakohta{$10x + 2x^2 = 0$}
        \alakohta{$2x^2 - x^3 = 0$}
        \alakohta{$-3x^2 + 8x = -2x$}
    \end{alakohdat}
    \begin{vastaus}
        \begin{alakohdat}
            \alakohta{$x=0$ tai $x=3$}
            \alakohta{$x=0$ tai $x=-5$}
            \alakohta{$x=0$ tai $x=2$}
            \alakohta{$x=0$ tai $x=\frac{10}{3}$}
        \end{alakohdat}
    \end{vastaus}
\end{tehtava}

\begin{tehtava}
    Elokuvassa \emph{Dredd} pudotetaan ihmisiä kuolemaan noin $1$ kilometrin korkeudesta. Ennen pudotusta heille annetaan huumausainetta, joka hidastaa aikakäsityksen $1$ prosenttiin normaalista. Vapaassa pudotuksessa pudottu matka ajanhetkellä $t$ on $\frac{1}{2} gt^2$, jossa $g$ on putoamiskiihtyvyytenä tunnettu vakio, jolle voimme tässä hyvin käyttää arviota $g \approx 10\frac{\text{m}}{\text{s}^2}$.
    \begin{alakohdat}
    \alakohta{Olettaen, että huumausaineen vaikutus kestää koko putoamisen ajan, kuinka pitkältä aika pudotuksesta kuolemaan \textbf{uhrista} tuntuu? (Oleta annetut arvot tarkoiksi ja muodosta relevantti toisen asteen yhtälö.)}
    \alakohta{$\star$ Mikä menee fataalisti pieleen, jos a-kohdan laskee suoraan kuvatulla tavalla?}
    \end{alakohdat}
    \begin{vastaus}
        \begin{alakohdat}
            \alakohta{Vastaukseksi saadaan $1\,414\,\text{s} = 23\,\text{min}\,34\,\text{s}$. Käytännössä hyvä vastaustarkkuus voisi olla esimerkiksi $25\,\text{min}$.}
            \alakohta{Tehtävä ei huomioi ilmanvastusta. Ihminen saavuttaa korkeimmillaan rajanopeuden $v_{raja} \approx 55\frac{\text{m}}{\text{s}}$. Tehtävän mallissa putoavan ihmisen nopeus nousee $v_{max} \approx 141\frac{\text{m}}{\text{s}}$ asti. Todellisuudessa putoaminen siis kestää vieläkin kauemmin.}
        \end{alakohdat}
    \end{vastaus}
\end{tehtava}

\begin{tehtava}
%täydentyy kahdeksi neliöksi, joiden summa on 0
    $\star$ Ratkaise $x$ ja $y$ yhtälöstä $y^2+2xy+x^4-3x^2+4=0$.
    \begin{vastaus}
        $x=\sqrt{2}, y=-\sqrt{2}$ tai $x=-\sqrt{2}, y=\sqrt{2}$
    \end{vastaus}
\end{tehtava}

\begin{tehtava} % Kaunis
    $\star$ Ratkaise $x$ ja $y$ yhtälöstä $2x^4+2y^4=4xy-1$. %lisää tai vähennä kiva termi puolittain
    \begin{vastaus}
        $x=\frac{\sqrt{2}}{2}, y=\frac{\sqrt{2}}{2}$ tai $x=-\frac{\sqrt{2}}{2}, y=-\frac{\sqrt{2}}{2}$
    \end{vastaus}
\end{tehtava}

\end{tehtavasivu}