\qrlinkki{http://opetus.tv/maa/maa2/polynomi-tekijoihin-nollakohtien-avulla/}{Opetus.tv: \emph{polynomin jakaminen tekijöihin nollakohtiensa avulla} ($6.48$)}

Polynomin $P(x)=(x-2)(x-3)$ nollakohdat ovat tulon nollasäännön nojalla $x=2$ ja $x=3$. Nollakohdat voi siis nähdä tekijöihin jaetusta polynomista suoraan. Tämä  yhteys toimii myös toisin päin: nollakohtien avulla voi selvittää polynomin tekijät.

Yleisesti on voimassa polynomien jakolause:

\laatikko{\textbf{Polynomien jakolause} \\
Jos $x=b$ on polynomin nollakohta, $x-b$ on polynomin tekijä.}

Polynomien jakolauseen todistus jätetään kurssiin 12.
%Polynomien jakolauseen todistus on hahmoteltu liitteessä
%\ref{tod:poljako}.

\begin{esimerkki}
Jaa tekijöihin polynomi $P(x)=x^2-3x+2$.
\begin{esimratk}
Ratkaistaan ensin polynomin nollakohdat.
\begin{align*}
x^2-3x+2&=0 \\
x&=\frac{-(-3) \pm \sqrt[]{(-3)^2-4 \cdot 1 \cdot 2}}{2 \cdot 1} \\
x&=\frac{3 \pm \sqrt[]{9-8}}{2} \\
x&=\frac{3 \pm 1}{2} \\
x&=1 \textrm{ tai } x = 2.
\end{align*}
Nollakohtien perusteella $(x-1)$ ja $(x-2)$ ovat polynomin $P(x)$ tekijöitä.
Tarkistetaan:
 $(x-1)(x-2)=x^2-2x-x+2= x^2-3x+2$.
\end{esimratk}
\begin{esimvast}
$x^2-3x+2 = (x-1)(x-2)$.
\end{esimvast}
\end{esimerkki}

Ensimmäisen asteen tekijöiden lisäksi saatetaan tarvita vakiokerroin:

\begin{esimerkki}
Jaa polynomi $-2x^2-x+1$ tekijöihinsä.
\begin{esimratk}
Ratkaistaan nollakohdat:
\begin{align*}
-2x^2-x+1&=0 \\
x&=\frac{-(-1) \pm \sqrt[]{(-1)^2-4 \cdot (-2) \cdot 1}}{2 \cdot (-2)} \\
x&=\frac{1 \pm \sqrt[]{1+8}}{-4} \\
x&=\frac{1 \pm 3}{-4} \\
x&=-1 \textrm{ tai } x = \frac{1}{2}.
\end{align*}
Jakolauseen mukaan $(x-\frac{1}{2})$ ja $(x-(-1))$ eli$(x+1)$ ovat kyseisen polynomin tekijöitä.
Ne keskenään kertomalla ei kuitenkaan saada oikeaa tulosta:
$$\left(x-\frac{1}{2}\right)(x+1)=x^2+\frac{1}{2}x-\frac{1}{2}.$$
Puuttuu vielä korkeimman asteen termin kerroin $-2$. Sillä kertomalla saadaan alkuperäinen polynomi:
$-2\left(x-\frac{1}{2}\right)(x+1)=-2x^2-x+1$.
\end{esimratk}
\begin{esimvast}
$-2x^2-x+1 = -2(x-\frac{1}{2})(x+1)$.
\end{esimvast}
\end{esimerkki}

\subsubsection*{Toisen asteen polynomin tekijät}

Polynomien jakolauseen mukaan

\laatikko{Jos toisen asteen polynomin $ax^2+bx+c$ nollakohdat ovat $x_1$ ja $x_2$,
\[ ax^2+bx+c=a(x-x_1)(x-x_2). \]}
Huomaa, että kerroin $a$ on edellisessä yhtälössä kummallakin puolella sama.
(Muuten korkeimman asteen termit eivät täsmää.)

\begin{esimerkki}
Jaetaan tekijöihin $P(x)=2x^2 + 4x-30$. \\
Ratkaistaan nollakohdat yhtälöstä $$2x^2 + 4x-30=0$$ toisen asteen yhtälön ratkaisukaavalla.
Nollakohdat ovat $x_1=3$ ja $x_2=-5$. Saadaan siis
$$P(x)= 2(x-3)(x-(-5)) = 2(x-3)(x+5).$$
\end{esimerkki}

\begin{esimerkki}
Toisen asteen polynomin $P$ nollakohdat ovat $x=1$ ja $x=3$. Lisäksi $P(2)=3$.
Määritä $P(x)$.
\begin{esimratk}
Koska polynomin nollakohdat ovat $x_1=1$ ja $x_2=3$, polynomi on muotoa
\begin{align*} P(x)=a(x-1)(x-3). \end{align*}
Lisäksi tiedetään $P(2)=3$, joten saadaan
\begin{align*}
P(2) = a(2-1)(2-3) &= 3 \\
	a \cdot 1 \cdot (-1) &= 3	&& \ppalkki : (-1) \\
	a = -3.
\end{align*}
\end{esimratk}
\begin{esimvast}
$P(x)=-3(x-1)(x-3)$.
\end{esimvast}
\end{esimerkki}

Jos toisen asteen polynomilla on vain yksi nollakohta, kyseessä on niin sanottu kaksinkertainen juuri. Voidaan tulkita, että nollakohdat $x_1$ ja $x_2$ ovat yhtäsuuret. Tällöin tekijöiksi saadaan $a(x-x_1)(x-x_1)=a(x-x_1)^2$.

\begin{esimerkki}
Jaa tekijöihin $P(x)=2x^2-4x+2$.
\begin{esimratk}
Ratkaistaan nollakohdat:
\begin{align*}
2x^2-4x+2 &= 0	\\
x &= \frac{4\pm \sqrt{(-4)^2-4\cdot 2 \cdot 2}}{2\cdot 2} \\
x &= \frac{4 \pm 0}{4} = 1.
\end{align*}
Yhtälöllä on vain yksi ratkaisu, joten se on kaksoisjuuri.
Polynomi voidaan siis jakaa tekijöihin seuraavasti: \\ $P(x)=2(x-1)(x-1)=2(x-1)^2$. 
\end{esimratk}
\begin{esimvast}
$P(x)=2(x-1)^2$.
\end{esimvast}
\end{esimerkki}

Jos toisen asteen polynomilla ei ole nollakohtia, sitä ei voi jakaa ensimmäisen asteen tekijöihin. (Sillä ensimmäisen asteen tekijällä on aina nollakohta.) Esimerkiksi polynomia $$x^2+4$$ ei voi jakaa tekijöihin. %kompleksiluvuila tämä kyllä onnistuu)