\qrlinkki{http://opetus.tv/maa/maa2/toisen-asteen-polynomifunktio/}{Opetus.tv: \emph{toisen asteen polynomifunktio} ($7.59$)}

Toisen asteen polynomifunktio on muotoa
\begin{align*}
P(x)=ax^2+bx+c,
\end{align*}
missä vakiot $b$ ja $c$ voivat olla mitä tahansa reaalilukuja $(b, \ c \in \rr)$ ja $a$ voi olla mikä tahansa reaaliluku, paitsi luku nolla $(a \in \rr, \ a \neq 0)$.

Toisen asteen polynomifunktion kuvaajaa nimitetään \termi{paraabeli}{paraabeliksi}.

Funktion $P(x)=ax^2+bx+c$ kuvaaja on joka ylös- tai alaspäin aukeava paraabeli. Paraabelin suuntaan vaikuttaa ainoastaan toisen asteen termin kerroin $a$: kun $a>0$, paraabeli aukeaa ylöpäin, ja kun $a<0$, paraabeli aukeaa alaspäin.

\begin{center}
\begin{tabular}{cc}

\begin{tabular}{c}
	\begin{lukusuora}{-2}{2}{4}
	\lukusuoraisobbox
	\lukusuorakuvaaja{x**2-1}
	\end{lukusuora}
	\\ ylöspäin aukeava paraabeli, \\ $a > 0$
\end{tabular}
&
\begin{tabular}{c}
	\begin{lukusuora}{-2}{2}{4}
	\lukusuoraisobbox
	\lukusuorakuvaaja{-x**2+1}
	\end{lukusuora}
	\\ alaspäin aukeava paraabeli, \\ $a < 0$
\end{tabular}

\end{tabular}
\end{center}

Vakiotermi $c$ vaikuttaa kuvaajan korkeuteen.

%FIXME: kuvaajassa x^2+3 ja x^2 menevät päällekäin 
\begin{luoKuva}{paraabelit}
	kuvaaja.pohja(-5, 5, -1, 8, leveys=7)
	
	kuvaaja.piirra("x**2+3", nimi="$x^2+3$")
	kuvaaja.piirra("x**2", nimi="$x^2$")
\end{luoKuva}

\begin{luoKuva}{paraabelit2}
	kuvaaja.pohja(-5, 5, -4, 6, leveys=7)
	
	kuvaaja.piirra("x**2-3*x", nimi="$x^2-3x$")
	kuvaaja.piirra("x**2", nimi="$x^2$")
\end{luoKuva}


%\begin{luoKuva}{vakio}
%	kuvaaja.pohja(-5, 5, -5, 5, leveys=7)
%	
%	kuvaaja.piirra("*x**2+2", nimi="$x^2+2$")
%	kuvaaja.piirra("x**2", nimi="$x^2$")
%\end{luoKuva}

\begin{center}
	\naytaKuva{paraabelit}
\end{center}

Ensimmäisen asteen termin kerroin $b$ vaikuttaa huipun sijaintiin sekä pysty- että vaakasuunnassa.

\begin{center}
	\naytaKuva{paraabelit2}
\end{center}

Tarkempi perustelu edellä esitetyille seikoille löytyy lisämateriaalista, sivulta \pageref{paraabeli_tod}.

%\begin{center}
%	\naytaKuva{vakio}
%\end{center}

%\begin{luoKuva}{ekaaste}
%	kuvaaja.pohja(-5, 5, -5, 5, leveys=7)
%	
%	kuvaaja.piirra("*x**2-3*x", nimi="$x^2-3x$")
%	kuvaaja.piirra("x**2", nimi="$x^2$")
%\end{luoKuva}
%
%\begin{center}
%	\naytaKuva{ekaaste}
%\end{center}