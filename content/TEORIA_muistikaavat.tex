\qrlinkki{http://opetus.tv/maa/maa2/muistikaavat/}{
Opetus.tv: \emph{muistikaavat} ($8.05$, $6.38$ ja $9.08$)}

Joitakin polynomien kertolaskuja tarvitaan niin usein, että niitä kutsutaan \termi{muistikaavat}{muistikaavoiksi}.

\laatikko[Muistikaavat]{
    \begin{tabular}{lrcl}
		& \\        
        {\bf Summan neliö} & $(a+b)^2$ &$=$& $a^2+2ab+b^2$\\
       {\bf Erotuksen neliö} & $(a-b)^2$ &$=$& $a^2-2ab+b^2$ \\
       {\bf Summan ja erotuksen tulo} & $(a+b)(a-b)$ &$=$& $a^2-b^2$ 
    \end{tabular} }

Nämä kaavat voidaan todistaa helposti laskemalla. %GEOMETRISET TODISTUKSET!

\subsubsection*{Summan neliö}

\begin{align*}
(a+b)^2 &= (a+b)(a+b) &\emph{neliön määritelmä} \\
% &= a(a+b)+b(a+b) &\emph{osittelulaki} \\
&= a^2+ab+ba+b^2 &\emph{osittelulaki} \\
&= a^2+ab+ab+b^2 &\emph{vaihdannaisuus ($ba=ab$)} \\
&= a^2+2ab+b^2
\end{align*}

\subsubsection*{Erotuksen neliö}

\begin{align*}
(a-b)^2 &= (a-b)(a-b) &\emph{neliön määritelmä} \\
% &= a(a-b)-b(a-b) &\emph{osittelulaki} \\
&= a^2-ab-ba+b^2 &\emph{osittelulaki} \\
&= a^2-ab-ab+b^2 &\emph{vaihdannaisuus ($ba=ab$)} \\
&= a^2-2ab+b^2
\end{align*}

Edellä todistettuja kahta muistikaavaa kutsutaan yhdessä nimellä \termi{binomin neliö}{binomin neliö}. Toisinaan ne kirjoitetaan yhtenä yhtälönä:
$$(a \pm b)^2=a^2 \pm 2ab+b^2.$$
Kaavoissa useasti esiintyviä $\pm$-merkkejä luetaan siten, että ylemmät ja alemmat täsmäävät keskenään. Ylempien merkkien (kaikki $+$:ia) valinta vastaa siis summan neliötä ja alempien merkkien (kaikki $-$:ia) valinta erotuksen neliötä.

\subsubsection*{Summan ja erotuksen tulo}

\begin{align*}
(a+b)(a-b) &= a^2-ab+ba-b^2 &\emph{osittelulaki} \\
&= a^2-ab+ab-b^2 &\emph{vaihdannaisuus ($ba=ab$)} \\
&= a^2-b^2
\end{align*}

\begin{esimerkki}
Sievennä $(x+5)^2$. \\
\quad\\
Käytetään muistikaavaa $(a+b)^2 = a^2+2ab+b^2$. Nyt $a = x$ ja $b = 5$.
Saadaan
        \[ (x+5)^2 = x^2-2\cdot x\cdot 5+5^2 = x^2+10x+25. \]
\end{esimerkki}

\begin{esimerkki}
Sievennä $(3x-2y)^2$. \\
\quad\\
Käytetään muistikaavaa $(a-b)^2 = a^2-2ab+b^2$. Nyt $a = 3x$ ja $b = 2y$.
Saadaan
        \[ (3x+2y)^2 = (3x)^2-2\cdot 3x\cdot 2y+(2y)^2 = 9x^2-12xy+4y^2. \]
\end{esimerkki}

\begin{esimerkki}
Laske ilman laskinta a) $995^2$ b) $104 \cdot 96$. \\
Käytetään ovelasti muistikaavoja $(a-b)^2 = a^2-2ab+b^2$ ja \mbox{$(a+b)(a-b) = a^2-b^2$}.
\begin{alakohdat}
\alakohta{$995^2 = (1\,000-5)^2 = 1\,000^2-2\cdot 1\,000\cdot 5+5^2 = 1\,000\,000-10\,000+25 = 990\,025 $}
\alakohta{$104\cdot 96 = (100+4)(100-4) = 100^2 - 4^2 = 10\,000 - 16 = 9\,984$.}
\end{alakohdat}
\end{esimerkki}

Huomaa, että kaikki polynomien kertolaskut voidaan sieventää käyttämällä osittelulakia. Muistikaavoja lukuun ottamatta ei ole tarkoitus opetella eri tapauksia erikseen (polynomin kertominen monomilla, polynomin kertominen polynomilla, ...), vaan oppia yleinen käytäntö.
%minne tuo? ^ aiemmaksi? Selvennystä?

%NYT LISÄINFOKSI PASCAL JA NEWTON!