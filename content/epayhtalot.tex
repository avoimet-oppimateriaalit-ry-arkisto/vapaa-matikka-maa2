\section{Epäyhtälöistä yleisesti}
Epäyhtälöllä tarkoitetaan ilmausta, jossa esitetään kahden lausekkeen arvon välinen suuruusjärjestys. Suuruusjärjestyksien esittämiseen käytetään seuraavia merkintöjä:

\begin{center}
\begin{tabular}{l|l}
\emph{Merkintä} & \emph{Merkitys} \\
\hline
$a<b$ &  $a$ on pienempi kuin $b$ \\
$a>b$ & $a$ on suurempi kuin $b$ \\
$a \leq b$ & $a$ on pienempi tai yhtäsuuri kuin $b$ \\
$a \geq b$ & $a$ on suurempi tai yhtäsuuri kuin $b$ \\
$a \neq b$ & $a$ ei ole yhtä suuri kuin $b$ \\
\end{tabular}
\end{center}

%yhtäsuuri vai yhtä suuri?

Sama epäyhtälö voidaan kirjoittaa kahdella tavalla: $a < b$ tarkoittaa samaa kuin $b > a$, ja $a \leq b$ tarkoittaa samaa kuin $b \geq a$. Epäyhtälö $a < b$ pätee täsmälleen silloin, kun epäyhtälö $a \geq b$ ei päde. Kahden luvun välinen epäyhtälöriippuvuus eli -relaatio voidaan esittää myös kielteisesti: $a \nless b $.

Epäyhtälön totuusarvo voi riippua epäyhtälön puolilla esiintyvien muuttujien arvoista. Tämän perusteella epäyhtälöt voidaan jakaa kolmeen tyyppiin:

\laatikko{
\begin{itemize}
\item \emph{Aina tosi} -- pätee kaikilla muuttujien arvoilla. Esimerkiksi epäyhtälöt $5 < 6$ tai $x + 1 \leq x + 3$ pätevät riippumatta muuttujan $x$ arvosta.
\item \emph{Ehdollisesti tosi} -- pätee vain joillain muuttujien arvoilla. Esimerkiksi epäyhtälö $x < 5$ pätee, kun $x = 4$, mutta ei päde, kun $x = 7$.
\item \emph{Aina epätosi} -- ei päde millään muuttujien arvoilla. Esimerkiksi epäyhtälöt $3 < 1$ ja $x < x$ eivät päde koskaan.
\end{itemize}
}

\subsection*{Epäyhtälöiden muokkaaminen}

Kuten yhtälöiden ratkaisemisessa, epäyhtälön ratkaisemisessa selvitetään ne muuttujien arvot, joilla epäyhtälö on tosi. Kuten yhtälöitä, myös epäyhtälöitä voidaan ratkaista muokkaamalla niitä sellaisilla operaatioilla, joilla muokattu epäyhtälö on yhtäpitävä alkuperäisen kanssa.

Kahden luvun kasvattaminen saman verran siirtää lukuja lukusuoralla, mutta säilyttää niiden keskinäisen järjestyksen:

\begin{kuva}
lukusuora.pohja(-1, 10, 11.5, n = 2)
lukusuora.kohta(0, "$0$", 0)

with vari("red"):
	lukusuora.nuoli(2, 2+4, 1, 2)
	lukusuora.nuoli(3, 3+4, 1, 2)

lukusuora.piste(2, "$2$", 1)
lukusuora.piste(3, "$3$", 1)

lukusuora.piste(2+4, "$2\!+\!4$", 2)
lukusuora.piste(3+4, "$3\!+\!4$", 2)
\end{kuva}

Tämän perusteella epäyhtälö, joka saadaan lisäämällä epäyhtälön molemmille puolille sama lauseke, on yhtäpitävä alkuperäisen epäyhtälön kanssa.

\begin{esimerkki}
Luvun lisääminen epäyhtälöön.
  \begin{align*}
     3 - x &< 5 - x && \ppalkki +x\\
     3 -x +x &< 5 -x +x \\
     3 &< 5 && \textrm{tosi}
  \end{align*}
\end{esimerkki}

Myös lukujen kertominen samalla positiivisella kertoimella säilyttää niiden keskinäisen järjestyksen. Jos kerroin on pienempi kuin yksi, luvut lähenevät toisiaan:

\begin{kuva}
lukusuora.pohja(-1, 10, 11.5, n = 2)
lukusuora.kohta(0, "$0$", 0)

with vari("red"):
	lukusuora.nuoli(2, 1, 1, 2)
	lukusuora.nuoli(4, 2, 1, 2)

lukusuora.piste(2, "$2$", 1)
lukusuora.piste(4, "$4$", 1)

lukusuora.piste(1, r"$2 \cdot \frac{1}{2}$", 2)
lukusuora.piste(2, r"$4 \cdot \frac{1}{2}$", 2)
\end{kuva}

Jos kerroin on suurempi kuin yksi, luvut etääntyvät toisistaan:

\begin{kuva}
lukusuora.pohja(-1, 10, 11.5, n = 2)
lukusuora.kohta(0, "$0$", 0)

with vari("red"):
	lukusuora.nuoli(2, 4, 1, 2)
	lukusuora.nuoli(2*2, 4*2, 1, 2)

lukusuora.piste(2, "$2$", 1)
lukusuora.piste(4, "$4$", 1)

lukusuora.piste(2*2, r"$2 \cdot 2$", 2)
lukusuora.piste(4*2, r"$4 \cdot 2$", 2)
\end{kuva}

Luvulla jakaminen on sama asia kuin jakajan käänteisluvulla kertominen, joten positiivisella luvulla jakaminen säilyttää järjestyksen kertolaskun tavoin. Näin ollen alkuperäisen epäyhtälön kanssa yhtäpitävä epäyhtälö saadaan kertomalla tai jakamalla molemmat puolet positiivisella luvulla.

\begin{esimerkki}
Epäyhtälön kertominen lukua yksi pienemmällä positiivisella luvulla.
\begin{align*}
     2 &< 4 && \ppalkki \cdot \frac{1}{2} \\
   2\cdot\frac{1}{2} &< 4\cdot\frac{1}{2}  \\
     1 &< 2 && \textrm{tosi}
\end{align*}
\end{esimerkki}

\begin{esimerkki}
Epäyhtälön kertominen lukua yksi suuremmalla luvulla.
\begin{align*}
     2 &< 4 && \ppalkki \cdot 2 \\
   2\cdot 2 &< 4\cdot 2  \\
     4 &< 8 && \textrm{tosi}
\end{align*}
\end{esimerkki}

Sen sijaan negatiivisella luvulla kertominen ei säilytä suuruusjärjestystä. Esimerkiksi kun lukuja 2 ja 5 kerrotaan luvulla $-1$, niiden suuruusjärjestys kääntyy:

\begin{kuva}
lukusuora.pohja(-6, 6, 11.5, n = 2)
lukusuora.kohta(0, "$0$", 0)

with vari("red"):
	lukusuora.nuoli(2, -2, 1, 2)
	lukusuora.nuoli(5, -5, 1, 2)

lukusuora.piste(2, "$2$", 1)
lukusuora.piste(5, "$5$", 1)

lukusuora.piste(-2, r"$2 \cdot (-1)$", 2)
lukusuora.piste(-5, r"$5 \cdot (-1)$", 2)
\end{kuva}

Jos epäyhtälöä kerrotaan tai jaetaan negatiivisella luvulla, epäyhtälömerkin suunta täytyy kääntää, jotta saataisiin yhtäpitävä epäyhtälö.
%: esimerkissä $2 < 5$ muuttuu muotoon $-2 > -5$.

\begin{esimerkki}
Epäyhtälön kertominen negatiivisella luvulla.
\begin{align*}
     2 &< 5 && \ppalkki \cdot (-1) \\
   2\cdot (-1) &> 4\cdot (-1)  \\
     -2 &> -5 && \textrm{tosi}
\end{align*}
\end{esimerkki}

\laatikko{Epäyhtälöstä saadaan yhtäpitävä epäyhtälö
\begin{itemize}
\item lisäämällä molemmille puolille sama lauseke,
\item kertomalla tai jakamalla molemmat puolet samalla positiivisella luvulla tai
\item kertomalla tai jakamalla molemmat puolet samalla negatiivisella luvulla ja kääntämällä epäyhtälömerkin suunta.
\end{itemize}
}
%tarkennus/laajennus! ^

Kuten yhtälöiden tapauksessa, epäyhtälön kertominen puolittain nollalla ei tuota yhtäpitävää epäyhtälöä, sillä esimerkiksi epäyhtälöstä $a \leq b$ tulee $0 \leq 0$, joka on aina tosi, ja epäyhtälöstä $a < b$ tulee $0 < 0$, joka on aina epätosi.

\begin{esimerkki}
Muokataan epäyhtälöä $-2x+4<6$ käyttämällä esitettyjä operaatioita.
\begin{align*}
-2x+4&<6 && \ppalkki -4 \\
-2x&<2 && \ppalkki :(-2) \\
x&>-1
\end{align*}
Tehdyt operaatiot tuottavat yhtäpitäviä epäyhtälöitä, joten epäyhtälö $-2x+4<6$ on yhtäpitävä epäyhtälön $x>-1$ kanssa. Voidaan päätellä, että lukua $-1$ suuremmat luvut ovat täsmälleen epäyhtälön ratkaisut.
\end{esimerkki}

\subsection*{Reaalilukuvälit}

Ratkaistaessa yhtälöitä ratkaisuksi saadaan yleensä pieni joukko lukuja. Epäyhtälöiden tapauksessa on tyypillistä, että ratkaisu on \termi{väli}{väli}, eli kaikki kahden luvun väliset luvut.

Reaalilukuvälejä merkitään usein laittamalla välin ala- ja ylärajat hakasulkujen sisään.  Esimerkiksi $[a, b]$ tarkoittaa lukuja $x$ jotka toteuttavat kaksoisepäyhtälön $a \leq x \leq b$. Mikäli ala- tai yläraja ei kuulu väliin, vastaava hakasulku käännetään. Esimerkiksi $[2,5[$ tarkoittaa lukuja $x$, joille pätee $2 \leq x < 5$.

Väliä kutsutaan \termi{suljettu väli}{suljetuksi väliksi}, mikäli ala- ja yläraja kuuluvat väliin, ja \termi{avoin väli}{avoimeksi väliksi}, mikäli ala- ja yläraja eivät kuulu väliin. Jos vain toinen rajoista kuuluu väliin, väli on \termi{puoliavoin väli}{puoliavoin}.

Väli voidaan piirtää lukusuoralle kahden luvun välisenä janana. Päätepisteet merkitään täytetyllä ympyrällä, mikäli luku kuuluu väliin, ja muuten tyhjällä ympyrällä. Esimerkiksi väli $[a, b[$ piirretään seuraavasti:

\begin{kuva}
lukusuora.pohja(0, 10, 8)
lukusuora.vali(2, 8, True, False, "$a$", "$b$")
\end{kuva}

Jos halutaan, että väli ei ole alhaalta tai ylhäältä rajoitettu, merkitään rajaksi $-\infty$ tai $\infty$. Koska ääretön ei ole reaaliluku eikä näin ollen kuulu väliin, on sitä vastaava hakasulku käännettävä, ja siten esimerkiksi väli $]{-\infty}, a[$ on avoin. 

Seuraavaan taulukkoon on koottu reaalilukuvälien olennainen käsitteistö ja merkinnät.

\begin{luoKuva}{vali1}
lukusuora.pohja(-5, 7, 3, varaa_tila = False)
lukusuora.kohta(0, "$0$")
lukusuora.vali(-3, 5, False, False, "$-3$", "$5$")
\end{luoKuva}
\begin{luoKuva}{vali2}
lukusuora.pohja(-5, 7, 3, varaa_tila = False)
lukusuora.kohta(0, "$0$")
lukusuora.vali(-3, 5, False, True, "$-3$", "$5$")
\end{luoKuva}
\begin{luoKuva}{vali3}
lukusuora.pohja(-5, 7, 3, varaa_tila = False)
lukusuora.kohta(0, "$0$")
lukusuora.vali(-3, 5, True, False, "$-3$", "$5$")
\end{luoKuva}
\begin{luoKuva}{vali4}
lukusuora.pohja(-5, 7, 3, varaa_tila = False)
lukusuora.kohta(0, "$0$")
lukusuora.vali(-3, 5, True, True, "$-3$", "$5$")
\end{luoKuva}
\begin{luoKuva}{vali5}
lukusuora.pohja(-5, 7, 3, varaa_tila = False)
lukusuora.kohta(0, "$0$")
lukusuora.vali(-3, None, True, False, "$-3$", "$5$")
\end{luoKuva}
\begin{luoKuva}{vali6}
lukusuora.pohja(-5, 7, 3, varaa_tila = False)
lukusuora.kohta(0, "$0$")
lukusuora.vali(-3, None, False, False, "$-3$", "$5$")
\end{luoKuva}
\begin{luoKuva}{vali7}
lukusuora.pohja(-5, 7, 3, varaa_tila = False)
lukusuora.kohta(0, "$0$")
lukusuora.vali(None, 5, False, True, "$-3$", "$5$")
\end{luoKuva}
\begin{luoKuva}{vali8}
lukusuora.pohja(-5, 7, 3, varaa_tila = False)
lukusuora.kohta(0, "$0$")
lukusuora.vali(None, 5, False, False, "$-3$", "$5$")
\end{luoKuva}

\begin{tabular}{|p{2.0cm}|p{2.0cm}|c|c|}
\hline
Epäyhtälö\-merkintä & Joukko-opillinen merkintä & Esitys lukusuoralla & Välin nimitys \\
\hline
 $-3<x<5$ & $x \in {]-3, 5[}$ & \naytaKuva{vali1} & Avoin väli  \\
\hline
 $-3<x \leq 5$ & $x \in {]-3, 5]}$ & \naytaKuva{vali2} & Puoliavoin väli  \\
\hline
 $-3\leq x < 5$ & $x \in {[-3, 5[}$ & \naytaKuva{vali3} & Puoliavoin väli  \\
\hline
$-3\leq x \leq 5$ & $x \in {[-3, 5]}$ & \naytaKuva{vali4} & Suljettu väli \\
\hline
$-3\leq x$ & $x \in {[-3, \infty[}$ & \naytaKuva{vali5} & Puoliavoin väli  \\
\hline
 $-3<x$ & ${x \in {]-3, \infty[}}$ & \naytaKuva{vali6} & Avoin väli \\
\hline
$x \leq 5$ & $x \in {]{-\infty}, 5]}$ & \naytaKuva{vali7} & Puoliavoin väli  \\
\hline
$x < 5$ & $x \in {]{-\infty}, 5[}$ & \naytaKuva{vali8} & Avoin väli  \\
\hline
\end{tabular}

\begin{esimerkki}
  a) Epäyhtälö $2<x<10$ vaatii, että $x$ saa arvoja kahden ja kymmenen väliltä, mutta se ei koskaan saa täsmälleen näitä reuna-arvoja. Kyseessä on avoin väli kahdesta kymmeneen, $]2,10[$. Annetulle epäyhtälölle yhtäpitävä ilmaisu on $x \in ]2,10[$. \\
 b) Epäyhtälö $0\leq y \leq 2$ rajaa muuttujan $y$ välille suljetulle välille $[0,2]$. Väli on suljettu, koska $y$ voi myös saada täsmälleen arvot $0$ ja $2$. \\
 c) Joskus kirjallisuudessa näkee äärettömyyssymbolin käyttöä myös kaksoisepäyhtälöissä, esimerkiksi $3<x<\infty $, mutta ilmaistaan yleisemmin muodossa $x \in ]3,\infty[$. Kyseessä on avoin väli. \\
 d) Epäyhtälöt $-100<k\leq 0$ ja $u\leq 90$ ovat puoliavoimia välejä, koska ne rajaavat muuttujan yhtäsuuruuden avulla vain toiselta puolelta. \\
 e) Kaksoisepäyhtälö $\frac{1}{5}\geq x>-\sqrt{3}$ tarkoittaa samaa kuin kaksoisepäyhtälö $-\sqrt{3}<x\leq \frac{1}{5}$. Kaksoisepäyhtälö vaatii, että muuttujalle $x$ pätee erikseen sekä epäyhtälö $-\sqrt{3}<x$ että $x\leq \frac{1}{5}$.
\end{esimerkki}

\begin{esimerkki}
Merkitse väli \quad a) $[-2,4]$ \quad b) $[4,5[$ \quad c) $]6,\infty[$ \quad
epäyhtälömerkintänä. Onko väli suljettu, avoin vai puoliavoin?
\begin{esimratk}
\begin{alakohdat}
\alakohta{Kun $x\in [-2,4]$, pätee $-2\leq x \leq 4$. Väli on suljettu, koska päätepisteet kuluvat väliin.}
\alakohta{Kun $x\in [4,5[$, pätee $4 \leq x < 5$. Väli on puoliavoin.}
\alakohta{Kun $x\in ]6,\infty[$, pätee $6<x$. Väli on avoin, koska kumpikaan päätepiste ei kuulu väliin.}
\end{alakohdat}
\end{esimratk}
\end{esimerkki}

Joskus, erityisesti englanninkielisissä matematiikan teksteissä, käytetään avoimen välin merkintänä käännetyn hakasulkeen sijasta tavallista kaarisuljetta.

\begin{esimerkki}
Avoin väli $]1,9[$ kirjoitetaan joskus $(1,9)$.
Puoliavoin väli $]3,4]$ kirjoitetaan joskus $(3,4]$.
\end{esimerkki}

%\newpage

\begin{tehtavasivu}

\begin{tehtava}
Ovatko seuraavat väitteet tosi vai epätosia?
	\begin{alakohdat}
		\alakohta{$2<4$}
		\alakohta{$4>2$}
		\alakohta{$-1\leq 10$}
		\alakohta{$3\geq 3$}
		\alakohta{$0,666 \cdot 6>4$}
	\end{alakohdat}
	\begin{vastaus}
		\begin{alakohdat}
		\alakohta{tosi}
		\alakohta{tosi}
		\alakohta{tosi}
		\alakohta{tosi}
		\alakohta{epätosi}
		\end{alakohdat}	
	\end{vastaus}	
\end{tehtava}

\begin{tehtava}
Ovatko seuraavat väitteet tosi, epätosia vai ehdollisesti tosia?
	\begin{alakohdat}
		\alakohta{$10^4<10^3$}
		\alakohta{$3^{-1}>3^{-2}$}
		\alakohta{$2^{10}\leq 10^3$}
		\alakohta{$x\leq y$}
		\alakohta{$x\geq x$}
		\alakohta{$x\geq x+1$}
	\end{alakohdat}
	\begin{vastaus}
	\begin{alakohdat}
		\alakohta{epätosi}
		\alakohta{tosi}
		\alakohta{epätosi}
		\alakohta{ehdollisesti tosi}
		\alakohta{ehdollisesti tosi}
		\alakohta{epätosi, mikään luku ei voi olla suurempi kuin luku$+1$.}
	\end{alakohdat}	
	\end{vastaus}
	
\end{tehtava}

\begin{tehtava}
Mitkä seuraavista väitteistä ovat tosia?
	\begin{alakohdat}
		\alakohta{$\pi=3,14$}		
		\alakohta{$\pi \approx 3,14$}
		\alakohta{$\pi \neq 3,14$}
		\alakohta{$\pi>3,14$}
		\alakohta{$\pi<3,14$}
		\alakohta{$\pi\leq 3,14$}
		\alakohta{$\pi\geq 3,14$}
	\end{alakohdat}
	\begin{vastaus}
	\begin{alakohdat}
		\alakohta{epätosi}	
		\alakohta{tosi (mutta subjektiivista)}
		\alakohta{tosi}
		\alakohta{tosi}
		\alakohta{epätosi}
		\alakohta{epätosi}
		\alakohta{tosi}
	\end{alakohdat}
	\end{vastaus}
\end{tehtava}

\begin{tehtava}
    Esitä joukko-opillisilla merkinnöillä ja lukusuoralla.
    \begin{alakohdat}
		\alakohta{$-10 < x < 10$}
        \alakohta{$-9<x \leq 7$}
        \alakohta{$5\leq c$}
        \alakohta{$5\leq s \leq 7\frac{1}{2}$}
        \alakohta{$5\geq x>1$}
    \end{alakohdat}
    \begin{vastaus}
        \begin{alakohdat}
			\alakohta{$x \in \aavali{-10}{10}$ \\  \naytaKuva{teht1}}
            \alakohta{$x \in \asvali{-9}{7}$ \\  \naytaKuva{teht2}}
            \alakohta{$c \in \savali{5}{\infty}$ \\  \naytaKuva{teht3}}
            \alakohta{$s \in \ssvali{5}{7\frac{1}{2}}$ \\ , \naytaKuva{teht4}}
            \alakohta{$x \in \asvali{1}{5}$ \\  \naytaKuva{teht5}}
        \end{alakohdat}
    \end{vastaus}
\end{tehtava}

\begin{tehtava}
Kuuluvatko luvut välille $[0,1;0,92]$?
	\begin{alakohdat}
		\alakohta{$0$}
		\alakohta{$\frac{1}{2}$}
		\alakohta{$\frac{\sqrt{2}}{2}$}
		\alakohta{$1$}		
		\alakohta{$0,1$}
		\alakohta{$0,92$}
		\alakohta{$0,09$}
		\alakohta{$0,9199$}
		\alakohta{$0,10$}
	\end{alakohdat}
	\begin{vastaus}	
		\begin{alakohdat}
		\alakohta{ei kuulu}
		\alakohta{kuuluu}
		\alakohta{kuuluu}
		\alakohta{ei kuulu}
		\alakohta{kuuluu}
		\alakohta{kuuluu}
		\alakohta{ei kuuluu}
		\alakohta{kuuluu}
		\alakohta{kuuluu}
	\end{alakohdat}
	\end{vastaus}
\end{tehtava}

%Pitääkö epäyhtälö paikkansa, kun k=... ... ...  ...

\begin{tehtava}
Monissa arkisissa ilmauksissa esitetään jonkinlaisia rajoja. Tarkoittavatko kyseiset ilmaisut aidon epäyhtälön tilannetta ($<$ tai $>$) vai sitä, että myös reuna-arvot ovat sallittuja ($\leq$ tai $\geq$)?
	\begin{alakohdat}
		\alakohta{"Lasten ateriat $12$-vuotiaaksi asti"}
		\alakohta{"Kauppa on auki jälleen kuun ensimmäisestä päivästä lähtien."}
		\alakohta{"Loton kierros on auki klo 20 asti."}
		\alakohta{"jäseniksi käyvät korkeintaan $29$-vuotiaat"}		
		\alakohta{"Lukekaa ensi tunnille kirjaa Potenssi-lukuun asti."}
	\end{alakohdat}
	\begin{vastaus}
		\begin{alakohdat}
		\alakohta{Yhtäsuuruus on mukana, eli $12$-vuotiaatkin saavat lasten aterian.}
		\alakohta{Kauppa on auki myös kuun ensimmäisenä päivänä, eli yhtäsuuruus on mukana.}
		\alakohta{Pelata voi vielä, kun kun kello on 19.59. Klo 20.00 on jo myöhäistä, joten ilmaisun ei käsitetä sisältävän reuna-arvoaan.}
		\alakohta{Yhtäsuuruus on mukana, myös $29$-vuotiaat kelpaavat jäseniksi.}
		\alakohta{Epäselvää. Se, pitääkö kotitehtävänä lukea myös Potenssi-luku, riippuu opettajasta.}
		\end{alakohdat}
	\end{vastaus}
\end{tehtava}
	
\subsubsection*{Hallitse kokonaisuus}
	
\begin{tehtava}
Säilyykö epäyhtälö totena, kun se korotetaan puolittain toiseen potenssiin?
	\begin{alakohdat}
		\alakohta{$0\leq 2$}
		\alakohta{$-2<2$}
		\alakohta{$3>-1$}
		\alakohta{$5\geq-10$}
		\alakohta{$1 \neq 2$}		
	\end{alakohdat}
	\begin{vastaus}	
	\begin{alakohdat}
		\alakohta{kyllä}
		\alakohta{ei}
		\alakohta{kyllä}
		\alakohta{ei}
		\alakohta{kyllä}
	\end{alakohdat}
	\end{vastaus}
\end{tehtava}

\begin{tehtava}
Säilyykö epäyhtälö totena, kun siitä otetaan puolittain neliöjuuri?
	\begin{alakohdat}
		\alakohta{$0\leq 2$}
		\alakohta{$1<2$}
		\alakohta{$999<1\,000$}
		\alakohta{$50\geq 25$}
		\alakohta{$1 \neq 2$}		
	\end{alakohdat}
	\begin{vastaus}	
		\begin{alakohdat}
		\alakohta{kyllä}
		\alakohta{kyllä}
		\alakohta{kyllä}
		\alakohta{kyllä}
		\alakohta{kyllä}
	\end{alakohdat}
	\end{vastaus}
\end{tehtava}	
	
\begin{tehtava}
Säilyykö epäyhtälö totena, kun epäyhtälön molemmille puolille käytetään 10-kantaista eksponenttifunktiota? (Eli epäyhtälön puolet sijoitetaan eksponenttifunktion muuttujan paikalle.)
	\begin{alakohdat}
		\alakohta{$0\leq 2$}
		\alakohta{$-2<2$}
		\alakohta{$3>-1$}
		\alakohta{$5\geq-10$}
		\alakohta{$1 \neq 2$}		
	\end{alakohdat}
	\begin{vastaus}	
		\begin{alakohdat}
		\alakohta{kyllä}
		\alakohta{kyllä}
		\alakohta{kyllä}
		\alakohta{kyllä}
		\alakohta{kyllä}
		\end{alakohdat}
	\end{vastaus}
\end{tehtava}	
	
	\begin{tehtava}
Piirrä funktion $P$ kuvaaja, kun funktion arvot määritellään kaavalla $P(x)=x^3-x$, ja funktion määrittelyjoukko on
		\begin{alakohdat}
		\alakohta{$[0,2]$}
		\alakohta{$[-2,2]$}
		\alakohta{$]-\infty, 0]$}
		\alakohta{$[-3,-2] \cup [2,3]$}
		%PUUTTUU VASTAUS
		\end{alakohdat}
\end{tehtava}
	
\subsubsection*{Lisää tehtäviä}

\begin{tehtava}
$\star$ Osoita, että $m^2+n^2 \geq mn$ kaikilla kokonaisluvuilla $m,n$.
		\begin{vastaus}
Käsitellään erilaiset tilanteet kolmessa eri osassa: 		
		Jos joko $m$ tai $n$ on nolla, niin $0^2+n^2 \geq 0\cdot n$ eli $n^2\geq 0$, mikä pitää paikkansa kaikilla kokonaisluvuille $n$. Symmetrian vuoksi sama pätee myös tapaukselle $m=0$. Jos sekä $m$ että $n$ ovat nollia, saadaan $0\geq 0$, mikä on myös tosi epäyhtälö. \\
		Jos $m$ ja $n$ ovat erimerkkiset, niin epäyhtälön vasenpuoli on edelleen neliöönkorotuksien vuoksi positiivinen. Epäyhtälön oikea puoli on kertolaskun merkkisäännön perusteella varmasti negatiivinen. Koska kaikki negatiiviset luvut ovat kaikkia positiivisia lukuja pienempiä, epäyhtälö pätee tässä tapauksessa. \\
		Viimeinen tapaus: $m$ ja $n$ on samanmerkkiset. Epäyhtälön $m^2+n^2 \geq mn$ vasen puoli muistuttaa binomin neliötä. Täydennetään se vähentämällä molemmilta puolilta epäyhtälöä lauseke $2mn$, jolloin saadaan: $m^2+n^2-2mn \geq mn-2mn$ eli $m^2-2mn+n^2 \geq -mn$, minkä voi kirjoittaa binomin neliön kaavalla muodossa $(m-n)^2 \geq -mn$. Neliöönkorotus varmistaa, että epäyhtälön vasen puoli on aina epänegatiivinen. Koska oletuksen mukaan $m$ ja $n$ ovat samanmerkkiset, tulo $-mn$ on negatiivinen. Jälleen epäyhtälö pitää paikkansa, ja kaikki mahdolliset $m$:n ja $n$:n yhdistelmät on käsitelty. 
		\end{vastaus}
\end{tehtava}

%parannettava ratkaisua!
\begin{tehtava}
$\star$ Tilapäivitys sosiaalisessa mediassa sanoo: "Vaimollani, alle $30$ vuotta, oli naurussa pitelemistä, kun hänelle valkeni, että täytän tänä vuonna $35$. Nauran takaisin, kun hän täyttää $35$ ja itse olen... yli neljäkymmentä." Pitääkö päättely välttämättä paikkansa?
		\begin{vastaus}
		Merkataan vaimon ikää $v$:llä ja puolisonsa ikää $p$:llä. Näiden avulla voimme kirjoittaa kaksoisepäyhtälön $v<30<p$. Huomataan myös, että $p=34$. Alle kolmekymppinen täyttää 35 vuotta aikaisintaan kuuden vuoden kuluttua. Lisätään kaksoisepäyhtälöön puolittain $+6$, jolloin saadaan $v+6<36<p+6$. Koska $p=34$, niin $v+6<36<40$, eli puolisonsa on välttämättäkin täyttänyt $40$ vuotta.
		\end{vastaus}
\end{tehtava}

\begin{tehtava}
$\star$ Osoita, että $x^2+\frac{1}{x^2}\geq 2$, kun $x \neq 0$.
    \begin{vastaus}
     Aloita tiedosta $\left(x-\frac{1}{x}\right)^2 \geq 0$ ja sievennä.
    \end{vastaus}
\end{tehtava}

\begin{tehtava} 
$\star$ Osoita, että kun $a \geq 0$ ja $b \geq 0$, pätee $\frac{a+b}{2} \geq \sqrt{ab}$. Milloin yhtäsuuruus on voimassa?
    \begin{vastaus}
     Opastus: Aloita tiedosta $\left(\sqrt{a}-\sqrt{b}\right)^2 \geq 0$ ja sievennä. Yhtäsuuruus pätee, kun $a = b$.
    \end{vastaus}
\end{tehtava}

\begin{tehtava}
$\star$ Merkitse suuruusjärjestys luvuille $10_9$ ja $9_{10}$.
%miten kantaluvun suurentaminen vaikuttaa luvun suuruuteen? 
	\begin{vastaus}
	$10_9=9_{10}$
	\end{vastaus}
\end{tehtava}

\end{tehtavasivu}

\newpage

\section{Ensimmäisen asteen epäyhtälö}

\qrlinkki{http://opetus.tv/maa/maa2/ensimmaisen-asteen-epayhtalo/}{Opetus.tv: \emph{ensimmäisen asteen epäyhtälö} (14:55 ja 8:21)}

%Harjoittelemme nyt erityisesti 1. asteen epäyhtälöiden ratkaisemista -- toisen asteen ja sitä korkeampien polynomiepäyhtälöiden ratkaisemista käsitellään toisen asteen yhtälön käsittelyn jälkeen.

Samoin kuin yhtälöiden kohdalla, epäyhtälö pyritään muuttamaan niin yksinkertaiseen muotoon kuin mahdollista, jotta yksinkertaisesta tilanteesta nähdään välittömästi, mitkä luvut kuuluvat ratkaisuun ja mitkä eivät. Tuntemattomat pyritään yhdistämään, ja epäyhtälöä muokataan niin, että tuntematon saadaan yksin omalle puolelleen yhtälöä.

\begin{esimerkki}
Ratkaise epäyhtälö $2x+1 < 0$.
\begin{esimratk}
\begin{align*}
2x+1 &< 0 && \ppalkki -1 \\
2x &< -1 && \ppalkki :2 \\
x &< -\frac{1}{2}
\end{align*}
\end{esimratk}
\begin{esimvast}
$x < -\frac{1}{2}$. 
Ratkaisu voidaan esittää myös muodossa $x \in ]-\infty, -\frac{1}{2}[$.
\end{esimvast}

Ratkaisua voidaan tulkita graafisesti tutkimalla lausekkeeseen $2x+1$ liittyvää kuvaajaa:

\begin{kuva}
kuvaaja.pohja(-2, 2, -1, 3, korkeus = 5, nimiX = "$x$")
kuvaaja.piirra("2*x+1", nimi = "$f(x) = 2x + 1$", kohta = 0.75)
\end{kuva}

Alkuperäinen epäyhtälö $2x+1<0$ vaatii, että lausekkeen $2x+1$ arvo on negatiivinen. Yhtälön ratkaisu on mahdollista nähdä katsomalla kuvasta, millä kaikilla $x$:n arvoilla funktion $2x+1$ kuvaaja laskee vaaka-akselin alapuolelle. Tällöin funktio, siis toisaalta lauseke $2x+1$, saa negatiivisia arvoja.
\end{esimerkki}

Ensimmäisen asteen epäyhtälö ratkaistaan muuten kuin yhtälö, mutta kun kerrotaan tai jaetaan negatiivisella luvulla, epäyhtälömerkki kääntyy.

\begin{esimerkki}
Ratkaise epäyhtälö $x < 5x-8$.
\begin{esimratk}
\begin{align*}
x &< 5x-8 && \ppalkki -5x \\
-4x &< -8 && \ppalkki :(-4), \ \ 
\text{merkki kääntyy, } -4 < 0 \\
x &> 2 &&
\end{align*}
\end{esimratk}
\begin{esimvast}
$x > 2$.
\end{esimvast}
\end{esimerkki}

% fixme tyhjää tilaa esimerkkien välissä

\begin{esimerkki} Millä $w$:n arvoilla pätee
$-8w-(8-w) \geq \frac12 w+5$?
\begin{esimratk}
\begin{align*}
-8w-(8-w) &\geq \frac12 w+5 \\
-8w-8+w &\geq \frac12 w+5 \\
-7w-8 &\geq \frac12 w+5  \ \ \ \ \ && \ppalkki -\frac12 w \\
-\frac{15}{2} w-8 &\geq 5  \ \ \ \ \ && \ppalkki +8 \\
-\frac{15}{2} w &\geq 13  \ \ \ \ \ && \ppalkki :(-\frac{15}{2}) \\
w &\leq 13:(-\frac{15}{2}) \\
w &\leq -13\cdot \frac{2}{15} \\
w &\leq -\frac{26}{15} \\
\end{align*}
\end{esimratk}
\begin{esimvast}
$w \leq -\frac{26}{15}$
\end{esimvast}
\end{esimerkki}

\begin{esimerkki}
Ratkaise kaksoisepäyhtälö $1\leq q+7<-5q-2$.
\begin{esimratk}
Ratkaistavana on itse asiassa kaksi epäyhtälöä: $1\leq q+7$ ja $q+7<-5q-2$. Halutaan siis löytää ne $q$:n arvot, joilla molemmat epäyhtälöt pätevät.
\begin{align*}
1&\leq q+7 \ \ \ \ \ && \ppalkki -7 \\
-6&\leq q
\end{align*}
Vastaavasti toiselle yhtälölle:
\begin{align*}
q+7&<-5q-2  \ \ \ \ \ && \ppalkki +5q \\
6q+7&<-2 && \ppalkki -7 \\
6q&<-9 && \ppalkki :6 \\
q&< -\frac{9}{6} = -1\frac{1}{2} \\
\end{align*}
Saatiin siis ehdot $-6\leq q$ ja $q< -1\frac12$, jotka täytyy yhdistää.

\begin{tabular}{cl}
\begin{lukusuora}{-8}{2}{6} \lukusuoravalisa{-6}{}{$-6$}{} \lukusuorapystyviiva{0}{$0$} \end{lukusuora} & $-6\leq q$ \\
\begin{lukusuora}{-8}{2}{6} \lukusuoravaliaa{}{-1.5}{}{$-1\frac12$} \lukusuorapystyviiva{0}{$0$} \end{lukusuora} & $q< -1\frac12$ \\
\begin{lukusuora}{-8}{2}{6} \lukusuoravalisa{-6}{-1.5}{$-6$}{$-1\frac12$} \lukusuorapystyviiva{0}{$0$} \end{lukusuora} & $-6\leq q < -1\frac12$ \\
\end{tabular}

Ehdot ovat yhtä aikaa voimassa, kun $-6\leq q < -1\frac12$.
\end{esimratk}

\begin{esimvast}
 $-6\leq q < -1\frac12$.
\end{esimvast}
\end{esimerkki}

\begin{tehtavasivu}

\paragraph*{Opi perusteet}

\begin{luoKuva}{teht1}
lukusuora.pohja(-12, 12, 3, varaa_tila = False)
lukusuora.kohta(0, "$0$")
lukusuora.vali(-10, 10, False, False, "$-10$", "$10$")
\end{luoKuva}

\begin{luoKuva}{teht2}
lukusuora.pohja(-11, 9, 3, varaa_tila = False)
lukusuora.kohta(0, "$0$")
lukusuora.vali(-9, 7, False, True, "$-9$", "$7$")
\end{luoKuva}

\begin{luoKuva}{teht3}
lukusuora.pohja(-2, 12, 3, varaa_tila = False)
lukusuora.kohta(0, "$0$")
lukusuora.vali(5, None, True, False, "$5$", "$turha$")
\end{luoKuva}

\begin{luoKuva}{teht4}
lukusuora.pohja(-1, 9, 3, varaa_tila = False)
lukusuora.kohta(0, "$0$")
lukusuora.vali(5, 7.5, True, True, "$5$", "$7,5$")
\end{luoKuva}

\begin{luoKuva}{teht5}
lukusuora.pohja(-1, 7, 3, varaa_tila = False)
lukusuora.kohta(0, "$0$")
lukusuora.vali(1, 5, False, True, "$1$", "$5$")
\end{luoKuva}

\begin{tehtava}
    Ratkaise seuraavat epäyhtälöt.
    \begin{alakohdat}
        \alakohta{$3x+6<4x$}
        \alakohta{$3x-6<2x+57$}
        \alakohta{$5y-2<12$}
        \alakohta{$3\leq y+9$}
        \alakohta{$z-5\geq-888$}
		\alakohta{$2z+5\leq 42z-995$}
    \end{alakohdat}
    \begin{vastaus}
        \begin{alakohdat}
            \alakohta{$x>6$}
            \alakohta{$x<63$}
            \alakohta{$y<2,8$}
            \alakohta{$y\geq -6$}
            \alakohta{$z\geq -883$}
			\alakohta{$z\geq 25$}
        \end{alakohdat}
    \end{vastaus}
\end{tehtava}

\begin{tehtava}
Maalipurkki sisältää $10$ litraa maalia. Maalin riittoisuus on noin $6\,\text{m}^2/\text{l}$. Talon ulkoseinän korkeus on $4,5$\,m. Ulkoseinälle tulevat laudat on maalattava kahteen kertaan. Riittääkö maali, jos maalattavan seinän pituus on
	\begin{alakohdat}
		\alakohta{$5$\,m}
		\alakohta{$10$\,m}
		\alakohta{Kuinka pitkälle seinälle yhden purkillisen sisältämä maali riittää?}
	\end{alakohdat}
	\begin{vastaus}
		\begin{alakohdat}
			\alakohta{riittää ($22,5~\text{m}^2 < 30~\text{m}^2$)}
			\alakohta{ei riitä ($45~\text{m}^2 > 30~\text{m}^2$)}
			\alakohta{noin $6,7$\,m seinälle}
		\end{alakohdat}

	\end{vastaus}
\end{tehtava}

\paragraph*{Hallitse kokonaisuus}

\begin{tehtava}
    Ratkaise seuraavat yhtälöt tai epäyhtälöt.
    \begin{alakohdat}
        \alakohta{$-2r+6=0$}
        \alakohta{$-2r+6\leq 0$}
        \alakohta{$5y-2<y+6$}
        \alakohta{$8(x+2)\geq -5(5-x)+3$}
        \alakohta{$\frac{x+3}{2}+\frac{-2x+1}{3}>\frac{x-9}{4}$}
    \end{alakohdat}
    \begin{vastaus}
        \begin{alakohdat}
            \alakohta{$r=3$}
            \alakohta{$r\geq 3$}
            \alakohta{$y<2$}
            \alakohta{$x=-12\frac{2}{3}$}
            \alakohta{$x<9\frac{4}{5}$}
        \end{alakohdat}
    \end{vastaus}
\end{tehtava}

\begin{tehtava}
	Tietyn auton käyttövoimavero on $450$\,€/vuosi, ja keskimääräinen kulutus dieselöljyä käytettäessä on $5$ litraa/$100$\,km. Saman valmistajan vastaava bensiinikäyttöinen auto kuluttaa $8$ litraa/$100$\,km. Diesel maksaa $1,55$\,€/litra, ja bensiini maksaa $1,65$\,€/litra. Kun vain annetut tiedot huomioidaan, niin kuinka paljon esimerkin dieselajoneuvolla tulee vähintään ajaa vuodessa, jotta se on edullisempi? Autojen ostohintoja ei huomioida.
    \begin{vastaus}
        $8\,257$\,km
    \end{vastaus}
\end{tehtava}

\begin{tehtava}
    Ratkaise epäyhtälöt.
    \begin{alakohdat}
        \alakohta{$3x+6<2x\leq 9-x$}
        \alakohta{$3x+6<2x\leq 1+3x$}
    \end{alakohdat}
    \begin{vastaus}
        \begin{alakohdat}
            \alakohta{$x<-6$}
            \alakohta{ei ratkaisua}
        \end{alakohdat}
    \end{vastaus}
\end{tehtava}

\begin{tehtava}
	Millä $x$:n arvoilla luvut $2x - 5$, $-x$ ja $x + 4$ ovat erisuuria ja $2x - 5$ on luvuista
	\begin{alakohdat}
		\alakohta{suurin}
		\alakohta{toiseksi suurin}
		\alakohta{pienin?}
	\end{alakohdat}
	\begin{vastaus}
		\begin{alakohdat}
			\alakohta{$x > 9$}
			\alakohta{$\frac{5}{3} < x < 9$}
			\alakohta{$x < \frac{5}{3}$}
		\end{alakohdat}
	\end{vastaus}
\end{tehtava}

\paragraph*{Lisää tehtäviä}

\begin{tehtava}
    Ratkaise seuraavat epäyhtälöt.
    \begin{alakohdat}
        \alakohta{$33x+2\geq 27x+6$}
        \alakohta{$3x-6\geq 4x-6$}
        \alakohta{$5y+5\geq 15$}
        \alakohta{$3y+2\geq 2y-1$}
        \alakohta{$z\geq 2z+1\,000$}
		\alakohta{$z-1\geq z+1$}
    \end{alakohdat}
    \begin{vastaus}
        \begin{alakohdat}
            \alakohta{$x\geq \frac{2}{3}$}
            \alakohta{$x\leq 0$}
            \alakohta{$y\geq 2$}
            \alakohta{$y\geq -3$}
            \alakohta{$z\leq -1000$}
			\alakohta{ei ratkaisuja}
        \end{alakohdat}
    \end{vastaus}
\end{tehtava}

\begin{tehtava}
Lukion päättötodistuksessa aineen arvosana määräytyy aineen pakollisten ja syventävien kurssien keskiarvosta pyöristettynä kokonaisluvuksi tavallisten sääntöjen mukaan. Opiskelija haluaa filosofian päättöarvosanakseen $7$ tai paremman. Opiskelija aikoo osallistua kolmelle filosofian kurssille. Kahden kurssin jälkeen hänen arvosanojensa keskiarvo on $6$. Mikä arvosana on opiskelijan vähintään saatava kolmannesta kurssista? Kurssit arvioidaan asteikolla 
$4$--$10$.
\begin{vastaus}
Muodostettava epäyhtälö on muotoa $\frac{2\cdot 6+x}{3}\geq 6.5$, josta ratkaisuna saadaan $x\geq7.5$. Opiskelija tarvitsee siis vähintään arvosanan $8$.
\end{vastaus}
\end{tehtava}

\begin{tehtava}
Ratkaise kaksoisepäyhtälö
\[ 2x-1 < x \leq 3+5x.  \]
    \begin{vastaus}
        $-\frac{3}{4} \leq x < 1$
    \end{vastaus}
\end{tehtava}

\end{tehtavasivu}
