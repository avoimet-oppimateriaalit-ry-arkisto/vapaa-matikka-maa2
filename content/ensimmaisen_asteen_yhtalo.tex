%% encoding: utf-8
\section{Kertausta: Ensimmäisen asteen yhtälö}

Yhtälöistä yksinkertaisin on ensimmäisen asteen yhtälö. Käydään se lyhyesti
läpi kertauksen vuoksi.

Ensimmäisen asteen yhtälössä ratkaistavana on vain mahdollisesti vakiolla
kerrottu muuttuja. Yhtälön ratkaisemiseksi riittää neljää
peruslaskutoimitusta: yhteen-, vähennys-, kerto ja jakolasku. 
Yhtälöä muokataan tekemällä yhtälön kummallekin puolelle sama laskutoimitus.

%Aluksi yhtälön molemmille puolille 
%lisätään tai vähennetään jokin luku, niin että 
%vasemmalle puolelle saadaan jäämään pelkkä vakiolla kerrottu muuttuja.
%Sen jälkeen jaetaan yhtälön molemmat puolet muuttujan kertoimella, jolloin
%yhtälön ratkaisu jää oikealle puolelle.

% fixme termiä juuri ei vielä esitellä


\begin{esimerkki}
Ratkaise yhtälö $4x + 5 = 2 + 2x$.

\begin{esimratk}
\begin{align*}
    4x + 5 &= 2 + 2x && \ppalkki -2x \\
    2x + 5 &= 2      && \ppalkki -5 \\
        2x &= -3     && \ppalkki :2 \\
         x &= -\frac{3}{2}
 \end{align*}
\end{esimratk}
\begin{esimvast}
$x=-\frac{3}{2}$
\end{esimvast}
\end{esimerkki}

\begin{esimerkki}
Ratkaise yhtälö $3x - 6 = 0$.

\begin{esimratk}
  \begin{align*}
    3x - 6 &= 0 && \ppalkki +6 \\
        3x &= 6 && \ppalkki :3 \\
         x &= \frac{6}{3}  = 2 \\
  \end{align*}
\end{esimratk}
\begin{esimvast}
$x=2$
\end{esimvast}
\end{esimerkki}

\begin{esimerkki}
Ratkaise yhtälö $\dfrac{x}{2}-\dfrac{x+3}{2}=3$.

\begin{esimratk}
  \begin{align*}
    \frac{x}{2}-\frac{x+10}{2}&=5 && \ppalkki \cdot 2 \\
        x-(x+10) &= 10  \\
         x-x-10 &= 10 \\
         -10 &= 10 && \text{ ristiriita, ei ratkaisua.}
  \end{align*}
\end{esimratk}
\begin{esimvast}
ei ratkaisua
\end{esimvast}
\end{esimerkki}

Yleisesti ensimmäisen asteen yhtälö on muotoa
\[
    ax + b = 0.
\]
Kaikki 1. asteen yhtälöt voidaan muokata tähän yleiseen
muotoon siirtämällä kaikki termit yhtälön vasemmalle puolelle ja
sieventämällä.

Yleisen yhtälön $ax + b = 0$ ratkaisu on

\begin{align*}
	 ax + b &= 0  && \ppalkki -b\\
	 ax &= -b  && \ppalkki : a \ (\neq 0)\\
  x &= -\frac{b}{a}.
\end{align*}

%Erityisesti kannattaa huomata, että kaikkia 1. asteen yhtälöitä ei tarvitse
%saattaa yleiseen muotoon. Esimerkiksi jos vakiotermit ovat valmiiksi
%oikealla puolella, yhtälön ratkaisemiseksi riittää luonnollisesti jakaa
%molemmat puolet muuttujan kertoimella.

\begin{tehtavasivu}

\paragraph*{Opi perusteet}

\begin{tehtava}
    Ratkaise yhtälöt.
    \begin{alakohdat}
        \alakohta{$x + 5 = 47$}
        \alakohta{$2x = 64$}
        \alakohta{$3x - 5 = 16$}
    \end{alakohdat}
    \begin{vastaus}
        \begin{alakohdat}
            \alakohta{$x = 42$}
            \alakohta{$x = 32$}
            \alakohta{$x = 7$}
        \end{alakohdat}
    \end{vastaus}
\end{tehtava}

\begin{tehtava}
    Ratkaise yhtälöt.
    \begin{alakohdat}
        \alakohta{$x + 8 = 2x - 1$}
        \alakohta{$2x + 4 = 60$}
        \alakohta{$3x - 5 = -x + 11$}
    \end{alakohdat}
    \begin{vastaus}
        \begin{alakohdat}
            \alakohta{$x = 9$}
            \alakohta{$x = 28$}
            \alakohta{$x = 4$}
        \end{alakohdat}
    \end{vastaus}
\end{tehtava}

\begin{tehtava}
    Antero pitää hauskaa keksimällä luvun mielikuvituksessaan. Hän kirjoittaa
    luvun mielikuvituspaperille vaaleanpunaisella mielikuvituskynällä. Tämän jälkeen hän
    kertoo luvun silmiensä lukumäärällä ja vähentää siitä varpaidensa lukumäärän. Antero
    on innoissaan, sillä hän saa tulokseksi luvun 8, joka on hänen ikänsä vuosina. Selvitä
    yhälönratkaisulla Anteron keksimä luku, kun oletetaan, että hänellä on yhtä monta silmää
    ja varvasta kuin hänen ikätovereillaan yleensä on.
    \begin{vastaus}
        $2x-10=8 \Leftrightarrow 2x=18 \Leftrightarrow x=9$. Vastaus: Antero keksi luvun 9.
    \end{vastaus}
\end{tehtava}

\begin{tehtava}
    Kun eräs luku kerrotaan kolmella, ja siihen sen jälkeen lisätään viisi, saadaan tulokseksi puolet alkuperäisestä luvusta.
    Kirjoita yhtälö ja selvitä kyseinen luku.
    \begin{vastaus}
        $3x+5=\frac12x$, $x=-2$.
    \end{vastaus}
\end{tehtava}

%ei ole yhtälötehtävä, mutta mallinnusharjoituksena ok? 
\begin{tehtava}
    Muodosta tilannetta kuvaavat lausekkeet.
    \begin{alakohdat}
        \alakohta{Kuinka paljon maksaa hilavitkuttimen vuokraus $x$ tunniksi, kun vuokra on 42 \euro /tunti. Vuokraajan tulee myös ottaa pakollinen 25 euron laitteistovakuutus.}
        \alakohta{Kuinka monta euroa saa $x$ dollarilla, kun 1~EUR vastaa 1,23~USD:a, ja halutaan vaihtaa dollareita euroiksi. Valuutanvaihtaja veloittaa lisäksi palvelumaksun 0,50 euroa.}
    \end{alakohdat}
    \begin{vastaus}
        \begin{alakohdat}
            \alakohta{$42x + 25$}
            \alakohta{$\frac{1}{1{,}23}x + 0{,}5$}
        \end{alakohdat}
    \end{vastaus}
\end{tehtava}



\paragraph*{Hallitse kokonaisuus}

\begin{tehtava}
    Ratkaise yhtälöt.
    \begin{alakohdat}
        \alakohta{$3(x+7)=7x$}
        \alakohta{$2(3x-1)=-7x $}
        \alakohta{$3-2x-(4-x)=2 $}
    \end{alakohdat}
    \begin{vastaus}
        \begin{alakohdat}
            \alakohta{$x = \frac{7}{6} =1\frac{1}{6} $}
            \alakohta{$x = \frac{2}{13}$}
            \alakohta{$x = -3$}
        \end{alakohdat}
    \end{vastaus}
\end{tehtava}

\begin{tehtava}
    Ratkaise yhtälöt.
    \begin{alakohdat}
        \alakohta{$-2\cdot\frac{x-5}{3}-\frac{5}{7}(1-x)=5x+3$}
        \alakohta{$\frac{4x-5}{3}-\frac{3}{2}(x-8)=-\frac{x+5}{6}$}
        \alakohta{$3(x-3)+x=4x-9$}
    \end{alakohdat}
    \begin{vastaus}
        \begin{alakohdat}
            \alakohta{$x = -\frac{1}{13}$}
            \alakohta{ei ratkaisuja}
            \alakohta{yhtälö on toteutuu kaikilla reaaliluvuilla}
        \end{alakohdat}
    \end{vastaus}
\end{tehtava}

%vaatii Pythagoraan lauseen, jota ei vielä ole käsitelty lukiossa.
\begin{tehtava}
    Tässä tehtävässä pitäisi muistaa peruskoulussa käsitelty Pythagoraan lause.
    Suorakulmaisen kolmion sivujen pituuden kateettien pituudet ovat $x+1$ ja $4$. Hypotenuusan pituus $x+3$. Mikä $x$ on?
    \begin{vastaus}
		$x=2$
    \end{vastaus}
\end{tehtava}

%hankala
\begin{tehtava}
    Määritä sekunnin tarkkuudella se ajanhetki, kun kellotaulun minuutti- ja tuntiviisarit ovat päällekkäin ensimmäisen kerran klo 12.00:n jälkeen.
    \begin{vastaus}
		Kello $13.05.27$
    \end{vastaus}
\end{tehtava}

\end{tehtavasivu}

% tämä alla oleva omaksi filukseen! ja näihin ratkaisut! ja lisää ?!?

\section{Testaa tietosi!}

\subsection*{Osaatko selittää?}
\begin{enumerate}
 \item Miksi epäyhtälön suunta muuttuu kun sitä kerrotaan negatiivisella luvulla?
\end{enumerate}

\subsection*{Kertauskysymyksiä}
\begin{enumerate}
 \item Pysyykö epäyhtälö yhtäpitävänä, jos
a) lisäämällä kummallekkin puolelle sama lauseke
b) kertomalla kummatkin puolet samalla luvulla
c) korotetaan kummatkin puolet potenssiin n
d) kerrotaan negatiivisellä luvulla ja käännetään epäyhtälön suunta.
\end{enumerate}