\begin{tehtavasivu}

\subsubsection*{Opi perusteet}

\begin{tehtava}
\begin{kuva}
	kuvaaja.pohja(-2,4.5,-2,8,5,nimiX="$x$",nimiY="$P(x)$")
	kuvaaja.piirra("x**2-4*x+3")
\end{kuva}
Kuvassa on polynomifunktion $P$ kuvaajaa. Määritä on kuvaajan perusteella
\alakohdat{
§ $P(-1)$
§ $P(1)$
§ $P(2)$
§ polynomin $P$ nollakohdat
§ millä $x$:n arvoilla $P(x)=3$.
}
\begin{vastaus}
\alakohdat{
§ $8$
§ $0$
§ $-1$
§ Nollakohdat ovat $x=1$ ja $x=3$.
§ Kun $x= 0$ tai $x=4$.
}
\end{vastaus}
\end{tehtava}

\begin{tehtava}
    Hahmottele polynomien kuvaajat koordinaatistoon käsin. Laske testipisteitä tarpeen mukaan. Tarkista tietokoneella tai graafisella laskimella.
    \alakohdat{
        § $P(x) = x-2$
        § $Q(x) = 4-x^2$
        § $R(x) = \frac{1}{4}x^3$
    }   
    \begin{vastaus}
    	\alakohdat{
    	§  \begin{kuvaajapohja}{0.4}{-4}{4}{-4}{4}
				\kuvaaja{2*x-2}{}{red}
			  \end{kuvaajapohja}
    	§  \begin{kuvaajapohja}{0.4}{-4}{4}{-3}{5}
				\kuvaaja{4-x**2}{}{red}
			  \end{kuvaajapohja}
		§  \begin{kuvaajapohja}{0.4}{-4}{4}{-4}{4}
				\kuvaaja{0.25*x**3}{}{red}
			  \end{kuvaajapohja}
		}
    \end{vastaus}
\end{tehtava}

\subsubsection*{Hallitse kokonaisuus}

\begin{tehtava}
Piirrä polynomifunktion kuvaaja ja päättele sen avulla polynomin nollakohdat. Käytä graafista laskinta, tietokonetta tai piirrä käsin.
	\alakohdat{
	§ $P(x)=x^3+x^2+x+1$
	§ $Q(x)=x^2-6x+2$
	}
	\begin{vastaus}
	\alakohdat{
	§ $x=-1$
	§ $x \approx 0,35$ ja $x \approx 5,65$
	}
	\end{vastaus}
\end{tehtava}

\begin{tehtava}
Piirrä funktion $f$ kuvaaja, kun funktion arvot määritellään kaavalla $f(x)=1-x^2$, ja funktion määrittelyjoukko on
		\alakohdat{
		§ $\mathbb{N}$
		§ $\mathbb{Z}$
		§ $\mathbb{R}$
		§ $\lbrace 0,1,2\rbrace$
	}		
		%PUUTTUU VASTAUS
		
\end{tehtava}

\subsubsection*{Lisää tehtäviä}

\begin{tehtava}
    Piirrä polynomien kuvaajat käsin.
    \alakohdat{
        § $x+4$
        § $2x-3$
        § $5$
        § $x^2+x-2$
%        § $x^3-x+3$
    }   
    \begin{vastaus}
    	\alakohdat{
    	§  \begin{kuvaajapohja}{0.4}{-4}{4}{-1}{7}
				\kuvaaja{x+4}{}{red}
			  \end{kuvaajapohja}
    	§  \begin{kuvaajapohja}{0.4}{-4}{4}{-5}{3}
				\kuvaaja{2*x-3}{}{red}
			  \end{kuvaajapohja}
    	§  \begin{kuvaajapohja}{0.4}{-4}{4}{-1}{7}
				\kuvaaja{5}{}{red}
			  \end{kuvaajapohja}
		§  \begin{kuvaajapohja}{0.4}{-4}{4}{-3}{5}
				\kuvaaja{x**2+x-2}{}{red}
			  \end{kuvaajapohja}
%		§  \begin{kuvaajapohja}{0.4}{-4}{4}{-3}{5
%				\kuvaaja{x**3-x+2}{}{red}
%			  \end{kuvaajapohja}}
		}
    \end{vastaus}
\end{tehtava}


\begin{tehtava}
    Piirrä polynomien kuvaajat tietokoneella tai laskimella.
    \alakohdat{
        § $x^2-6x+3$
        § $\frac{1}{5}x^3-x^2+2$
        § $-0,5x^4+0,25x^3+2,5x^2-1$
    }
    \begin{vastaus}
    	\alakohdat{
		§  \begin{kuvaajapohja}{0.4}{-1}{7}{-7}{5}
				\kuvaaja{x**2-6*x+3}{}{red}
			  \end{kuvaajapohja}
    	§  \begin{kuvaajapohja}{0.4}{-2.5}{5.5}{-4}{4}
				\kuvaaja{0.2*x**3-x**2+2}{}{red}
			  \end{kuvaajapohja}
		§  \begin{kuvaajapohja}{0.4}{-3}{3}{-4}{4}
				\kuvaaja{-0.5*x**4+0.25*x**3+2.5*x**2-1}{}{red}
			  \end{kuvaajapohja}
		}
    \end{vastaus}
\end{tehtava}

%\begin{tehtava}
%    Piirrä polynomien kuvaajat.
%    \alakohdat{
%        § $x+4$
%        § $2x-9$
%        § $5x+2$
%        § $6x+1$
%    }
%    \begin{vastaus}
%    	\alakohdat{
%        §  \begin{kuvaajapohja}{0.4}{-4}{4}{-1}{7
%				\kuvaaja{x+4}{}{red}
%			  \end{kuvaajapohja}}
%    	§  \begin{kuvaajapohja}{0.4}{-2}{6}{-6}{2
%				\kuvaaja{2*x-9}{}{red}
%			  \end{kuvaajapohja}}
%		§  \begin{kuvaajapohja}{0.4}{-4}{4}{-2}{6
%				\kuvaaja{5*x+2}{}{red}
%			  \end{kuvaajapohja}}
%		§  \begin{kuvaajapohja}{0.4}{-4}{4}{-2}{6
%				\kuvaaja{6*x+1}{}{red}
%			  \end{kuvaajapohja}}
%		}
%    \end{vastaus}
%\end{tehtava}
%
%\begin{tehtava}
%    Piirrä polynomien kuvaajat.
%    \alakohdat{
%        § $x^2-1$
%        § $2x^2$
%        § $4x^2+4$
%        § $x^2-6x+3$
%    }
%    \begin{vastaus}
%    	\alakohdat{
%        §  \begin{kuvaajapohja}{0.4}{-4}{4}{-2}{6
%				\kuvaaja{x+4}{}{red}
%			  \end{kuvaajapohja}}
%    	§  \begin{kuvaajapohja}{0.4}{-4}{4}{-1}{7
%				\kuvaaja{2*x-9}{}{red}
%			  \end{kuvaajapohja}}
%		§  \begin{kuvaajapohja}{0.4}{-4}{4}{-1}{11
%				\kuvaaja{5*x+2}{}{red}
%			  \end{kuvaajapohja}}
%		§  \begin{kuvaajapohja}{0.4}{-1}{7}{-7}{5
%				\kuvaaja{6*x+1}{}{red}
%			  \end{kuvaajapohja}}
%		}
%    \end{vastaus}
%\end{tehtava}

\begin{tehtava} $\star$
	Monia funktioita voidaan esittää likimääräisesti polynomeina (ns. Taylorin polynomi). Esimerkiksi (kun $-1<x<1$)

	\begin{tabular}{lcll}
	$\frac{1}{1+x^2}$ &$\approx$ & $1-x^2+x^4-x^6+x^8-x^{10}$ \\
	$\sqrt{1+x}$ & $\approx $ & $ 1+\frac{x}{2}
	-\frac{x^2}{8}+\frac{x^3}{16}-\frac{5x^4}{128}$
	\end{tabular}

	Piirrä alkuperäinen funktio ja polynomi samaan kuvaajaan tietokoneella tai graafisella laskimella. Kokeile, kuinka polynomin viimeisten termien pois jättäminen vaikuttaa tarkkuuteen. Mitä havaitset? (Termejä voi laskea lisääkin, mutta siihen ei puututa tässä.)
	\begin{vastaus}
		Mitä enemmän termejä, sitä parempi vastaavuus.
	\end{vastaus}
\end{tehtava}

\end{tehtavasivu}