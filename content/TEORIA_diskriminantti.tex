\qrlinkki{http://opetus.tv/maa/maa2/diskriminantti/}{Opetus.tv: \emph{diskriminantti 2. asteen yhtälölle} (7:56 ja 8:30)}

\begin{esimerkki}
    Ratkaistaan toisen asteen yhtälö $3x^2-5x+10=0$.\\
    
%    \begin{align*}
%        \underbrace{3}_{=a}x^2\underbrace{-5}_{=b}x+\underbrace{10}_{=c}=0
%    \end{align*}
    Sijoitetaan $a=3$, $b=-5$ ja $c=10$ toisen asteen yhtälön ratkaisukaavaan $x=\frac{-b \pm \sqrt[]{b^2-4ac}}{2a}$.
    \begin{align*}
        x &=\frac{-(-5) \pm \sqrt[]{(-5)^2-4\cdot 3 \cdot 10}}{2 \cdot 3} \\
        x &=-\frac{5 \pm \sqrt[]{25-120}}{6} \\
          x &=-\frac{5 \pm \sqrt[]{-95}}{6} 
    \end{align*}
    Koska juurrettava on negatiivinen, yhtälöllä ei ole ratkaisuja.
\end{esimerkki}

Edellisen esimerkin tulokseen päästäisiin nopeamminkin. Koska ratkaisukaavassa esiintyvä lauseke $b^2-4ac$ on negatiivinen, ratkaisua ei ole.

Lauseketta $b^2-4ac$ kutsutaan \termi{diskriminantti}{diskriminantiksi}. Sen arvo kertoo yhtälön ratkaisujen lukumäärän. Jos $D<0$, ratkaisuja ei ole, sillä negatiivisella luvulla ei ole neliöjuurta. Jos $D=0$, neliöjuuren arvoksikin tulee $0$ ja ratkaisuja saadaan vain yksi ($\pm 0 = 0$). Jos $D>0$, neliöjuuri saa positiivisen arvon ja ratkaisuja on kaksi.

Diskriminantin avulla voidaan siis tutkia yhtälön ratkaisujen lukumäärää ilman että yhtälöä tarvitsee ratkaista.

\newpage
\laatikko{Toisen asteen yhtälön $ax^2+bx+c=0$ ratkaisujen lukumäärä voidaan laskea diskriminantin $D=b^2-4ac$ avulla seuraavasti:
\begin{itemize}
\item
Jos $D<0$, yhtälöllä ei ole reaalisia ratkaisuja.
\item
Jos $D=0$, yhtälöllä on tasan yksi reaalinen ratkaisu.
\item
Jos $D>0$, yhtälöllä on kaksi erisuurta reaaliratkaisua.
\end{itemize}
}
Tapauksessa $D=0$ yhtälön ainoaa ratkaisua kutsutaan sen \termi{kaksoisjuuri}{kaksoisjuureksi}.

\begin{esimerkki}
\ \\
\parbox{4.5cm}{
\begin{kuvaajapohja}{1}{-1}{3}{-1}{3}
  \kuvaaja{2*x**2-2*x+1}{}{blue}
\end{kuvaajapohja}
}
\parbox{6cm}{$2x^2-2x+1=0$:\\$D=(-2)^2-4 \cdot 2 \cdot 1=4-8=-4$, eli $D <0$. Ei reaalisia ratkaisuja.}
\\
\parbox{4.5cm}{
\begin{kuvaajapohja}{1}{-1}{3}{-1}{3}
  \kuvaaja{x**2-2*x+1}{}{blue}
\end{kuvaajapohja}
}
\parbox{6cm}{$x^2-2x+1=0$:\\$D=(-2)^2-4 \cdot 1 \cdot 1=4-4=0$, eli $D = 0$. Yksi reaaliratkaisu.}
\\
\parbox{4.5cm}{
\begin{kuvaajapohja}{1}{-1}{3}{-2}{2}
  \kuvaaja{2*x**2-4*x+1}{}{blue}
\end{kuvaajapohja}
}
\parbox{6cm}{$2x^2-4x+1=0$:\\$D=(-4)^2-4 \cdot 2 \cdot 1=16-8=8$, eli $D > 0$. Kaksi eri reaaliratkaisua.}
\end{esimerkki}

\begin{esimerkki}
Selvitetään, onko yhtälöllä $x^2+x+2=0$ ratkaisuja.

Tutkitaan diskriminanttia.
\[D=1^2-4\cdot 1 \cdot 2 = 1-8 = -7\]
Koska $D<0$, yhtälöllä ei ole ratkaisuja.

Jos yhtälön ratkaisemista yrittäisi ratkaisukaavan avulla, tulisi neliöjuuren alle negatiivinen luku.
\end{esimerkki}

\begin{esimerkki} %olihan tästä epäyhtälöesimerkkejä myöhemmin?
Millä $k$:n arvolla yhtälöllä $9x^2+kx+1$ on tasan yksi ratkaisu?

Jotta ratkaisuja olisi tasan yksi, on diskriminantin oltava $0$.
\begin{align*}
D &= 0\\
k^2-4\cdot 9\cdot 1 &= 0\\
k^2-36 &= 0\\
k^2 &= 36\\
k &= \pm 6
\end{align*}
Yhtälöllä on täsmälleen yksi ratkaisu, kun $k=-6$ tai $k=6$.
\end{esimerkki}

%\begin{esimerkki}
%Perustele, kuinka monta reaalista nollakohtaa polynomilla $-\frac{1}{2}y^3-y^2-2y$ on.
%	\begin{esimratk}
%Yksi nollakohta ($y=0$), joka saadaan jakamalla polynomi tekijöihin. Toisen asteen polynomitekijällä ei ole diskriminantin perusteella reaalisia nollakohtia.
%	\end{esimratk}
%\end{esimerkki}