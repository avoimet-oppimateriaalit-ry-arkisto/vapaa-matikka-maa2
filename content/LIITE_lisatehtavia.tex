\begin{tehtavasivu}

\subsection*{Polynomi}

\begin{tehtava} 
Laske.
		\begin{alakohdat}
		\alakohta{$x+x+x$  }
		\alakohta{$x\cdot x \cdot x$  }
		\alakohta{$-5x^2-5x^2$  }
		\alakohta{$2x\cdot 7x$  }
		\alakohta{$\frac{1}{2}x^2\cdot(-6x^3)$  }						
		\end{alakohdat}
    \begin{vastaus}
		\begin{alakohdat}
		\alakohta{$3x$}
		\alakohta{$x^3$}
		\alakohta{$-10x^2$}
		\alakohta{$14x^2$}
		\alakohta{$-3x^5$}						
		\end{alakohdat}
    \end{vastaus}
\end{tehtava}

\begin{tehtava} 
Laske.
		\begin{alakohdat}
		\alakohta{$-2x^3+4x-x^2-x+5x^3$  }
		\alakohta{$x-3-(5-x)$  }
		\alakohta{$(x^2+5x+2)-(-x^2+3x-5)$  }
		\alakohta{$3x(x-7)$  }
		\alakohta{$-2x^2(3x^5-12x^2+1)$  }						
		\end{alakohdat}
    \begin{vastaus}
		\begin{alakohdat}
		\alakohta{$3x^3-x^2+3x$}
		\alakohta{$2x-8$}
		\alakohta{$2x^2+2x+7$}
		\alakohta{$3x^2-21x$}
		\alakohta{$-6x^7+24x^4-2x^2$}						
		\end{alakohdat}
    \end{vastaus}
\end{tehtava}

\begin{tehtava} 
Kerro sulut auki.
		\begin{alakohdat}
		\alakohta{$(x-2)(x+4)$  }
		\alakohta{$(3a+b)(a-b)$  }
		\alakohta{$(a+3)^2$  }
		\alakohta{$(x-1)^2$  }
		\alakohta{$(4x+1)(4x-1)$  }						
		\end{alakohdat}
    \begin{vastaus}
		\begin{alakohdat}
		\alakohta{$x^2+2x-8$}
		\alakohta{$3a^2-2ab-b^2$}
		\alakohta{$a^2+6a+9$}
		\alakohta{$x^2-2x+1$}
		\alakohta{$16x^2-1$}						
		\end{alakohdat}
    \end{vastaus}
\end{tehtava}

\begin{tehtava} 
Jaa tekijöihin.
		\begin{alakohdat}
		\alakohta{$4x^2+x$  }
		\alakohta{$5x^3y+10xy^2$  }
		\alakohta{$y^2-9$  }
		\alakohta{$x^2-4x+4$  }
		\alakohta{$x^4-5x^3+10x-50$  }						
		\end{alakohdat}
    \begin{vastaus}
		\begin{alakohdat}
		\alakohta{$x(4x+1)$}
		\alakohta{$5xy(x^2+2y)$}
		\alakohta{$(y+3)(y-3)$}
		\alakohta{$(x-2)^2$}
		\alakohta{$(x^3+10)(x-5)$}						
		\end{alakohdat}
    \end{vastaus}
\end{tehtava}

\begin{tehtava}
Sievennä muistikaavan avulla
    \begin{alakohdat}
            \alakohta{$(x^2-1)^2$} 
	        \alakohta{$(a^6+3b^3)^2$}
            \alakohta{$(-12-3x)(12-3x)$}
            \alakohta{$(x+\frac{1}{x})^2$}
    \end{alakohdat}
    \begin{vastaus}
        \begin{alakohdat}
            \alakohta{$x^4-2x^2+1$} 
            \alakohta{$a^{12}+6a^6b^3+9b^6$}
            \alakohta{$-(144-9x^2)=9x^2-144$}
            \alakohta{$x^2+2+\frac{1}{x^2}$}
         \end{alakohdat}
    \end{vastaus}
\end{tehtava}

\begin{tehtava} 
Ratkaise yhtälö \\ $(x-3)(x+2)(x-1)=0$.
    \begin{vastaus}
		$x=3$ tai $x=-2$ tai $x=1$.
    \end{vastaus}
\end{tehtava}

\begin{tehtava} 
Miksi polynomi $x^6+3x^2+5$ ei voi saada negatiivisia arvoja?
    \begin{vastaus}
		Koska $x^6\geq 0$ ja $x^2 \geq 0$. (Parilliset potenssit.)
    \end{vastaus}
\end{tehtava}

\begin{tehtava} 
		\begin{alakohdat}
		\alakohta{Osoita oikeaksi kaavat \\
$a^3-b^3=(a-b)(a^2+ab+b^2)$ ja \\
$a^3+b^3=(a+b)(a^2-ab+b^2)$}
		\alakohta{Jaa $x^6-y^6$ neljään tekijään.}						
		\end{alakohdat}
    \begin{vastaus}
		\begin{alakohdat}
		\alakohta{Opastus: Kerro sulut auki.}
		\alakohta{$(x-y)(x^2+xy+y^2)(x+y)(x^2-xy+y^2)$}						
		\end{alakohdat}
    \end{vastaus}
\end{tehtava}

\begin{tehtava} 
Kuvassa on polynomin $P(x)$ kuvaaja.
\begin{kuvaajapohja}{0.8}{-3}{3}{-2}{4}
				\kuvaaja{x*(x-1)*(x+2)}{}{black}
\end{kuvaajapohja}

		\begin{alakohdat}
		\alakohta{Mitkä ovat polynomin nollakohdat?}
		\alakohta{Millä muuttujan $x$ arvoilla $P(x)>0$?}
		\alakohta{Kuinka monta ratkaisua yhtälöllä $P(x)=1$ on?}
		\end{alakohdat}
    \begin{vastaus}
		\begin{alakohdat}
		\alakohta{$x=-2$, $x=0$ ja $x=1$}
		\alakohta{$-2<x<0$ tai $1 < x$}
		\alakohta{Vähintään 3. (Kuvaajassa näkyvän alueen 
		ulkopuolella voisi olla lisää.)}
		\end{alakohdat}
    \end{vastaus}
\end{tehtava}

\begin{tehtava} 
Jaa tekijöihin \\ $(a+b)^2-16$.
    \begin{vastaus}
		$(a+b+4)(a+b-4)$. \\
    Opastus: Älä kerro aluksi sulkuja auki. $16=4^2$
    \end{vastaus}
\end{tehtava}

\subsection*{Ensimmäinen aste}

\begin{tehtava} 
Ratkaise yhtälöt.
		\begin{alakohdat}
		\alakohta{$3x-5(x-2)=3-(-x)$ }
		\alakohta{$x(x-6) = x^2+5$  }
		\alakohta{$ \frac{2x}{3}-\frac{x-3}{4}=7$ }
		\end{alakohdat}
    \begin{vastaus}
		\begin{alakohdat}
		\alakohta{$x = \frac{7}{3}$}
		\alakohta{$x=-\frac{5}{6}$}
		\alakohta{$15$}			
		\end{alakohdat}
    \end{vastaus}
\end{tehtava}

\begin{tehtava} 
Mitä tarkoittaa
		\begin{alakohdat}
		\alakohta{$x \in [2,7]$ }
		\alakohta{$y \in ]-3,0]$  }
		\alakohta{$z \in ]-\infty, 5[$ ?}
		\end{alakohdat}
    \begin{vastaus}
		\begin{alakohdat}
		\alakohta{$2 \leq x \leq 7$}
		\alakohta{$-3 < y \leq 0$}
		\alakohta{$z < 5$}			
		\end{alakohdat}
    \end{vastaus}
\end{tehtava}

\begin{tehtava} 
Ratkaise epäyhtälöt.
		\begin{alakohdat}
		\alakohta{$-3x < 6$ }		
		\alakohta{$5x-2 > 7x+3$ }
		\alakohta{$ -2(x+3)  \leq x-(5-x)$  }
		\end{alakohdat}
    \begin{vastaus}
		\begin{alakohdat}
		\alakohta{$ x > -2$}	
		\alakohta{$x < -\frac{5}{2}=-2\frac{1}{2}$}
		\alakohta{$x \geq -\frac{1}{4}$}
		\end{alakohdat}
    \end{vastaus}
\end{tehtava}

\begin{tehtava} 
Polkupyörävuokraamo A laskuttaa pyörän vuokrasta $5$~\euro \ ja $2,5$~\euro \
jokaisesta täydestä tunnista. Vuokraamo B laskuttaa $11,50$~\euro \ ja $1,5$~\euro \ jokaisesta
täydestä tunnista. Kuinka monen tunnin vuokrassa vuokraamo B on edullisempi?
    \begin{vastaus}
	7 h tai sitä pidemmissä vuokrissa.
    \end{vastaus}
\end{tehtava}

\begin{tehtava} 
Ratkaise epäyhtälö \\
$ax \geq a-2x$ \\ parametrin $a$ kaikilla arvoilla.
    \begin{vastaus}
        $x \geq \frac{a}{a+2}$, kun $a > -2$ \\
        $x \leq \frac{a}{a+2}$, kun $a < -2$ \\
    $x \in \mathbb{R}$, kun $a = -2$ \\
	\end{vastaus}
\end{tehtava}

\subsection*{Toinen aste}

\begin{tehtava} 
Ratkaise yhtälöt.
		\begin{alakohdat}
		\alakohta{$x^2-13=0$ }
		\alakohta{$5x^+2x=0$  }
		\alakohta{$2x^2+5x-3=0$}
		\end{alakohdat}
    \begin{vastaus}
		\begin{alakohdat}
		\alakohta{$x= \pm \sqrt{13}$}
		\alakohta{$x=0$ tai $x=-\frac{2}{5}$}
		\alakohta{$x=-3$ tai $x= \frac{1}{2}$}			
		\end{alakohdat}
    \end{vastaus}
\end{tehtava}

\begin{tehtava} 
Ratkaise epäyhtälöt.
		\begin{alakohdat}
		\alakohta{$x^2-x-6<0$ }
		\alakohta{$x^2 \geq 5x$  }
		\end{alakohdat}
    \begin{vastaus}
		\begin{alakohdat}
		\alakohta{$ -2 < x < 3 $}
		\alakohta{$x \leq 0$ tai $x \geq 5$}
	\end{alakohdat}
    \end{vastaus}
\end{tehtava}

\begin{tehtava} 
Maijan ja Veeran ikien summa on 30. Ikien tulo on yli 186. Minkä ikäisiä tytöt voivat olla?
    \begin{vastaus}
	Vähintään 9, korkeintaan 21. Ratkeaa epäyhtälöstä $x(30-x)>186$.
	\end{vastaus}
\end{tehtava}

\begin{tehtava} 
Millä vakion $k$ arvolla yhtälöllä \\ $kx^2+kx=x-3$ on tasan yksi ratkaisu?
    \begin{vastaus}
		$k = 7 \pm 4 \sqrt{3}$. Diskriminantti on $D = (k-1)^2-4\cdot k \cdot 3$.
    \end{vastaus}
\end{tehtava}

\begin{tehtava} 
Polynomilla $P(x)=x^2-3x+c$ on tekijä $x+5$. Mikä on $c$?
    \begin{vastaus}
		$c=-40$
    \end{vastaus}
\end{tehtava}

\begin{tehtava} 
Suorakulmion $A$ sivut ovat $9$ ja $x^2+1$, suorakulmion $B$ sivut $5x+5$
ja $5-x$. Millä luvun $x$ arvoilla suorakulmion $A$ ala on suurempi?
    \begin{vastaus}
	$-1 < x < -\frac{4}{7}$ tai $2 < x < 5$. Huomioi, että suorakulmion $B$
    sivut ovat positiiviset vain, kun $-1<x<5$.
    \end{vastaus}
\end{tehtava}

\subsection*{Korkeampi aste}

\begin{tehtava} 
Ratkaise yhtälö $2x^7=5x^2$.
    \begin{vastaus}
		$x=0$ tai $x=\sqrt[5]{\frac{5}{2}}$
    \end{vastaus}
\end{tehtava}

\begin{tehtava} 
Anna esimerkki (jos mahdollista)
		\begin{alakohdat}
		\alakohta{4. asteen yhtälöstä, jolla ei ole ratkaisua }
		\alakohta{5. asteen yhtälöstä, jolla on tasan kaksi ratkaisua}
		\alakohta{3. asteen yhtälöstä, jolla on tasan neljä ratkaisua}
		\end{alakohdat}
    \begin{vastaus}
		\begin{alakohdat}
		\alakohta{esimerkiksi $x^4=-1$ }
		\alakohta{esimerkiksi $x^4(x+1)=0$ }
		\alakohta{mahdotonta}		
		\end{alakohdat}
    \end{vastaus}
\end{tehtava}

\begin{tehtava} 
Ratkaise yhtälö $x^4=x^2+6$.
    \begin{vastaus}
		$x=\pm \sqrt{3}$
    \end{vastaus}
\end{tehtava}

\begin{tehtava} 
Ratkaise yhtälö
$x^7=5x^5-x^6$.
     \begin{vastaus}
		$x=0$ tai $x\frac{-1 \pm \sqrt{21}}{2}$ 
    \end{vastaus}
\end{tehtava}

\begin{tehtava} % Korkeamman asteen epäyhtälö
Mitkä luvut ovat kuutiotaan suurempia?
    \begin{vastaus}
	Luvut, jotka ovat pienempiä kuin $-1$ ja luvut välillä $]0,1[$.
    \end{vastaus}
\end{tehtava}

\begin{tehtava} 
Ratkaise epäyhtälöt
		\begin{alakohdat}
		\alakohta{$x^4 < 5x $}
		\alakohta{$2x^3 \leq 3x-5x^2$}
		\end{alakohdat}
     \begin{vastaus}
		\begin{alakohdat}
		\alakohta{$0<x<\sqrt[3]{5}$}
		\alakohta{$x<-3$ tai $0<x<\frac{1}{2}$}
		\end{alakohdat}
    \end{vastaus}
\end{tehtava}

\begin{tehtava} 
Kuution tilavuus (kuutiometreinä) on sama kuin sen särmien pituuksien summa (metreinä). Kuinka pitkä on kuution särmä?
    \begin{vastaus}
		$\sqrt{12}$~m $\approx 3,46$~m
    \end{vastaus}
\end{tehtava}

\subsection*{Sekalaisia}

\begin{tehtava} 
Sievennä:
		\begin{alakohdat}
			\alakohta{$(a^2-1)^2+(a^2+1)^2-2(a^4+1)$}
			\alakohta{$(x+y)^2-4xy$}
		\end{alakohdat}
	\begin{vastaus}
		\begin{alakohdat}
			\alakohta{$0$}
			\alakohta{$(x-y)^2$}
		\end{alakohdat}
    \end{vastaus}
\end{tehtava}

\begin{tehtava} 
Ratkaise yhtälö
$2x^5-(x+6)^5=0$.
    \begin{vastaus}
	$x=\frac{6}{\sqrt[5]{2}-1}$
    \end{vastaus}
\end{tehtava}

\begin{tehtava} 
Suorakulmion piiri on 34 ja lävistäjä 13. Ratkaise suorakulmion sivut.
    \begin{vastaus}
	Sivut ovat $5$ ja $12$.
    \end{vastaus}
\end{tehtava}

\begin{tehtava} % Korkeamman asteen yhtälö
Etsi yhtälön $x^5-5x^4+6x^3-1=0$ kaikkien kolmen ratkaisun likiarvot
laskimella tai tietokoneen avulla. Anna vastaukset
kahden desimaalin tarkkuudella.
    \begin{vastaus}
	$x \approx 0,69$, $x \approx 1,86$ tai $x \approx 3,03$.
    \end{vastaus}
\end{tehtava}


\end{tehtavasivu}

