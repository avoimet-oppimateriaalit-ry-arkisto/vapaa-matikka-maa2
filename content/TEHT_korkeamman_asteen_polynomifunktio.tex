\begin{tehtavasivu}

\subsubsection*{Opi perusteet}

\begin{tehtava}
    Anna esimerkki
    \alakohdat{
	§ neljännen asteen polynomista, jolla on neljä nollakohtaa
	§ kolmannen asteen polynomista, jolla on kaksi nollakohtaa
	§ neljännen asteen polynomista, jolla on yksi nollakohta
	§ neljänen asteen polynomista, jolla ei ole nollakohtia
}
    \begin{vastaus}
	Esimerkiksi    
    \alakohdat{
	§ $P(x)=x(x-1)(x-2)(x-3)$ (nollakohdat $x= 0$, $x = 1$, $x=2$ ja $x=3$)
	§ $P(x)=x^2(x-1)$ (nollakohdat $x= 0$ ja $x = 1$)
	§ $Q(x)=x^4$ (nollakohta $x=0$)
	§ $R(x)=x^4+1$
}
    \end{vastaus}
\end{tehtava}

\begin{tehtava}
Alla on polynomin $P(x)$ kuvaaja. \\
\begin{kuvaajapohja}{1}{-3}{3}{-2}{5}
  \kuvaaja{-x**5+3*x**3+2}{$P(x)$}{black}
\end{kuvaajapohja} \\
Kaikki polynomin nollakohdat näkyvät kuvaajassa.
\alakohdat{
§ Mitä voidaan sanoa polynomin $P$ asteesta?
§ Mikä on polynomin $P$ vakiotermi?
}
\begin{vastaus}
\alakohdat{
§ Nollakohtia on kolme, joten polynomin aste on vähintään $3$. (Itse asiassa todellinen aste on $5$, mutta sitä on vaikea päätellä silmämääräisesti kuvaajasta.)
§ Vakiotermi on 2, koska $P(0)=2$
}
\end{vastaus}
\end{tehtava}

\subsubsection*{Hallitse kokonaisuus}

\begin{tehtava}
    Mikä on se kolmannen asteen polynomi, jonka nollakohdat ovat $x=2$, $x=-1$ ja $x=3$, ja jonka vakiotermi on $3$?
    \begin{vastaus}
        $P(x)=\frac{1}{2}(x-2)(x+1)(x-3)$
    \end{vastaus}
\end{tehtava}

\begin{tehtava}
    Kolmannen asteen polynomille $P$ pätee $P(1)=P(2)=P(3)=-2$ ja $P(0)=16$. Ratkaise $P(x)$.
    \begin{vastaus}
        $P(x)=-3(x-1)(x-2)(x-3)-2=-3x^3+18x^2-33x+18$
    \end{vastaus}
\end{tehtava}

\begin{tehtava}
   	$\star$ Osoita, että jos $n$ asteen polynomeilla $P(x)$ ja $Q(x)$ on $n+1$ yhteistä pistettä, ne ovat sama polynomi.
    \begin{vastaus}
        Tarkastele polynomien erotuksen nollakohtia.
    \end{vastaus}
\end{tehtava}

\end{tehtavasivu}